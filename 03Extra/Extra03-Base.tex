\documentclass[a4,11pt]{aleph-notas}

% -- Paquetes adicionales
\usepackage{enumitem}
\usepackage{aleph-comandos}
\usepackage{systeme}

% -- Datos  
\institucion{Escuela de Ciencias Físicas y Matemática}
\carrera{Ciencia de datos}
\asignatura{Álgebra lineal}
\tema{Material extra no. 3: Cómputo de bases}
\autor{Andrés Merino}
\fecha{Semestre 2024-1}

\logouno[0.14\textwidth]{Logos/logoPUCE_04_ac}
\definecolor{colortext}{HTML}{0030A1}
\definecolor{colordef}{HTML}{0030A1}
\fuente{montserrat}

% -- Comandos adicionales
\setlist[enumerate]{label=\roman*.}

\begin{document}

\encabezado

\begin{ejer}
    En $\R^3$, consideremos el subespacio vectorial
    \[
        E = \{(x,y,z)\in \R^3 : x+y+z=0 \}.
    \]
    Demuestre que 
    \[
        B = \{(-1,1,0) , (-1,0,1)\}
    \]
    es una base de $E$.
\end{ejer}

\begin{proof}
    Debemos demostrar que $B$ genera a $E$ y que $B$ es linealmente independiente.
    \begin{itemize}[leftmargin=*]
    \item
        Sea $(x,y,z)\in E$, se tiene que
        \[
            x+y+z=0.
        \]
        Debemos determinar si existen $\alpha,\beta\in\R$ tal que
        \[
            (x,y,z) = \alpha(-1,1,0)+\beta(-1,0,1),
        \]
        es decir, debemos determinar si el sistema
        \[
            \systeme[\alpha\beta]{
                -\alpha-\beta = x{,},
                \alpha = y{,},
                \beta = z{,}
            }
        \]
        tiene solución. La matriz ampliada asociada a este sistema es
        \[
            \begin{pmatrix}
                -1 & -1 & | & x\\
                1 & 0 & | & y\\
                0 & 1 & | & z
            \end{pmatrix}
        \]
        además,
        \[
            \begin{pmatrix}
                -1 & -1 & | & x\\
                1 & 0 & | & y\\
                0 & 1 & | & z
            \end{pmatrix}
            \sim
            \begin{pmatrix}
                1 & 1 & | & -x\\
                0 & 1 & | & -x-y\\
                0 & 0 & | & x+y+z
            \end{pmatrix}
            =
            \begin{pmatrix}
                1 & 1 & | & -x\\
                0 & 1 & | & -x-y\\
                0 & 0 & | & 0
            \end{pmatrix},
        \]
        por lo tanto, el sistema sí tiene solución, por lo tanto, 
        \[
            \spn(B) = E.
        \]
    \item
        Sean $\alpha,\beta\in\R$ tal que
        \[
            (0,0,0) = \alpha(-1,1,0)+\beta(-1,0,1),
        \]
        se tiene $\alpha,\beta$ son soluciones del sistema
        \[
            \systeme[\alpha\beta]{
                -\alpha-\beta = 0{,},
                \alpha = 0{,},
                \beta = 0{.}
            }
        \]
        La matriz ampliada asociada a este sistema es
        \[
            \begin{pmatrix}
                -1 & -1 & | & 0\\
                1 & 0 & | & 0\\
                0 & 1 & | & 0
            \end{pmatrix}
        \]
        además,
        \[
            \begin{pmatrix}
                -1 & -1 & | & 0\\
                1 & 0 & | & 0\\
                0 & 1 & | & 0
            \end{pmatrix}
            \sim
            \begin{pmatrix}
                1 & 0 & | & 0\\
                0 & 1 & | & 0\\
                0 & 0 & | & 0
            \end{pmatrix},
        \]
        por lo tanto, $\alpha=\beta=0$, de donde, se concluye que $B$ es linealmente independiente.
    \end{itemize}
    Con esto, se tiene que $B$ es una base para $E$.
\end{proof}

\begin{advertencia}
    Como podemos observar, los sistemas que se obtiene son similares, es más, basta demostrar que la solución del primer sistema es única, de ahí, se puede concluir que el conjunto genera y el linealmente independiente.
\end{advertencia}


\begin{ejer}
    En $\R_n[t]$, consideremos el subespacio vectorial
    \[
        E = \{at^2+b\in \R_n[t] : a,b\in\R \}.
    \]
    Demuestre que 
    \[
        B = \{t^2 , t^2+2\}
    \]
    es una base de $E$.
\end{ejer}

\begin{proof}
    Sea $p(t)\in E$, se tiene que
    \[
        p(t) = at^2+b
    \]
    con $a,b\in\R$. Debemos determinar si existen únicos $\alpha,\beta\in\R$ tal que
    \begin{align*}
        at^2+b 
            & = \alpha(t^2)+\beta(t^2+2)\\
            & = (\alpha+\beta)t^2 + 2\beta,
    \end{align*}
    es decir, debemos determinar si el sistema
    \[
        \systeme[\alpha\beta]{
            \alpha+\beta = a{,},
            2\beta = b{,}
        }
    \]
    tiene solución única. La matriz ampliada asociada a este sistema es
    \[
        \begin{pmatrix}
            1 & 1 & | & a\\
            0 & 2 & | & b
        \end{pmatrix}
    \]
    por lo tanto, el sistema sí tiene solución única, por lo tanto, $B$ es base para $E$.
\end{proof}


\begin{ejer}
    En $\R^3$, consideremos el subespacio vectorial
    \[
        E = \spn(D)
    \]
    donde
    \[
        D = \{v_1,v_2,v_3\} = \{(1,0,1), (1,1,1), (0,1,0)\}.
    \]
    Determinar una base para $E$.
\end{ejer}

\begin{proof}[Solución]
    Dado que $D$ genera a $E$, existe un subconjunto de $E$ que es base para $E$. Para esto, tomemos $\alpha,\beta,\gamma\in\R$ tales que
    \[
        (0,0,0) = \alpha(1,0,1)+\beta(1,1,1)+\gamma(0,1,0),
    \]
    es decir, $\alpha,\beta,\gamma\in\R$ tales que sean soluciones del sistema
    \[
        \systeme[\alpha\beta\gamma]{
        \alpha + \beta = 0{,},
        \beta+\gamma = 0{,},
        \alpha + \beta = 0{,}
        }
    \]
    La matriz ampliada asociada a este sistema es
    \[
        \begin{pmatrix}
            1 & 1 & 0 & | & 0\\
            0 & 1 & 1 & | & 0\\
            1 & 1 & 0 & | & 0
        \end{pmatrix}
    \]
    además,
    \[
        \begin{pmatrix}
            1 & 1 & 0 & | & 0\\
            0 & 1 & 1 & | & 0\\
            1 & 1 & 0 & | & 0
        \end{pmatrix}
        \sim
        \begin{pmatrix}
            1 & 1 & 0 & | & 0\\
            0 & 1 & 1 & | & 0\\
            0 & 0 & 0 & | & 0
        \end{pmatrix}.
    \]
    Dado que la matriz resultante tiene los unos principales en la primera y segunda columna, una base para $E$ puede estar conformada por
    \[
        \{v_1,v_2\}=\{(1,0,1), (1,1,1)\}.\qedhere
    \]
\end{proof}



\begin{ejer}
    En $\R^3$, consideremos el subespacio vectorial
    \[
        E = \{(x,y,z)\in \R^3 : x+y+z=0 \}.
    \]
    Determinar una base para $E$.
\end{ejer}

\begin{proof}[Solución]
    Sea $(x,y,z)\in E$, se tiene que
    \[
        x+y+z=0
    \]
    de donde se obtiene que 
    \[
        x = -y - z,
    \]
    por lo tanto
    \begin{align*}
        (x,y,z)
            & = (-y - z,y,z) \\
            & = y(-1,1,0) + z(-1,0,1).
    \end{align*}
    Con esto, se obtiene que
    \[
        \{(-1,1,0) , (-1,0,1)\}
    \]
    es un conjunto generador de $E$, a partir de este conjunto hallamos una base para $E$ (como en el ejercicio anterior), y obtenemos que 
    \[
        B = \{(-1,1,0) , (-1,0,1)\}
    \]
    es una base para $E$.
\end{proof}


\begin{ejer}
    En $\R_3[t]$, consideremos el subespacio vectorial
    \[
        E = \{at^3 + (a+b)t + 2b\in \R^3 : a,b\in\R \}.
    \]
    Determinar una base para $E$.
\end{ejer}

\begin{proof}[Solución]
    Sea $at^3 + (a+b)t + 2b\in E$, con $a,b\in\R$, se tiene que
    \begin{align*}
        at^3 + (a+b)t + 2b
            & = a(t^3+t) + b(t+2).
    \end{align*}
    Con esto, se obtiene que
    \[
        \{t^3+t , t+2\}
    \]
    es un conjunto generador de $E$, a partir de este conjunto hallamos una base para $E$ (como en el ejercicio anterior), y obtenemos que 
    \[
        B = \{t^3+t , t+2\}
    \]
    es una base para $E$.
\end{proof}



\end{document}