\documentclass[a4,11pt]{aleph-notas}

% -- Paquetes adicionales
\usepackage{enumitem}
\usepackage{aleph-comandos}
\usepackage{systeme}

% -- Datos  
\institucion{Escuela de Ciencias Físicas y Matemática}
\carrera{Ciencia de datos}
\asignatura{Álgebra lineal}
\tema{Material extra no. 2: Factorización de LU}
\autor{Andrés Merino}
\fecha{Semestre 2024-1}

\logouno[0.14\textwidth]{Logos/logoPUCE_04_ac}
\definecolor{colortext}{HTML}{0030A1}
\definecolor{colordef}{HTML}{0030A1}
\fuente{montserrat}

% -- Comandos adicionales
\setlist[enumerate]{label=\roman*.}

\begin{document}

\encabezado

\begin{ejer}
    Determinar, si existe, una factorización $LU$ de la matrix
    \[
        A = 
        \begin{pmatrix}
             2 &  1 & 1\\
             4 & -6 & 0\\
            -2 &  7 & 2
        \end{pmatrix}.
    \]
\end{ejer}

\begin{proof}[Solución]
    Empecemos determinando una matriz triangula superior $U$ que sea equivalente por filas a $A$ y en la cual solo utilicemos el múltiplo de una fila más otra 
    \begin{align*}
        \begin{pmatrix}
             2 &  1 & 1\\
             4 & -6 & 0\\
            -2 &  7 & 2
        \end{pmatrix}
        & \sim 
        \begin{pmatrix}
             2 &  1 & 1\\
             0 & -8 & -2\\
            -2 &  7 & 2
        \end{pmatrix}
        && -2F_1\leftrightarrow F_2\\
        & \sim 
        \begin{pmatrix}
             2 &  1 & 1\\
             0 & -8 & -2\\
             0 &  8 & 3
        \end{pmatrix}
        && 1F_1+F_3\to F_3\\
        & \sim 
        \begin{pmatrix}
             2 &  1 & 1\\
             0 & -8 & -2\\
             0 &  0 & 1
        \end{pmatrix}
        && 1F_2 + F_3 \to F_3
    \end{align*}

    Las matrices elementales correspondientes a las operaciones por filas realizadas son
    \[
        E_1 = 
        \begin{pmatrix}
             1 &  0 & 0\\
            -2 &  1 & 0\\
             0 &  0 & 1
        \end{pmatrix},
        \qquad
        E_2 = 
        \begin{pmatrix}
             1 &  0 & 0\\
             0 &  1 & 0\\
             1 &  0 & 1
        \end{pmatrix}
        \texty
        E_3 = 
        \begin{pmatrix}
             1 &  0 & 0\\
             0 &  1 & 0\\
             0 &  1 & 1
        \end{pmatrix}.
    \]
    Con esto, tenemos que
    \[
        U = E_3E_2E_1 A,
    \]
    así, podemos tomar
    \[
        L = (E_3E_2E_1)^{-1} = E_1^{-1}E_2^{-1}E_3^{-1} = 
        \begin{pmatrix}
             1 &  0 & 0\\
             2 &  1 & 0\\
            -1 & -1 & 1
        \end{pmatrix}.
    \]
    Con lo cual, obtenemos
    \[
        A = 
        \begin{pmatrix}
             2 &  1 & 1\\
             4 & -6 & 0\\
            -2 &  7 & 2
        \end{pmatrix}
        =
        \begin{pmatrix}
             1 &  0 & 0\\
             2 &  1 & 0\\
            -1 & -1 & 1
        \end{pmatrix}
        \begin{pmatrix}
             2 &  1 & 1\\
             0 & -8 & -2\\
             0 &  0 & 1
        \end{pmatrix}
        = LU.\qedhere
    \]
\end{proof}


\begin{advertencia}
    Notemos que las entradas de la matriz $L$ son los inversos aditivos de los múltiplos que fueron utilizados para ``eliminar'' las entradas bajo la diagonal correspondientes.
\end{advertencia}


%%%%%%%%%%%%%%%%%%%%%%%%%%%%%%%%%%%%%%%%
\begin{ejer}
    Dado el sistema de ecuaciones lineales:
    \[
        \begin{pmatrix}
             2 &  1 & 1\\
             4 & -6 & 0\\
            -2 &  7 & 2
        \end{pmatrix}
        \begin{pmatrix}
            x_1 \\ x_2 \\ x_3
        \end{pmatrix}
        =
        \begin{pmatrix}
            5 \\ -2 \\ 9
        \end{pmatrix}
    \]
    determinar su solución utilizando factorización $LU$.
\end{ejer}

\begin{proof}[Solución]
    Tenemos que las matrices
    \[
        A = \begin{pmatrix}
            0&-1&3\\1&1&1\\1&-1&-1
        \end{pmatrix},
        \qquad
        x = \begin{pmatrix}
            x_1 \\ x_2\\ x_3
        \end{pmatrix}.
        \texty
        b = \begin{pmatrix}
            5 \\ -2 \\ 9
        \end{pmatrix}.
    \]
    Así, el sistema en forma matricial es
    \[
        Ax = b.
    \]
    Dado que $A$ tiene factorización $LU$ con
    \[
        L = 
        \begin{pmatrix}
             1 &  0 & 0\\
             2 &  1 & 0\\
            -1 & -1 & 1
        \end{pmatrix}
        \texty
        U = 
        \begin{pmatrix}
             2 &  1 & 1\\
             0 & -8 & -2\\
             0 &  0 & 1
        \end{pmatrix},
    \]
    resolvamos el sistema
    \[
        Ly = b,
    \]
    cuya matriz ampliada es
    \[
        \begin{pmatrix}
             1 &  0 & 0 & | & 5 \\
             2 &  1 & 0 & | & -2\\
            -1 & -1 & 1 & | & 9
        \end{pmatrix}.
    \]
   Podemos resolver este sistema utilizando sustitución progresiva y obtenemos
    \[
        y =
        \begin{pmatrix}
            5 \\ -12 \\ 2
        \end{pmatrix}.
    \]
    Ahora, resolvemos el sistema
    \[
        Ux = y
    \]
    cuya matriz ampliada es
    \[
        \begin{pmatrix}
             2 &  1 & 1 & | & 5 \\
            0  & -8 &-2 & | & -12\\
             0 &  0 & 1 & | & 2
        \end{pmatrix}.
    \]
   Podemos resolver este sistema utilizando sustitución regresiva y obtenemos
    \[
        x =
        \begin{pmatrix}
            1 \\ 1 \\ 2
        \end{pmatrix}.\qedhere
    \]
\end{proof}



\end{document}