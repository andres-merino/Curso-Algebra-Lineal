\documentclass[a4,11pt]{aleph-notas}

% -- Paquetes adicionales
\usepackage{enumitem}
\usepackage{aleph-comandos}
\usepackage{systeme}

% -- Datos  
\institucion{Escuela de Ciencias Físicas y Matemática}
\carrera{Ciencia de datos}
\asignatura{Álgebra lineal}
\tema{Material extra no. 1: Eliminación de Gauss-Jordan}
\autor{Andrés Merino}
\fecha{Semestre 2024-1}

\logouno[0.14\textwidth]{Logos/logoPUCE_04_ac}
\definecolor{colortext}{HTML}{0030A1}
\definecolor{colordef}{HTML}{0030A1}
\fuente{montserrat}

% -- Comandos adicionales
\setlist[enumerate]{label=\roman*.}

\begin{document}

\encabezado

\begin{ejer}
    Dado el sistema de ecuaciones lineales:
    \[
        \systeme{
        -v+3w=1, u+v+w=1, u-v-w=-1
        }
    \]
    utilizar la eliminación de Gauss-Jordan para determinar el conjunto de soluciones del sistema.
\end{ejer}

\begin{proof}[Solución]
    Tenemos que las matrices
    \[
        A = \begin{pmatrix}
            0&-1&3\\1&1&1\\1&-1&-1
        \end{pmatrix}
        \texty
        b = \begin{pmatrix}
            1 \\ 1\\ -1
        \end{pmatrix}.
    \]
    Así, el sistema en forma matricial es
    \[
        \begin{pmatrix}
            0&-1&3\\1&1&1\\1&-1&-1
        \end{pmatrix}
        \begin{pmatrix}
            u \\ v\\ w
        \end{pmatrix}
        =
        \begin{pmatrix}
            1 \\ 1\\ -1
        \end{pmatrix}.
    \]
    De donde, la matriz ampliada es
    \[
        \begin{pmatrix}
            0&-1&3&1&|&1\\
            1&1&1&1&|&1\\
            1&-1&-1&1&|&-1
        \end{pmatrix}
    \]
    
    Ahora, procedemos a realizar la eliminación de Gauss-Jordan sobre la matriz ampliada:
    \begin{align*}
        \begin{pmatrix}
            0&-1&3&|&1\\
            1&1&1&|&1\\
            1&-1&-1&|&-1
        \end{pmatrix}
        & \sim 
        \begin{pmatrix}
            1&1&1&|&1\\
            0&-1&3&|&1\\
            1&-1&-1&|&-1
        \end{pmatrix}
        && F_1\leftrightarrow F_2\\
        & \sim 
        \begin{pmatrix}
            1&1&1&|&1\\
            0&-1&3&|&1\\
            0&-2&-2&|&-2
        \end{pmatrix}
        && -1F_1+F_3\to F_3\\
        & \sim 
        \begin{pmatrix}
            1&1&1&|&1\\
            0&1&-3&|&-1\\
            0&-2&-2&|&-2
        \end{pmatrix}
        && -1F_2 \to F_2\\
        & \sim 
        \begin{pmatrix}
            1&0&4&|&2\\
            0&1&-3&|&-1\\
            0&-2&-2&|&-2
        \end{pmatrix}
        && -1F_2+F_1\to F_1\\
        & \sim 
        \begin{pmatrix}
            1&0&4&|&2\\
            0&1&-3&|&-1\\
            0&0&-8&|&-4
        \end{pmatrix}
        && 2F_2+F_3\to F_3\\
        & \sim 
        \begin{pmatrix}
            1&0&4&|&2\\
            0&1&-3&|&-1\\
            0&0&1&|&\frac 1 2
        \end{pmatrix}
        && -\frac 1 8F_3 \to F_3\\
        & \sim
        \begin{pmatrix}
            1&0&0&|&0\\
            0&1&-3&|&-1\\
            0&0&1&|&\frac 1 2
        \end{pmatrix}
        && -4F_3+F_1\to F_1\\
        & \sim 
        \begin{pmatrix}
            1&0&0&|&0\\
            0&1&0&|&\frac 1 2\\
            0&0&1&|&\frac 1 2
        \end{pmatrix}
        && 3F_3+F_2\to F_2
    \end{align*}

    Con esto, tenemos que
    \[
        \begin{pmatrix}
            0&-1&3&|&1\\
            1&1&1&|&1\\
            1&-1&-1&|&-1
        \end{pmatrix}
        \sim
        \begin{pmatrix}
            1&0&0&|&0\\
            0&1&0&|&\frac 1 2\\
            0&0&1&|&\frac 1 2
        \end{pmatrix},
    \]
    con esto, tenemos que $\rang(A)=\rang(A|b)$, de donde el sistema tiene solución, además, $\rang(A)=3$, por lo tanto, la solución es única. Además, los sistemas
    \[
        \systeme{
        -v+3w=1, u+v+w=1, u-v-w=-1
        }
    \]
    y
    \[
        \systeme{
        u=0,v=\frac 1 2 , w = \frac 1 2
        }
    \]
    tienen las mismas soluciones, es decir, la solución es $u=0$, $v=\frac 1 2$ y $w=\frac 1 2$. Es decir, el conjunto de soluciones del sistema es
    \[  
        \{(0,1/2,1/2)\}.\qedhere
    \]
\end{proof}

%%%%%%%%%%%%%%%%%%%%%%%%%%%%%%%%%%%%%%%%
\begin{ejer}
    Dado el sistema de ecuaciones lineales:
    \[
        \systeme{
        -v+3w=1, u+v+w=1
        }
    \]
    utilizar la eliminación de Gauss-Jordan para determinar el conjunto de soluciones del sistema.
\end{ejer}

\begin{proof}[Solución]
    Tenemos que las matrices
    \[
        A = \begin{pmatrix}
            0&-1&3\\1&1&1\\1&-1&-1
        \end{pmatrix}
        \texty
        b = \begin{pmatrix}
            1 \\ 1\\ -1
        \end{pmatrix}.
    \]
    Así, el sistema en forma matricial es
    \[
        \begin{pmatrix}
            0&-1&3\\1&1&1\\1&-1&-1
        \end{pmatrix}
        \begin{pmatrix}
            u \\ v\\ w
        \end{pmatrix}
        =
        \begin{pmatrix}
            1 \\ 1\\ -1
        \end{pmatrix}.
    \]
    De donde, la matriz ampliada es
    \[
        \begin{pmatrix}
            0&-1&3&1&|&1\\
            1&1&1&1&|&1\\
            1&-1&-1&1&|&-1
        \end{pmatrix}
    \]
    
    Ahora, luego de realizar la eliminación de Gauss-Jordan sobre la matriz ampliada, tenemos
    \[
        \begin{pmatrix}
            0&-1&3&|&1\\
            1&1&1&|&1
        \end{pmatrix}
        \sim 
        \begin{pmatrix}
            1&0&4&|&2\\
            0&1&-3&|&-1
        \end{pmatrix},
    \]
    con esto, tenemos que $\rang(A)=\rang(A|b)$, de donde el sistema tiene solución, además, $\rang(A)=2<3$, por lo tanto, tiene infinitas soluciones. Además, los sistemas
    \[
        \systeme{
        -v+3w=1, u+v+w=1
        }
    \]
    y
    \[
        \systeme{
        u+4w=2,v-3w=-1
        }
    \]
    tienen las mismas soluciones, así, se tiene que
    \begin{align*}
        u & = 2-4\alpha\\
        v & = -1+3\alpha\\
        w & = \alpha,
    \end{align*}
    es solución del sistema para todo $\alpha\in\R$, de donde el conjunto de soluciones del sistema es
    \[  
        \{(2-4\alpha,-1+3\alpha,\alpha):\alpha\in\R\}.\qedhere
    \]
\end{proof}


\end{document}