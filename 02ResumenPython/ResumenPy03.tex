\documentclass[a4,11pt]{aleph-notas}

% -- Paquetes adicionales
\usepackage{enumitem}
\usepackage{aleph-comandos}


% -- Datos  
\institucion{Escuela de Ciencias Físicas y Matemática}
\carrera{Ciencia de datos}
\asignatura{Álgebra lineal}
\tema{Python no. 3: Cofactores, Vectores en el plano y $R^n$}
\autor{Andrés Merino}
\fecha{Semestre 2024-1}

\logouno[0.14\textwidth]{Logos/logoPUCE_04_ac}
\definecolor{colortext}{HTML}{0030A1}
\definecolor{colordef}{HTML}{0030A1}
\fuente{montserrat}


% -- Comandos adicionales
\setlist[enumerate]{label=\roman*.}
% Caga del paquete
\usepackage{listings}

% Recuadro para código
\definecolor{cellborder}{HTML}{CFCFCF}
\definecolor{cellbackground}{HTML}{F7F7F7}
\newtcolorbox{ipynbcodigo}
    {breakable, size=fbox, boxrule=1pt, pad at break*=1mm,colback=cellbackground, colframe=cellborder,top=0mm,bottom=-3.3mm, before upper=\LineaIn}
% Recuadro para salida
\newtcolorbox{ipynbsalida}[1][0mm]
    {breakable, size=fbox, boxrule=1pt, pad at break*=1mm,colback=cellbackground, colframe=cellborder,top=#1,bottom=0mm, before upper=\footnotesize\LineaOut}
% Recuadro para código
\definecolor{colcod}{RGB}{174,218,255}
\newtcolorbox{pycodigo}
    {icono=\faKeyboardO,color=colcod,
     postit,left=15mm,bottom=1.8mm}
% Texto de In y Out
\definecolor{incolor}{HTML}{303F9F}
\definecolor{outcolor}{HTML}{D84315}
\newcommand{\LineaIn}{
    \llap{\scriptsize\ttfamily\color{incolor}[In]:\hspace{6pt}}%
    \vspace{-1.24\baselineskip}
    }
\newcommand{\LineaOut}{
    \llap{\scriptsize\ttfamily\color{outcolor}[Out]:\hspace{6pt}}%
    \vspace{-0.95\baselineskip}
    }
    

% Estilo para python
\lstloadlanguages{Python}

% Configuración de estilo
\definecolor{codverde}{rgb}{0,0.6,0}
\definecolor{codgris}{rgb}{0.5,0.5,0.5}
\definecolor{coduva}{rgb}{0.58,0,0.82}
\definecolor{codazul}{rgb}{0.0,0,0.8}
\lstset{
    language=Python,
    basicstyle=\small\ttfamily,
    stringstyle=\color{coduva},
    commentstyle=\color{codgris},
    emph={[2]from,import,pass,return}, emphstyle={[2]\color{codverde}},
    emph={[3]range,print,display,Matrix}, emphstyle={[3]\color{codazul}},
    emph={[4]for,in,def}, emphstyle={[4]\color{blue}},
    breaklines=true,
    prebreak=\mbox{{\color{gray}\tiny$\searrow$}},
    extendedchars=true,
    literate=   {á}{{\'a}}1
                {é}{{\'e}}1 
                {í}{{\'i}}1 
                {ó}{{\'o}}1 
                {ú}{{\'u}}1 
                {ñ}{{\~n}}1,
    showstringspaces=false,
}

\usepackage{fancyvrb} % verbatim replacement that allows latex


\begin{document}

\encabezado

\vspace*{-10mm}
%%%%%%%%%%%%%%%%%%%%%%%%%%%%%%%%%%%%%%
\section{Cofactores}
%%%%%%%%%%%%%%%%%%%%%%%%%%%%%%%%%%%%%%

Podemos calcular los cofactores de una matriz con el método \texttt{cofactor}:

\begin{pycodigo}
    \begin{ipynbcodigo}\begin{lstlisting}[language=Python]
A = Matrix([[1, 2, 3], [4, 5, 6], [-1, 0, 2]])
print("Matriz A:")
display(A)
print("Cofactor 1 2 de A:")
display(A.cofactor(0,1))
    \end{lstlisting}\end{ipynbcodigo}
    %
    \begin{ipynbsalida}
    \begin{Verbatim}[commandchars=\\\{\}]
Matriz A:
    \end{Verbatim}

    $\displaystyle \left[\begin{matrix}1 & 2 & 3\\4 & 5 & 6\\-1 & 0 & 2\end{matrix}\right]$

    
    \begin{Verbatim}[commandchars=\\\{\}]
Cofactor 1 2 de A:
    \end{Verbatim}

    $\displaystyle -14$
    \end{ipynbsalida}
\end{pycodigo}
    
Si queremos la matriz de cofactores, usamos el método \texttt{cofactor\_matrix}:
\begin{pycodigo}
    \begin{ipynbcodigo}\begin{lstlisting}[language=Python]
print("Matriz de cofactores de A:")
display(A.cofactor_matrix())
    \end{lstlisting}\end{ipynbcodigo}
    %
    \begin{ipynbsalida}
    \begin{Verbatim}[commandchars=\\\{\}]
Matriz de cofactores de A:
    \end{Verbatim}

    $\displaystyle \left[\begin{matrix}10 & -14 & 5\\-4 & 5 & -2\\-3 & 6 & -3\end{matrix}\right]$
    \end{ipynbsalida}
\end{pycodigo}

Finalmente, la matriz adjunta se obtiene con el método \texttt{adjugate}:
\begin{pycodigo}
    \begin{ipynbcodigo}\begin{lstlisting}[language=Python]
print("Matriz de adjunta de A:")
display(A.adjugate())
    \end{lstlisting}\end{ipynbcodigo}
    %
    \begin{ipynbsalida}
    \begin{Verbatim}[commandchars=\\\{\}]
Matriz de adjunta de A:
    \end{Verbatim}

    $\displaystyle \left[\begin{matrix}10 & -4 & -3\\-14 & 5 & 6\\5 & -2 & -3\end{matrix}\right]$
    \end{ipynbsalida}
\end{pycodigo}

%%%%%%%%%%%%%%%%%%%%%%%%%%%%%%%%%%%%%%
\section{El espacio $\mathbb{R}^n$}
%%%%%%%%%%%%%%%%%%%%%%%%%%%%%%%%%%%%%%

Podemos generar vectores como matrices de una columna:
\begin{pycodigo}
    \begin{ipynbcodigo}\begin{lstlisting}[language=Python]
x = Matrix([1, 2, 3])
print("El vector x:")
display(x)
y = Matrix([-1, 0, 2])
print("El vector y:")
display(y)
    \end{lstlisting}\end{ipynbcodigo}
    %
    \begin{ipynbsalida}
    \begin{Verbatim}[commandchars=\\\{\}]
El vector x:
    \end{Verbatim}

    $\displaystyle \left[\begin{matrix}1\\2\\3\end{matrix}\right]$

    
    \begin{Verbatim}[commandchars=\\\{\}]
El vector y:
    \end{Verbatim}

    $\displaystyle \left[\begin{matrix}-1\\0\\2\end{matrix}\right]$
    \end{ipynbsalida}
\end{pycodigo}

Para calcular el producto punto entre dos vectores, tenemos el método \texttt{dot}:
\begin{pycodigo}
    \begin{ipynbcodigo}\begin{lstlisting}[language=Python]
x.dot(y)
    \end{lstlisting}\end{ipynbcodigo}
    %
    \begin{ipynbsalida}[2mm]
$\displaystyle 5$
    \end{ipynbsalida}
\end{pycodigo}

Para la norma, podemos usar el método \texttt{norm}:
\begin{pycodigo}
    \begin{ipynbcodigo}\begin{lstlisting}[language=Python]
x.norm()
    \end{lstlisting}\end{ipynbcodigo}
    %
    \begin{ipynbsalida}
$\displaystyle \sqrt{14}$
    \end{ipynbsalida}
\end{pycodigo}

Finalmente, para el producto cruz, tenemos el método \texttt{cross}:
\begin{pycodigo}
    \begin{ipynbcodigo}\begin{lstlisting}[language=Python]
x.cross(y)
    \end{lstlisting}\end{ipynbcodigo}
    %
    \begin{ipynbsalida}
$\displaystyle \left[\begin{matrix}4\\-5\\2\end{matrix}\right]$
    \end{ipynbsalida}
\end{pycodigo}



\end{document}











\begin{pycodigo}
    \begin{ipynbcodigo}\begin{lstlisting}[language=Python]
Codigo
    \end{lstlisting}\end{ipynbcodigo}
    %
    \begin{ipynbsalida}
Salida
    \end{ipynbsalida}
\end{pycodigo}