\documentclass[a4,11pt]{aleph-notas}

% -- Paquetes adicionales
\usepackage{enumitem}
\usepackage{aleph-comandos}


% -- Datos  
\institucion{Escuela de Ciencias Físicas y Matemática}
\carrera{Ciencia de datos}
\asignatura{Álgebra lineal}
\tema{Python no. 1: Matrices y sistemas de ecuaciones}
\autor{Andrés Merino}
\fecha{Semestre 2024-1}

\logouno[0.14\textwidth]{Logos/logoPUCE_04_ac}
\definecolor{colortext}{HTML}{0030A1}
\definecolor{colordef}{HTML}{0030A1}
\fuente{montserrat}


% -- Comandos adicionales
\setlist[enumerate]{label=\roman*.}
% Caga del paquete
\usepackage{listings}

% Recuadro para código
\definecolor{cellborder}{HTML}{CFCFCF}
\definecolor{cellbackground}{HTML}{F7F7F7}
\newtcolorbox{ipynbcodigo}
    {breakable, size=fbox, boxrule=1pt, pad at break*=1mm,colback=cellbackground, colframe=cellborder,top=0mm,bottom=-3.3mm, before upper=\LineaIn}
% Recuadro para salida
\newtcolorbox{ipynbsalida}[1][0mm]
    {breakable, size=fbox, boxrule=1pt, pad at break*=1mm,colback=cellbackground, colframe=cellborder,top=#1,bottom=0mm, before upper=\footnotesize\LineaOut}
% Recuadro para código
\definecolor{colcod}{RGB}{174,218,255}
\newtcolorbox{pycodigo}
    {icono=\faKeyboardO,color=colcod,
     postit,left=15mm,bottom=1.8mm}
% Texto de In y Out
\definecolor{incolor}{HTML}{303F9F}
\definecolor{outcolor}{HTML}{D84315}
\newcommand{\LineaIn}{
    \llap{\scriptsize\ttfamily\color{incolor}[In]:\hspace{6pt}}%
    \vspace{-1.24\baselineskip}
    }
\newcommand{\LineaOut}{
    \llap{\scriptsize\ttfamily\color{outcolor}[Out]:\hspace{6pt}}%
    \vspace{-0.95\baselineskip}
    }
    

% Estilo para python
\lstloadlanguages{Python}

% Configuración de estilo
\definecolor{codverde}{rgb}{0,0.6,0}
\definecolor{codgris}{rgb}{0.5,0.5,0.5}
\definecolor{coduva}{rgb}{0.58,0,0.82}
\definecolor{codazul}{rgb}{0.0,0,0.8}
\lstset{
    language=Python,
    basicstyle=\small\ttfamily,
    stringstyle=\color{coduva},
    commentstyle=\color{codgris},
    emph={[2]from,import,pass,return}, emphstyle={[2]\color{codverde}},
    emph={[3]range,print,display,Matrix}, emphstyle={[3]\color{codazul}},
    emph={[4]for,in,def}, emphstyle={[4]\color{blue}},
    breaklines=true,
    prebreak=\mbox{{\color{gray}\tiny$\searrow$}},
    extendedchars=true,
    literate=   {á}{{\'a}}1
                {é}{{\'e}}1 
                {í}{{\'i}}1 
                {ó}{{\'o}}1 
                {ú}{{\'u}}1 
                {ñ}{{\~n}}1,
    showstringspaces=false,
}

\usepackage{fancyvrb} % verbatim replacement that allows latex


\begin{document}

\encabezado

\vspace*{-10mm}
%%%%%%%%%%%%%%%%%%%%%%%%%%%%%%%%%%%%%%
\section{Matrices}
%%%%%%%%%%%%%%%%%%%%%%%%%%%%%%%%%%%%%%


Una manera para trabajar con matrices en Python, es utilizando la librería \texttt{sympy}, con la siguiente línea de comandos:

\begin{pycodigo}
    \begin{ipynbcodigo}\begin{lstlisting}[language=Python]
from sympy import *
    \end{lstlisting}\end{ipynbcodigo}
\end{pycodigo}

Con esto, podemos almacenar y visualizar una matriz de la siguiente manera:
\begin{pycodigo}
    \begin{ipynbcodigo}\begin{lstlisting}[language=Python]
A = Matrix([[1, 2, 3], [4, 5, 6], [-1, 0, 2]])
display(A)
    \end{lstlisting}\end{ipynbcodigo}
    %
    \begin{ipynbsalida}[2mm]
        $\displaystyle \left[\begin{matrix}1 & 2 & 3\\4 & 5 & 6\\-1 & 0 & 2\end{matrix}\right]$
    \end{ipynbsalida}
\end{pycodigo}

    
Cuando se quiera visualizar un único objeto, basta con escribir sunombre al final de la celda:
\begin{pycodigo}
    \begin{ipynbcodigo}\begin{lstlisting}[language=Python]
A
    \end{lstlisting}\end{ipynbcodigo}
    %
    \begin{ipynbsalida}[2mm]
        $\displaystyle \left[\begin{matrix}1 & 2 & 3\\4 & 5 & 6\\-1 & 0 & 2\end{matrix}\right]$
    \end{ipynbsalida}
\end{pycodigo}
    
Para obtener un elemento de la matriz realizamos lo siguiente:
\begin{pycodigo}
    \begin{ipynbcodigo}\begin{lstlisting}[language=Python]
# Entrada de la primera fila y primera columna
A[0, 0]
    \end{lstlisting}\end{ipynbcodigo}
    %
    \begin{ipynbsalida}
        $1$
    \end{ipynbsalida}
    %
    \begin{ipynbcodigo}\begin{lstlisting}[language=Python]
# Entrada de la segunda fila y primera columna
A[1, 0]
    \end{lstlisting}\end{ipynbcodigo}
    %
    \begin{ipynbsalida}
        $4$
    \end{ipynbsalida}
\end{pycodigo}

Para obtener una fila o columna de una matriz realizamos lo siguiente:
\begin{pycodigo}
    \begin{ipynbcodigo}\begin{lstlisting}[language=Python]
# Primera fila
A[0, :]
    \end{lstlisting}\end{ipynbcodigo}
    %
    \begin{ipynbsalida}[2mm]
        $\displaystyle \left[\begin{matrix}1 & 2 & 3\end{matrix}\right]$
    \end{ipynbsalida}
    %
    \begin{ipynbcodigo}\begin{lstlisting}[language=Python]
# Segunda fila
A[1, :]
    \end{lstlisting}\end{ipynbcodigo}
    %
    \begin{ipynbsalida}[2mm]
        $\displaystyle \left[\begin{matrix}4 & 5 & 6\end{matrix}\right]$
    \end{ipynbsalida}
    %
    \begin{ipynbcodigo}\begin{lstlisting}[language=Python]
# Primera columna
A[:, 0]
    \end{lstlisting}\end{ipynbcodigo}
    %
    \begin{ipynbsalida}[2mm]
        $\displaystyle \left[\begin{matrix}1\\4\\-1\end{matrix}\right]$
    \end{ipynbsalida}
    %
    \begin{ipynbcodigo}\begin{lstlisting}[language=Python]
# Segunda columna
A[:, 1]
    \end{lstlisting}\end{ipynbcodigo}
    %
    \begin{ipynbsalida}[2mm]
        $\displaystyle \left[\begin{matrix}2\\5\\0\end{matrix}\right]$
    \end{ipynbsalida}
\end{pycodigo}

Otra forma válida es:
\begin{pycodigo}
    \begin{ipynbcodigo}\begin{lstlisting}[language=Python]
# Primera fila
A.row(0)
    \end{lstlisting}\end{ipynbcodigo}
    %
    \begin{ipynbsalida}[2mm]
        $\displaystyle \left[\begin{matrix}1 & 2 & 3\end{matrix}\right]$
    \end{ipynbsalida}
    %
    \begin{ipynbcodigo}\begin{lstlisting}[language=Python]
# Segunda columna
A.col(1)
    \end{lstlisting}\end{ipynbcodigo}
    %
    \begin{ipynbsalida}[2mm]
        $\displaystyle \left[\begin{matrix}2\\5\\0\end{matrix}\right]$
    \end{ipynbsalida}
\end{pycodigo}

Para comparar la igualdad entre dos matrices, podemos usar el operador \texttt{==}:

\begin{pycodigo}
    \begin{ipynbcodigo}\begin{lstlisting}[language=Python]
# Definimos las matrices
A = Matrix([[1, 0], [0, 1]])
print("A:")
display(A)
B = Matrix([[1, 2], [3, 4]])
print("B:")
display(B)
C = Matrix([[1, 2]])
print("C:")
display(C)
# Comparamos las matrices
print("Es A igual a B: ", A == B)
print("Es B igual a C: ", B == C)
print("Es A igual a A: ", A == A)
    \end{lstlisting}\end{ipynbcodigo}
    %
    \begin{ipynbsalida}
    \begin{Verbatim}
A:
    \end{Verbatim}
    $\displaystyle \left[\begin{matrix}1 & 0\\0 & 1\end{matrix}\right]$
    \begin{Verbatim}
B:
    \end{Verbatim}
    $\displaystyle \left[\begin{matrix}1 & 2\\3 & 4\end{matrix}\right]$
    \begin{Verbatim}
C:
    \end{Verbatim}
    $\displaystyle \left[\begin{matrix}1 & 2\end{matrix}\right]$
    \begin{Verbatim}
Es A igual a B:  False
Es B igual a C:  False
Es A igual a A:  True
    \end{Verbatim}
    \end{ipynbsalida}
\end{pycodigo}

%%%%%%%%%%%%%%%%%%%%%%%%%%%%%%%%%%%%%%
\section{Operaciones de matrices}
%%%%%%%%%%%%%%%%%%%%%%%%%%%%%%%%%%%%%%

Podemos realizar las distintas operaciones de matrices de la siguiente manera:

\begin{pycodigo}
    \begin{ipynbcodigo}\begin{lstlisting}[language=Python]
# Suma de matrices
A+B
    \end{lstlisting}\end{ipynbcodigo}
    %
    \begin{ipynbsalida}[2mm]
        $\displaystyle \left[\begin{matrix}2 & 2\\3 & 5\end{matrix}\right]$
    \end{ipynbsalida}
    %
    \begin{ipynbcodigo}\begin{lstlisting}[language=Python]
# Multiplicación por un escalar
2*B
    \end{lstlisting}\end{ipynbcodigo}
    %
    \begin{ipynbsalida}[2mm]
        $\displaystyle \left[\begin{matrix}2 & 4\\6 & 8\end{matrix}\right]$
    \end{ipynbsalida}
    %
    \begin{ipynbcodigo}\begin{lstlisting}[language=Python]
# Transpuesta
B.T
    \end{lstlisting}\end{ipynbcodigo}
    %
    \begin{ipynbsalida}[2mm]
        $\displaystyle \left[\begin{matrix}1 & 3\\2 & 4\end{matrix}\right]$
    \end{ipynbsalida}
\end{pycodigo}

%%%%%%%%%%%%%%%%%%%%%%%%%%%%%%%%%%%%%%
\subsection{Multiplicación de matrices}

Definamos algunas matrices:
\begin{pycodigo}
    \begin{ipynbcodigo}\begin{lstlisting}[language=Python]
A = Matrix([[1, 2, 3], [3, 2, 1], [1, -1, 0]])
print("A:")
display(A)
B = Matrix([[1, 0, 2], [2, 1, 0], [0, 1, 0]])
print("B:")
display(B)
    \end{lstlisting}\end{ipynbcodigo}
    %
    \begin{ipynbsalida}
    \begin{Verbatim}
A:
    \end{Verbatim}
    $\displaystyle \left[\begin{matrix}1 & 2 & 3\\3 & 2 & 1\\1 & -1 & 0\end{matrix}\right]$
    \begin{Verbatim}
B:
    \end{Verbatim}
    $\displaystyle \left[\begin{matrix}1 & 0 & 2\\2 & 1 & 0\\0 & 1 & 0\end{matrix}\right]$
    \end{ipynbsalida}
\end{pycodigo}

La multiplicación es:
\begin{pycodigo}
    \begin{ipynbcodigo}\begin{lstlisting}[language=Python]
A*B
    \end{lstlisting}\end{ipynbcodigo}
    %
    \begin{ipynbsalida}[2mm]
        $\displaystyle \left[\begin{matrix}5 & 5 & 2\\7 & 3 & 6\\-1 & -1 & 2\end{matrix}\right]$
    \end{ipynbsalida}
\end{pycodigo}


Podemos elevar una matriz a un exponente con el operador \texttt{**}:
\begin{pycodigo}
    \begin{ipynbcodigo}\begin{lstlisting}[language=Python]
print('A^0:')
display(A**0)
print('A^1:')
display(A**1)
print('A^2:')
display(A**2)
    \end{lstlisting}\end{ipynbcodigo}
    %
    \begin{ipynbsalida}[2mm]
    \begin{Verbatim}
A^0:
    \end{Verbatim}

    $\displaystyle \left[\begin{matrix}1 & 0 & 0\\0 & 1 & 0\\0 & 0 & 1\end{matrix}\right]$

    
    \begin{Verbatim}
A^1:
    \end{Verbatim}

    $\displaystyle \left[\begin{matrix}1 & 2 & 3\\3 & 2 & 1\\1 & -1 & 0\end{matrix}\right]$

    
    \begin{Verbatim}
A^2:
    \end{Verbatim}

    $\displaystyle \left[\begin{matrix}10 & 3 & 5\\10 & 9 & 11\\-2 & 0 & 2\end{matrix}\right]$
    \end{ipynbsalida}
\end{pycodigo}
 
\section{Tipos de matrices}\label{tipos-de-matrices}

Podemos obtener las dimensiones de una matriz con el siguiente código:

\begin{pycodigo}
    \begin{ipynbcodigo}\begin{lstlisting}[language=Python]
# Definimos las matrices
A = Matrix([[1, 0, 1], [0, 1, 1]])
print("A:")
display(A)
print("Las dimensiones de A son:", A.shape)
print("El número de filas de A es:", A.shape[0])
print("El número de columnas de A es:", A.shape[1])
    \end{lstlisting}\end{ipynbcodigo}
    %
    \begin{ipynbsalida}
    \begin{Verbatim}
A:
    \end{Verbatim}
    $\displaystyle \left[\begin{matrix}1 & 0 & 1\\0 & 1 & 1\end{matrix}\right]$
    \begin{Verbatim}
Las dimensiones de A son: (2, 3)
El número de filas de A es: 2
El número de columnas de A es: 3
    \end{Verbatim}
    \end{ipynbsalida}
\end{pycodigo}

Ahora, veamos cómo crear algunas matrices especiales:
\begin{itemize}
\item
    Matriz diagonal:
\begin{pycodigo}
    \begin{ipynbcodigo}\begin{lstlisting}[language=Python]
diag(1, 2, 3)
    \end{lstlisting}\end{ipynbcodigo}
    %
    \begin{ipynbsalida}[2mm]
$\displaystyle \left[\begin{matrix}1 & 0 & 0\\0 & 2 & 0\\0 & 0 & 3\end{matrix}\right]$
    \end{ipynbsalida}
\end{pycodigo}

\item
    Matriz identidad:
\begin{pycodigo}
    \begin{ipynbcodigo}\begin{lstlisting}[language=Python]
eye(4)
    \end{lstlisting}\end{ipynbcodigo}
    %
    \begin{ipynbsalida}[2mm]
$\displaystyle \left[\begin{matrix}1 & 0 & 0 & 0\\0 & 1 & 0 & 0\\0 & 0 & 1 & 0\\0 & 0 & 0 & 1\end{matrix}\right]$
    \end{ipynbsalida}
\end{pycodigo}

\item
    Matriz nula:
\begin{pycodigo}
    \begin{ipynbcodigo}\begin{lstlisting}[language=Python]
zeros(2, 4)
    \end{lstlisting}\end{ipynbcodigo}
    %
    \begin{ipynbsalida}[2mm]
$\displaystyle \left[\begin{matrix}0 & 0 & 0 & 0\\0 & 0 & 0 & 0\end{matrix}\right]$
    \end{ipynbsalida}
\end{pycodigo}

\item
    Matriz formada de unos:
\begin{pycodigo}
    \begin{ipynbcodigo}\begin{lstlisting}[language=Python]
ones(4, 2)
    \end{lstlisting}\end{ipynbcodigo}
    %
    \begin{ipynbsalida}[2mm]
$\displaystyle \left[\begin{matrix}1 & 1\\1 & 1\\1 & 1\\1 & 1\end{matrix}\right]$
    \end{ipynbsalida}
\end{pycodigo}
\end{itemize}


%%%%%%%%%%%%%%%%%%%%%%%%%%%%%%%%%%%%%%%%
\section{Sistemas de ecuaciones}
%%%%%%%%%%%%%%%%%%%%%%%%%%%%%%%%%%%%%%%%

 Podemos generar una matriz ampliada de la siguiente manera:
\begin{pycodigo}
    \begin{ipynbcodigo}\begin{lstlisting}[language=Python]
# Defino las matrices
A = Matrix([[1, 2, 3], [3, 2, 1], [1, -1, 0]])
print("A:")
display(A)
B = Matrix([[1, 0, 2], [2, 1, 0], [0, 1, 0]])
print("B:")
display(B)
# Genero la matriz ampliada
print("(A|B):")
Matrix(BlockMatrix([A, B]))
    \end{lstlisting}\end{ipynbcodigo}
    %
    \begin{ipynbsalida}
    \begin{Verbatim}
A:
    \end{Verbatim}
$\displaystyle \left[\begin{matrix}1 & 2 & 3\\3 & 2 & 1\\1 & -1 & 0\end{matrix}\right]$
    \begin{Verbatim}
B:
    \end{Verbatim}
$\displaystyle \left[\begin{matrix}1 & 0 & 2\\2 & 1 & 0\\0 & 1 & 0\end{matrix}\right]$

    \begin{Verbatim}
(A|B):
    \end{Verbatim}
$\displaystyle \left[\begin{matrix}1 & 2 & 3 & 1 & 0 & 2\\3 & 2 & 1 & 2 & 1 & 0\\1 & -1 & 0 & 0 & 1 & 0\end{matrix}\right]$
    \end{ipynbsalida}
\end{pycodigo}

%%%%%%%%%%%%%%%%%%%%%%%%%%%%%%%%%%%%%%%%
\subsection{Operaciones por filas}

\begin{itemize}
\item
    Intercambio de filas:
\begin{pycodigo}
    \begin{ipynbcodigo}\begin{lstlisting}[language=Python]
# Imprimo la matriz original
print("A:")
display(A)
# Operación de filas
print("Intercambio la fila 0 con la 2:")
A.elementary_row_op('n<->m', row1=0, row2=2)
    \end{lstlisting}\end{ipynbcodigo}
    %
    \begin{ipynbsalida}
    \begin{Verbatim}
A:
    \end{Verbatim}
    $\displaystyle \left[\begin{matrix}1 & 2 & 3\\3 & 2 & 1\\1 & -1 & 0\end{matrix}\right]$
    \begin{Verbatim}
Intercambio la fila 0 con la 2:
    \end{Verbatim}
    $\displaystyle \left[\begin{matrix}1 & -1 & 0\\3 & 2 & 1\\1 & 2 & 3\end{matrix}\right]$
    \end{ipynbsalida}
\end{pycodigo}

\item
    Multiplicar una fila por un escalar:
\begin{pycodigo}
    \begin{ipynbcodigo}\begin{lstlisting}[language=Python]
# Imprimo la matriz original
print("A:")
display(A)
# Operación de filas
print("Multiplico por 3 la fila 2:")
A.elementary_row_op('n->kn', k=3, row=2)
    \end{lstlisting}\end{ipynbcodigo}
    %
    \begin{ipynbsalida}
    \begin{Verbatim}
A:
    \end{Verbatim}
    $\displaystyle \left[\begin{matrix}1 & 2 & 3\\3 & 2 & 1\\1 & -1 & 0\end{matrix}\right]$
    \begin{Verbatim}
Multiplico por 3 la fila 2:
    \end{Verbatim}
    $\displaystyle \left[\begin{matrix}1 & 2 & 3\\3 & 2 & 1\\3 & -3 & 0\end{matrix}\right]$
    \end{ipynbsalida}
\end{pycodigo}

\item
    Sumar un múltiplo de una fila con otra:
\begin{pycodigo}
    \begin{ipynbcodigo}\begin{lstlisting}[language=Python]
# Imprimo la matriz original
print("A:")
display(A)
# Operación de filas
print("Multiplico por 3 la fila 0 y la sumo a la fila 2:")
A.elementary_row_op('n->n+km', k=3, row1=2, row2=0)
    \end{lstlisting}\end{ipynbcodigo}
    %
    \begin{ipynbsalida}
    \begin{Verbatim}
A:
    \end{Verbatim}
    $\displaystyle \left[\begin{matrix}1 & -2 & 1\\2 & 0 & -4\\3 & 1 & -2\end{matrix}\right]$
    \begin{Verbatim}
Multiplico por 3 la fila 0 y la sumo a la fila 2:
    \end{Verbatim}
    $\displaystyle \left[\begin{matrix}1 & -2 & 1\\2 & 0 & -4\\6 & -5 & 1\end{matrix}\right]$
    \end{ipynbsalida}
\end{pycodigo}

\end{itemize}

%%%%%%%%%%%%%%%%%%%%%%%%%%%%%%%%%%%%%%%%
\subsubsection{Ejemplo de Gauss-Jordan}

En caso de cometer algún error en uno de los pasos, deberá compilarse todas las celdas desde este punto.

\begin{pycodigo}
    \begin{ipynbcodigo}\begin{lstlisting}[language=Python]
# Defino la matriz
A = Matrix([[1, -2, 1], [2, 0, -4], [3, 1, -2]])
print("A:")
display(A)
# Guardo la matriz en una temporal
temp = A
    \end{lstlisting}\end{ipynbcodigo}
    %
    \begin{ipynbsalida}
    \begin{Verbatim}
A:
    \end{Verbatim}
    $\displaystyle \left[\begin{matrix}1 & -2 & 1\\2 & 0 & -4\\3 & 1 & -2\end{matrix}\right]$
    \end{ipynbsalida}
\end{pycodigo}


\begin{itemize}
\item
    Primer paso: Multiplico por \(-2\) la primera fila y lo sumo a la segunda.
\begin{pycodigo}
    \begin{ipynbcodigo}\begin{lstlisting}[language=Python]
temp = temp.elementary_row_op('n->n+km', k=-2, row1=1, row2=0)
display(temp)
    \end{lstlisting}\end{ipynbcodigo}
    %
    \begin{ipynbsalida}[2mm]
$\displaystyle \left[\begin{matrix}1 & -2 & 1\\0 & 4 & -6\\3 & 1 & -2\end{matrix}\right]$
    \end{ipynbsalida}
\end{pycodigo}

\item
    Segundo paso: Multiplico por \(-3\) la primera fila y lo sumo a la tercera.
\begin{pycodigo}
    \begin{ipynbcodigo}\begin{lstlisting}[language=Python]
temp = temp.elementary_row_op('n->n+km', k=-3, row1=2, row2=0)
display(temp)
    \end{lstlisting}\end{ipynbcodigo}
    %
    \begin{ipynbsalida}[2mm]
$\displaystyle \left[\begin{matrix}1 & -2 & 1\\0 & 4 & -6\\0 & 7 & -5\end{matrix}\right]$
    \end{ipynbsalida}
\end{pycodigo}

\item
    Tercer paso: multiplico por \(\frac{1}{4}\) la segunda fila.
\begin{pycodigo}
    \begin{ipynbcodigo}\begin{lstlisting}[language=Python]
temp = temp.elementary_row_op('n->kn', k=1/4, row=1)
display(temp)
    \end{lstlisting}\end{ipynbcodigo}
    %
    \begin{ipynbsalida}[2mm]
$\displaystyle \left[\begin{matrix}1 & -2 & 1\\0 & 1.0 & -1.5\\0 & 7 & -5\end{matrix}\right]$
    \end{ipynbsalida}
\end{pycodigo}

\item
    Cuarto paso: Multiplico por \(-7\) la segunda fila y lo sumo a la tercera.
\begin{pycodigo}
    \begin{ipynbcodigo}\begin{lstlisting}[language=Python]
temp = temp.elementary_row_op('n->n+km', k=-7, row1=2, row2=1)
display(temp)
    \end{lstlisting}\end{ipynbcodigo}
    %
    \begin{ipynbsalida}[2mm]
$\displaystyle \left[\begin{matrix}1 & -2 & 1\\0 & 1.0 & -1.5\\0 & 0 & 5.5\end{matrix}\right]$
    \end{ipynbsalida}
\end{pycodigo}

\item
    Quinto paso: Multiplico por \(\frac{1}{5.5}\) la tercera fila.
\begin{pycodigo}
    \begin{ipynbcodigo}\begin{lstlisting}[language=Python]
temp = temp.elementary_row_op('n->kn', k=1/5.5, row=2)
display(temp)
    \end{lstlisting}\end{ipynbcodigo}
    %
    \begin{ipynbsalida}[2mm]
$\displaystyle \left[\begin{matrix}1 & -2 & 1\\0 & 1.0 & -1.5\\0 & 0 & 1.0\end{matrix}\right]$
    \end{ipynbsalida}
\end{pycodigo}
\end{itemize}

%%%%%%%%%%%%%%%%%%%%%%%%%%%%%%%%%%%%%%%%
\subsubsection{Funciones para reducir una matriz}

\begin{pycodigo}
    \begin{ipynbcodigo}\begin{lstlisting}[language=Python]
# Defino la matriz
A = Matrix([[1, -2, 1], [2, 0, -4], [3, 1, -2]])
print("A:")
display(A)
    \end{lstlisting}\end{ipynbcodigo}
    %
    \begin{ipynbsalida}
    \begin{Verbatim}
A:
    \end{Verbatim}
    $\displaystyle \left[\begin{matrix}1 & -2 & 1\\2 & 0 & -4\\3 & 1 & -2\end{matrix}\right]$
    \end{ipynbsalida}
\end{pycodigo}

    
\begin{itemize}
\item
    Forma escalonada sin unos principales.
\begin{pycodigo}
    \begin{ipynbcodigo}\begin{lstlisting}[language=Python]
A.echelon_form()
    \end{lstlisting}\end{ipynbcodigo}
    %
    \begin{ipynbsalida}[2mm]
$\displaystyle \left[\begin{matrix}1 & -2 & 1\\0 & 4 & -6\\0 & 0 & 22\end{matrix}\right]$
    \end{ipynbsalida}
\end{pycodigo}

\item
    Forma escalonada reducida por fila.
\begin{pycodigo}
    \begin{ipynbcodigo}\begin{lstlisting}[language=Python]
A.rref(pivots=False)
    \end{lstlisting}\end{ipynbcodigo}
    %
    \begin{ipynbsalida}[2mm]
$\displaystyle \left[\begin{matrix}1 & 0 & 0\\0 & 1 & 0\\0 & 0 & 1\end{matrix}\right]$
    \end{ipynbsalida}
\end{pycodigo}

\item
    Rango de una matriz.
\begin{pycodigo}
    \begin{ipynbcodigo}\begin{lstlisting}[language=Python]
A.rank()
    \end{lstlisting}\end{ipynbcodigo}
    %
    \begin{ipynbsalida}
\begin{Verbatim}
3
\end{Verbatim}
    \end{ipynbsalida}
\end{pycodigo}
\end{itemize}

%%%%%%%%%%%%%%%%%%%%%%%%%%%%%%%%%%%%%%%%
\subsection{Resolución de sistemas de ecuaciones}

Consideremos el sistema de ecuaciones \(Ax=b\), donde 
\[
    A = \begin{pmatrix}1 & 2 & 1\\0 & 1 & 2\\2 & 4 & 0\end{pmatrix}
    \quad\text{y}\quad
    b=\begin{pmatrix}7\\12\\4\end{pmatrix}.
\]
Podemos resolverlo con el comando \texttt{solve} de la siguiente manera:
\begin{pycodigo}
    \begin{ipynbcodigo}\begin{lstlisting}[language=Python]
# Defino las matrices
A = Matrix([[1, 2, 1], [0, 1, 2], [2, 4, 0]])
b = Matrix([7, 12, 4])
# Solución
A.solve(b)
    \end{lstlisting}\end{ipynbcodigo}
    %
    \begin{ipynbsalida}[2mm]
$\displaystyle \left[\begin{matrix}-2\\2\\5\end{matrix}\right]$
    \end{ipynbsalida}
\end{pycodigo}

Sin embargo, si el sistema no tiene solución única, nos dará un error. En estos casos, podemos utilizar \texttt{gauss\_jordan\_solve}:

\begin{pycodigo}
    \begin{ipynbcodigo}\begin{lstlisting}[language=Python]
# Solución
sol, par = A.gauss_jordan_solve(b)
sol
    \end{lstlisting}\end{ipynbcodigo}
    %
    \begin{ipynbsalida}[2mm]
$\displaystyle \left[\begin{matrix}-2\\2\\5\end{matrix}\right]$
    \end{ipynbsalida}
\end{pycodigo}

Este comando, a más de determinar la solución, presenta los parámetros que esta tendrá en el caso de no tener solución única. Para esto, consideremos el sistema de ecuaciones \(Ax=b\), con 
\[
    A=\begin{pmatrix}1 & 2 & 1 & 1\\1 & 2 & 2 & -1\\2 & 4 & 0 & 6\end{pmatrix}
    \quad\text{y}\quad
    b=\begin{pmatrix}7\\12\\4\end{pmatrix}.
\]

\begin{pycodigo}
    \begin{ipynbcodigo}\begin{lstlisting}[language=Python]
# Definimos la matrices
A = Matrix([[1, 2, 1, 1], [1, 2, 2, -1], [2, 4, 0, 6]])
b = Matrix([7, 12, 4])
# Solución
sol, par = A.gauss_jordan_solve(b)
print("Solución:")
display(sol)
print("Parámetros:")
display(par)
    \end{lstlisting}\end{ipynbcodigo}
    %
    \begin{ipynbsalida}[2mm]
    \begin{Verbatim}
Solución:
    \end{Verbatim}
    $\displaystyle \left[\begin{matrix}- 2 \tau_{0} - 3 \tau_{1} + 2\\\tau_{0}\\2 \tau_{1} + 5\\\tau_{1}\end{matrix}\right]$
    \begin{Verbatim}
Parámetros:
    \end{Verbatim}
    $\displaystyle \left[\begin{matrix}\tau_{0}\\\tau_{1}\end{matrix}\right]$
    \end{ipynbsalida}
\end{pycodigo}



    



\end{document}