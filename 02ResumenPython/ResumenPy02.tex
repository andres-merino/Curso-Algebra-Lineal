\documentclass[a4,11pt]{aleph-notas}

% -- Paquetes adicionales
\usepackage{enumitem}
\usepackage{aleph-comandos}


% -- Datos  
\institucion{Escuela de Ciencias Físicas y Matemática}
\carrera{Ciencia de datos}
\asignatura{Álgebra lineal}
\tema{Python no. 2: Inversa de matrices y Determinantes}
\autor{Andrés Merino}
\fecha{Semestre 2024-1}

\logouno[0.14\textwidth]{Logos/logoPUCE_04_ac}
\definecolor{colortext}{HTML}{0030A1}
\definecolor{colordef}{HTML}{0030A1}
\fuente{montserrat}


% -- Comandos adicionales
\setlist[enumerate]{label=\roman*.}
% Caga del paquete
\usepackage{listings}

% Recuadro para código
\definecolor{cellborder}{HTML}{CFCFCF}
\definecolor{cellbackground}{HTML}{F7F7F7}
\newtcolorbox{ipynbcodigo}
    {breakable, size=fbox, boxrule=1pt, pad at break*=1mm,colback=cellbackground, colframe=cellborder,top=0mm,bottom=-3.3mm, before upper=\LineaIn}
% Recuadro para salida
\newtcolorbox{ipynbsalida}[1][0mm]
    {breakable, size=fbox, boxrule=1pt, pad at break*=1mm,colback=cellbackground, colframe=cellborder,top=#1,bottom=0mm, before upper=\footnotesize\LineaOut}
% Recuadro para código
\definecolor{colcod}{RGB}{174,218,255}
\newtcolorbox{pycodigo}
    {icono=\faKeyboardO,color=colcod,
     postit,left=15mm,bottom=1.8mm}
% Texto de In y Out
\definecolor{incolor}{HTML}{303F9F}
\definecolor{outcolor}{HTML}{D84315}
\newcommand{\LineaIn}{
    \llap{\scriptsize\ttfamily\color{incolor}[In]:\hspace{6pt}}%
    \vspace{-1.24\baselineskip}
    }
\newcommand{\LineaOut}{
    \llap{\scriptsize\ttfamily\color{outcolor}[Out]:\hspace{6pt}}%
    \vspace{-0.95\baselineskip}
    }
    

% Estilo para python
\lstloadlanguages{Python}

% Configuración de estilo
\definecolor{codverde}{rgb}{0,0.6,0}
\definecolor{codgris}{rgb}{0.5,0.5,0.5}
\definecolor{coduva}{rgb}{0.58,0,0.82}
\definecolor{codazul}{rgb}{0.0,0,0.8}
\lstset{
    language=Python,
    basicstyle=\small\ttfamily,
    stringstyle=\color{coduva},
    commentstyle=\color{codgris},
    emph={[2]from,import,pass,return}, emphstyle={[2]\color{codverde}},
    emph={[3]range,print,display,Matrix}, emphstyle={[3]\color{codazul}},
    emph={[4]for,in,def}, emphstyle={[4]\color{blue}},
    breaklines=true,
    prebreak=\mbox{{\color{gray}\tiny$\searrow$}},
    extendedchars=true,
    literate=   {á}{{\'a}}1
                {é}{{\'e}}1 
                {í}{{\'i}}1 
                {ó}{{\'o}}1 
                {ú}{{\'u}}1 
                {ñ}{{\~n}}1,
    showstringspaces=false,
}

\usepackage{fancyvrb} % verbatim replacement that allows latex


\begin{document}

\encabezado

\vspace*{-10mm}
%%%%%%%%%%%%%%%%%%%%%%%%%%%%%%%%%%%%%%
\section{Inversa de una matriz}
%%%%%%%%%%%%%%%%%%%%%%%%%%%%%%%%%%%%%%

Podemos calcular la inversa de una matriz con el método \texttt{inv}:

\begin{pycodigo}
    \begin{ipynbcodigo}\begin{lstlisting}[language=Python]
A = Matrix([[1, 2, 3], [4, 5, 6], [-1, 0, 2]])
display(A)
display(A.inv())
    \end{lstlisting}\end{ipynbcodigo}
    %
    \begin{ipynbsalida}[2mm]
$\displaystyle \left[\begin{matrix}1 & 2 & 3\\4 & 5 & 6\\-1 & 0 & 2\end{matrix}\right]$

$\displaystyle \left[\begin{matrix}- \frac{10}{3} & \frac{4}{3} & 1\\\frac{14}{3} & - \frac{5}{3} & -2\\- \frac{5}{3} & \frac{2}{3} & 1\end{matrix}\right]$
    \end{ipynbsalida}
\end{pycodigo}
    
Si la matriz no tiene inversa, obtenemos el siguiente error:
\begin{pycodigo}
    \begin{ipynbcodigo}\begin{lstlisting}[language=Python]
B = Matrix([[1, 2], [1, 2]])
display(B)
display(B.inv())
    \end{lstlisting}\end{ipynbcodigo}
    %
    \begin{ipynbsalida}[2mm]
$\displaystyle \left[\begin{matrix}1 & 2\\1 & 2\end{matrix}\right]$

\begin{Verbatim}[commandchars=\\\{\}]
\textcolor{magenta}{NonInvertibleMatrixError}  Traceback (most recent call last)
.
.
.
\textcolor{magenta}{NonInvertibleMatrixError}: Matrix det == 0; not invertible.
    \end{Verbatim}
    \end{ipynbsalida}
\end{pycodigo}


%%%%%%%%%%%%%%%%%%%%%%%%%%%%%%%%%%%%%%
\section{Factorizaciones}
%%%%%%%%%%%%%%%%%%%%%%%%%%%%%%%%%%%%%%

Podemos generar la descomposición LU con el método \texttt{LUdecomposition}:
\begin{pycodigo}
    \begin{ipynbcodigo}\begin{lstlisting}[language=Python]
A = Matrix([[1, 2, 3], [4, 5, 6], [-1, 0, 2]])
display(A)
L, U, _= A.LUdecomposition()
display(L)
display(U)
    \end{lstlisting}\end{ipynbcodigo}
    %
    \begin{ipynbsalida}[2mm]
    $\displaystyle \left[\begin{matrix}1 & 2 & 3\\4 & 5 & 6\\-1 & 0 & 2\end{matrix}\right]$

    
    $\displaystyle \left[\begin{matrix}1 & 0 & 0\\4 & 1 & 0\\-1 & - \frac{2}{3} & 1\end{matrix}\right]$

    
    $\displaystyle \left[\begin{matrix}1 & 2 & 3\\0 & -3 & -6\\0 & 0 & 1\end{matrix}\right]$
    \end{ipynbsalida}
\end{pycodigo}

En este código, hemos colocado un \texttt{\_} pues este método devuelve tres valores, pero el tercero no es de interés por el momento.

También la descomposición QR con el método \texttt{QRdecomposition}:
\begin{pycodigo}
    \begin{ipynbcodigo}\begin{lstlisting}[language=Python]
A = Matrix([[1, 2, 3], [4, 5, 6], [-1, 0, 2]])
display(A)
Q, R = A.QRdecomposition()
display(Q)
display(R)
    \end{lstlisting}\end{ipynbcodigo}
    %
    \begin{ipynbsalida}[2mm]
    $\displaystyle \left[\begin{matrix}1 & 2 & 3\\4 & 5 & 6\\-1 & 0 & 2\end{matrix}\right]$

    
    $\displaystyle \left[\begin{matrix}\frac{\sqrt{2}}{6} & \frac{7 \sqrt{19}}{57} & - \frac{5 \sqrt{38}}{38}\\\frac{2 \sqrt{2}}{3} & \frac{\sqrt{19}}{57} & \frac{\sqrt{38}}{19}\\- \frac{\sqrt{2}}{6} & \frac{11 \sqrt{19}}{57} & \frac{3 \sqrt{38}}{38}\end{matrix}\right]$

    
    $\displaystyle \left[\begin{matrix}3 \sqrt{2} & \frac{11 \sqrt{2}}{3} & \frac{25 \sqrt{2}}{6}\\0 & \frac{\sqrt{19}}{3} & \frac{49 \sqrt{19}}{57}\\0 & 0 & \frac{3 \sqrt{38}}{38}\end{matrix}\right]$
    \end{ipynbsalida}
\end{pycodigo}


%%%%%%%%%%%%%%%%%%%%%%%%%%%%%%%%%%%%%%
\section{Determinantes}
%%%%%%%%%%%%%%%%%%%%%%%%%%%%%%%%%%%%%%

Podemos obtener el menor de una matriz con el siguiente código:
\begin{pycodigo}
    \begin{ipynbcodigo}\begin{lstlisting}[language=Python]
A = Matrix([[1, 2, 3], [4, 5, 6], [-1, 0, 2]])
display(A)
# Menor 1,1
print("El menor A_11 es")
display(A.minor_submatrix(0,0))
# Menor 1,1
print("El menor A_23 es")
display(A.minor_submatrix(1,2))
    \end{lstlisting}\end{ipynbcodigo}
    %
    \begin{ipynbsalida}[2mm]
$\displaystyle \left[\begin{matrix}1 & 2 & 3\\4 & 5 & 6\\-1 & 0 & 2\end{matrix}\right]$

    
    \begin{Verbatim}[commandchars=\\\{\}]
El menor A\_11 es
    \end{Verbatim}

    $\displaystyle \left[\begin{matrix}5 & 6\\0 & 2\end{matrix}\right]$

    
    \begin{Verbatim}[commandchars=\\\{\}]
El menor A\_23 es
    \end{Verbatim}

    $\displaystyle \left[\begin{matrix}1 & 2\\-1 & 0\end{matrix}\right]$
    \end{ipynbsalida}
\end{pycodigo}

Finalmente, podemos calcular el determinante de una matriz con el método
\texttt{det}:
\begin{pycodigo}
    \begin{ipynbcodigo}\begin{lstlisting}[language=Python]
A.det()
    \end{lstlisting}\end{ipynbcodigo}
    %
    \begin{ipynbsalida}
-3
    \end{ipynbsalida}
\end{pycodigo}





\end{document}

\begin{pycodigo}
    \begin{ipynbcodigo}\begin{lstlisting}[language=Python]
Codigo
    \end{lstlisting}\end{ipynbcodigo}
    %
    \begin{ipynbsalida}
Salida
    \end{ipynbsalida}
\end{pycodigo}