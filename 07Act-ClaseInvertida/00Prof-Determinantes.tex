\documentclass[a4,11pt]{aleph-notas}

% -- Paquetes adicionales
\usepackage{enumitem}
\usepackage{aleph-comandos}
\usepackage{aleph-moodle}

% -- Datos  
\institucion{Escuela de Ciencias Físicas y Matemática}
\carrera{Bioingeniería}
\asignatura{Álgebra lineal}
\tema{Clase invertida no. 1: Cálculo de determinantes}
\autor[A. Merino]{Andrés Merino}
\fecha{Semestre 2024-1}

\logouno[0.14\textwidth]{Logos/logoPUCE_04_ac}
\definecolor{colortext}{HTML}{0030A1}
\definecolor{colordef}{HTML}{0030A1}
\fuente{montserrat}

% -- Comandos adicionales
\begin{document}

\encabezado

\vspace*{-10mm}
\section*{Introducción}

\begin{itemize}
    \item \textbf{Tema:} Determinantes
    \item \textbf{Resultado de Aprendizaje:} Calcula determinantes de matrices usando menores.
\end{itemize}

\section{Lección en casa}

\subsection{Adquisición de concepto}

Para la adquisición del concepto, se solicitará al estudiante interactuar con ChatGPT mediante la siguiente serie de \textit{prompts} específicos:

\begin{enumerate}[label=\textit{Prompt \arabic*.},leftmargin=2.1cm]
    \item Vas a ser mi profesor de la asignatura de Álgebra Lineal, te iré dando indicaciones y me irás explicando de manera formal y luego de manera intuitiva los conceptos. Vas a tener mucho cuidado al escribir la parte matemática para que se visualice bien. ¿Entendido?
    \item Dada una matriz A, ¿qué es la matriz (i,j)-menor que la llamaremos A\_ij? Dame un ejemplo revisado de manera cuidadosa para A\_11 y A\_12.
    \item ¿Qué es el determinante de una matriz? Quiero solo la idea intuitiva.
    \item ¿Cómo se calcula el determinante de una matriz de 1$\times$1? Dame un ejemplo numérico.
    \item ¿Cómo se calcula el determinante de una matriz de 2$\times$2 usando expansión por menores de la primera fila? Dame un ejemplo numérico.
    \item ¿Cómo se calcula el determinante de una matriz de 3$\times$3 usando expansión por menores de la primera fila? Dame un ejemplo numérico.
    \item Explícame el cálculo del determinante (usando expansión por menores de la primera fila) de la siguiente matriz\\
    1 5 3\\
    -4 8 9\\
    4 2 1
    \item Evalúame para determinar si he comprendido. Escríbeme una pregunta y te daré la respuesta, luego me darás retroalimentación y proseguirás con otra pregunta.
\end{enumerate}

\subsection{Personalización de la actividad}

Se la consigue solicitando al estudiante continuar la interacción hasta que se sienta preparado en el cálculo de determinantes.

\subsection{Solventación de dudas}

En caso de tener dudas sobre el tema, se solicitará al estudiante interactuar con sus compañeros de clase para solventarlas.

\subsection{Micro-tarea}

Para realizar un seguimiento de la actividad, se solicitará al estudiante copiar el enlace del chat como evidencia del proceso. Adicionalmente, se le pedirá realizar el cuestionario del aula virtual. El cuestionario se encuentra detallado en el Anexo.

\section{Tareas en clase}

\subsection{Visión conjunta}

Se muestra la relación entre las actividades realizadas en casa y las tareas a realizar en clase. De manera específica, cómo el cálculo de determinantes se relaciona con sus diferentes propiedades y con la inversión de matrices.

\subsection{Retroalimentación}

Se brinda retroalimentación a los estudiantes sobre las respuestas dadas en la micro-tarea.

\subsection{Actividad de aplicación}

Se solicitará a los estudiantes resolver los siguientes ejercicios:
\begin{enumerate}
    \item Calcular el determinante de la matriz 
    \[
        A = \begin{pmatrix}
            1 & -4\\
            4 & 5 \
        \end{pmatrix}
    \] 
    \item Calcular el determinante de la matriz
    \[
        B = \begin{pmatrix}
            2 & -1 & 3\\
            0 & 4 & 5\\
            1 & 2 & 3
        \end{pmatrix}
    \]
    \item Calcular el determinante de la matriz
    \[
        C = \begin{pmatrix}
            1 & 2 & 0 & 0\\
            -4 & 1 & 2 & 3\\
            0 & 0 & 1 & 2\\
            0 & 0 & 0 & 1
        \end{pmatrix}
    \]
\end{enumerate}

\subsection{Micro-evaluación}

No se realizará micro-evaluación en esta clase.


\section*{Anexo}

\begin{quiz}{Cálculo de Determinantes 2 por 2}
    
\begin{numerical}[tolerance=0]%
    % - Identificador
    {DeterminantesCalc-01}
    % - Enunciado
    Calcular el determinante de la matriz 
    \[
        A = \begin{pmatrix}
            1 & -4\\
            4 & 5 
        \end{pmatrix}
    \] 
    \item 21
\end{numerical}

\begin{numerical}[tolerance=0]%
    % - Identificador
    {DeterminantesCalc-03}
    % - Enunciado
    Calcular el determinante de la matriz 
    \[
        C = \begin{pmatrix}
            2 & 3\\
            4 & 1 
        \end{pmatrix}
    \]
    \item -10
\end{numerical}

\begin{numerical}[tolerance=0]%
    % - Identificador
    {DeterminantesCalc-04}
    % - Enunciado
    Calcular el determinante de la matriz 
    \[
        D = \begin{pmatrix}
            -1 & 2\\
            3 & -4 
        \end{pmatrix}
    \]
    \item -10
\end{numerical}

\begin{numerical}[tolerance=0]%
    % - Identificador
    {DeterminantesCalc-05}
    % - Enunciado
    Calcular el determinante de la matriz 
    \[
        E = \begin{pmatrix}
            5 & -2\\
            1 & 3 
        \end{pmatrix}
    \]
    \item 17
\end{numerical}

\begin{numerical}[tolerance=0]%
    % - Identificador
    {DeterminantesCalc-06}
    % - Enunciado
    Calcular el determinante de la matriz 
    \[
        F = \begin{pmatrix}
            0 & 1\\
            -3 & 2 
        \end{pmatrix}
    \]
    \item 3
\end{numerical}

\begin{numerical}[tolerance=0]%
    % - Identificador
    {DeterminantesCalc-07}
    % - Enunciado
    Calcular el determinante de la matriz 
    \[
        G = \begin{pmatrix}
            2 & 0\\
            -1 & 3 
        \end{pmatrix}
    \]
    \item 6
\end{numerical}

\end{quiz}

\begin{quiz}{Cálculo de Determinantes 3 por 3}

\begin{numerical}[tolerance=0]%
    % - Identificador
    {DeterminantesCalc-08}
    % - Enunciado
    Calcular el determinante de la matriz
    \[
        B = \begin{pmatrix}
            2 & -1 & 3\\
            0 & 4 & 5\\
            1 & 2 & 3
        \end{pmatrix}
    \]
    \item 42
\end{numerical}

\begin{numerical}[tolerance=0]%
    % - Identificador
    {DeterminantesCalc-09}
    % - Enunciado
    Calcular el determinante de la matriz
    \[
        C = \begin{pmatrix}
            1 & 2 & 0\\
            -4 & 1 & 2\\
            0 & 0 & 1
        \end{pmatrix}
    \]
    \item 9
\end{numerical}

\begin{numerical}[tolerance=0]%
    % - Identificador
    {DeterminantesCalc-10}
    % - Enunciado
    Calcular el determinante de la matriz
    \[
        D = \begin{pmatrix}
            1 & 2 & 3\\
            4 & 5 & 6\\
            7 & 8 & 9
        \end{pmatrix}
    \]
    \item 0
\end{numerical}

\begin{numerical}[tolerance=0]%
    % - Identificador
    {DeterminantesCalc-11}
    % - Enunciado
    Calcular el determinante de la matriz
    \[
        E = \begin{pmatrix}
            1 & 5 & -1\\
            0 & 2 & 0\\
            0 & 0 & 1
        \end{pmatrix}
    \]
    \item 2
\end{numerical}

\begin{numerical}[tolerance=0]%
    % - Identificador
    {DeterminantesCalc-12}
    % - Enunciado
    Calcular el determinante de la matriz
    \[
        F = \begin{pmatrix}
            1 & 2 & 2\\
            0 & 5 & 0\\
            0 & 0 & -1
        \end{pmatrix}
    \]
    \item -15
\end{numerical}

\begin{numerical}[tolerance=0]%
    % - Identificador
    {DeterminantesCalc-13}
    % - Enunciado
    Calcular el determinante de la matriz
    \[
        G = \begin{pmatrix}
            1 & 2 & 4\\
            0 & 1 & 0\\
            0 & 0 & 3
        \end{pmatrix}
    \]
    \item 3
\end{numerical}

\end{quiz}

\begin{quiz}{Clase Invertida Determinantes}
    
\begin{essay}[response format=text, response field lines=5]%
    % - Identificador
    {ClaseInvertida-Chat}
    % - Enunciado
    Copia el enlace del chat con ChatGPT como evidencia de la actividad realizada en casa.
    \item Acceder al enlace.
\end{essay}

\begin{essay}[response format=text, response field lines=5]%
    % - Identificador
    {ClaseInvertida-Sol}
    % - Enunciado
    En caso de que algún compañero te haya ayudado a resolver tus dudas, indica aquí quién o quienes te ayudaron.
    \item Solo para registro.
\end{essay}

\begin{essay}[response format=text, response field lines=5]%
    % - Identificador
    {ClaseInvertida-Dudas}
    % - Enunciado
    ¿Qué dudas tuviste sobre el cálculo de determinantes?
    \item Solo para registro.
\end{essay}



\end{quiz}


\end{document}