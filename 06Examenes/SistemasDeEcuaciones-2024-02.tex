% \documentclass[11pt,respuestas,a4]{aleph-examen}
\documentclass[12pt,a4]{aleph-examen}

% -- Paquetes extra
\usepackage{aleph-comandos}
\usepackage{booktabs}
\usepackage{systeme}
\usepackage{booktabs}
\usepackage{multicol}


% -- Datos
\institucion{Escuela de Ciencias Físicas y Matemática}
\carrera{Ciencia de datos}
\asignatura{Álgebra lineal}
\tema{Examen no. 1-2: Sistemas de ecuaciones}
\autor{Andrés Merino}
\fecha{Semestre 2024-1}

\logouno[0.14\textwidth]{Logos/logoPUCE_04_ac}
\definecolor{colortext}{HTML}{0030A1}
\definecolor{colordef}{HTML}{0030A1}
\fuente{montserrat}


\begin{document}

\encabezado

\section*{Indicaciones}
\begin{itemize}[leftmargin=*]
\item 
    En esta actividad se evalúa si el estudiante aplica diferentes métodos para resolver sistemas de ecuaciones lineales, como el método de eliminación de Gauss y técnicas matriciales.
\item 
    Las resoluciones de los ejercicios deben ser entregadas en un archivo \texttt{.ipynb} mediante el aula virtual.
\item
    Para el desarrollo se podrán utilizar los resúmenes que constan en el aula virtual.
\item
    Se encuentra prohibido el uso de cualquier otra fuente de información durante todo el examen (incluye los ejercicios resueltos del aula virtual).
\item
    En caso de considerar que existe un error en la pregunta o que esta se encuentra mal planteada, se debe indicar cuál es el error y justificarlo.
\item
    Todas las soluciones deben estar correctamente redactadas y explicadas.
\item 
    Se debe utilizar el formato proporcionado, trabajando en VSCode (no en Google Colab) y utilizando Markdown para la redacción de las respuestas. No se deben agregar celdas adicionales, salvo que se indique lo contrario.
\item 
    Se asignarán 4 puntos de excelencia si, a más de tener las soluciones correctas, el examen se destaca por su orden y escritura.
\end{itemize}

\section*{Ejercicios}

\begin{preguntas}

%%%%%%%%%%%%%%%%%%%%%%%%%%%%%%%%%%%%%%%%
%% Rangos
%%%%%%%%%%%%%%%%%%%%%%%%%%%%%%%%%%%%%%%%
\item 
    Considere el sistema lineal de ecuaciones:
    \[
        \systeme{x+5y-4z=0{,}, x-2y+z=0{,}, 3x+y-2z=0{.}}
    \]
    \begin{enumerate}
        \item Calcule el rango de la matriz de coeficientes y el de la matriz ampliada. Con esto, determine la naturaleza del sistema.\puntaje{5}
        \item Determine la forma escalonada reducida por filas de la matriz ampliada.\puntaje{3}
        \item Escriba la solución del sistema (si el sistema tiene infinitas soluciones, escribir de forma paramétrica y el conjunto solución).\puntaje{5}
    \end{enumerate}

%%%%%%%%%%%%%%%%%%%%%%%%%%%%%%%%%%%%%%%%
%% Inversa
%%%%%%%%%%%%%%%%%%%%%%%%%%%%%%%%%%%%%%%%
\item 
    Considere el sistema lineal de ecuaciones:
    \[
        \systeme{x+y+4 z=2, 3x+4y+2z=4, 2x+3y-z=1}
    \]
    \begin{enumerate}
        \item Calcule el determinante de la matriz de coeficientes. ¿Esta matriz tiene inversa?\puntaje{5}
        \item Calcule la inversa de la matriz de coeficientes.\puntaje{3}
        \item Utilice la inversa de la matriz de coeficientes para calcular la solución del sistema.\puntaje{5}
    \end{enumerate}

%%%%%%%%%%%%%%%%%%%%%%%%%%%%%%%%%%%%%%%%%%
%%%%%%%%%%%%%%%%%%%%%%%%%%%%%%%%%%%%%%%%%%
%%%%%%%%%%%%%%%%%%%%%%%%%%%%%%%%%%%%%%%%%%
\item Sean $r,s \in \R$. Considere el sistema lineal de ecuaciones:\puntaje{20}
    \[
        \systeme[xyzw]{x + z = r{,}, 
        y + 2w = 0{,}, 
        x + 2z + 2w = 0{,}, 
        -y + 4z + sw =2{.}}
    \]
    Utilizando la eliminación de Gauss-Jordan (indicando lo que realiza paso a paso), determine los valores de $r$ y $s$ tales que el sistema
    \begin{itemize}
        \item tiene una única solución,
        \item tiene infinitas soluciones, y
        \item no tiene solución.
    \end{itemize}
    Recuerde que, para definir una variable simbólica en Python, se utiliza, por ejemplo:\linebreak \verb+r = Symbol("r")+.

\begin{respuesta}
    Definimos la matriz ampliada:
\begin{pycodigo}
    \begin{ipynbcodigo}\begin{lstlisting}[language=Python]
r = Symbol("r")
s = Symbol("s")
print("Matriz ampliada:")
Ampl = Matrix([[1, 0, 1, 0, r], [0, 1, 0, 2, 0], [1, 0, 2, 2, 0], [0, -1, 4, s, 2]])
display(Ampl)
    \end{lstlisting}\end{ipynbcodigo}
    %
    \begin{ipynbsalida}[2mm]
    \begin{Verbatim}[commandchars=\\\{\}]
Matriz ampliada:
    \end{Verbatim}

    $\displaystyle \left[\begin{matrix}1 & 0 & 1 & 0 & r\\0 & 1 & 0 & 2 & 0\\1 & 0 & 2 & 2 & 0\\0 & -1 & 4 & s & 2\end{matrix}\right]$
    \end{ipynbsalida}
\end{pycodigo}
    
Tenemos listo el segundo \(1\) principal, procedemos a hacer \(0\) el
resto de elementos de la segunda columna, para esto, tenemos la
siguiente operación:
\begin{itemize}
    \item \(1F_2+F_4\to F_4\).
\end{itemize}
\begin{pycodigo}
    \begin{ipynbcodigo}\begin{lstlisting}[language=Python]
Ampl = Ampl.elementary_row_op("n->n+km", k=-1, row1=2, row2=0)
display(Ampl)
    \end{lstlisting}\end{ipynbcodigo}
    %
    \begin{ipynbsalida}[2mm]
    $\displaystyle \left[\begin{matrix}1 & 0 & 1 & 0 & r\\0 & 1 & 0 & 2 & 0\\0 & 0 & 1 & 2 & - r\\0 & -1 & 4 & s & 2\end{matrix}\right]$
    \end{ipynbsalida}
\end{pycodigo}

Tenemos listo el segundo \(1\) principal, procedemos a hacer \(0\) el
resto de elementos de la segunda columna, para esto, tenemos la
siguiente operación: 
\begin{itemize}
    \item \(1F_2+F_4\to F_4\).
\end{itemize}
\begin{pycodigo}
    \begin{ipynbcodigo}\begin{lstlisting}[language=Python]
Ampl = Ampl.elementary_row_op("n->n+km", k=1, row1=3, row2=1)
display(Ampl)
    \end{lstlisting}\end{ipynbcodigo}
    %
    \begin{ipynbsalida}[2mm]
    $\displaystyle \left[\begin{matrix}1 & 0 & 1 & 0 & r\\0 & 1 & 0 & 2 & 0\\0 & 0 & 1 & 2 & - r\\0 & 0 & 4 & s + 2 & 2\end{matrix}\right]$
    \end{ipynbsalida}
\end{pycodigo}

Tenemos listo el tercer \(1\) principal, procedemos a hacer \(0\) el
resto de elementos de la tercera columna, para esto, tenemos la
siguiente operación: 
\begin{itemize}
    \item \(-1F_3+F_1\to F_1\),
    \item \(-4F_3+F_4\to F_4\).
\end{itemize}
\begin{pycodigo}
    \begin{ipynbcodigo}\begin{lstlisting}[language=Python]
Ampl = Ampl.elementary_row_op("n->n+km", k=-1, row1=0, row2=2)
Ampl = Ampl.elementary_row_op("n->n+km", k=-4, row1=3, row2=2)
display(Ampl)
    \end{lstlisting}\end{ipynbcodigo}
    %
    \begin{ipynbsalida}[2mm]
    $\displaystyle \left[\begin{matrix}1 & 0 & 0 & -2 & 2 r\\0 & 1 & 0 & 2 & 0\\0 & 0 & 1 & 2 & - r\\0 & 0 & 0 & s - 6 & 4 r + 2\end{matrix}\right]$
    \end{ipynbsalida}
\end{pycodigo}

Con esto tenemos varios casos:

\begin{itemize}
\item
  Si \(s-6=0\) y \(4r+2=0\), el sistema sería consistente, pero se
  tendría infinitas soluciones; es decir, el sistema tiene infinitas
  soluciones cuando \(s=6\) y \(r=-1/2\).
\item
  Si \(s-6=0\) y \(4r+2\neq 0\), el sistema sería inconsistente; es
  decir, el sistema no tiene solución cuando \(s=6\) y \(r\neq -1/2\).
\item
  Si \(s-6\neq 0\), el sistema sería consistente y tendría solución
  única; es decir, el sistema tiene única solución cuando \(s\neq 6\).\qedhere
\end{itemize}

\end{respuesta}



\end{preguntas}

\end{document}