% \documentclass[11pt,respuestas,a4]{aleph-examen}
\documentclass[11pt,a5]{aleph-examen}

% -- Paquetes extra
\usepackage{aleph-comandos}
\usepackage{booktabs}
\usepackage{multicol}
\usepackage{systeme}

% -- Datos 
\institucion{Escuela de Ciencias Físicas y Matemática}
\carrera{Ciencia de datos}
\asignatura{Álgebra lineal}
\tema{Examen no. 2: Espacios Vectoriales}
\autor{Andrés Merino}
\fecha{Semestre 2024-1}

\logouno[0.14\textwidth]{Logos/logoPUCE_04_ac}
\definecolor{colortext}{HTML}{0030A1}
\definecolor{colordef}{HTML}{0030A1}
\fuente{montserrat}


\begin{document}

\encabezado

\section*{Ejercicios}

\begin{preguntas}

%%%%%%%%%%%%%%%%%%%%%%%%%%%%%%%%%%%%%%%%
%% Independencia lineal
%%%%%%%%%%%%%%%%%%%%%%%%%%%%%%%%%%%%%%%%
\item
    En $\R^4$, determine el conjunto generado por:
    \[
        S = \big\{
            (1,0,1,0),\ 
            (2,1,2,0),\  
            (1,1, 1,1)\big\}.
    \]
    Escriba el conjunto en función de sus restricciones.\puntaje{10}


\begin{respuesta}
    Tomemos $(w,x,y,z)\in\R^4$ tal que
    \begin{align*}
        (w,x,y,z) 
        & = a(1,0,1,0) + b(2,1,2,0) + c(1,1,1,1)\\
        & = (a+2b+c,\ b+c,\ a+2b+c,\ c).
    \end{align*}
    Obtenemos el sistema
    \[
        \systeme[abc]{
            a+2b+c = w{,},
            b+c = x{,},
            a+2b+c = y{,},
            c = z{.}
        }
    \]
    Planteamos la matriz ampliada y realizamos reducción por filas:
    \[
        \begin{pmatrix}
            1 & 2 & 1 & | & w\\
            0 & 1 & 1 & | & x\\
            1 & 2 & 1 & | & y\\
            0 & 0 & 1 & | & z
        \end{pmatrix}
        \sim
        \begin{pmatrix}
            1 & 2 & 1 & | & w\\
            0 & 1 & 1 & | & x\\
            0 & 0 & 1 & | & z\\
            0 & 0 & 0 & | & - w + y
        \end{pmatrix}.
    \]
    Este sistema tiene solución si y solo si $-w + y = 0$. Por lo tanto, el conjunto generado por $S$ es
    \[
        \{
            (w,x,y,z)\in\R^4 : -w + y = 0
        \}.\qedhere
    \]
\end{respuesta}

\item
    En $\R^3$, para $\alpha \in \R$, se toma el conjunto
    \[
        S = \big\{
            (1,0,1),\ 
            (2\alpha,1,2),\  
            (1, \alpha, 1)\big\}.
    \]
    ¿Para qué valores de $\alpha$ se tiene que $S$ es un conjunto linealmente independiente?\puntaje{10}


\begin{respuesta}
    Tomemos $a,b,c,d\in\R$ tales que
    \[
        a(1,0,1) + b(2\alpha,1,2) + c(1,\alpha,1) = (0,0,0),
    \]
    operando, se tiene que
    \[ 
        (a + 2\alpha b + c, b + \alpha c, a + 2b + c) = (0,0,0).
    \]
    Con esto, se obtiene el sistema
    \[
        \systeme[abc]{
            a + 2\alpha b + c = 0{,},
            b + \alpha c = 0{,},
            a + 2b + c = 0{.}
        }
    \]
    Para que el conjunto sea linealmente independiente se necesita que el sistema tenga solución única, para esto, planteamos la matriz ampliada y realizamos reducción por filas
    \[
        \begin{pmatrix}
            1 & 2\alpha & 1 & | & 0\\
            0 & 1 & \alpha & | & 0\\
            1 & 2 & 1 & | & 0
        \end{pmatrix}
        \sim
        \begin{pmatrix}
            1 & 2\alpha & 1 & | & 0\\
            0 & 1 & \alpha & | & 0\\
            0 & 0 & 2\alpha^2 -2\alpha & | & 0
        \end{pmatrix}
    \]
    Tenemos que el sistema tiene solución única, si y solo si 
    \[
        2\alpha^2 - 2\alpha = 0,
    \]
    es decir
    \[
        \alpha = 0 \texto \alpha = 1.
    \]
    Con esto, se obtiene que $S$ es linealmente independiente para todo $\alpha$ en $\R\setminus\{0,1\}$.
\end{respuesta}


%%%%%%%%%%%%%%%%%%%%%%%%%%%%%%%%%%%%%%%%
%% Subespacios vectoriales y base
%%%%%%%%%%%%%%%%%%%%%%%%%%%%%%%%%%%%%%%%
\item
    En $\R^{2\times2}$, se define el conjunto
    \[
        F = \left\{ \begin{pmatrix} a & b\\ c & d \end{pmatrix} : a-b+c=0\ \land\ b+2d=0 \right\}.
    \]
    \begin{enumerate}
    \item
        Determine una base para $F$.\puntaje{12}
    \item
        ¿Cuál es la dimensión de $F$?\puntaje{3}
    \end{enumerate}

\begin{respuesta}\hspace{0pt}
    \begin{enumerate}
    \item 
        Para la matriz
        \[
            \begin{pmatrix}
                0 & 0 \\
                0 & 0 
            \end{pmatrix}    
        \],
        se tiene que
        \[
            0 - 0 = 0 \texty 0+2(0)= 0,
        \]
        por lo tanto, $0\in F$.
        
        Ahora, tomemos 
        \[
            A=\begin{pmatrix}
                a_{11} & a_{12} \\
                a_{21} & a_{22}
            \end{pmatrix}
            \texty
            B\begin{pmatrix}
                b_{11} & b_{12} \\
                b_{21} & b_{22}
            \end{pmatrix}
        \]
        en $F$. Tenemos que
        \[
            a_{11}-a_{21}=0,\qquad a_{12} + 2a_{22}=0
            \texty
            b_{11}-b_{21}=0,\qquad b_{12} + 2b_{22}=0
        \]
        Calculemos: 
        \[
            A + B = \begin{pmatrix}
                a_{11} & a_{12} \\
                a_{21} & a_{22}
            \end{pmatrix} + \begin{pmatrix}
                b_{11} & b_{12} \\
                b_{21} & b_{22}
            \end{pmatrix}  = \begin{pmatrix}
                 a_{11} + b_{11} &  a_{12} + b_{12} \\
                 a_{21} + b_{21} &  a_{22} + b_{22}
            \end{pmatrix}.
        \]
        Con esto, tenemos
        \[
            a_{11}+b_{11} -( a_{21}+b_{21} )  = (a_{11}-a_{21}) + (b_{11} - b_{21}) =  0
        \]
        y
        \[
            a_{12} + b_{12} + 2(a_{22} + b_{22}) =  (a_{12} + 2a_{22}) + (b_{12} + 2b_{22}) = 0.
        \]
        Por lo tanto, se sigue que $A + B \in F$.

        Finalmente, tomemos $\alpha\in \R$, vamos a probar que $\alpha A+B \in F$. Notemos que 
        \[
            \alpha A = \alpha \begin{pmatrix}
                a_{11} & a_{12} \\
                a_{21} & a_{22}
            \end{pmatrix}  = \begin{pmatrix}
                \alpha a_{11} & \alpha a_{12} \\
                \alpha a_{21} & \alpha a_{22} 
            \end{pmatrix};
        \]
        entonces
        \[
            \alpha a_{11} - \alpha a_{21}  = \alpha (a_{11}-a_{21})  =  \alpha(0) = 0
        \]
        y
        \[
            \alpha a_{12} + 2\alpha a_{22} = \alpha (a_{12} + 2a_{22})  = \alpha(0) = 0.
        \]
        Por lo tanto, se sigue que $\alpha A \in F$, es decir, $F$ es un subespacio vectorial de $\R^{2\times 2}$.
    \item 
        Tomemos un elemento $\begin{pmatrix}
                a_{11} & a_{12} \\
                a_{21} & a_{22}
        \end{pmatrix}$ arbitrario de $F$, se tiene que
        \[
            a_{11} - a_{21} = 0 \texty a_{12} + 2a_{22} = 0,
        \]
        es decir, se tiene que 
        \[
            a_{11} =  a_{21} \texty a_{12} = -2 a_{22}.
        \]
        De esta manera, se tiene que 
        \[
            \begin{pmatrix}
                a_{21} & -2 a_{22} \\
                a_{21} & a_{22}
            \end{pmatrix} = \begin{pmatrix}
                a_{21} & 0 \\
                a_{21} & 0
            \end{pmatrix} + \begin{pmatrix}
                0 & -2a_{22}\\
                0 & a_{22}
            \end{pmatrix} = a_{21} \begin{pmatrix}
                1 & 0 \\
                1 & 0 
            \end{pmatrix} + a_{22} \begin{pmatrix}
                0 & -2 \\
                0 & 1
            \end{pmatrix}.
        \]
        De esta manera, una posible base para $F$ está dada por 
        \[
            \left\{ \begin{pmatrix}
                1 & 0 \\
                1 & 0 
            \end{pmatrix},\begin{pmatrix}
                0 & -2 \\
                0 & 1
            \end{pmatrix} \right\},
        \]
        pues este conjunto genera a $F$. Debemos comprobar que es linealmente independiente, para esto, tomemos $\alpha_1$ y $\alpha_2$ números reales tales que:
        \begin{align*}
            \alpha_1\begin{pmatrix}
                1 & 0 \\
                1 & 0 
            \end{pmatrix}+\alpha_2\begin{pmatrix}
                0 & -2 \\
                0 & 1
            \end{pmatrix}
            = \begin{pmatrix}
                0 & 0 \\
                0 & 0
            \end{pmatrix}
        \end{align*}
        se sigue que $\alpha_1=\alpha_2=0$, por lo tanto, es linealmente independiente. Así, el conjunto es una base para $F$
    \item 
        Por el literal anterior, se concluye que la dimensión de $F$ es $2$. \qedhere
\end{enumerate}
\end{respuesta}


%%%%%%%%%%%%%%%%%%%%%%%%%%%%%%%%%%%%%%%%
%% Suma de espacios
%%%%%%%%%%%%%%%%%%%%%%%%%%%%%%%%%%%%%%%%
\item
    Sea $T: \mathbb{R}^3 \to \mathbb{R}^1$ una aplicación dada por $T(x,y,z) = (x+2y,\ 3x+4z)$.    \begin{enumerate}
    \item 
        Sin realizar ningún cálculo, indique si esta aplicación podría ser inyectiva o si podría ser sobreyectiva.\puntaje{3}
    \item 
        Calcule el núcleo de la aplicación. \puntaje{5}
    \item 
        Determine la imagen de la aplicación. \puntaje{5}
    \item 
        Con lo calculado en los puntos anteriores conteste: ¿La aplicación en inyectiva? ¿Es sobreyectiva? ¿Es isomorfismo? \puntaje{2}
    \end{enumerate}

\begin{respuesta}
\begin{enumerate}[leftmargin=*]
\item 
    Tomemos dos vectores $(x_1,y_1)$ y $(x_2,y_2)$ y un escalar $\lambda$. Calculemos:
    \begin{align*}
        T\big((x_1,y_1)&+(x_2,y_2)\big)\\
        &=
        T\big((x_1+x_2,y_1+y_2)\big)
        \\
        &=
        \big((x_1+x_2)+2(y_1+y_2),\ 3(x_1+x_2)+4(y_1+y_2),\ (x_1+x_2)-(y_1+y_2)\big)
        \\
        &=
        \big(x_1+2y_1+x_2+2y_2,\ 3x_1+4y_1+3x_2+4y_2,\ x_1-x_2+y_1-y_2\big)
    \end{align*}
    \begin{align*}
        T\big((x_1,y_1)\big)& + T\big((x_2,y_2)\big)\\
        &=
        \big(x_1+2y_1,\ 3x_1+4y_1,\ x_1-y_1\big) + \big(x_2+2y_2,\ 3x_2+4y_2,\ x_2-y_2\big)
        \\
        &=
        \big(x_1+2y_1+x_2+2y_2,\ 3x_1+4y_1+3x_2+4y_2,\ x_1-x_2+y_1-y_2\big)
    \end{align*}
    Como ambos resultados son iguales, la aplicación cumple la primera propiedad de aplicación lineal. Ahora, calculemos:
    \begin{align*}
        T\big(\lambda(x_1,y_1)\big)
        &=
        T\big((\lambda x_1,\lambda y_1)\big)
        \\
        &=
        \big((\lambda x_1)+2(\lambda y_1),\ 3(\lambda x_1)+4(\lambda y_1),\ (\lambda x_1)-(\lambda y_1)\big)
        \\
        &=
        \big(\lambda x_1+2\lambda y_1,\ 3\lambda x_1+4\lambda y_1,\ \lambda x_1-\lambda y_1\big);
    \end{align*}
    \begin{align*}
        \lambda T\big((x_1,y_1)\big)
        &=
        \lambda\big((x_1+2y_1,\ 3x_1+4y_1,\ x_1-y_1)\big)
        \\
        &=
        \big(\lambda(x_1+2y_1),\ \lambda(3x_1+4y_1),\ \lambda(x_1-y_1)\big) \\
        &=
        \big(\lambda x_1+2\lambda y_1,\ 3\lambda x_1+4\lambda y_1,\ \lambda x_1-\lambda y_1\big);
    \end{align*}
    Como ambos resultados son iguales, la aplicación cumple la segunda propiedad de aplicación lineal. Por lo tanto, $T$ es una aplicación lineal.
\item
    La aplicación no puede ser sobreyectiva dado que $\dim(\R^3)=3$ es mayor que $\dim(\R^2)=2$. Por otro lado, la aplicación sí podría ser inyectiva.
\item
    Tomemos un vector $(x,y)$ tal que $T(x,y)=(0,0,0)$. Entonces:
    \[
        T(x,y)=(x+2y,\ 3x+4y,\ x-y)=(0,0,0).
    \]
    y tenemos el sistema
    \[
        \systeme{
            x+2y = 0,
            3x+4y = 0,
            x-y = 0.
        }
    \]
    cuya matriz ampliada es
    \[
        \begin{pmatrix}
            1 & 2 &|& 0\\
            3 & 4 &|& 0\\
            1 & -1 &|& 0
        \end{pmatrix}
        \sim
        \begin{pmatrix}
            1 & 0 & |& 0\\
            0 & 1 & |& 0\\
            0 & 0 & |& 0
        \end{pmatrix}
    \]
    Obtenemos las ecuaciones
    \[
        \systeme{
            x=0,
            y=0
        }
    \]
    y por lo tanto, el núcleo de la aplicación es
    \[
        \ker(T)=\{(0,0)\}.
    \]
\item
    Tomemos un vector $(a,b,c)\in\R^3$. Entonces, existe un vector $(x,y)\in\R^2$ tal que $T(x,y)=(x+2y,3x+4y,x-y)=(a,b,c)$. Es decir, tenemos el sistema
    \[
        \systeme{
            x+2y = a,
            3x+4y = b,
            x-y = c.
        }
    \]
    cuya matriz ampliada es
    \[
        \begin{pmatrix}
            1 & 2 &|& a\\
            3 & 4 &|& b\\
            1 & -1 &|& c
        \end{pmatrix}
        \sim
        \begin{pmatrix}
            1 & 0 & | & \frac{1}{3}(a+2c) \\
            0 & 1 & | & \frac{1}{4}(b-a+2c) \\
            0 & 0 & | & -\frac{5a}{4}+\frac{b}{4}+\frac{c}{2}
        \end{pmatrix}
    \]
    Obtenemos las ecuaciones
    \[
        \systeme{
            x=\frac{1}{3}(a+2c),
            y= \frac{1}{4}(b-a+2c),
            0=-\frac{5a}{4}+\frac{b}{4}+\frac{c}{2}
        }    
    \]
    Dado que este sistema no siempre tiene solución, se tiene que $\img(T)\subsetneq\R^2$.
\item
    Como el núcleo de la aplicación es igual a $\{(0,0,0)\}$, la aplicación es inyectiva. Por otro lado, como la imagen de la aplicación es un subconjunto propio de $\R^3$ , la aplicación no es sobreyectiva. Con esto, tenemos que la aplicación no es biyectiva, por lo que no es un isomorfismo.\qedhere
\end{enumerate} 
\end{respuesta}



\end{preguntas}


\end{document}