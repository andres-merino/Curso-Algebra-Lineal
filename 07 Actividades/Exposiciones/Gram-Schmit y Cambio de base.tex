\documentclass[a4,11pt]{aleph-notas}

% -- Paquetes adicionales
\usepackage{aleph-comandos}


% -- Datos 
\institucion{Escuela de Ciencias Físicas y Matemática}
\carrera{Ciencia de datos}
\asignatura{Álgebra lineal}
\tema{Video exposición no. 2: Gram-Schmit y Cambio de base}
\autor{Andrés Merino}
\fecha{Semestre 2024-1}

\logouno[0.14\textwidth]{Logos/logoPUCE_04_ac}
\definecolor{colortext}{HTML}{0030A1}
\definecolor{colordef}{HTML}{0030A1}
\fuente{montserrat}


% -- Otros comandos




\begin{document}

\encabezado

%%%%%%%%%%%%%%%%%%%%%%%%%%%%%%%%%%%%%%%%
\section{Resultado de aprendizaje}
%%%%%%%%%%%%%%%%%%%%%%%%%%%%%%%%%%%%%%%%

\begin{itemize}
\item 
    Comprende los conceptos de espacios vectoriales y aplicaciones lineales, incluyendo la base y dimensión, transformaciones lineales y sus propiedades.
\end{itemize}


%%%%%%%%%%%%%%%%%%%%%%%%%%%%%%%%%%%%%%%%
\section{Instrucciones}
%%%%%%%%%%%%%%%%%%%%%%%%%%%%%%%%%%%%%%%%

\begin{enumerate}
    \item Preparación del Material: Se deberá preparar una presentación en formato de video donde expliquen de manera clara y concisa los conceptos de «Matriz de cambio de base» y «Proceso de Gram-Schmit».
    \item Contenido del Video:
    \begin{itemize}
        \item Introducción: Se deberá introducir el tema explicando qué la «Matriz de cambio de base» y el «Proceso de Gram-Schmit», y por qué es importante comprenderlos.
        \item Conceptos Clave: Se deberá explicar los conceptos de «Matriz de cambio de base», destacando sus propiedades más importantes.
        \item Ejemplos Prácticos: Se incluirán ejemplos numéricos sobre el cálculo de «Matriz de cambio de base» y la aplicación del «Proceso de Gram-Schmit».
    \end{itemize}
    \item Claridad y Profundidad: Se valorará la claridad en la exposición y la profundidad en la explicación de los conceptos. Los estudiantes deben asegurarse de que su presentación sea comprensible para un público que tenga conocimientos básicos de álgebra lineal.
    \item Duración del Video: La duración del video no deberá exceder los 10 minutos.
\end{enumerate}

%%%%%%%%%%%%%%%%%%%%%%%%%%%%%%%%%%%%%%%%
\section{Entrega de la actividad}
%%%%%%%%%%%%%%%%%%%%%%%%%%%%%%%%%%%%%%%%

Se deberá entregar el enlace al video de la exposición a través del aula virtual antes de la fecha límite establecida.


%%%%%%%%%%%%%%%%%%%%%%%%%%%%%%%%%%%%%%%%
\section{Guía de calificación}
%%%%%%%%%%%%%%%%%%%%%%%%%%%%%%%%%%%%%%%%

\begin{itemize}
\item
    Comprensión de Conceptos (20 puntos):
    \begin{itemize}
        \item 20 puntos: La exposición comprende una explicación clara y completa.
        \item 10 puntos: La exposición presenta una explicación adecuada de los conceptos, pero puede haber algunas deficiencias en la claridad o completitud de la explicación.
        \item 0 puntos: La explicación de los conceptos es insuficiente o incorrecta en la mayoría de los casos.
    \end{itemize}

\item
    Ejemplos Prácticos «Matriz de cambio de base» (10 puntos):
    \begin{itemize}
        \item 12 puntos: Se presentan ejemplos prácticos claros, y bien explicados, que muestran la aplicación de los conceptos enseñados.
        \item 6 puntos: Se incluyen ejemplos prácticos, pero puede haber algunas deficiencias en la claridad de la explicación asociada.
        \item 0 puntos: No se presentan ejemplos prácticos o los ejemplos presentados no son relevantes o están mal explicados.
    \end{itemize}

\item
    Ejemplos Prácticos «Proceso de Gram-Schmit» (10 puntos):
    \begin{itemize}
        \item 12 puntos: Se presentan ejemplos prácticos claros, y bien explicados, que muestran la aplicación de los conceptos enseñados.
        \item 6 puntos: Se incluyen ejemplos prácticos, pero puede haber algunas deficiencias en la claridad de la explicación asociada.
        \item 0 puntos: No se presentan ejemplos prácticos o los ejemplos presentados no son relevantes o están mal explicados.
    \end{itemize}

\item
    Excelencia (6 puntos), este puntaje es dado únicamente si se tiene el puntaje completo en los otros puntos. Se premiará la creatividad en la presentación del contenido, la originalidad en la elección de ejemplos o la inclusión de elementos visuales innovadores que enriquezcan la experiencia del espectador. Se evaluará la claridad en la comunicación de los conceptos, la fluidez en la presentación y la capacidad para mantener el interés del espectador a lo largo del video.
    \begin{itemize}
        \item 6 puntos: El video es excelente.
        \item 0 puntos: El video no es excelente.
    \end{itemize}

\end{itemize}

\end{document}




\end{document}