\documentclass[a4,11pt]{aleph-notas}


% -- Paquetes adicionales
\usepackage{aleph-comandos}


% -- Datos 
\institucion{Escuela de Ciencias Físicas y Matemática}
\carrera{Ciencia de datos}
\asignatura{Álgebra lineal}
\tema[Taller no. 2: Generación e independencia]{Taller no. 3: ¿Cómo automatizar la comprobación de generación e independencia?}
\autor{Andrés Merino - Mario Cueva}
\fecha{Semestre 2023-1}

\logouno[0.14\textwidth]{Logos/logoPUCE_04_ac}
\definecolor{colortext}{HTML}{0030A1}
\definecolor{colordef}{HTML}{0030A1}
\fuente{montserrat}


% -- Otros comandos




\begin{document}

\encabezado

%%%%%%%%%%%%%%%%%%%%%%%%%%%%%%%%%%%%%%%%
\section{Objetivos}
%%%%%%%%%%%%%%%%%%%%%%%%%%%%%%%%%%%%%%%%

\begin{itemize}
\item 
    Automatizar la comprobación de generación e independencia lineal de un conjunto de vectores.
\item
    Generar funciones en Python.
\end{itemize}


%%%%%%%%%%%%%%%%%%%%%%%%%%%%%%%%%%%%%%%%
\section{Descripción}
%%%%%%%%%%%%%%%%%%%%%%%%%%%%%%%%%%%%%%%%

Ya conoces los conceptos de que un conjunto de vectores genere el espacio y que un conjunto de vectores sea linealmente independiente. Dado que se deben resolver y analizar varios sistemas de ecuaciones, deseas generar unas funciones en Python que te ayuden a automatizar esto. 

%%%%%%%%%%%%%%%%%%%%%%%%%%%%%%%%%%%%%%%%
\section{Producto}
%%%%%%%%%%%%%%%%%%%%%%%%%%%%%%%%%%%%%%%%

Se deberá producir una Jupyter Notebook, en el cual se definan las siguientes funciones:

\begin{itemize}
\item
    \texttt{genera(«lista de vectores», x)}: a esta función se le da como entrada una lista de vectores, todos de la misma dimensión y un vector extra denominado $x$; como resultado, imprime en pantalla una de las siguientes opciones:
    \begin{itemize}
    \item 
        En caso de que se pueda generar de manera única, debe indicarlo e imprimir los coeficientes.
    \item
        En caso de que se pueda generar de múltiples maneras, debe indicarlo e imprimir dos conjuntos de coeficientes
    \item
        En caso de que no se pueda generar, debe indicarlo.
    \end{itemize}
\item
    \texttt{si\_li(«lista de vectores»)}: a esta función se le da como entrada una lista de vectores, todos de la misma dimensión; devuelve \texttt{True} si la lista de vectores es linealmente independiente y \textbf{False} caso contrario.
\item
    \texttt{si\_genera(«lista de vectores»)}: a esta función se le da como entrada una lista de vectores, todos de la misma dimensión; devuelve \texttt{True} si la lista de vectores genera todo el espacio y \textbf{False} caso contrario.
\end{itemize}

Se debe entregar el archivo .ipynb mediante el aula virtual, dentro del archivo, debe constar un enlace al video de presentación.

%%%%%%%%%%%%%%%%%%%%%%%%%%%%%%%%%%%%%%%%
\section{Rúbrica de evaluación}
%%%%%%%%%%%%%%%%%%%%%%%%%%%%%%%%%%%%%%%%

\begin{itemize}
\item
    Función \texttt{genera(«lista de vectores», x)}: calificado sobre 3 puntos bajo los siguientes parámetros:
    \begin{itemize}
        \item El producto da la respuesta correcta, en todos los casos, 3 puntos.
        \item El producto da la respuesta correcta, en la mayoría de casos, pero no en todos, 1 punto.
        \item El producto no da la respuesta correcta, en la mayoría de casos, 0 puntos.
    \end{itemize}
\item
    Función \texttt{si\_li(«lista de vectores»)}: calificado sobre 3 puntos bajo los siguientes parámetros:
    \begin{itemize}
        \item El producto da la respuesta correcta, en todos los casos, 3 puntos.
        \item El producto da la respuesta correcta, en la mayoría de casos, pero no en todos, 1 punto.
        \item El producto no da la respuesta correcta, en la mayoría de casos, 0 puntos.
    \end{itemize}
\item
    Función \texttt{si\_genera(«lista de vectores»)}: calificado sobre 3 puntos bajo los siguientes parámetros:
    \begin{itemize}
        \item El producto da la respuesta correcta, en todos los casos, 3 puntos.
        \item El producto da la respuesta correcta, en la mayoría de casos, pero no en todos, 1 punto.
        \item El producto no da la respuesta correcta, en la mayoría de casos, 0 puntos.
    \end{itemize}


\item
    Documentación del Jupyter Notebook: calificado sobre 3 puntos bajo los siguientes parámetros:
    \begin{itemize}
        \item Existen comentarios que explican detalladamente lo que hace el producto, 3 puntos.
        \item No se comenta a detalle la totalidad de lo que hace el producto, 1 punto.
        \item No existen comentarios, 0 puntos.
    \end{itemize}

\item
    Presentación: calificado sobre 3 puntos bajo los siguientes parámetros:
    \begin{itemize}
        \item Se expone de manera comprensible lo que hace el producto y se presenta ejemplos, 3 puntos.
        \item La presentación del producto no está completa, 1 punto.
        \item No se existe presentación, 0 puntos.
    \end{itemize}
    
\end{itemize}

\end{document}