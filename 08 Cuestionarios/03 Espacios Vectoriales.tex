\documentclass[a4,11pt]{aleph-notas}
% Actualizado en febrero de 2024
% Funciona con TeXLive 2022
% Para obtener solo el pdf, compilar con pdfLaTeX. 
%  latexmk -pdf 01\ Cuestionarios/01\ Matrices.tex -output-directory="01 Cuestionarios"
% Para obtener el xml compilar con XeLaTeX.
% latexmk -xelatex 01\ Cuestionarios/01\ Matrices.tex -output-directory="01 Cuestionarios"

% -- Paquetes adicionales
\usepackage{aleph-moodle}
\moodleregisternewcommands
% Todos los comandos nuevos deben ir luego del comando anterior
\usepackage{aleph-comandos}


% -- Datos 
\institucion{Escuela de Ciencias Físicas y Matemática}
\carrera{Ciencia de datos / Bioingeniería}
\asignatura{Álgebra Lineal}
\tema{Cuestionario No. 2: Espacios vectoriales y aplicaciones lineales}
\autor{Andrés Merino}
\fecha{Semestre 2024-1}

\logouno[0.14\textwidth]{Logos/logoPUCE_04_ac}
\definecolor{colortext}{HTML}{0030A1}
\definecolor{colordef}{HTML}{0030A1}
\fuente{montserrat}

% -- Otros comandos



\begin{document}

\encabezado

\vspace*{-8mm}
\tableofcontents

%%%%%%%%%%%%%%%%%%%%%%%%%%%%%%%%%%%%%%%%
\section{Indicaciones}
%%%%%%%%%%%%%%%%%%%%%%%%%%%%%%%%%%%%%%%%

Se plantean bancos de preguntas orientados a evaluar el \textbf{criterio}: «Comprende los conceptos de espacios vectoriales y aplicaciones lineales, incluyendo la base y dimensión, transformaciones lineales y sus propiedades», correspondiente al \textbf{resultado de aprendizaje}: Comprender los conceptos fundamentales del Álgebra Lineal, incluyendo el estudio de matrices, determinantes, sistemas de ecuaciones lineales y espacios vectoriales, destacando su importancia en el análisis y resolución de problemas matemáticos.

%%%%%%%%%%%%%%%%%%%%%%%%%%%%%%%%%%%%%%%%
\section{Banco de preguntas}
%%%%%%%%%%%%%%%%%%%%%%%%%%%%%%%%%%%%%%%%

%%%%%%%%%%%%%%%%%%%%%%%%%%%%%%%%%%%%%%%%
\begin{quiz}{Vectores en el plano y en Rn}
%%%%%%%%%%%%%%%%%%%%%%%%%%%%%%%%%%%%%%%%

%%%%%%%%%%%%%%%%%%%%%%%%%%%%%%%%%%%%%%%%
\begin{multi}[]%
    % - Identificador
    {Rn-01}
    % - Enunciado
    ¿Cuáles de los siguientes conjuntos representan una base canónica de \( \mathbb{R}^2 \)?
    \item* $\{ (1,0), (0,1) \}$
    \item $\{ (1,1), (0,1) \}$
    \item $\{ (1,0) \}$
    \item $\{ (1,1), (-1,-1) \}$
\end{multi}

\begin{multi}[]%
    % - Identificador
    {Rn-02}
    % - Enunciado
    ¿Cuáles de los siguientes conjuntos representan una base canónica de \( \mathbb{R}^3 \)?
    \item* $\{ (1,0,0), (0,1,0), (0,0,1) \}$
    \item $\{ (1,1,1), (0,1,1), (0,0,1) \}$
    \item $\{ (1,0,0), (1,1,0) \}$
    \item $\{ (1,1,0), (0,1,1), (1,0,1) \}$
\end{multi}

\begin{multi}[]%
    % - Identificador
    {Rn-03}
    % - Enunciado
    Considerando dos vectores \( u \) y \( v \) en \( \mathbb{R}^n \), ¿cuál de las siguientes afirmaciones sobre el producto punto \( u \cdot v \) es correcta?
    \item* El producto punto es conmutativo, es decir, \( u \cdot v = v \cdot u \).
    \item El producto punto es anticonmutativo, es decir, \( u \cdot v = -v \cdot u \).
    \item El producto punto es siempre cero.
    \item El producto punto depende del orden de los vectores.
\end{multi}

\begin{multi}[]%
    % - Identificador
    {Rn-04}
    % - Enunciado
    ¿Qué relación correcta describe el producto punto de un vector \( u \) en \( \mathbb{R}^n \) consigo mismo?
    \item* El producto punto \( u \cdot u \) es igual al cuadrado de la norma de \( u \).
    \item El producto punto \( u \cdot u \) es igual a la norma de \( u \).
    \item El producto punto \( u \cdot u \) es siempre igual a 1.
    \item El producto punto \( u \cdot u \) es independiente de la norma de \( u \).
\end{multi}

\begin{multi}[]%
    % - Identificador
    {Rn-05}
    % - Enunciado
    ¿Cuál de los siguientes pares de vectores en \( \mathbb{R}^2 \) es ortogonal?
    \item* $\{ (2, -2), (1, 1) \}$
    \item $\{ (1, 2), (-2, 3) \}$
    \item $\{ (3, -3), (1, -1) \}$
    \item $\{ (0, 1), (1, 1) \}$
\end{multi}

\begin{multi}[]%
    % - Identificador
    {Rn-06}
    % - Enunciado
    ¿Cuál es la definición correcta de vectores ortogonales en un espacio vectorial?
    \item* Dos vectores son ortogonales si su producto punto es igual a cero.
    \item Dos vectores son ortogonales si su suma es igual a cero.
    \item Dos vectores son ortogonales si son linealmente dependientes.
    \item Dos vectores son ortogonales si ambos tienen la misma norma.
\end{multi}

\end{quiz}

%%%%%%%%%%%%%%%%%%%%%%%%%%%%%%%%%%%%%%%%
\begin{quiz}{Subespacios vectoriales}
%%%%%%%%%%%%%%%%%%%%%%%%%%%%%%%%%%%%%%%%

\begin{multi}[]%
    % - Identificador
    {EspVec-01}
    % - Enunciado
    ¿Cuál de las siguientes opciones describe correctamente un espacio vectorial?
    \item* Un conjunto de vectores junto con las operaciones de suma y multiplicación por escalares, que cumple con ciertas propiedades.
    \item Un conjunto de vectores que pueden ser sumados o multiplicados entre sí.
    \item Un conjunto de vectores que forman un sistema linealmente independiente.
    \item Un conjunto de vectores donde la suma de cualquier par de vectores siempre resulta en un vector fuera del conjunto original.
\end{multi}

\begin{multi}[]%
    % - Identificador
    {EspVec-02}
    % - Enunciado
    ¿Cuál de las siguientes opciones describe correctamente un subespacio vectorial de un espacio vectorial dado?
    \item* Un conjunto no vacío de vectores que es cerrado bajo la suma y la multiplicación por escalar.
    \item Un conjunto de todos los vectores linealmente independientes en un espacio vectorial.
    \item Cualquier conjunto de vectores que incluya al menos un vector del espacio vectorial mayor.
    \item Un conjunto de vectores donde cada vector es ortogonal a un vector dado del espacio vectorial mayor.
\end{multi}

\begin{multi}[]%
    % - Identificador
    {EspVec-03}
    % - Enunciado
    ¿Cuál de las siguientes afirmaciones describe correctamente una combinación lineal de vectores?
    \item* Una combinación lineal de vectores es una expresión de la forma \( a_1v_1 + a_2v_2 + \ldots + a_nv_n \), donde \( v_1, v_2, \ldots, v_n \) son vectores y \( a_1, a_2, \ldots, a_n \) son escalares.
    \item Una combinación lineal de vectores ocurre cuando todos los vectores involucrados son mutuamente ortogonales.
    \item Una combinación lineal es cualquier suma de vectores sin incluir multiplicación por escalares.
    \item Una combinación lineal de vectores se define como la suma de dos vectores que son linealmente independientes.
\end{multi}

\begin{multi}[]%
    % - Identificador
    {EspVec-04}
    % - Enunciado
    ¿Qué es el espacio generado por un conjunto de vectores?
    \item* Es el conjunto de todas las combinaciones lineales posibles de esos vectores.
    \item Es el conjunto de todos los vectores que son ortogonales a esos vectores.
    \item Es el conjunto formado únicamente por los vectores más largos del grupo inicial.
    \item Es el conjunto que contiene solo los vectores linealmente independientes del grupo inicial.
\end{multi}


\end{quiz}

%%%%%%%%%%%%%%%%%%%%%%%%%%%%%%%%%%%%%%%%
\begin{quiz}{Independencia lineal}
%%%%%%%%%%%%%%%%%%%%%%%%%%%%%%%%%%%%%%%%

\begin{multi}[]%
    % - Identificador
    {IndepLin-01}
    % - Enunciado
    ¿Cuál de las siguientes afirmaciones es correcta respecto a la independencia lineal de un conjunto de vectores?
    \item* Un conjunto de vectores es linealmente independiente si la única combinación lineal que produce el vector cero es aquella en la que todos los coeficientes son cero.
    \item Un conjunto de vectores es linealmente dependiente si todos sus vectores son ortogonales entre sí.
    \item Un conjunto de vectores es linealmente independiente si se puede expresar cada vector como una combinación lineal de los demás vectores en el conjunto.
    \item Un conjunto de vectores es linealmente independiente si su número es igual al número de dimensiones del espacio vectorial al que pertenecen.
\end{multi}

\begin{multi}[]%
    % - Identificador
    {IndepLin-02}
    % - Enunciado
    ¿Cuál es la idea intuitiva detrás de la independencia lineal de un conjunto de vectores?
    \item* Un conjunto de vectores es linealmente independiente si todos apuntan en direcciones diferentes.
    \item Un conjunto de vectores es linealmente independiente si todos los vectores apuntan en la misma dirección.
    \item Un conjunto de vectores es linealmente independiente si todos los vectores son de la misma longitud.
    \item Un conjunto de vectores es linealmente independiente si pueden ser sumados para obtener cualquier vector en el espacio.
\end{multi}


\end{quiz}


%%%%%%%%%%%%%%%%%%%%%%%%%%%%%%%%%%%%%%%%
\begin{quiz}{Bases}
%%%%%%%%%%%%%%%%%%%%%%%%%%%%%%%%%%%%%%%%

\begin{multi}[]%
    % - Identificador
    {BasesEspVec-01}
    % - Enunciado
    ¿Cuál de las siguientes afirmaciones describe correctamente una base de un espacio vectorial?
    \item* Una base es un conjunto de vectores linealmente independientes que genera todo el espacio vectorial.
    \item Una base es cualquier conjunto de vectores que incluye el vector cero del espacio.
    \item Una base es un conjunto de vectores donde cada vector es ortogonal a los otros.
    \item Una base es un conjunto de vectores que son todos mutuamente ortogonales y de la misma longitud.
\end{multi}

\begin{multi}[]%
    % - Identificador
    {BasesEspVec-02}
    % - Enunciado
    Considera el espacio de polinomios \( \mathbb{R}_2[x] \) de grado menor o igual a 2. ¿Cuál de los siguientes conjuntos constituye una base para \( \mathbb{R}_2[x] \)?
    \item* $\{1, x, x^2\}$
    \item $\{1, x, 2x, x^2\}$
    \item $\{x, x^2\}$
    \item $\{0, x, x^2\}$
\end{multi}

\begin{multi}[]%
    % - Identificador
    {BasesEspVec-03}
    % - Enunciado
    ¿Cuál de estos conjuntos puede ser base de \( \mathbb{R}^2 \)?
    \item $\{(1, 2)\}$
    \item* $\{(1, 0), (0, 1), (1, 1)\}$
    \item $\{(1, 0), (0, 0)\}$
    \item $\{(1, 1), (1, -1)\}$
\end{multi}

\begin{multi}[]%
    % - Identificador
    {BasesEspVec-04}
    % - Enunciado
    ¿Cuál de estos conjuntos puede ser base del espacio de todas las matrices de 2x2?
    \item $\left\{ \begin{pmatrix} 1 & 0 \\ 0 & 1 \end{pmatrix}, \begin{pmatrix} 1 & 0 \\ 1 & 1 \end{pmatrix}, \begin{pmatrix} 1 & 1 \\ 0 & 1 \end{pmatrix} \right\}$
    \item $\left\{ \begin{pmatrix} 1 & 0 \\ 0 & 0 \end{pmatrix}, \begin{pmatrix} 0 & 1 \\ 0 & 0 \end{pmatrix}, \begin{pmatrix} 0 & 0 \\ 1 & 0 \end{pmatrix}, \begin{pmatrix} 0 & 0 \\ 0 & 1 \end{pmatrix} \right\}$
    \item $\left\{ \begin{pmatrix} 1 & 0 \\ 0 & 0 \end{pmatrix}, \begin{pmatrix} 0 & 1 \\ 0 & 0 \end{pmatrix}, \begin{pmatrix} 0 & 0 \\ 0 & 0 \end{pmatrix}, \begin{pmatrix} 0 & 0 \\ 1 & 0 \end{pmatrix} \right\}$
    \item* $\left\{ \begin{pmatrix} 1 & 1 \\ 1 & 1 \end{pmatrix}, \begin{pmatrix} 2 & 1 \\ 2 & 2 \end{pmatrix}, \begin{pmatrix} 3 & 3 \\ 1 & 1 \end{pmatrix}, \begin{pmatrix} 0 & 0 \\ 1 & 1 \end{pmatrix}, \begin{pmatrix} 3 & 3 \\ 0 & 0 \end{pmatrix} \right\}$
\end{multi}

\end{quiz}

%%%%%%%%%%%%%%%%%%%%%%%%%%%%%%%%%%%%%%%%
\begin{quiz}{Dimensión}
%%%%%%%%%%%%%%%%%%%%%%%%%%%%%%%%%%%%%%%%
\begin{numerical}[tolerance=0]%
    % - Identificador
    {Dimension-01}
    % - Enunciado
    ¿Cuál es la dimensión del espacio vectorial \( \mathbb{R}^3 \)?
    \item 3
\end{numerical}

\begin{numerical}[tolerance=0]%
    % - Identificador
    {Dimension-02}
    % - Enunciado
    ¿Cuál es la dimensión del espacio vectorial $\mathbb{R}^{2\times 2}$?
    \item 4
\end{numerical}

\begin{numerical}[tolerance=0]%
    % - Identificador
    {Dimension-03}
    % - Enunciado
    ¿Cuál es la dimensión del espacio vectorial \( \mathbb{R}_2[x] \), que consiste en todos los polinomios de grado menor o igual a 2?
    \item 3
\end{numerical}


\end{quiz}


%%%%%%%%%%%%%%%%%%%%%%%%%%%%%%%%%%%%%%%%
\begin{quiz}{Coordenadas y cambio de base}
%%%%%%%%%%%%%%%%%%%%%%%%%%%%%%%%%%%%%%%%

\begin{multi}[]%
    % - Identificador
    {CoordVec-01}
    % - Enunciado
    ¿Cuál es la definición correcta de un vector de coordenadas respecto a una base dada?
    \item* El vector de coordenadas de un vector respecto a una base es el conjunto de escalares que, multiplicados por los vectores de la base y sumados, dan como resultado el vector original.
    \item El vector de coordenadas de un vector respecto a una base es simplemente la suma de los vectores de la base.
    \item El vector de coordenadas es el mayor vector que puede ser formado usando los vectores de la base.
    \item El vector de coordenadas de un vector respecto a una base es el vector que resulta de multiplicar cada vector de la base por sí mismo.
\end{multi}


\begin{multi}[]%
    % - Identificador
    {CoordVec-01}
    % - Enunciado
    Dado el espacio vectorial \( \mathbb{R}^2 \) y la base \( B = \{(1, 1), (-1, 2)\} \), si $[v]_B = \begin{pmatrix}2\\-2\end{pmatrix}$, ¿cuál es el vector $v$?
    \item* $(4, -2)$
    \item $(2,-2)$
    \item $(-4, 2)$
    \item $(-2,2)$
\end{multi}


\end{quiz}


%%%%%%%%%%%%%%%%%%%%%%%%%%%%%%%%%%%%%%%%
\begin{quiz}{Espacios con producto interno}
%%%%%%%%%%%%%%%%%%%%%%%%%%%%%%%%%%%%%%%%

\begin{numerical}[tolerance=0]%
    % - Identificador
    {ProdInt-01}
    % - Enunciado
    Considera los vectores \( u = (3, -2) \) y \( v = (1, 4) \) en \( \mathbb{R}^2 \). ¿Cuál es el valor del producto interno \( \langle u , v\rangle \)?
    \item -5
\end{numerical}

\begin{numerical}[tolerance=0]%
    % - Identificador
    {ProdInt-02}
    % - Enunciado
     Considera las matrices \( A = \begin{pmatrix} 1 & 2 \\ 3 & 4 \end{pmatrix} \) y \( B = \begin{pmatrix} 1 & 0 \\ 2 & 0 \end{pmatrix} \). 
     ¿Cuál es el valor del producto interno \( \langle A , B\rangle \)?
    \item 7
\end{numerical}

\end{quiz}


%%%%%%%%%%%%%%%%%%%%%%%%%%%%%%%%%%%%%%%%
\begin{quiz}{Aplicaciones lineales}
%%%%%%%%%%%%%%%%%%%%%%%%%%%%%%%%%%%%%%%%

\begin{multi}[]%
    % - Identificador
    {ApLin-01}
    % - Enunciado
    ¿Qué es el núcleo de una aplicación lineal \( T: V \rightarrow W \) entre dos espacios vectoriales \( V \) y \( W \)?
    \item* El núcleo de \( T \) es el conjunto de todos los vectores \( v \) en \( V \) tales que \( T(v) = 0 \) en \( W \).
    \item El núcleo de \( T \) es el conjunto de todas las imágenes \( w \) en \( W \) tales que existe un \( v \) en \( V \) con \( T(v) = w \).
    \item El núcleo de \( T \) incluye todos los vectores en \( V \) y todas sus imágenes en \( W \).
    \item El núcleo de \( T \) es el conjunto de todos los escalares que se pueden usar para multiplicar vectores en \( V \) para obtener vectores en \( W \).
\end{multi}

\begin{multi}[]%
    % - Identificador
    {ApLin-02}
    % - Enunciado
    ¿Qué es la imagen de una aplicación lineal \( T: V \rightarrow W \) entre dos espacios vectoriales \( V \) y \( W \)?
    \item La imagen de \( T \) es el conjunto de todos los vectores \( v \) en \( V \) tales que \( T(v) = 0 \) en \( W \).
    \item* La imagen de \( T \) es el conjunto de todas las imágenes \( w \) en \( W \) tales que existe un \( v \) en \( V \) con \( T(v) = w \).
    \item La imagen de \( T \) incluye todos los vectores en \( V \) y todas sus imágenes en \( W \).
    \item La imagen de \( T \) es el conjunto de todos los escalares que se pueden usar para multiplicar vectores en \( V \) para obtener vectores en \( W \).
\end{multi}

\begin{multi}[]%
    % - Identificador
    {ApLin-03}
    % - Enunciado
    Dada una aplicación lineal \( T: \mathbb{R}^3 \rightarrow \mathbb{R}^2 \), el núcleo de $T$ es un subconjunto de
    \item* $\mathbb{R}^3$
    \item $\mathbb{R}^2$
\end{multi}

\begin{multi}[]%
    % - Identificador
    {ApLin-04}
    % - Enunciado
    Dada una aplicación lineal \( T: \mathbb{R}^3 \rightarrow \mathbb{R}^2 \), la imagen de $T$ es un subconjunto de
    \item $\mathbb{R}^3$
    \item* $\mathbb{R}^2$
\end{multi}

\begin{multi}[]%
    % - Identificador
    {ApLin-05}
    % - Enunciado
    Considera la aplicación lineal \( T: \mathbb{R}^3 \rightarrow \mathbb{R}^2 \) definida por \( T(x, y, z) = (x + y, y-z, z+x) \). ¿Cuál de los siguientes vectores es parte del núcleo de \( T \)?
    \item $(0, 0, 1)$
    \item* $(1, -1, -1)$
    \item $(1, 0, 1)$
    \item $(-1, -1, -1)$
\end{multi}

\end{quiz}

\end{document}