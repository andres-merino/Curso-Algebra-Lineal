\documentclass[11pt,respuestas,a4]{aleph-examen}
% \documentclass[11pt,a4]{aleph-examen}

% -- Paquetes extra
\usepackage{aleph-comandos}
\usepackage{systeme}
% Caga del paquete
\usepackage{listings}

% Recuadro para código
\definecolor{cellborder}{HTML}{CFCFCF}
\definecolor{cellbackground}{HTML}{F7F7F7}
\newtcolorbox{ipynbcodigo}
    {breakable, size=fbox, boxrule=1pt, pad at break*=1mm,colback=cellbackground, colframe=cellborder,top=0mm,bottom=-3.3mm, before upper=\LineaIn}
% Recuadro para salida
\newtcolorbox{ipynbsalida}[1][0mm]
    {breakable, size=fbox, boxrule=1pt, pad at break*=1mm,colback=cellbackground, colframe=cellborder,top=#1,bottom=0mm, before upper=\footnotesize\LineaOut}
% Recuadro para código
\definecolor{colcod}{RGB}{174,218,255}
\newtcolorbox{pycodigo}
    {icono=\faKeyboardO,color=colcod,
     postit,left=15mm,bottom=1.8mm}
% Texto de In y Out
\definecolor{incolor}{HTML}{303F9F}
\definecolor{outcolor}{HTML}{D84315}
\newcommand{\LineaIn}{
    \llap{\scriptsize\ttfamily\color{incolor}[In]:\hspace{6pt}}%
    \vspace{-1.24\baselineskip}
    }
\newcommand{\LineaOut}{
    \llap{\scriptsize\ttfamily\color{outcolor}[Out]:\hspace{6pt}}%
    \vspace{-0.95\baselineskip}
    }
    

% Estilo para python
\lstloadlanguages{Python}

% Configuración de estilo
\definecolor{codverde}{rgb}{0,0.6,0}
\definecolor{codgris}{rgb}{0.5,0.5,0.5}
\definecolor{coduva}{rgb}{0.58,0,0.82}
\definecolor{codazul}{rgb}{0.0,0,0.8}
\lstset{
    language=Python,
    basicstyle=\small\ttfamily,
    stringstyle=\color{coduva},
    commentstyle=\color{codgris},
    emph={[2]from,import,pass,return}, emphstyle={[2]\color{codverde}},
    emph={[3]range,print,display,Matrix}, emphstyle={[3]\color{codazul}},
    emph={[4]for,in,def}, emphstyle={[4]\color{blue}},
    breaklines=true,
    prebreak=\mbox{{\color{gray}\tiny$\searrow$}},
    extendedchars=true,
    literate=   {á}{{\'a}}1
                {é}{{\'e}}1 
                {í}{{\'i}}1 
                {ó}{{\'o}}1 
                {ú}{{\'u}}1 
                {ñ}{{\~n}}1,
    showstringspaces=false,
}

\usepackage{fancyvrb} % verbatim replacement that allows latex


% -- Datos 
\institucion{Escuela de Ciencias Físicas y Matemática}
\carrera{Ciencia de datos}
\asignatura{Álgebra lineal}
\tema{Examen no. 1-1: Matrices}
\autor{Andrés Merino}
\fecha{Semestre 2024-1}

\logouno[0.14\textwidth]{Logos/logoPUCE_04_ac}
\definecolor{colortext}{HTML}{0030A1}
\definecolor{colordef}{HTML}{0030A1}
\fuente{montserrat}


\begin{document}

\encabezado

\section*{Indicaciones}
\begin{itemize}[leftmargin=*]
\item 
    En esta actividad se evalúa si el estudiante resuelve operaciones con matrices, incluyendo productos y cálculo de inversas; además del cálculo de determinantes de matrices y su significado en el contexto del Álgebra Lineal.
\item 
    Los ejercicios deben ser resueltos «a mano», sin ayuda de resúmenes o asistentes computacionales.
\item
    Las resoluciones de los ejercicios deben ser entregadas en hojas de forma física al docente. No importa el orden de resolución de los ejercicios.
\item
    Se encuentra prohibido el uso de cualquier otra fuente de información durante todo el examen.
\item
    En caso de considerar que existe un error en la pregunta o que esta se encuentra mal planteada, se debe indicar cuál es el error y justificarlo.
\item
    Todas las soluciones deben estar correctamente redactadas y explicadas.
\item 
    Se asignarán 4 puntos de excelencia si, a más de tener las soluciones correctas, el examen se destaca por su orden y escritura.
\end{itemize}

\section*{Ejercicios}

\begin{preguntas}

%%%%%%%%%%%%%%%%%%%%%%%%%%%%%%%%%%%%%%%%%%
%%%%%%%%%%%%%%%%%%%%%%%%%%%%%%%%%%%%%%%%%%
%%%%%%%%%%%%%%%%%%%%%%%%%%%%%%%%%%%%%%%%%%
\item
    Dada la siguiente matriz
    \[
        A=\begin{pmatrix}
                2 & 4 & -4 \\
                -3 & -2 & 1 \\
                2 & 2 & 3 \\
        \end{pmatrix},
    \]
    realice las siguientes operaciones elementales por filas de forma consecutiva:
    \begin{itemize}
        \item $\frac{1}{2}F_1\rightarrow F_1$ \puntaje{1.5}
        \item $3F_1+F_2\rightarrow F_2$ \puntaje{1.5}
        \item $-F_1+F_3\rightarrow F_3$ \puntaje{1.5}
        \item $2F_2+F_1\rightarrow F_3$ \puntaje{1.5}
    \end{itemize}

\begin{respuesta}
    \begin{itemize}
    \item 
        $\frac{1}{2}F_1\rightarrow F_1$:
        \[
            \begin{pmatrix}1 & 2 & -2\\-3 & 2 & 3\\2 & -2 & 1\end{pmatrix}
        \]
    \item 
        $3F_1+F_2\rightarrow F_2$:
        \[
            \begin{pmatrix}1 & 2 & -2\\3 & 10 & -11\\2 & -2 & 1\end{pmatrix}
        \]
    \item 
        $-F_1+F_3\rightarrow F_3$:
        \[
            \begin{pmatrix}1 & 2 & -2\\-4 & 0 & 5\\1 & 0 & 3\end{pmatrix}
        \]
    \item
        $2F_2+F_1\rightarrow F_3$: esta no es una operación elemental por filas válida.\qedhere
    \end{itemize}
\end{respuesta}

%%%%%%%%%%%%%%%%%%%%%%%%%%%%%%%%%%%%%%%%%%
%%%%%%%%%%%%%%%%%%%%%%%%%%%%%%%%%%%%%%%%%%
%%%%%%%%%%%%%%%%%%%%%%%%%%%%%%%%%%%%%%%%%%
\item
    Considere $k\in \R$ y las siguientes matrices:
    \[
    	M=\begin{pmatrix}
    	  	k & 2 & 0\\ 1 & -1 & 2k
    	  	\end{pmatrix}
    	\qquad\text{y}\qquad
    	N=\begin{pmatrix}
    	  	0 & -2\\
                k & 1\\
                2 & 3\\
    	  \end{pmatrix},
    \]
    determine para qué valores de $k$ el producto $MN$ tiene inversa.\puntaje{10}

\begin{respuesta}
    Primero, calculemos el producto de las matrices; así, notemos que 
    \[
        MN = \begin{pmatrix}
            2k & -2k+2 \\
            3k & 6k-3
        \end{pmatrix}.
    \]
    Luego, como la matriz $MN$ tiene inversa si y solo si su determinante es distinto de cero; entonces, como 
    \[
        \det(MN) = \begin{vmatrix}
            2k & -2k+2 \\
            3k & 6k-3
        \end{vmatrix} = -12k+18k^2.
    \]
    Por lo tanto, se tiene que para $k\in\R\setminus\left\{0,\dfrac{2}{3}\right\}$, la matriz $MN$ tiene inversa. 
\end{respuesta}

%%%%%%%%%%%%%%%%%%%%%%%%%%%%%%%%%%%%%%%%%%
%%%%%%%%%%%%%%%%%%%%%%%%%%%%%%%%%%%%%%%%%%
%%%%%%%%%%%%%%%%%%%%%%%%%%%%%%%%%%%%%%%%%%
\item
    Considere $\alpha\in \R$ y la siguiente matriz:
    \[
    	A=\begin{pmatrix}
    	  1 & 2 & 3\alpha \\
            4 & -2 & 2\alpha \\
            0 & 2 & -4 \\
    	\end{pmatrix},
    \]
    Calcula el determinante de $A$ realizando expansión por cofactores a la tercera fila.\puntaje{10}

\begin{respuesta}
    Aplicando la expansión por cofactores a la tercera fila, se tiene que
    \[
        \det(A) = + (0) \begin{vmatrix}
            2 & 3\alpha \\
            -2 & 2\alpha
        \end{vmatrix} - (2) \begin{vmatrix}
            1 & 3\alpha \\
            4 & 2\alpha
        \end{vmatrix} + (-4)\begin{vmatrix}  
            1 & 2 \\
            4 & -2
        \end{vmatrix}.
    \]
    Resolviendo los determinantes, se obtiene que
    \[
        \det(A) = 20\alpha + 40. \qedhere
    \]
\end{respuesta}

%%%%%%%%%%%%%%%%%%%%%%%%%%%%%%%%%%%%%%%%%%
%%%%%%%%%%%%%%%%%%%%%%%%%%%%%%%%%%%%%%%%%%
%%%%%%%%%%%%%%%%%%%%%%%%%%%%%%%%%%%%%%%%%%
\item
    Calcule, usando reducción por filas, la inversa de la siguiente matriz:\puntaje{10}
    \[
    	A=\begin{pmatrix}
    	  1 & 2 & 3 \\
            4 & -2 & 2 \\
            0 & 0 & -1 \\
    	\end{pmatrix}.
    \]

\begin{respuesta}
    Para saber si es posible calcular la inversa de la matriz, primero se debe calcular el determinante de la misma. Así, se tiene que
    \[
        \det(A) = -1\begin{vmatrix}
            1 & 2 \\
            4 & -2    
        \end{vmatrix} = 10.
    \]
    Por lo tanto, la matriz $A$ tiene inversa y se tiene que
    \[
        (A|I_3)\sim (I_3|A^{-1}).
    \]
    Entonces, notando que 
    \[
        (A|I_3) = \begin{pmatrix}
            1 & 2 & 3 & | & 1 & 0 & 0 \\
            4 & -2 & 2 & | & 0 & 1 & 0 \\
            0 & 0 & -1 & | & 0 & 0 & 1
        \end{pmatrix},    
    \]
    y aplicando operaciones de filas, se tiene que
    \begin{align*}
        \begin{pmatrix}
            1 & 2 & 3 & | & 1 & 0 & 0 \\
            4 & -2 & 2 & | & 0 & 1 & 0 \\
            0 & 0 & -1 & | & 0 & 0 & 1
        \end{pmatrix} & \sim \begin{pmatrix}
            1 & 2 & 3 & | & 1 & 0 & 0 \\
            1 & -\frac{1}{2} & \frac{1}{2} & | & 0 & \frac{1}{4} & 0 \\
            0 & 0 & -1 & | & 0 & 0 & 1
        \end{pmatrix} && \frac{1}{4} F_2 \rightarrow F_2 \\
        & \sim \begin{pmatrix}
            1 & 2 & 3 & | & 1 & 0 & 0 \\
            0 & -\frac{5}{2} & -\frac{5}{2} & | & -1 & \frac{1}{4} & 0 \\
            0 & 0 & -1 & | & 0 & 0 & 1
        \end{pmatrix} && - F_1 + F_2 \rightarrow F_2 \\
        & \sim \begin{pmatrix}
            1 & 2 & 3 & | & 1 & 0 & 0 \\
            0 & 1 & 1 & | & \frac{2}{5} & -\frac{1}{10} & 0 \\
            0 & 0 & -1 & | & 0 & 0 & 1
        \end{pmatrix} && -\frac{2}{5} F_2 \rightarrow F_2 \\
        & \sim \begin{pmatrix}
            1 & 0 & 1 & | & \frac{1}{5} & \frac{1}{5} & 0 \\
            0 & 1 & 1 & | & \frac{2}{5} & -\frac{1}{10} & 0 \\
            0 & 0 & -1 & | & 0 & 0 & 1
        \end{pmatrix} && - 2F_2 + F_1 \rightarrow F_1 \\
        & \sim \begin{pmatrix}
            1 & 0 & 0 & | & \frac{1}{5} & \frac{1}{5} & 1 \\
            0 & 1 & 1 & | & \frac{2}{5} & -\frac{1}{10} & 0 \\
            0 & 0 & -1 & | & 0 & 0 & 1
        \end{pmatrix} &&  F_3 + F_1 \rightarrow F_1 \\
        & \sim \begin{pmatrix}
            1 & 0 & 0 & | & \frac{1}{5} & \frac{1}{5} & 1 \\
            0 & 1 & 0 & | & \frac{2}{5} & -\frac{1}{10} & 1 \\
            0 & 0 & -1 & | & 0 & 0 & 1
        \end{pmatrix} && F_3 + F_2 \rightarrow F_2 \\
        & \sim \begin{pmatrix}
            1 & 0 & 0 & | & \frac{1}{5} & \frac{1}{5} & 1 \\
            0 & 1 & 0 & | & \frac{2}{5} & -\frac{1}{10} & 1 \\
            0 & 0 & 1 & | & 0 & 0 & -1
        \end{pmatrix} && - F_3 \rightarrow F_3,
    \end{align*}
    y, por lo tanto
    \[
        A^{-1} = \begin{pmatrix}
            \frac{1}{5} & \frac{1}{5} & 1 \\
            \frac{2}{5} & -\frac{1}{10} & 1 \\
            0 & 0 & -1
        \end{pmatrix}.    \qedhere
    \]

\end{respuesta}
%%%%%%%%%%%%%%%%%%%%%%%%%%%%%%%%%%%%%%%%%%
%%%%%%%%%%%%%%%%%%%%%%%%%%%%%%%%%%%%%%%%%%
%%%%%%%%%%%%%%%%%%%%%%%%%%%%%%%%%%%%%%%%%%
\item
    Determine si existen valores de $\alpha\in\R$ y $\beta\in\R$ tales que las matrices $A$ y $B$ conmutan, donde
    \[
    A = \begin{pmatrix}
        \alpha & 1 \\
        1 & 0
    \end{pmatrix} \texty B = \begin{pmatrix}
        1 & \beta \\
        0 & 1+\beta
    \end{pmatrix}.
    \]
    En caso de existir, determínelos.\puntaje{10}

\begin{respuesta}
    Tenemos que 
    \[
        AB = \begin{pmatrix}
            \alpha & 1 + \beta + \alpha\beta \\
            1 & \beta
        \end{pmatrix} \texty BA = \begin{pmatrix}
            \alpha+\beta & 1 \\
            1+\beta & 0
        \end{pmatrix},
    \]
    de donde se tiene que 
    \begin{align*}
        \alpha &= \alpha + \beta, \\
        1 + \beta + \alpha\beta &= 1, \\
        1 &= 1 + \beta, \\
        \beta &= 0.
    \end{align*}
    Observamos entonces que la única posibilidad es que $\beta = 0$ y en tal caso, todas las ecuaciones anteriores se verifican. Por ende, para $\beta=0$ y para todo $\alpha\in\R$, las matrices $A$ y $B$ conmutan.
\end{respuesta}

\end{preguntas}

\end{document}