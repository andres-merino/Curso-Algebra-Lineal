\documentclass[a4,11pt]{aleph-notas}
% Actualizado en febrero de 2024
% Funciona con TeXLive 2022
% Para obtener solo el pdf, compilar con pdfLaTeX. 
%  latexmk -pdf 01\ Cuestionarios/01\ Matrices.tex -output-directory="01 Cuestionarios"
% Para obtener el xml compilar con XeLaTeX.
% latexmk -xelatex 01\ Cuestionarios/01\ Matrices.tex -output-directory="01 Cuestionarios"

% -- Paquetes adicionales
\usepackage{aleph-moodle}
\moodleregisternewcommands
% Todos los comandos nuevos deben ir luego del comando anterior
\usepackage{aleph-comandos}


% -- Datos  
\institucion{Escuela de Ciencias Físicas y Matemática}
\carrera{Ciencia de datos / Bioingeniería}
\asignatura{Álgebra Lineal}
\tema{control de lectura: Coordenadas y Cambio de base}
\autor{Andrés Merino}
\fecha{Semestre 2024-1}

\logouno[0.14\textwidth]{Logos/logoPUCE_04_ac}
\definecolor{colortext}{HTML}{0030A1}
\definecolor{colordef}{HTML}{0030A1}
\fuente{montserrat}

% -- Otros comandos



\begin{document}

\encabezado

\vspace*{-8mm}
\tableofcontents

%%%%%%%%%%%%%%%%%%%%%%%%%%%%%%%%%%%%%%%%
\section{Indicaciones}
%%%%%%%%%%%%%%%%%%%%%%%%%%%%%%%%%%%%%%%%

Se plantean bancos de preguntas orientados a realizar el control de lectura de la sección 4.7 del libro de Larson.

%%%%%%%%%%%%%%%%%%%%%%%%%%%%%%%%%%%%%%%%
\section{Banco de preguntas}
%%%%%%%%%%%%%%%%%%%%%%%%%%%%%%%%%%%%%%%%

%%%%%%%%%%%%%%%%%%%%%%%%%%%%%%%%%%%%%%%%
\begin{quiz}{Control - Coordenadas y cambio de base}
%%%%%%%%%%%%%%%%%%%%%%%%%%%%%%%%%%%%%%%%

\begin{multi}[]{Q1}
    ¿Qué afirmación es verdadera respecto a la representación de coordenadas en $\mathbb{R}^n$?
    \item* Todo vector $x$ en un espacio vectorial $V$ puede expresarse de manera única como una combinación lineal de vectores en una base $B$.
    \item Todo vector en $\mathbb{R}^n$ tiene la misma representación de coordenadas en cualquier base.
    \item Las coordenadas de un vector en $\mathbb{R}^n$ siempre corresponden a sus componentes en la base estándar.
    \item La representación de coordenadas no depende del orden de los vectores en la base.
\end{multi}

\begin{multi}[]{Q2}
    ¿Cuál es la matriz de coordenadas del vector $x = (–2, 1, 3)$ en $\mathbb{R}^3$ con respecto a la base estándar?
    \item* $\begin{pmatrix} -2 \\ 1 \\ 3 \end{pmatrix}$
    \item $\begin{pmatrix} 2 \\ 1 \\ 3 \end{pmatrix}$
    \item $\begin{pmatrix} 1 \\ -2 \\ 3 \end{pmatrix}$
    \item $\begin{pmatrix} 3 \\ 1 \\ -2 \end{pmatrix}$
\end{multi}

\begin{multi}[]{Q3}
    En el proceso de cambio de base, ¿qué representa la matriz $P$ en la ecuación matricial $x_{B} = P x_{B'}$?
    \item* La matriz de transición de $B'$ a $B$.
    \item La matriz de coordenadas de $x$ relativa a $B$.
    \item La matriz de coordenadas de $x$ relativa a $B'$.
    \item La inversa de la matriz de transición de $B$ a $B'$.
\end{multi}

\begin{multi}[]{Q4}
    ¿Qué propiedad tiene la matriz de transición $P$?
    \item* $P$ es invertible y $P^{-1}$ es la matriz de transición de $B$ a $B'$.
    \item $P$ no es invertible en la mayoría de los casos.
    \item $P$ y $P^{-1}$ representan la misma transición de bases.
    \item $P$ siempre es igual a la matriz identidad independientemente de las bases.
\end{multi}

\begin{multi}[]{Q5}
    ¿Qué método se utiliza para encontrar la matriz de transición $P^{-1}$?
    \item* Eliminación de Gauss-Jordan en la matriz $[B' \; B]$.
    \item Multiplicación directa de las matrices $B$ y $B'$.
    \item Suma de las matrices de las bases $B$ y $B'$.
    \item Inversión de la matriz de coordenadas de $x$.
\end{multi}

\begin{multi}[]{Q7}
    ¿Cuál es el propósito del cambio de base en espacios vectoriales?
    \item* Encontrar las coordenadas de un vector relativas a otra base.
    \item Cambiar las dimensiones del espacio vectorial.
    \item Simplificar las operaciones de vectores para cálculos básicos.
    \item Identificar las bases que no son estándar.
\end{multi}

\begin{multi}[]{Q8}
    ¿Cuál es la representación de coordenadas de $p = 3x^3 - 2x^2 + 4$ en $P_3$ con respecto a la base estándar $S = \{1, x, x^2, x^3\}$?
    \item* $\begin{pmatrix} 4 \\ 0 \\ -2 \\ 3 \end{pmatrix}$
    \item $\begin{pmatrix} 3 \\ -2 \\ 4 \\ 0 \end{pmatrix}$
    \item $\begin{pmatrix} 0 \\ 4 \\ -2 \\ 3 \end{pmatrix}$
    \item $\begin{pmatrix} -2 \\ 3 \\ 4 \\ 0 \end{pmatrix}$
\end{multi}

\begin{multi}[]{Q9}
    ¿Cómo se generaliza el concepto de coordenadas para representar vectores en espacios $n$-dimensionales?
    \item* Permite representar vectores de cualquier espacio n-dimensional usando la notación de $\mathbb{R}^n$.
    \item Restringe el uso de coordenadas a los espacios vectoriales que son isomorfos a $\mathbb{R}^n$.
    \item Elimina la necesidad de bases para definir las coordenadas de un vector.
    \item Solo permite representar vectores en espacios de dimensiones superiores a $\mathbb{R}^3$.
\end{multi}

\begin{multi}[]{Q10}
    De acuerdo con el teorema 4.21, ¿cómo se determina la matriz de transición $P^{-1}$ de una base $B$ a otra base $B'$?
    \item* Mediante la eliminación de Gauss-Jordan aplicada a la matriz $[B' \; B]$.
    \item Utilizando la multiplicación matricial entre las matrices de las bases $B$ y $B'$.
    \item Sumando las matrices de las bases $B$ y $B'$.
    \item Calculando la inversa de la matriz de coordenadas del vector $x$.
\end{multi}

\begin{multi}[]{Q11}
    ¿Qué afirmación describe correctamente el proceso de cambio de base de una base no estándar a una base estándar en $\mathbb{R}^n$?
    \item* La matriz de transición puede ser más sencilla de determinar debido a la estructura de la base estándar.
    \item Siempre se requiere la inversión de la matriz de coordenadas del vector para completar el cambio de base.
    \item No es necesario conocer las coordenadas del vector en la base original.
    \item El cambio de base de una base no estándar a una estándar es matemáticamente imposible.
\end{multi}

\begin{multi}[]{Q13}
    ¿Cómo se representa la matriz de coordenadas de un vector en el espacio de polinomios de grado menor o igual a 3?
    \item* Como un vector en $\mathbb{R}^4$.
    \item Como un vector en $\mathbb{R}^3$.
    \item Como una matriz 3x3 en $\mathbb{R}^3$.
    \item Como un polinomio de grado 3.
\end{multi}

\begin{multi}[]{Q14}
    En el contexto de las coordenadas y el cambio de base, ¿qué importancia tiene el orden de los vectores en una base?
    \item* El orden es crucial porque afecta la representación de las coordenadas del vector.
    \item El orden de los vectores en una base es irrelevante y no afecta las coordenadas.
    \item Solo es importante en bases estándar y no en bases no estándar.
    \item Afecta únicamente la visualización gráfica, pero no las operaciones matemáticas.
\end{multi}

\begin{multi}[]{Q15}
    ¿Qué método se utiliza para determinar la matriz de transición $P^{-1}$ de una base $B$ a otra base $B'$, según el Teorema 4.21?
    \item* Eliminación de Gauss-Jordan en la matriz aumentada $[B' | B]$.
    \item Factorización LU de la matriz $[B' | B]$.
    \item Descomposición en valores singulares de la matriz $[B' | B]$.
    \item Multiplicación directa de las matrices $B$ y $B'$.
\end{multi}

\begin{multi}[]{Q16}
    ¿Cuál es la principal ventaja de representar coordenadas en espacios $n$-dimensionales generales, según el texto?
    \item* Permite la representación de vectores de un espacio $n$-dimensional arbitrario con la misma notación usada en $\mathbb{R}^n$.
    \item Facilita el cálculo de dimensiones adicionales en espacios $n$-dimensionales.
    \item Reduce la complejidad computacional de las operaciones vectoriales en dimensiones altas.
    \item Elimina la necesidad de bases para definir coordenadas en espacios vectoriales.
\end{multi}


\end{quiz}


\end{document}