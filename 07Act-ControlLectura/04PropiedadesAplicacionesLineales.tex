\documentclass[a4,11pt]{aleph-notas}
% Actualizado en febrero de 2024
% Funciona con TeXLive 2022
% Para obtener solo el pdf, compilar con pdfLaTeX. 
%  latexmk -pdf 01\ Cuestionarios/01\ Matrices.tex -output-directory="01 Cuestionarios"
% Para obtener el xml compilar con XeLaTeX.
% latexmk -xelatex 01\ Cuestionarios/01\ Matrices.tex -output-directory="01 Cuestionarios"

% -- Paquetes adicionales
\usepackage{aleph-moodle}
\moodleregisternewcommands
% Todos los comandos nuevos deben ir luego del comando anterior
\usepackage{aleph-comandos}


% -- Datos
\institucion{Escuela de Ciencias Físicas y Matemática}
\carrera{Ciencia de datos / Bioingeniería}
\asignatura{Álgebra Lineal}
\tema{Control de lectura: Propiedades de las Aplicaciones Lineales}
\autor{Andrés Merino}
\fecha{Semestre 2024-1}

\logouno[0.14\textwidth]{Logos/logoPUCE_04_ac}
\definecolor{colortext}{HTML}{0030A1}
\definecolor{colordef}{HTML}{0030A1}
\fuente{montserrat}

% -- Otros comandos



\begin{document}

\encabezado

\vspace*{-8mm}
\tableofcontents

%%%%%%%%%%%%%%%%%%%%%%%%%%%%%%%%%%%%%%%%
\section{Indicaciones}
%%%%%%%%%%%%%%%%%%%%%%%%%%%%%%%%%%%%%%%%

Se plantean bancos de preguntas orientados a realizar el control de lectura de la sección 6.2 del libro de Larson y sección 5.3 del libro de Aranda.

%%%%%%%%%%%%%%%%%%%%%%%%%%%%%%%%%%%%%%%%
\section{Banco de preguntas}
%%%%%%%%%%%%%%%%%%%%%%%%%%%%%%%%%%%%%%%%

%%%%%%%%%%%%%%%%%%%%%%%%%%%%%%%%%%%%%%%%
\begin{quiz}{Control - Propiedades Aplicacion Lineal}
%%%%%%%%%%%%%%%%%%%%%%%%%%%%%%%%%%%%%%%%

\begin{multi}[]%
    {P2}     
    ¿Cuál de las siguientes afirmaciones sobre la operación suma de aplicaciones lineales es correcta?     
    \item* Si \( f \) y \( g \) son aplicaciones lineales de \( V \) en \( W \), entonces \( (f + g)(v) = f(v) + g(v) \)     
    \item Si \( f \) y \( g \) son aplicaciones lineales de \( V \) en \( W \), entonces \( (f + g)(v) = f(g(v)) \)     
    \item Si \( f \) y \( g \) son aplicaciones lineales de \( V \) en \( W \), entonces \( (f + g)(v) = fg(v) \)     
    \item Si \( f \) y \( g \) son aplicaciones lineales de \( V \) en \( W \), entonces \( (f + g)(v) = f(v) - g(v) \)
\end{multi}

\begin{multi}[]%
    {P3}     
    ¿Cuál es el resultado de la composición \( g \circ f \) si \( f \) y \( g \) son aplicaciones lineales entre los espacios indicados y sus matrices son \( A \) y \( B \), respectivamente?     
    \item* La matriz de \( g \circ f \) es \( BA \)     
    \item La matriz de \( g \circ f \) es \( AB \)     
    \item La matriz de \( g \circ f \) es \( A + B \)     
    \item La matriz de \( g \circ f \) es \( A - B \)
\end{multi}

\begin{multi}[]%
    {P4}     
    ¿Qué característica debe tener una aplicación lineal para que sea invertible?     
    \item* La aplicación debe ser biyectiva     
    \item La aplicación debe ser solo inyectiva     
    \item La aplicación debe ser solo sobreyectiva     
    \item Ninguna de las anteriores es necesaria
\end{multi}

\begin{multi}[]%
    {P5}     
    ¿Cuál es la propiedad de la inversa de una aplicación lineal?     
    \item* \( f^{-1}(\alpha x + \beta y) = \alpha f^{-1}(x) + \beta f^{-1}(y) \)     
    \item \( f^{-1}(\alpha x + \beta y) = \alpha f(x) + \beta f(y) \)     
    \item \( f^{-1}(\alpha x + \beta y) = f(\alpha x + \beta y) \)     
    \item \( f^{-1}(\alpha x + \beta y) = f^{-1}(\alpha) + f^{-1}(\beta) \)
\end{multi}

\begin{multi}[]%
    {P6}     
    Si \( A \) es la matriz asociada a la aplicación biyectiva \( f : V \rightarrow W \), ¿cuál es la matriz asociada a la inversa de \( f \)?     
    \item* \( A^{-1} \)     
    \item \( A^{T} \)
    \item \( 2A \)     
    \item \( A^2 \)
\end{multi}


\begin{multi}[]%
    {P7}     
    ¿Qué conjunto se denomina kernel de una transformación lineal \(T: V \rightarrow W\)?     
    \item* El conjunto de todos los vectores en \(V\) que son mapeados al vector cero en \(W\)     
    \item El conjunto de todos los vectores en \(W\) que son mapeados al vector cero en \(V\)     
    \item El conjunto de todos los vectores en \(V\) que son mapeados a ellos mismos en \(W\)     
    \item El conjunto de todas las transformaciones posibles de \(V\) a \(W\)
\end{multi}

\begin{multi}[]%
    {P8}     
    ¿Cuál de las siguientes afirmaciones sobre el rango de una transformación lineal \(T: V \rightarrow W\) es correcta?     
    \item* El rango de \(T\) es un subespacio de \(W\)     
    \item El rango de \(T\) es un subespacio de \(V\)     
    \item El rango de \(T\) incluye todos los vectores de \(V\)     
    \item El rango de \(T\) es siempre igual a la dimensión de \(V\)
\end{multi}

\begin{multi}[]%
    {P9}     
    ¿Bajo qué condición una transformación lineal \(T: V \rightarrow W\) es inyectiva?     
    \item* \(T\) es uno a uno si y solo si el kernel de \(T\) contiene únicamente el vector cero     
    \item \(T\) es uno a uno si el kernel de \(T\) incluye al menos un vector diferente de cero     
    \item \(T\) es uno a uno si su rango es igual al dominio \(V\)     
    \item \(T\) es uno a uno si cada vector en \(V\) tiene una preimagen única en \(W\)
\end{multi}

\begin{multi}[]%
    {P10}     
    ¿Cuál es la relación entre la dimensión del dominio \(V\) y la suma de las dimensiones de la imagen y el kernel de una transformación lineal \(T: V \rightarrow W\)?     
    \item* La dimensión del dominio es igual a la suma de las dimensiones de la imagen y el kernel     
    \item La dimensión del dominio es mayor que la suma de las dimensiones de la imagen y el kernel     
    \item La dimensión del dominio es menor que la suma de las dimensiones de la imagen y el kernel     
    \item No existe relación entre la dimensión del dominio y la suma de las dimensiones de la imagen y el kernel
\end{multi}

\begin{multi}[]%
    {P11}     
    Si una transformación lineal \(T: V \rightarrow W\) es un isomorfismo, ¿qué se puede afirmar sobre las dimensiones de \(V\) y \(W\)?     
    \item* \(V\) y \(W\) tienen la misma dimensión     
    \item \(V\) tiene una dimensión mayor que \(W\)     
    \item \(W\) tiene una dimensión mayor que \(V\)     
    \item Las dimensiones de \(V\) y \(W\) no están relacionadas
\end{multi}

\begin{multi}[]%
    {P12}     
    ¿Qué es una transformación lineal en el contexto de álgebra lineal?     
    \item* Una función que toma un vector y devuelve otro, manteniendo líneas rectas y el origen fijo.     
    \item Una función que convierte vectores en matrices.     
    \item Una operación que siempre curva las líneas y mueve el origen.     
    \item Una transformación que convierte matrices en vectores.
\end{multi}

\begin{multi}[]%
    {P13}     
    ¿Cómo se visualiza una transformación lineal?     
    \item* Imaginando los puntos de una cuadrícula infinita moviéndose hacia nuevos puntos sin cambiar su paralelismo o distancia relativa.     
    \item Pensando en vectores como flechas que apuntan de un espacio a otro.     
    \item Visualizando cómo los vectores se curvan y el espacio se distorsiona.     
    \item Considerando solo los vectores que se mueven hacia la derecha o hacia arriba.
\end{multi}

\begin{multi}[]%
    {P14}     
    ¿Qué característica clave debe cumplir una transformación para ser considerada lineal?     
    \item* Las líneas de la cuadrícula deben permanecer paralelas y a la misma distancia unas de otras.     
    \item Todas las líneas deben converger en el origen.     
    \item Las líneas rectas deben convertirse en curvas.     
    \item El origen debe desplazarse a una nueva posición.
\end{multi}

\begin{multi}[]%
    {P15}     
    ¿Cómo se pueden describir numéricamente las transformaciones lineales?     
    \item* A través de las coordenadas de los vectores de la base después de la transformación, organizadas en una matriz.     
    \item Usando una serie de funciones trigonométricas para calcular las nuevas posiciones.     
    \item Mediante el cálculo diferencial de cada componente del vector.     
    \item Con un conjunto de ecuaciones diferenciales que describen el movimiento de cada punto.
\end{multi}

\begin{multi}[]%
    {P16}     
    ¿Cuál es un método para pensar en el producto de matrices?     
    \item* Como la combinación lineal de los vectores de la base transformados, donde cada columna de la matriz representa el resultado de aplicar la transformación a cada vector de la base.     
    \item Memorizar las reglas de multiplicación de matrices sin entender el significado subyacente.     
    \item Visualizando el producto de matrices como una serie de operaciones aritméticas sin relación geométrica.     
    \item Ignorando las propiedades espaciales y concentrándose solo en los cálculos numéricos.
\end{multi}

\begin{multi}[]%
    {P17}     
    ¿Por qué se suele trabajar principalmente en dos dimensiones?     
    \item* Porque es más fácil visualizar y entender los conceptos en dos dimensiones antes de extenderlos a dimensiones superiores.     
    \item Porque las transformaciones en dos dimensiones son más complejas y requieren mayor atención.     
    \item Porque las matrices en dos dimensiones son más fáciles de calcular que en tres dimensiones.     
    \item Porque las transformaciones en dos dimensiones son menos útiles en aplicaciones prácticas.
\end{multi}

\begin{multi}[]%
    {P18}     
    ¿Cómo se puede visualizar una transformación que maneja vectores tridimensionales?     
    \item* Moviendo puntos en un espacio tridimensional de manera que las líneas de las cuadrículas se mantengan paralelas y el origen fijo.     
    \item Utilizando simulaciones computacionales que generan imágenes en 4D.     
    \item Ignorando las dimensiones y enfocándose solo en los cambios de magnitud de los vectores.     
    \item Representando cada vector como un punto en un plano bidimensional.
\end{multi}

\begin{multi}[]%
    {P19}     
    ¿Cómo se describen completamente las transformaciones tridimensionales?     
    \item* Especificando a dónde van a parar los tres vectores de la base unitaria y organizando estas coordenadas en una matriz de tres por tres.     
    \item Definiendo la orientación y longitud de cada vector después de la transformación.     
    \item Calculando el producto cruz de cada vector con los vectores unitarios estándar.     
    \item Usando una serie de transformaciones lineales en dos dimensiones.
\end{multi}

\begin{multi}[]%
    {P21}     
    ¿Por qué son importantes las matrices de tres dimensiones en campos como el diseño gráfico computacional y la robótica?     
    \item* Porque permiten describir transformaciones complejas como rotaciones en tres dimensiones de manera más comprensible y descomponible.     
    \item Porque simplifican las ecuaciones necesarias para renderizar gráficos en tres dimensiones.     
    \item Porque reducen el costo computacional al trabajar con grandes volúmenes de datos.     
    \item Porque permiten una mayor precisión en la modelación de fenómenos físicos en simulaciones.
\end{multi}


\end{quiz}


\end{document}