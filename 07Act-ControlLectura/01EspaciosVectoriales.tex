\documentclass[a4,11pt]{aleph-notas}
% Actualizado en febrero de 2024
% Funciona con TeXLive 2022
% Para obtener solo el pdf, compilar con pdfLaTeX. 
%  latexmk -pdf 01\ Cuestionarios/01\ Matrices.tex -output-directory="01 Cuestionarios"
% Para obtener el xml compilar con XeLaTeX.
% latexmk -xelatex 01\ Cuestionarios/01\ Matrices.tex -output-directory="01 Cuestionarios"

% -- Paquetes adicionales
\usepackage{aleph-moodle}
\moodleregisternewcommands
% Todos los comandos nuevos deben ir luego del comando anterior
\usepackage{aleph-comandos}


% -- Datos  
\institucion{Escuela de Ciencias Físicas y Matemática}
\carrera{Ciencia de datos / Bioingeniería}
\asignatura{Álgebra Lineal}
\tema{control de lectura: Espacios Vectoriales}
\autor{Andrés Merino}
\fecha{Semestre 2024-1}

\logouno[0.14\textwidth]{Logos/logoPUCE_04_ac}
\definecolor{colortext}{HTML}{0030A1}
\definecolor{colordef}{HTML}{0030A1}
\fuente{montserrat}

% -- Otros comandos



\begin{document}

\encabezado

\vspace*{-8mm}
\tableofcontents

%%%%%%%%%%%%%%%%%%%%%%%%%%%%%%%%%%%%%%%%
\section{Indicaciones}
%%%%%%%%%%%%%%%%%%%%%%%%%%%%%%%%%%%%%%%%

Se plantean bancos de preguntas orientados a realizar el control de lectura de las secciones 4.2 del libro de Larson y 4.1 del libro de Aranda.

%%%%%%%%%%%%%%%%%%%%%%%%%%%%%%%%%%%%%%%%
\section{Banco de preguntas}
%%%%%%%%%%%%%%%%%%%%%%%%%%%%%%%%%%%%%%%%

%%%%%%%%%%%%%%%%%%%%%%%%%%%%%%%%%%%%%%%%
\begin{quiz}{Control Espacios Vectoriales}
%%%%%%%%%%%%%%%%%%%%%%%%%%%%%%%%%%%%%%%%

%%%%%%%%%%%%%%%%%%%%%%%%%%%%%%%%%%%%%%%%
\begin{multi}{Contr. Espacio Vectorial 1}%
    {¿Qué es un espacio vectorial?}
    \item* Un conjunto $V$ junto con dos operaciones (suma de vectores y producto por escalar) que cumple ciertas propiedades.
    \item Un conjunto de vectores sin ninguna operación definida.
    \item Una colección de objetos que pueden sumarse entre sí o multiplicarse por números, sin reglas específicas.
    \item Una estructura matemática que incluye únicamente la operación de suma de vectores.
\end{multi}

\begin{multi}{Contr. Espacio Vectorial 2}%
    {Según el texto, ¿cuál de las siguientes opciones describe correctamente al cuerpo $\mathbb{K}$ en el contexto de espacios vectoriales?}
    \item* $\mathbb{K}$ puede ser el conjunto de los números reales $\mathbb{R}$ o el conjunto de los números complejos $\mathbb{C}$.
    \item $\mathbb{K}$ es exclusivamente el conjunto de los números enteros.
    \item $\mathbb{K}$ representa cualquier conjunto de matrices.
    \item $\mathbb{K}$ es un vector específico dentro del espacio vectorial.
\end{multi}

\begin{multi}{Contr. Espacio Vectorial 3}%
    {¿Cuál de las siguientes propiedades NO es requerida para que un conjunto $V$ sea considerado un espacio vectorial?}
    \item La multiplicación de un vector por un escalar está bien definida.
    \item* Todos los vectores en $V$ deben tener magnitud unitaria.
    \item La suma de vectores en $V$ está bien definida y es cerrada dentro de $V$.
    \item Existe un vector cero que actúa como el elemento neutro en la suma de vectores.
\end{multi}

\begin{multi}{Contr. Espacio Vectorial 4}%
    {¿Qué se afirma sobre los elementos neutros y opuestos en un espacio vectorial?}
    \item* El elemento neutro y el opuesto en un espacio vectorial son únicos.
    \item Hay múltiples elementos neutros y opuestos para cualquier espacio vectorial.
    \item El elemento neutro y el opuesto pueden variar dependiendo del vector en cuestión.
    \item No existen elementos neutros ni opuestos en los espacios vectoriales.
\end{multi}

\begin{multi}{Contr. Espacio Vectorial 5}%
    {De acuerdo con el texto, ¿cuál de los siguientes conjuntos NO es un ejemplo de un espacio vectorial?}
    \item Los polinomios de grado $n$ con coeficientes reales.
    \item* El conjunto de polinomios con coeficientes reales de grado exactamente $n$.
    \item Las matrices de tamaño $m\times n$ sobre el cuerpo $\mathbb{K}$.
    \item Las secuencias reales infinitas.
\end{multi}

\begin{multi}{Contr. Espacio Vectorial 6}%
    {¿Cuál de las siguientes opciones es una condición necesaria para que un conjunto $V$ sea un espacio vectorial?}
    \item* $V$ contiene un vector cero \(0\) tal que para todo \(u\) en \(V\), \(u + 0 = u\).
    \item $V$ contiene al menos un vector cuyo módulo es 1.
    \item Todos los vectores en $V$ deben ser ortogonales entre sí.
    \item La suma de cualquier par de vectores en $V$ debe resultar en un vector cuyo módulo es igual a la suma de los módulos de los vectores originales.
\end{multi}


\begin{multi}{Contr. Espacio Vectorial 7}%
    {¿Cuál de las siguientes afirmaciones es verdadera para cualquier elemento \(v\) de un espacio vectorial \(V\) y cualquier escalar \(c\)?}
    \item* Si \(cv = 0\), entonces \(c = 0\) o \(v = 0\).
    \item Si \(cv = 0\), entonces \(c\) y \(v\) deben ser ambos distintos de cero.
    \item La multiplicación de cualquier vector por cero siempre resulta en un vector cuyo módulo es 1.
    \item Un vector multiplicado por su inverso aditivo siempre da como resultado el vector cero.
\end{multi}

\begin{multi}{Contr. Espacio Vectorial 8}%
    {¿Cuál de los siguientes conjuntos no forma un espacio vectorial debido a la falta de cerradura bajo la suma?}
    \item* El conjunto de todos los polinomios de grado exactamente 2.
    \item El conjunto de todos los polinomios de grado menor o igual que 2.
    \item El conjunto de todas las matrices de 2$\times$ 3.
    \item El conjunto de todas las funciones continuas definidas sobre toda la recta numérica.
\end{multi}

\begin{multi}{Contr. Espacio Vectorial 9}%
    {¿Cuál es el vector cero en el espacio vectorial de todos los polinomios de grado menor o igual que 2, \(P_2\)?}
    \item El polinomio \(p(x) = 1\).
    \item* El polinomio \(p(x) = 0\).
    \item El polinomio \(p(x) = x^2\).
    \item El polinomio \(p(x) = x\).
\end{multi}


\end{quiz}


\end{document}