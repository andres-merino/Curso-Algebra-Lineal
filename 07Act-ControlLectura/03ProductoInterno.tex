\documentclass[a4,11pt]{aleph-notas}
% Actualizado en febrero de 2024
% Funciona con TeXLive 2022
% Para obtener solo el pdf, compilar con pdfLaTeX. 
%  latexmk -pdf 01\ Cuestionarios/01\ Matrices.tex -output-directory="01 Cuestionarios"
% Para obtener el xml compilar con XeLaTeX.
% latexmk -xelatex 01\ Cuestionarios/01\ Matrices.tex -output-directory="01 Cuestionarios"

% -- Paquetes adicionales
\usepackage{aleph-moodle}
\moodleregisternewcommands
% Todos los comandos nuevos deben ir luego del comando anterior
\usepackage{aleph-comandos}


% -- Datos
\institucion{Escuela de Ciencias Físicas y Matemática}
\carrera{Ciencia de datos / Bioingeniería}
\asignatura{Álgebra Lineal}
\tema{Control de lectura: Producto Interno}
\autor{Andrés Merino}
\fecha{Semestre 2024-1}

\logouno[0.14\textwidth]{Logos/logoPUCE_04_ac}
\definecolor{colortext}{HTML}{0030A1}
\definecolor{colordef}{HTML}{0030A1}
\fuente{montserrat}

% -- Otros comandos



\begin{document}

\encabezado

\vspace*{-8mm}
\tableofcontents

%%%%%%%%%%%%%%%%%%%%%%%%%%%%%%%%%%%%%%%%
\section{Indicaciones}
%%%%%%%%%%%%%%%%%%%%%%%%%%%%%%%%%%%%%%%%

Se plantean bancos de preguntas orientados a realizar el control de lectura de la sección 5.2 del libro de Larson y sección 8.1 del libro de Aranda.

%%%%%%%%%%%%%%%%%%%%%%%%%%%%%%%%%%%%%%%%
\section{Banco de preguntas}
%%%%%%%%%%%%%%%%%%%%%%%%%%%%%%%%%%%%%%%%

%%%%%%%%%%%%%%%%%%%%%%%%%%%%%%%%%%%%%%%%
\begin{quiz}{Control - Producto Interno}
%%%%%%%%%%%%%%%%%%%%%%%%%%%%%%%%%%%%%%%%

\begin{multi}[]%     
    {P1}     
    ¿Qué propiedad del producto escalar permite afirmar que \(\langle x,y \rangle = \langle y,x \rangle\) para todos los vectores \(x,y\) en el espacio \(E\)?     
    \item* Simetría     
    \item Linealidad     
    \item Positividad     
    \item Ortogonalidad 
\end{multi}

\begin{multi}[]%     
    {P2}     
    ¿Cuál de las siguientes afirmaciones es verdadera respecto a la propiedad de linealidad del producto escalar?     
    \item* \(\langle x + y, z \rangle = \langle x, z \rangle + \langle y, z \rangle\) para cualquier \(x, y, z\) en \(E\)     
    \item \(\langle x, y + z \rangle = \langle x, y \rangle \cdot \langle x, z \rangle\)     
    \item \(\langle x, y \rangle = \langle x + z, y + z \rangle\)     
    \item \(\langle x, y \rangle = 0\) siempre que \(x \neq y\) 
\end{multi}

\begin{multi}[]%     
    {P3}     
    Si \(\alpha\) es un escalar real y \(x, y\) son vectores en \(E\), ¿cuál es la correcta expresión del producto escalar involucrando \(\alpha\)?     
    \item* \(\langle \alpha x, y \rangle = \alpha \langle x, y \rangle\)     
    \item \(\langle \alpha x, y \rangle = \langle x, \alpha y \rangle\)     
    \item \(\langle \alpha x, y \rangle = \alpha^2 \langle x, y \rangle\)     
    \item \(\langle \alpha x, y \rangle = \langle x, y \rangle/\alpha\) 
\end{multi}

\begin{multi}[]%     
    {P4}     
    ¿Qué implica sobre el vector \(x\) si \(\langle x, x \rangle = 0\) en un espacio vectorial euclídeo \(E\)?     
    \item* \(x\) es el vector nulo     
    \item \(x\) tiene norma uno     
    \item \(x\) es ortogonal a sí mismo     
    \item \(x\) puede ser cualquier vector 
\end{multi}

\begin{multi}[]%     
    {P5}     
    ¿Cuál es una correcta interpretación de la propiedad de positividad del producto escalar?     
    \item* \(\langle x, x \rangle \geq 0\) para todo \(x\) en \(E\)     
    \item \(\langle x, y \rangle > 0\) para todo \(x, y\) en \(E\)     
    \item \(\langle x, y \rangle \geq 0\) sólo si \(x\) y \(y\) son ortogonales     
    \item \(\langle x, y \rangle = 1\) siempre que \(x\) y \(y\) sean unitarios 
\end{multi}

\begin{multi}[]%     
    {P6}     
    ¿Qué establece la desigualdad de Cauchy-Schwarz en un espacio vectorial euclídeo?     
    \item* \(-1 \leq \frac{\langle x, y \rangle}{\|x\| \cdot \|y\|} \leq 1\)     
    \item \(\langle x, y \rangle^2 \leq \|x\|^2 + \|y\|^2\)     
    \item \(\langle x, y \rangle \leq \|x\| + \|y\|\)     
    \item \(\|x + y\| \leq \|x\| \|y\|\) 
\end{multi}

\begin{multi}[]%     
    {P8}     
    ¿Qué indica la propiedad de positividad respecto al valor de \(\langle x, x \rangle\) cuando \(x\) es diferente del vector nulo?     
    \item* \(\langle x, x \rangle > 0\)     
    \item \(\langle x, x \rangle = 0\)     
    \item \(\langle x, x \rangle < 0\)     
    \item \(\langle x, x \rangle\) puede ser cualquier número real 
\end{multi}

\begin{multi}[]%     
    {P9}     
    ¿Cuál de las siguientes es una aplicación del producto escalar en espacios vectoriales euclídeos?     
    \item* Calcular distancias entre vectores     
    \item Definir operaciones de suma de vectores     
    \item Establecer el número máximo de dimensiones de un espacio     
    \item Determinar la cantidad de vectores en una base 
\end{multi}


\begin{multi}[]%     
    {P11}     
    ¿Cuál de las siguientes afirmaciones describe correctamente un producto interno en un espacio vectorial?     
    \item* Asocia un número real con cada par de vectores cumpliendo ciertos axiomas.     
    \item Asocia un vector con cada par de vectores cumpliendo ciertos axiomas.     
    \item Asocia un número complejo con cada par de vectores sin requerimientos adicionales.     
    \item Asocia un número real con cada vector individual en el espacio. 
\end{multi}

\begin{multi}[]%     
    {P12}     
    ¿Cuál es un ejemplo de un producto interno diferente en \(\mathbb{R}^2\) que no es el producto interno euclidiano estándar?     
    \item* \(\langle u, v \rangle = u_1v_1 + 2u_2v_2\)     
    \item \(\langle u, v \rangle = u_1v_1 + u_2v_2\)     
    \item \(\langle u, v \rangle = u_1^2 + v_1^2\)     
    \item \(\langle u, v \rangle = u_1v_2 + u_2v_1\) 
\end{multi}

\begin{multi}[]%     
    {P13}     
    En el contexto de productos internos, ¿qué implica el axioma que indica que \(\langle v, v \rangle \geq 0\) para todo vector \(v\)?     
    \item* El producto interno nunca es negativo.     
    \item El producto interno es siempre positivo.     
    \item El producto interno es nulo.     
    \item El producto interno siempre es igual a 1. 
\end{multi}

\begin{multi}[]%     
    {P14}     
    Si se define un producto interno por la integral de dos funciones, ¿qué propiedad básica de la integral garantiza que \(\langle f, f \rangle \geq 0\) para toda función \(f\) en \(C[a, b]\)?     
    \item* El cuadrado de una función real no es negativo.     
    \item La integral de una función siempre es positiva.     
    \item La integral de una función es independiente de sus límites.     
    \item El resultado de la integral siempre es una función. 
\end{multi}


\begin{multi}[]%     
    {P16}     
    ¿Qué describe la desigualdad de Cauchy-Schwarz en espacios con producto interno?     
    \item* La magnitud del producto interno de dos vectores no excede el producto de sus normas.     
    \item La suma de los productos internos de dos vectores es igual a su norma.     
    \item El producto interno de dos vectores es siempre menor que su suma.     
    \item El producto interno de dos vectores es directamente proporcional a sus normas. 
\end{multi}

\begin{multi}[]%     
    {P17}     
    ¿Qué indica el teorema de Pitágoras en el contexto de espacios con producto interno?     
    \item* La suma de los cuadrados de las normas de dos vectores ortogonales es igual a la norma de su suma al cuadrado.     
    \item La norma de la suma de dos vectores es igual al producto de sus normas.     
    \item La norma de la suma de dos vectores siempre es menor que la suma de sus normas.     
    \item La suma de los cuadrados de dos vectores es siempre positiva. 
\end{multi}


\begin{multi}[]%     
    {P19}     
    ¿Qué representa la proyección ortogonal de un vector \(u\) sobre un vector \(v\) en espacios con producto interno?     
    \item* Un múltiplo escalar de \(v\) que es la componente de \(u\) en la dirección de \(v\).     
    \item La suma de \(u\) y \(v\) que minimiza la distancia a \(u\).     
    \item Un vector que es ortogonal a \(v\) y paralelo a \(u\).     
    \item La diferencia entre \(u\) y \(v\) que maximiza la longitud. 
\end{multi}

\begin{multi}[]%     
    {P20}     
    En la definición de norma en espacios con producto interno, ¿cómo se expresa la norma de un vector \(u\)?     
    \item* \(\sqrt{\langle u, u \rangle}\)     
    \item \(\langle u, u \rangle\)     
    \item \(|\langle u, u \rangle|\)     
    \item \(\langle u, u \rangle^2\) 
\end{multi}

\begin{multi}[]%     
    {P21}     
    ¿Qué caracteriza a un vector unitario en un espacio con producto interno?     
    \item* Su norma es igual a 1.     
    \item Su norma es igual a 0.     
    \item Su norma es mayor que 1.     
    \item No es ortogonal a ningún otro vector. 
\end{multi}

\begin{multi}[]%     
    {P22}     
    ¿Qué propiedad de los productos internos se verifica al calcular el producto interno de un vector con el vector nulo?     
    \item* El producto interno de cualquier vector con el vector nulo es 0.     
    \item El producto interno resulta en el mismo vector.     
    \item El producto interno es siempre negativo.     
    \item El producto interno es igual a la norma del vector. 
\end{multi}

\begin{multi}[]%     
    {P24}     
    ¿Cuál de las siguientes opciones describe adecuadamente la desigualdad del triángulo en contextos de producto interno?     
    \item* La norma de la suma de dos vectores es menor o igual a la suma de sus normas.     
    \item La norma de la suma de dos vectores es igual a la suma de sus normas.     
    \item La norma de la suma de dos vectores es siempre mayor que la suma de sus normas.     
    \item La norma de la suma de dos vectores es siempre menor que la diferencia de sus normas. 
\end{multi}

\begin{multi}[]%     
    {P26}     
    ¿Qué característica importante tiene la matriz de Gram en el contexto de productos internos?     
    \item* Es simétrica.     
    \item Es siempre diagonal.     
    \item Sus elementos son siempre positivos.     
    \item Es siempre invertible. 
\end{multi}

\end{quiz}


\end{document}