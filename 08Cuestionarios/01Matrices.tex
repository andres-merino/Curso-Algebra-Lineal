\documentclass[a4,11pt]{aleph-notas}
% Actualizado en febrero de 2024
% Funciona con TeXLive 2022
% Para obtener solo el pdf, compilar con pdfLaTeX. 
%  latexmk -pdf 01\ Cuestionarios/01\ Matrices.tex -output-directory="01 Cuestionarios"
% Para obtener el xml compilar con XeLaTeX.
% latexmk -xelatex 01\ Cuestionarios/01\ Matrices.tex -output-directory="01 Cuestionarios"

% -- Paquetes adicionales
\usepackage{aleph-moodle}
\moodleregisternewcommands
% Todos los comandos nuevos deben ir luego del comando anterior
\usepackage{aleph-comandos}


% -- Datos  
\institucion{Escuela de Ciencias Físicas y Matemática}
\carrera{Ciencia de datos / Bioingeniería}
\asignatura{Álgebra Lineal}
\tema{Cuestionario No. 1-1: Matrices}
\autor{Andrés Merino}
\fecha{Semestre 2024-1}

\logouno[0.14\textwidth]{Logos/logoPUCE_04_ac}
\definecolor{colortext}{HTML}{0030A1}
\definecolor{colordef}{HTML}{0030A1}
\fuente{montserrat}

% -- Otros comandos



\begin{document}

\encabezado

\vspace*{-8mm}
\tableofcontents

%%%%%%%%%%%%%%%%%%%%%%%%%%%%%%%%%%%%%%%%
\section{Indicaciones}
%%%%%%%%%%%%%%%%%%%%%%%%%%%%%%%%%%%%%%%%

Se plantean bancos de preguntas orientados a evaluar el \textbf{criterio}: «Identifica las propiedades fundamentales de las matrices, incluyendo tipos, operaciones y la inversión de matrices», correspondiente al \textbf{resultado de aprendizaje}: Comprender los conceptos fundamentales del Álgebra Lineal, incluyendo el estudio de matrices, determinantes, sistemas de ecuaciones lineales y espacios vectoriales, destacando su importancia en el análisis y resolución de problemas matemáticos.

%%%%%%%%%%%%%%%%%%%%%%%%%%%%%%%%%%%%%%%%
\section{Banco de preguntas}
%%%%%%%%%%%%%%%%%%%%%%%%%%%%%%%%%%%%%%%%


%%%%%%%%%%%%%%%%%%%%%%%%%%%%%%%%%%%%%%%%
\begin{quiz}{Propiedades Matrices}
%%%%%%%%%%%%%%%%%%%%%%%%%%%%%%%%%%%%%%%%

%%%%%%%%%%%%%%%%%%%%%%%%%%%%%%%%%%%%%%%%
\begin{multi}[]%
    % - Identificador
    {Mat-DefProp-01}
    % - Enunciado
    ¿Cuál de las siguientes no es una propiedad fundamental de la multiplicación de matrices?
    \item* Conmutatividad
    \item Asociatividad
    \item Distributividad
    \item Inversión
\end{multi}

%%%%%%%%%%%%%%%%%%%%%%%%%%%%%%%%%%%%%%%%
\begin{multi}[]%
    % - Identificador
    {Mat-DefProp-02}
    % - Enunciado
    ¿Cómo se conoce a las matrices que tienen el mismo número de filas y columnas?
    \item* Matriz cuadrada
    \item Matriz rectangular
    \item Matriz identidad
    \item Matriz diagonal
\end{multi}

%%%%%%%%%%%%%%%%%%%%%%%%%%%%%%%%%%%%%%%%
\begin{multi}[]%
    % - Identificador
    {Mat-DefProp-03}
    % - Enunciado
    ¿Cuál de las siguientes operaciones no se puede realizar con matrices?
    \item Multiplicación
    \item Suma
    \item Resta
    \item* División
\end{multi}

%%%%%%%%%%%%%%%%%%%%%%%%%%%%%%%%%%%%%%%%
\begin{multi}[]%
    % - Identificador
    {Mat-DefProp-04}
    % - Enunciado
    ¿Cuál de las siguientes propiedades no es cierta para la matriz identidad?
    \item Es una matriz cuadrada.
    \item Su diagonal principal está compuesta por unos.
    \item Es el elemento neutro de la multiplicación.
    \item* La matriz identidad es una matriz nula
\end{multi}

%%%%%%%%%%%%%%%%%%%%%%%%%%%%%%%%%%%%%%%%
\begin{multi}[]%
    % - Identificador
    {Mat-DefProp-05}
    % - Enunciado
    ¿Cuál de las siguientes propiedades no es cierta para la matriz nula?
    \item Es una matriz cuadrada.
    \item Todos sus elementos son cero.
    \item Es el elemento neutro de la suma.
    \item* La matriz nula es una matriz identidad.
\end{multi}

%%%%%%%%%%%%%%%%%%%%%%%%%%%%%%%%%%%%%%%%
\begin{multi}[]%
    % - Identificador
    {Mat-DefProp-06}
    % - Enunciado
    ¿Cuál de las siguientes propiedades no es cierta para una matriz triangula superior?
    \item Todos los elementos debajo de la diagonal principal son cero.
    \item* Todos los elementos sobre de la diagonal principal son diferentes de cero.
    \item La diagonal principal puede tener elementos distintos de cero.
    \item Es una matriz cuadrada.
\end{multi}

%%%%%%%%%%%%%%%%%%%%%%%%%%%%%%%%%%%%%%%%
\begin{multi}[]%
    % - Identificador
    {Mat-DefProp-07}
    % - Enunciado
    ¿Cuál de las siguientes propiedades no es cierta para una matriz triangular inferior?
    \item Todos los elementos sobre de la diagonal principal son cero.
    \item* Todos los elementos debajo de la diagonal principal son diferentes de cero.
    \item La diagonal principal puede tener elementos distintos de cero.
    \item Es una matriz cuadrada.
\end{multi}

%%%%%%%%%%%%%%%%%%%%%%%%%%%%%%%%%%%%%%%%
\begin{multi}[]%
    % - Identificador
    {Mat-DefProp-08}
    % - Enunciado
    Si el determinante de una matriz $A$ es cero, ¿qué se puede decir de la matriz?
    \item* La matriz no tiene inversa.
    \item La matriz tiene rango completo.
    \item El sistema de ecuaciones asociado a la matriz tiene solución única.
\end{multi}

%%%%%%%%%%%%%%%%%%%%%%%%%%%%%%%%%%%%%%%%
\begin{multi}[]%
    % - Identificador
    {Mat-DefProp-09}
    % - Enunciado
    Si el determinante de una matriz $A$ es diferente de cero, ¿qué se puede decir de la matriz?
    \item La matriz no tiene inversa.
    \item* La matriz tiene rango completo.
    \item El sistema de ecuaciones asociado a la matriz tiene infinitas soluciones.
\end{multi}

%%%%%%%%%%%%%%%%%%%%%%%%%%%%%%%%%%%%%%%%
\begin{multi}[]%
    % - Identificador
    {Mat-DefProp-10}
    % - Enunciado
    Si una matriz $A$ cuadrada tiene rango completo, ¿qué se puede decir de la matriz?
    \item El determinante de la matriz es cero.
    \item* La matriz tiene inversa.
    \item El sistema de ecuaciones asociado a la matriz no tiene solución.
\end{multi}

%%%%%%%%%%%%%%%%%%%%%%%%%%%%%%%%%%%%%%%%
\begin{multi}[]%
    % - Identificador
    {Mat-DefProp-11}
    % - Enunciado
    Si una matriz $A$ cuadrada no tiene rango completo, ¿qué se puede decir de la matriz?
    \item* El determinante de la matriz es cero.
    \item La matriz tiene inversa.
    \item El sistema de ecuaciones asociado a la matriz tiene solución única.
\end{multi}

%%%%%%%%%%%%%%%%%%%%%%%%%%%%%%%%%%%%%%%%
\begin{multi}[]%
    % - Identificador
    {Mat-DefProp-12}
    % - Enunciado
    Si una matriz $A$ tiene inversa, ¿qué se puede decir del determinante de la matriz?
    \item* El determinante de la matriz es diferente de cero.
    \item La matriz no tiene rango completo.
    \item El sistema de ecuaciones asociado a la matriz tiene infinitas soluciones.
\end{multi}

%%%%%%%%%%%%%%%%%%%%%%%%%%%%%%%%%%%%%%%%
\begin{multi}[]%
    % - Identificador
    {Mat-DefProp-13}
    % - Enunciado
    Si $A$ es una matriz cuadrada triangular superior o triangular inferior, ¿qué se puede decir de su determinante?
    \item* El determinante de la matriz es igual al producto de los elementos de la diagonal principal.
    \item El determinante de la matriz es cero.
    \item El determinante de la matriz es igual a la suma de los elementos de la diagonal principal.
\end{multi}

%%%%%%%%%%%%%%%%%%%%%%%%%%%%%%%%%%%%%%%%
\begin{multi}[]%
    % - Identificador
    {Mat-DefProp-14}
    % - Enunciado
    Si $A$ es una matriz cuadrada con dos filas o columnas iguales, ¿qué se puede decir de su determinante?
    \item* Es cero.
    \item Es distinto de cero.
    \item No se puede determinar.
    \item Es negativo.
    \item Es positivo.
\end{multi}

%%%%%%%%%%%%%%%%%%%%%%%%%%%%%%%%%%%%%%%%
\begin{multi}[]%
    % - Identificador
    {Mat-DefProp-15}
    % - Enunciado
    Si $A$ es una matriz cuadrada esta se dice diagonal si verifica que $a_{ij}=0$ para:
    \item* $i\in I$ y $j\in J$ con $i\neq j$.
    \item $i\in I$ y $j\in J$ con $i=j$.
    \item $i\in $ y $j\in I$.
\end{multi}

%%%%%%%%%%%%%%%%%%%%%%%%%%%%%%%%%%%%%%%%
\begin{multi}[]%
    % - Identificador
    {Mat-DefProp-16}
    % - Enunciado
    Si $A$ es y $B$ son matrices cuadradas de la misma dimensión, entonces el determinante de $AB$ es igual a:
    \item* El producto de los determinantes de $A$ y $B$.
    \item La suma de los determinantes de $A$ y $B$.
    \item La resta de los determinantes de $A$ y $B$.
    \item El cociente de los determinantes de $A$ y $B$.
\end{multi}

%%%%%%%%%%%%%%%%%%%%%%%%%%%%%%%%%%%%%%%%
\begin{multi}[]%
    % - Identificador
    {Mat-DefProp-17}
    % - Enunciado
    Si $A$ es una matriz cuadrada no singular, entonces el determinante de $A^{-1}$ es igual a:
    \item* El inverso del determinante de $A$.
    \item El determinante de $A$.
    \item Uno menos el determinante de $A$.
    \item El cuadrado del determinante de $A$.
\end{multi}

\end{quiz}


%%%%%%%%%%%%%%%%%%%%%%%%%%%%%%%%%%%%%%%%
\begin{quiz}{Gauss-Jordan}
%%%%%%%%%%%%%%%%%%%%%%%%%%%%%%%%%%%%%%%%

%%%%%%%%%%%%%%%%%%%%%%%%%%%%%%%%%%%%%%%%
\begin{multi}[]%
    % - Identificador
    {Gauss-01}
    % - Enunciado
    Dada la matriz 
    \[
        A= \begin{pmatrix} 1 & 3 & 2 & -1 \\ -2 & 4 & 7 & 4  \\ -4 & -2 & 0 & -1 \\ 2 & 0 & 2 & 3\end{pmatrix} 
    \] 
    la operación elemental por filas que vuelve $0$ a la primera entrada de la segunda fila es:
    \item* $2F_1+F_2\to F_2$
    \item $-2F_1+F_2\to F_2$
    \item $F_1+\frac{1}{2}F_2\to F_2$
    \item $0F_2\to F_2$
\end{multi}

%%%%%%%%%%%%%%%%%%%%%%%%%%%%%%%%%%%%%%%%
\begin{multi}{Gauss-02}
    Dada la matriz \[A= \begin{pmatrix} 1 & 3 & 2 & -1 \cr -2 & 4 & 7 & 4  \cr -4 & -2 & 0 & -1 \cr 2 & 0 & 2 & 3\end{pmatrix} \] la operación elemental por filas que vuelve $0$ a la primera entrada de la tercera fila es:
    \item* $4F_1+F_3\to F_3$
    \item $-4F_1+F_3\to F_3$
    \item $F_1+\frac{1}{4}F_3\to F_3$
    \item $0F_3\to F_3$
\end{multi}

%%%%%%%%%%%%%%%%%%%%%%%%%%%%%%%%%%%%%%%%
\begin{multi}{Gauss-OpMult-03}
    Dada la matriz \[A= \begin{pmatrix} 1 & 3 & 2 & -1 \cr -2 & 4 & 7 & 4  \cr -4 & -2 & 0 & -1 \cr 2 & 0 & 2 & 3\end{pmatrix} \] la operación elemental por filas que vuelve $0$ a la primera entrada de la cuarta fila es:
    \item* $-2F_1+F_4\to F_4$
    \item $2F_1+F_4\to F_4$
    \item $F_1-\frac{1}{2}F_4\to F_4$
    \item $0F_4\to F_4$
\end{multi}

%%%%%%%%%%%%%%%%%%%%%%%%%%%%%%%%%%%%%%%%
\begin{multi}{Gauss-OpMult-04}
    Dada la matriz \[A= \begin{pmatrix} 1 & 3 & 2 & -1 \cr 3 & 4 & 7 & 4  \cr -4 & -2 & 0 & -1 \cr 2 & 0 & 2 & 3\end{pmatrix} \] la operación elemental por filas que vuelve $0$ a la primera entrada de la segunda fila es:
    \item* $-3F_1+F_2\to F_2$
    \item $3F_1+F_2\to F_2$
    \item $F_1-\frac{1}{3}F_2\to F_2$
    \item $0F_2\to F_2$
\end{multi}

%%%%%%%%%%%%%%%%%%%%%%%%%%%%%%%%%%%%%%%%
\begin{multi}{Gauss-OpMult-05}
    Dada la matriz \[A= \begin{pmatrix} 1 & 3 & 2 & -1 \cr -2 & 4 & 7 & 4  \cr -1 & -2 & 0 & -1 \cr 2 & 0 & 2 & 3\end{pmatrix} \] la operación elemental por filas que vuelve $0$ a la primera entrada de la tercera fila es:
    \item* $F_1+F_3\to F_3$
    \item $-F_1+F_3\to F_3$
    \item $-F_1+F_3\to F_3$
    \item $0F_3\to F_3$
\end{multi}

%%%%%%%%%%%%%%%%%%%%%%%%%%%%%%%%%%%%%%%%
\begin{multi}{Gauss-OpMult-06}
    Dada la matriz \[A= \begin{pmatrix} 1 & 3 & 2 & -1 \cr 3 & 4 & 7 & 4  \cr -4 & -2 & 0 & -1 \cr 3 & 0 & 2 & 3\end{pmatrix} \] la operación elemental por filas que vuelve $0$ a la primera entrada de la cuarta fila es:
    \item* $-3F_1+F_4\to F_4$
    \item $3F_1+F_4\to F_4$
    \item $F_1+\frac{1}{3}F_4\to F_4$
    \item $0F_4\to F_4$
\end{multi}
 
%%%%%%%%%%%%%%%%%%%%%%%%%%%%%%%%%%%%%%%%
\begin{multi}{Gauss-OpMult-07}
    Dada la matriz
    \[
        A= \begin{pmatrix} 3 & 3 & 2 & -1 \cr 4 & 4 & 7 & 4  \cr -2 & -2 & 0 & -1 \cr 1 & 0 & 2 & 3\end{pmatrix} 
    \] la operación elemental por filas que vuelve $0$ a la primera entrada de la cuarta fila es:
    \item* $3F_4-F_1\to F_4$
    \item $3F_4+F_4\to F_4$
    \item $F_1-\frac{1}{3}F_4\to F_4$
    \item $0F_4\to F_4$
\end{multi}

%%%%%%%%%%%%%%%%%%%%%%%%%%%%%%%%%%%%%%%%
\begin{multi}{Gauss-OpMult-08}
    Dada la matriz
    \[
        A= \begin{pmatrix} 3 & 3 & 2 & -1 \cr 4 & 4 & 7 & 4  \cr -2 & -2 & 0 & -1 \cr 1 & 0 & 2 & 3\end{pmatrix} 
    \] la operación elemental por filas que vuelve $0$ a la primera entrada de la segunda fila es:
    \item* $2F_3+F_2\to F_2$
    \item $2F_2+F_3\to F_2$
    \item $\frac{2}{3}F_1+\frac{1}{2}F_2\to F_2$
    \item $0F_2\to F_2$
\end{multi}

%%%%%%%%%%%%%%%%%%%%%%%%%%%%%%%%%%%%%%%%
\begin{multi}{Gauss-OpMult-09}
    Dada la matriz
    \[
        A= \begin{pmatrix} 2 & 3 & 2 & -1 \cr 1 & 4 & 7 & 4  \cr -2 & -2 & 0 & -1 \cr 1 & 0 & 2 & 3\end{pmatrix} 
    \] la operación elemental por filas que vuelve $0$ a la primera entrada de la segunda fila es:
    \item* $F_3+2F_2\to F_2$
    \item $3F_2+F_2\to F_2$
    \item $\frac{1}{2}F_1+\frac{1}{2}F_2\to F_2$
    \item $0F_2\to F_2$
\end{multi}

%%%%%%%%%%%%%%%%%%%%%%%%%%%%%%%%%%%%%%%%
\begin{multi}{Gauss-OpMult-10}
    Dada la matriz
    \[
        A= \begin{pmatrix} 2 & 3 & 2 & -1 \cr 1 & 4 & 7 & 4  \cr -2 & -2 & 0 & -1 \cr 1 & 0 & 2 & 3\end{pmatrix} 
    \] la operación elemental por filas que vuelve $0$ a la primera entrada de la primera fila es:
    \item* $F_3+F_1\to F_1$
    \item $2F_2+F_1\to F_1$
    \item $F_1+2F_3\to F_1$
    \item $0F_1\to F_1$
\end{multi}

%%%%%%%%%%%%%%%%%%%%%%%%%%%%%%%%%%%%%%%%
\begin{multi}{Gauss-OpMult-11}
    Dada la matriz
    \[
        A= \begin{pmatrix} 2 & 3 & 2 & -1 \cr 1 & 4 & 7 & 4  \cr -2 & -2 & 0 & -1 \cr 1 & 0 & 2 & 3\end{pmatrix} 
    \] la operación elemental por filas que vuelve $0$ a la cuarta entrada de la primera fila es:
    \item* $F_1-F_2\to F_1$
    \item $2F_2+F_1\to F_1$
    \item $F_2-2F_1\to F_1$
    \item $0F_1\to F_1$
\end{multi}

%%%%%%%%%%%%%%%%%%%%%%%%%%%%%%%%%%%%%%%%
\begin{multi}{Gauss-OpMult-12}
    Dada la matriz
    \[
        A= \begin{pmatrix} 2 & 3 & 2 & -1 \cr 1 & 4 & 7 & 4  \cr -2 & -2 & 0 & -1 \cr 1 & 0 & 2 & 3\end{pmatrix} 
    \] la operación elemental por filas que vuelve $0$ a la tercera entrada de la cuarta fila es:
    \item* $F_4-F_1\to F_4$
    \item $2F_2+F_1\to F_1$
    \item $F_1-2F_4\to F_4$
    \item $0F_3\to F_3$
\end{multi}

%%%%%%%%%%%%%%%%%%%%%%%%%%%%%%%%%%%%%%%%
\begin{multi}{Gauss-OpMult-13}
    Dada la matriz
    \[
        A= \begin{pmatrix} 2 & 3 & 2 & -1 \cr 1 & 4 & 7 & 4  \cr -2 & -2 & 0 & -1 \cr 1 & 0 & 2 & 3\end{pmatrix} 
    \] 
    la operación elemental por filas que vuelve $0$ a la segunda entrada de la segunda fila es:
    \item* $F_2+2F_3\to F_2$
    \item $2F_2+F_1\to F_1$
    \item $F_3-2F_2\to F_3$
    \item $0F_3\to F_3$
\end{multi}

\end{quiz}


%%%%%%%%%%%%%%%%%%%%%%%%%%%%%%%%%%%%%%%%
\begin{quiz}{Operaciones por Filas}
%%%%%%%%%%%%%%%%%%%%%%%%%%%%%%%%%%%%%%%%

%%%%%%%%%%%%%%%%%%%%%%%%%%%%%%%%%%%%%%%%
\begin{multi}[]%
    % - Identificador
    {OpFilas-01}
    % - Enunciado
    ¿Cuál de las siguientes operaciones de filas no es válida para una matriz $A$ de $3\times 3$?
    \item* $2F_1+F_2\to F_3$
    \item $-2F_1+F_2\to F_2$
    \item $\frac{1}{2}F_1+F_2\to F_2$
    \item $2F_2\to F_2$
\end{multi}

%%%%%%%%%%%%%%%%%%%%%%%%%%%%%%%%%%%%%%%%
\begin{multi}[]%
    % - Identificador
    {OpFilas-02}
    % - Enunciado
    ¿Cuál de las siguientes operaciones de filas no es válida para una matriz $A$ de $3\times 3$?
    \item* $2F_4+F_2\to F_2$
    \item $-2F_3+F_2\to F_2$
    \item $F_2+\frac{1}{2}F_1\to F_2$
    \item $2F_3\to F_3$
\end{multi}

%%%%%%%%%%%%%%%%%%%%%%%%%%%%%%%%%%%%%%%%
\begin{multi}[]%
    % - Identificador
    {OpFilas-03}
    % - Enunciado
    ¿Cuál de las siguientes operaciones de filas no es válida para una matriz $A$ de $3\times 3$?
    \item $F_1+F_2\to F_2$
    \item $F_2-2F_3\to F_2$
    \item $\frac{1}{2}F_1+F_2\to F_2$
    \item* $2F_4\to F_4$
\end{multi}

%%%%%%%%%%%%%%%%%%%%%%%%%%%%%%%%%%%%%%%%
\begin{multi}[]%
    % - Identificador
    {OpFilas-04}
    % - Enunciado
    ¿Cuál de las siguientes operaciones de filas no es válida para una matriz $A$ de $4\times 4$?
    \item* $2F_4+F_2\to F_1$
    \item $-2F_3+F_2\to F_2$
    \item $F_2+\frac{1}{2}F_1\to F_2$
    \item $2F_3\to F_3$
\end{multi}

%%%%%%%%%%%%%%%%%%%%%%%%%%%%%%%%%%%%%%%%
\begin{multi}[]%
    % - Identificador
    {OpFilas-06}
    % - Enunciado
    ¿Cuál de las siguientes operaciones de filas no es válida para una matriz $A$ de $4\times 4$?
    \item* $F_1+3F_3\to F_3$
    \item $F_2+F_4\to F_2$
    \item $\frac{1}{2}F_1+F_2\to F_2$
    \item $F_3\leftrightarrow F_3$
\end{multi}

%%%%%%%%%%%%%%%%%%%%%%%%%%%%%%%%%%%%%%%%
\begin{multi}[]%
    % - Identificador
    {OpFilas-05}
    % - Enunciado
    ¿Cuál de las siguientes operaciones de filas no es válida para una matriz $A$ de $4\times 4$?
    \item $F_3+3F_4\to F_3$
    \item $F_2+4F_4\to F_2$
    \item $\frac{1}{2}F_1+F_2\to F_2$
    \item* $2F_3\leftrightarrow F_3$
\end{multi}

%%%%%%%%%%%%%%%%%%%%%%%%%%%%%%%%%%%%%%%%
\begin{multi}[]%
    % - Identificador
    {OpFilas-06}
    % - Enunciado
    ¿Cuál de las siguientes operaciones de filas no es válida para una matriz $A$ de $4\times 4$?
    \item $F_3+3F_4\to F_3$
    \item $F_2+\frac{1}{2}F_4\to F_2$
    \item $\frac{1}{2}F_1+F_2\to F_2$
    \item* $\frac{1}{2}F_3\leftrightarrow F_3$
\end{multi}

%%%%%%%%%%%%%%%%%%%%%%%%%%%%%%%%%%%%%%%%
\begin{multi}[]%
    % - Identificador
    {OpFilas-07}
    % - Enunciado
    ¿Cuál de las siguientes operaciones de filas no es válida para una matriz $A$ de $3\times 3$?
    \item* $F_4+2F_2\to F_4$
    \item $-2F_3+F_2\to F_2$
    \item $F_2+\frac{1}{2}F_1\to F_2$
    \item $2F_3\to F_3$
\end{multi}

%%%%%%%%%%%%%%%%%%%%%%%%%%%%%%%%%%%%%%%%
\begin{multi}[]%
    % - Identificador
    {OpFilas-08}
    % - Enunciado
    ¿Cuál de las siguientes operaciones de filas no es válida para una matriz $A$ de $3\times 3$?
    \item $F_1+F_2\to F_2$
    \item $F_2-2F_3\to F_2$
    \item* $\frac{1}{2}F_1+F_2\to F_1$
    \item $2F_3\to F_3$
\end{multi}

%%%%%%%%%%%%%%%%%%%%%%%%%%%%%%%%%%%%%%%%
\begin{multi}[]%
    % - Identificador
    {OpFilas-09}
    % - Enunciado
    ¿Cuál de las siguientes operaciones de filas no es válida para una matriz $A$ de $2\times 2$?
    \item $F_1+F_2\to F_2$
    \item* $F_2-2F_3\to F_2$
    \item $F_1+\frac{1}{2}F_2\to F_1$
    \item $2F_2\to F_2$
\end{multi}

%%%%%%%%%%%%%%%%%%%%%%%%%%%%%%%%%%%%%%%%
\begin{multi}[]%
    % - Identificador
    {OpFilas-10}
    % - Enunciado
    ¿Cuál de las siguientes operaciones de filas no es válida para una matriz $A$ de $3\times 3$?
    \item $F_1+F_2\to F_2$
    \item $F_2-2F_3\to F_2$
    \item $F_1+\frac{1}{2}F_2\to F_1$
    \item* $2F_2\leftrightarrow F_2$
\end{multi}

%%%%%%%%%%%%%%%%%%%%%%%%%%%%%%%%%%%%%%%%
\begin{multi}[]%
    % - Identificador
    {OpFilas-11}
    % - Enunciado
    ¿Cuál de las siguientes operaciones de filas no es válida para una matriz $A$ de $3\times 3$?
    \item $F_1+F_2\to F_2$
    \item $F_2-2F_3\to F_2$
    \item $F_1+\frac{1}{2}F_2\to F_1$
    \item* $2F_2\leftrightarrow F_2$
\end{multi}

%%%%%%%%%%%%%%%%%%%%%%%%%%%%%%%%%%%%%%%%
\begin{multi}[]%
    % - Identificador
    {OpFilas-12}
    % - Enunciado
    ¿Cuál de las siguientes operaciones de filas no es válida para una matriz $A$ de $4\times 4$?
    \item $F_1+F_2\to F_2$
    \item $F_2-2F_3\to F_2$
    \item $F_1+\frac{1}{2}F_2\to F_1$
    \item* $2F_3\leftrightarrow F_3$
\end{multi}

\end{quiz}



%%%%%%%%%%%%%%%%%%%%%%%%%%%%%%%%%%%%%%%%
\begin{quiz}{Rango}
%%%%%%%%%%%%%%%%%%%%%%%%%%%%%%%%%%%%%%%%
    
%%%%%%%%%%%%%%%%%%%%%%%%%%%%%%%%%%%%%%%%
\begin{multi}[]%
    % - Identificador
    {Rango-01}
    % - Enunciado
    A simple vista, ¿Cuál es el rango de la siguiente matriz?
    \[
        A=\begin{pmatrix} 2 & 2 & 3 \\ 0 & 4 & 6 \\ 0 & 0 & 0 \end{pmatrix}
    \]
    \item 0
    \item 1
    \item* 2
    \item 3
    \item No se puede determinar
\end{multi}

%%%%%%%%%%%%%%%%%%%%%%%%%%%%%%%%%%%%%%%%
\begin{multi}[]%
    % - Identificador
    {Rango-02}
    % - Enunciado
    A simple vista, ¿Cuál es el rango de la siguiente matriz?
    \[
        A=\begin{pmatrix} 1 & 2 & 3 \\ 0 & 0 & 0 \\ 0 & 0 & 0 \end{pmatrix}
    \]
    \item 0
    \item* 1
    \item 2
    \item 3
    \item No se puede determinar a simple vista.
\end{multi}

%%%%%%%%%%%%%%%%%%%%%%%%%%%%%%%%%%%%%%%%
\begin{multi}[]%
    % - Identificador
    {Rango-03}
    % - Enunciado
    A simple vista, ¿Cuál es el rango de la siguiente matriz?
    \[
        A=\begin{pmatrix} 1 & 2 & 3 \\ 0 & 1 & 0 \\ 0 & 0 & 1 \end{pmatrix}
    \]
    \item 0
    \item 1
    \item 2
    \item* 3
    \item No se puede determinar a simple vista.
\end{multi}

%%%%%%%%%%%%%%%%%%%%%%%%%%%%%%%%%%%%%%%%
\begin{multi}[]%
    % - Identificador
    {Rango-04}
    % - Enunciado
    A simple vista, ¿Cuál es el rango de la siguiente matriz?
    \[
        A=\begin{pmatrix} 1 & 2 & 3 \\ 2 & 1 & 0 \\ 0 & 0 & 0 \end{pmatrix}
    \]
    \item 0
    \item 1
    \item 2
    \item 3
    \item* No se puede determinar a simple vista.
\end{multi}

%%%%%%%%%%%%%%%%%%%%%%%%%%%%%%%%%%%%%%%%
\begin{multi}[]%
    % - Identificador
    {Rango-05}
    % - Enunciado
    A simple vista, ¿Cuál es el rango de la siguiente matriz?
    \[
        A=\begin{pmatrix} 1 & 2 & 3 \\ 3 & 7 & 6 \\ 0 & 0 & 0 \end{pmatrix}
    \]
    \item 0
    \item 1
    \item 2
    \item 3
    \item* No se puede determinar a simple vista.
\end{multi}

%%%%%%%%%%%%%%%%%%%%%%%%%%%%%%%%%%%%%%%%
\begin{multi}[]%
    % - Identificador
    {Rango-06}
    % - Enunciado
    A simple vista, ¿Cuál es el rango de la siguiente matriz?
    \[
        A=\begin{pmatrix} 1 & 2 & 3 \\ 2 & 5 & 6 \\ 5 & 6 & 9 \end{pmatrix}
    \]
    \item 0
    \item 1
    \item 2
    \item 3
    \item* No se puede determinar a simple vista.
\end{multi}

%%%%%%%%%%%%%%%%%%%%%%%%%%%%%%%%%%%%%%%%
\begin{multi}[]%
    % - Identificador
    {Rango-07}
    % - Enunciado
    A simple vista, ¿Cuál es el rango de la siguiente matriz?
    \[
        A=\begin{pmatrix} 1 & 2 & 3 \\ 0 & 0 & 0 \\ 3 & 7 & 9 \end{pmatrix}
    \]
    \item 0
    \item 1
    \item 2
    \item 3
    \item* No se puede determinar a simple vista.
\end{multi}

%%%%%%%%%%%%%%%%%%%%%%%%%%%%%%%%%%%%%%%%
\begin{multi}[]%
    % - Identificador
    {Rango-08}
    % - Enunciado
    A simple vista, ¿Cuál es el rango de la siguiente matriz?
    \[
        A=\begin{pmatrix} 1 & 2 & 3 \\ 0 & 4 & 6 \\ 0 & 0 & 9 \end{pmatrix}
    \]
    \item 0
    \item 1
    \item 2
    \item* 3
    \item No se puede determinar a simple vista.
\end{multi}

%%%%%%%%%%%%%%%%%%%%%%%%%%%%%%%%%%%%%%%%
\begin{multi}[]%
    % - Identificador
    {Rango-09}
    % - Enunciado
    A simple vista, ¿Cuál es el rango de la siguiente matriz?
    \[
        A=\begin{pmatrix} 1 & 2 & 3 \\ 0 & 0 & 0 \\ 0 & 0 & 0 \end{pmatrix}
    \]
    \item 0
    \item* 1
    \item 2
    \item 3
    \item No se puede determinar a simple vista.
\end{multi}

%%%%%%%%%%%%%%%%%%%%%%%%%%%%%%%%%%%%%%%%
\begin{multi}[]%
    % - Identificador
    {Rango-10}
    % - Enunciado
    A simple vista, ¿Cuál es el rango de la siguiente matriz?
    \[
        A=\begin{pmatrix} 0 & 1 & 0 \\ 0 & 0 & 1 \\ 0 & 0 & 0 \end{pmatrix}
    \]
    \item 0
    \item 1
    \item* 2
    \item 3
    \item No se puede determinar a simple vista.
\end{multi}

%%%%%%%%%%%%%%%%%%%%%%%%%%%%%%%%%%%%%%%%
\begin{multi}[]%
    % - Identificador
    {Rango-11}
    % - Enunciado
    A simple vista, ¿Cuál es el rango de la siguiente matriz?
    \[
        A=\begin{pmatrix} 3 & 1 & 4 \\ 6 & 3 & 1 \\ 2 & 5 & 9 \end{pmatrix}
    \]
    \item 0
    \item 1
    \item 2
    \item 3
    \item* No se puede determinar a simple vista.
\end{multi}

%%%%%%%%%%%%%%%%%%%%%%%%%%%%%%%%%%%%%%%%
\begin{multi}[]%
    % - Identificador
    {Rango-12}
    % - Enunciado
    A simple vista, ¿Cuál es el rango de la siguiente matriz?
    \[
        A=\begin{pmatrix} 3 & 1 & 4 \\ 0 & 0 & 1 \\ 0 & 0 & 9 \end{pmatrix}
    \]
    \item 0
    \item 1
    \item* 2
    \item 3
    \item No se puede determinar a simple vista.
\end{multi}

%%%%%%%%%%%%%%%%%%%%%%%%%%%%%%%%%%%%%%%%
\begin{multi}[]%
    % - Identificador
    {Rango-13}
    % - Enunciado
    A simple vista, ¿Cuál es el rango de la siguiente matriz?
    \[
        A=\begin{pmatrix} 8 & 6 & 4 \\ 4 & 3 & 0 \\ 0 & 0 & 1 \end{pmatrix}
    \]
    \item 0
    \item 1
    \item* 2
    \item 3
    \item No se puede determinar a simple vista.
\end{multi}

%%%%%%%%%%%%%%%%%%%%%%%%%%%%%%%%%%%%%%%%
\begin{multi}[]%
    % - Identificador
    {Rango-14}
    % - Enunciado
    A simple vista, ¿Cuál es el rango de la siguiente matriz?
    \[
        A=\begin{pmatrix} 4 & 3 & 0 \\ 2 & 0 & 1\\ 8 & 6 & 4 \end{pmatrix}
    \]
    \item 0
    \item 1
    \item 2
    \item* 3
    \item No se puede determinar a simple vista.
\end{multi}

%%%%%%%%%%%%%%%%%%%%%%%%%%%%%%%%%%%%%%%%
\begin{multi}[]%
    % - Identificador
    {Rango-15}
    % - Enunciado
    A simple vista, ¿Cuál es el rango de la siguiente matriz?
    \[
        A=\begin{pmatrix} 4 & 3 & 9 \\ 2 & 3 & 1\\ 8 & 6 & 4 \end{pmatrix}
    \]
    \item 0
    \item 1
    \item 2
    \item 3
    \item* No se puede determinar a simple vista.
\end{multi}

\end{quiz}


%%%%%%%%%%%%%%%%%%%%%%%%%%%%%%%%%%%%%%%%
\begin{quiz}{Determinantes}
%%%%%%%%%%%%%%%%%%%%%%%%%%%%%%%%%%%%%%%%

%%%%%%%%%%%%%%%%%%%%%%%%%%%%%%%%%%%%%%%%
\begin{numerical}[tolerance=0.01]%
    % - Identificador
    {Determinantes-01}
    % - Enunciado
    Si el determinante de una matriz $A$, de $2\times 2$ es $-2$, entonces el determinante de $3A$ es:
    \item -18
\end{numerical}

%%%%%%%%%%%%%%%%%%%%%%%%%%%%%%%%%%%%%%%%
\begin{numerical}[tolerance=0.01]%
    % - Identificador
    {Determinantes-02}
    % - Enunciado
    Si el determinante de una matriz $A$, de $3\times 3$ es $4$, entonces el determinante de $2A$ es:
    \item 32
\end{numerical}

%%%%%%%%%%%%%%%%%%%%%%%%%%%%%%%%%%%%%%%%
\begin{numerical}[tolerance=0.01]%
    % - Identificador
    {Determinantes-03}
    % - Enunciado
    Si el determinante de una matriz $A$, de $3\times 3$ es $-1$, entonces el determinante de $2A$ es:
    \item -8
\end{numerical}

%%%%%%%%%%%%%%%%%%%%%%%%%%%%%%%%%%%%%%%%
\begin{numerical}[tolerance=0.01]%
    % - Identificador
    {Determinantes-04}
    % - Enunciado
    Si el determinante de una matriz $A$, de $4\times 4$ es $3$, entonces el determinante de $-2A$ es:
    \item 48
\end{numerical}

%%%%%%%%%%%%%%%%%%%%%%%%%%%%%%%%%%%%%%%%
\begin{numerical}[tolerance=0.01]%
    % - Identificador
    {Determinantes-05}
    % - Enunciado
    Si el determinante de una matriz $A$, de $4\times 4$ es $-1$, entonces el determinante de $-2A$ es:
    \item -16
\end{numerical}

%%%%%%%%%%%%%%%%%%%%%%%%%%%%%%%%%%%%%%%%
\begin{numerical}[tolerance=0.01]%
    % - Identificador
    {Determinantes-06}
    % - Enunciado
    Si el determinante de una matriz $A$, de $2\times 2$ es $-2$, entonces el determinante de $A^2$ es:
    \item 4
\end{numerical}

%%%%%%%%%%%%%%%%%%%%%%%%%%%%%%%%%%%%%%%%
\begin{numerical}[tolerance=0.01]%
    % - Identificador
    {Determinantes-07}
    % - Enunciado
    Si el determinante de una matriz $A$, de $2\times 2$ es $-2$ y el determinante de una matriz $B$, de $2\times 2$ es $3$, entonces el determinante de $AB$ es:
    \item -6
\end{numerical}

%%%%%%%%%%%%%%%%%%%%%%%%%%%%%%%%%%%%%%%%
\begin{numerical}[tolerance=0.01]%
    % - Identificador
    {Determinantes-08}
    % - Enunciado
    Si el determinante de una matriz $A$, de $3\times 3$ es $-2$ y el determinante de una matriz $B$, de $3\times 3$ es $4$, entonces el determinante de $AB$ es:
    \item -8
\end{numerical}

%%%%%%%%%%%%%%%%%%%%%%%%%%%%%%%%%%%%%%%%
\begin{numerical}[tolerance=0.01]%
    % - Identificador
    {Determinantes-09}
    % - Enunciado
    Si el determinante de una matriz $A$, de $3\times 3$ es $1$ y el determinante de una matriz $B$, de $3\times 3$ es $3$, entonces el determinante de $BA$ es:
    \item 3
\end{numerical}

%%%%%%%%%%%%%%%%%%%%%%%%%%%%%%%%%%%%%%%%
\begin{numerical}[tolerance=0.01]%
    % - Identificador
    {Determinantes-10}
    % - Enunciado
    Si el determinante de una matriz $A$, de $4\times 4$ es $3$ y el determinante de una matriz $B$, de $4\times 4$ es $3$, entonces el determinante de $BA$ es:
    \item 9
\end{numerical}

%%%%%%%%%%%%%%%%%%%%%%%%%%%%%%%%%%%%%%%%
\begin{numerical}[tolerance=0.01]%
    % - Identificador
    {Determinantes-11}
    % - Enunciado
    Si el determinante de una matriz $A$, de $2\times 2$ es $-2$, entonces el determinante de $A^\intercal$ es:
    \item -2
\end{numerical}

%%%%%%%%%%%%%%%%%%%%%%%%%%%%%%%%%%%%%%%%
\begin{numerical}[tolerance=0.01]%
    % - Identificador
    {Determinantes-12}
    % - Enunciado
    Si el determinante de una matriz $A$, de $3\times 3$ es $2$, entonces el determinante de $A^\intercal$ es:
    \item 2
\end{numerical}

%%%%%%%%%%%%%%%%%%%%%%%%%%%%%%%%%%%%%%%%
\begin{numerical}[tolerance=0.01]%
    % - Identificador
    {Determinantes-13}
    % - Enunciado
    Si el determinante de una matriz $A$, de $3\times 3$ es $2$, y $B$ es la matriz obtenida al realizar el intercambio de dos filas o columnas de $A$, entonces el determinante de $B$ es:
    \item -2
\end{numerical}

%%%%%%%%%%%%%%%%%%%%%%%%%%%%%%%%%%%%%%%%
\begin{numerical}[tolerance=0.01]%
    % - Identificador
    {Determinantes-14}
    % - Enunciado
    Si el determinante de una matriz $A$, de $4\times 4$ es $3$, y $B$ es la matriz obtenida al realizar el intercambio de dos filas o columnas de $A$, entonces el determinante de $B$ es:
    \item -3
\end{numerical}

%%%%%%%%%%%%%%%%%%%%%%%%%%%%%%%%%%%%%%%%
\begin{numerical}[tolerance=0.01]%
    % - Identificador
    {Determinantes-15}
    % - Enunciado
    Si el determinante de una matriz $A$, de $4\times 4$ es $3$, y $B$ es la matriz obtenida al realizar la multiplicación de una fila o columna de $A$ por el escalar $2$, entonces el determinante de $B$ es:
    \item 6
\end{numerical}

%%%%%%%%%%%%%%%%%%%%%%%%%%%%%%%%%%%%%%%%
\begin{numerical}[tolerance=0.01]%
    % - Identificador
    {Determinantes-16}
    % - Enunciado
    Si el determinante de una matriz $A$, de $4\times 4$ es $3$, y $B$ es la matriz obtenida al realizar la multiplicación de una fila o columna de $A$ por el escalar $-2$, entonces el determinante de $B$ es:
    \item -6
\end{numerical}

%%%%%%%%%%%%%%%%%%%%%%%%%%%%%%%%%%%%%%%%
\begin{numerical}[tolerance=0.01]%
    % - Identificador
    {Determinantes-17}
    % - Enunciado
    Si el determinante de una matriz $A$, de $2\times 2$ es $-2$, entonces el determinante de $A^{-1}$ es:
    \item -0.5
\end{numerical}

%%%%%%%%%%%%%%%%%%%%%%%%%%%%%%%%%%%%%%%%
\begin{numerical}[tolerance=0.01]%
    % - Identificador
    {Determinantes-18}
    % - Enunciado
    Si el determinante de una matriz $A$, de $3\times 3$ es $4$, entonces el determinante de $A^{-1}$ es:
    \item 0.25
\end{numerical}

%%%%%%%%%%%%%%%%%%%%%%%%%%%%%%%%%%%%%%%%
\begin{numerical}[tolerance=0.01]%
    % - Identificador
    {Determinantes-19}
    % - Enunciado
    Si el determinante de una matriz $A$, de $3\times 3$ es $2$, y el determinante de una matriz $B$, de $3\times 3$ es $4$, entonces el determinante de $A^{-1}B$ es:
    \item 2
\end{numerical}

%%%%%%%%%%%%%%%%%%%%%%%%%%%%%%%%%%%%%%%%
\begin{numerical}[tolerance=0.01]%
    % - Identificador
    {Determinantes-20}
    % - Enunciado
    Si el determinante de una matriz $A$, de $4\times 4$ es $3$, y el determinante de una matriz $B$, de $4\times 4$ es $3$, entonces el determinante de $A^{-1}B$ es:
    \item 1
\end{numerical}

\end{quiz}

\end{document}