\documentclass[a4,11pt]{aleph-notas}
% Actualizado en febrero de 2024
% Funciona con TeXLive 2022
% Para obtener solo el pdf, compilar con pdfLaTeX. 
%  latexmk -pdf 01\ Cuestionarios/01\ Matrices.tex -output-directory="01 Cuestionarios"
% Para obtener el xml compilar con XeLaTeX.
% latexmk -xelatex 01\ Cuestionarios/01\ Matrices.tex -output-directory="01 Cuestionarios"

% -- Paquetes adicionales
\usepackage{aleph-moodle}
\moodleregisternewcommands
% Todos los comandos nuevos deben ir luego del comando anterior
\usepackage{aleph-comandos}


% -- Datos  
\institucion{Escuela de Ciencias Físicas y Matemática}
\carrera{Ciencia de datos / Bioingeniería}
\asignatura{Álgebra Lineal}
\tema{Cuestionario No. 1-2: Sistemas de ecuaciones}
\autor{Andrés Merino}
\fecha{Semestre 2024-1}

\logouno[0.14\textwidth]{Logos/logoPUCE_04_ac}
\definecolor{colortext}{HTML}{0030A1}
\definecolor{colordef}{HTML}{0030A1}
\fuente{montserrat}

% -- Otros comandos



\begin{document}

\encabezado

\vspace*{-8mm}
\tableofcontents

%%%%%%%%%%%%%%%%%%%%%%%%%%%%%%%%%%%%%%%%
\section{Indicaciones}
%%%%%%%%%%%%%%%%%%%%%%%%%%%%%%%%%%%%%%%%

Se plantean bancos de preguntas orientados a evaluar el \textbf{criterio}: «Identifica los elementos de los sistemas de ecuaciones lineales y la clasificación de estos por su tipo de soluciones», correspondiente al \textbf{resultado de aprendizaje}: Comprender los conceptos fundamentales del Álgebra Lineal, incluyendo el estudio de matrices, determinantes, sistemas de ecuaciones lineales y espacios vectoriales, destacando su importancia en el análisis y resolución de problemas matemáticos.

%%%%%%%%%%%%%%%%%%%%%%%%%%%%%%%%%%%%%%%%
\section{Banco de preguntas}
%%%%%%%%%%%%%%%%%%%%%%%%%%%%%%%%%%%%%%%%

%%%%%%%%%%%%%%%%%%%%%%%%%%%%%%%%%%%%%%%%
\begin{quiz}{Propiedades Sistemas de Ecuaciones}
%%%%%%%%%%%%%%%%%%%%%%%%%%%%%%%%%%%%%%%%

%%%%%%%%%%%%%%%%%%%%%%%%%%%%%%%%%%%%%%%%
\begin{multi}[]%
    % - Identificador
    {SisEc-DefProp-01}
    % - Enunciado
    ¿Cuál de las siguientes afirmaciones es verdadera respecto a un sistema lineal de ecuaciones?
    \item Si el rango de la matriz de coeficientes es igual al rango de la matriz ampliada, entonces el sistema tiene una única solución.
    \item Si el rango de la matriz de coeficientes es menor que el rango de la matriz ampliada, entonces el sistema tiene infinitas soluciones.
    \item Si el rango de la matriz de coeficientes es mayor que el rango de la matriz ampliada, entonces el sistema no tiene solución.
    \item* Si el rango de la matriz de coeficientes es igual al número de incógnitas y es igual al rango de la matriz ampliada, entonces el sistema tiene solución única.
\end{multi}

%%%%%%%%%%%%%%%%%%%%%%%%%%%%%%%%%%%%%%%%    
\begin{multi}[]%
    % - Identificador
    {SisEc-DefProp-02}
    % - Enunciado
    ¿Cuál es el rango máximo que puede tener una matriz de coeficientes de un sistema lineal de 4 ecuaciones con 3 incógnitas?
    \item* 3
    \item 4
    \item 2
    \item 1
\end{multi}
    
%%%%%%%%%%%%%%%%%%%%%%%%%%%%%%%%%%%%%%%%  
\begin{multi}[]%
    % - Identificador
    {SisEc-DefProp-03}
    % - Enunciado
    ¿Cuál de las siguientes afirmaciones es verdadera respecto a un sistema lineal homogéneo?
    \item* Siempre tiene al menos una solución trivial.
    \item Siempre tiene una única solución no trivial.
    \item Siempre tiene infinitas soluciones.
    \item Nunca tiene solución.
\end{multi}
    
%%%%%%%%%%%%%%%%%%%%%%%%%%%%%%%%%%%%%%%%  
    \begin{multi}[]%
    % - Identificador
    {SisEc-DefProp-04}
    % - Enunciado
    ¿Cuál de las siguientes afirmaciones es verdadera para un sistema lineal inconsistente?
    \item Tiene exactamente una solución.
    \item Tiene al menos una solución.
    \item Tiene infinitas soluciones.
    \item* No tiene solución.
    \end{multi}
    
%%%%%%%%%%%%%%%%%%%%%%%%%%%%%%%%%%%%%%%%  
\begin{multi}[]%
    % - Identificador
    {SisEc-DefProp-05}
    % - Enunciado
    ¿Cuál de las siguientes afirmaciones es correcta respecto a la matriz aumentada de un sistema lineal?
    \item* La última columna corresponde a los términos independientes de las ecuaciones.
    \item La primera columna corresponde a los términos independientes de las ecuaciones.
    \item La última columna está compuesta por ceros.
\end{multi}

%%%%%%%%%%%%%%%%%%%%%%%%%%%%%%%%%%%%%%%%  
\begin{multi}[]%
    % - Identificador
    {SisEc-DefProp-06}
    % - Enunciado
    ¿Cuál es la condición necesaria para que un sistema lineal tenga solución única?
    \item Que el número de ecuaciones sea igual al número de incógnitas.
    \item Que el número de ecuaciones sea menor que el número de incógnitas.
    \item* Que el rango de la matriz de coeficientes sea igual al rango de la matriz ampliada y coincida con el número de incógnitas.
    \item Que el rango de la matriz de coeficientes sea menor que el número de incógnitas.
\end{multi}

%%%%%%%%%%%%%%%%%%%%%%%%%%%%%%%%%%%%%%%%  
\begin{multi}[]%
    % - Identificador
    {SisEc-DefProp-07}
    % - Enunciado
    ¿Cuál de las siguientes afirmaciones es verdadera respecto a un sistema lineal con más incógnitas que ecuaciones?
    \item Siempre tiene una solución única.
    \item* Puede tener infinitas soluciones.
    \item Siempre es inconsistente.
    \item No tiene solución.
\end{multi}

%%%%%%%%%%%%%%%%%%%%%%%%%%%%%%%%%%%%%%%%  
\begin{multi}[]%
    % - Identificador
    {SisEc-DefProp-08}
    % - Enunciado
    ¿Cuál de las siguientes afirmaciones es verdadera respecto a un sistema lineal homogéneo en el cual el número de incógnitas es mayor que el número de ecuaciones?
    \item Solamente tiene la solución trivial.
    \item* Tiene solución no trivial.
    \item Siempre es inconsistente.
    \item No tiene solución.
\end{multi}

%%%%%%%%%%%%%%%%%%%%%%%%%%%%%%%%%%%%%%%%  
\begin{multi}[]%
    % - Identificador
    {SisEc-DefProp-09}
    % - Enunciado
    ¿Cuál de las siguientes afirmaciones es verdadera respecto a un sistema lineal consistente?
    \item Siempre tiene solución única.
    \item* Siempre tiene solución.
    \item Siempre tiene infinitas soluciones.
    \item No tiene solución.
\end{multi}

%%%%%%%%%%%%%%%%%%%%%%%%%%%%%%%%%%%%%%%%  
\begin{multi}[]%
    % - Identificador
    {SisEc-DefProp-10}
    % - Enunciado
    ¿Cuál de las siguientes afirmaciones es verdadera respecto a dos sistemas de ecuaciones lineales?
    \item Tienen las mismas soluciones si y solo si tienen el mismo número de ecuaciones.
    \item* Tienen las mismas soluciones si y solo si las matrices aumentadas de los sistemas son equivalentes por filas.
    \item Tienen las mismas soluciones si y solo si tienen el mismo número de incógnitas.
    \item Tienen las mismas soluciones si y solo si las matrices de coeficientes de los sistemas son equivalentes por filas.
\end{multi}

%%%%%%%%%%%%%%%%%%%%%%%%%%%%%%%%%%%%%%%%  
\begin{multi}[]%
    % - Identificador
    {SisEc-DefProp-11}
    % - Enunciado
    ¿Cuál de las siguientes afirmaciones es verdadera respecto a un sistema de ecuaciones lineales homogéneo?
    \item La columna de constantes es no nula.
    \item* La columna de constantes es nula.
    \item La columna de incógnitas es nula.
    \item La columna de incógnitas es no nula.
\end{multi}


\end{quiz}


%%%%%%%%%%%%%%%%%%%%%%%%%%%%%%%%%%%%%%%%
\begin{quiz}{Soluciones}
%%%%%%%%%%%%%%%%%%%%%%%%%%%%%%%%%%%%%%%%
    
%%%%%%%%%%%%%%%%%%%%%%%%%%%%%%%%%%%%%%%%
\begin{multi}[]%
    % - Identificador
    {Soluciones-01}
    % - Enunciado
    ¿Cuál de las siguientes opciones es una solución del sistema de ecuaciones lineales?
    \[
        \begin{cases}
            2x + 3y = 5 \\
            4x - 2y = 10
        \end{cases}
    \]
    \item $(x, y) = (1, 1)$
    \item $(x, y) = (2, 1)$
    \item* $(x, y) = (5/2, 0)$
\end{multi}
        
%%%%%%%%%%%%%%%%%%%%%%%%%%%%%%%%%%%%%%%%
\begin{multi}[]%
    % - Identificador
    {Soluciones-02}
    % - Enunciado
    Encuentra la solución del sistema de ecuaciones lineales:
    \[
    \begin{cases}
    3x - 2y = 5 \\
    6x + 2y = 4
    \end{cases}
    \]
    \item $(x, y) = (2, 1)$
    \item $(x, y) = (2, -1)$
    \item* $(x, y) = (1, -1)$
\end{multi}
        
%%%%%%%%%%%%%%%%%%%%%%%%%%%%%%%%%%%%%%%%
\begin{multi}[]%
    % - Identificador
    {Soluciones-03}
    % - Enunciado
    Resuelve el siguiente sistema de ecuaciones:
    \[
    \begin{cases}
    4x - 3y = -17 \\
    2x + 5y = 11
    \end{cases}
    \]
    \item $(x, y) = (3, -1)$
    \item $(x, y) = (1, 2)$
    \item* $(x, y) = (-2, 3)$
\end{multi}
        
%%%%%%%%%%%%%%%%%%%%%%%%%%%%%%%%%%%%%%%%
\begin{multi}[]%
    % - Identificador
    {Soluciones-04}
    % - Enunciado
    Encuentra la solución del sistema de ecuaciones lineales:
    \[
    \begin{cases}
    2x + y = 5 \\
    x - 3y = 6
    \end{cases}
    \]
    \item* $(x, y) = (3, -1)$
    \item $(x, y) = (-1, 3)$
    \item $(x, y) = (1, -2)$
\end{multi}
        
%%%%%%%%%%%%%%%%%%%%%%%%%%%%%%%%%%%%%%%%
\begin{multi}[]%
    % - Identificador
    {Soluciones-05}
    % - Enunciado
    Resuelve el siguiente sistema de ecuaciones:
    \[
    \begin{cases}
    3x + 2y = 13 \\
    4x - y = 10
    \end{cases}
    \]
    \item $(x, y) = (2, 1)$
    \item* $(x, y) = (3, 2)$
    \item $(x, y) = (1, 2)$
\end{multi}

%%%%%%%%%%%%%%%%%%%%%%%%%%%%%%%%%%%%%%%%
\begin{multi}[]%
    % - Identificador
    {Soluciones-06}
    % - Enunciado
    Resuelve el siguiente sistema de ecuaciones:
    \[
    \begin{cases}
    2x + y = 1 \\
    4x - y = 4
    \end{cases}
    \]
    \item $(x, y) = (2, 1)$
    \item* $(x, y) = (1, 0)$
    \item $(x, y) = (1, 2)$
\end{multi}

%%%%%%%%%%%%%%%%%%%%%%%%%%%%%%%%%%%%%%%%
\begin{multi}[]%
    % - Identificador
    {Soluciones-07}
    % - Enunciado
    Resuelve el siguiente sistema de ecuaciones:
    \[
    \begin{cases}
    -3x + 7y = 4 \\
    x - y = 4
    \end{cases}
    \]
    \item $(x, y) = (2, 5)$
    \item* $(x, y) = (8, 4)$
    \item $(x, y) = (6, 1)$
\end{multi}

%%%%%%%%%%%%%%%%%%%%%%%%%%%%%%%%%%%%%%%%
\begin{multi}[]%
    % - Identificador
    {Soluciones-08}
    % - Enunciado
    Resuelve el siguiente sistema de ecuaciones:
    \[
    \begin{cases}
    -2x + y = -1 \\
    -x - 2y = 8
    \end{cases}
    \]
    \item $(x, y) = (8, 4)$
    \item* $(x, y) = (2, 3)$
    \item $(x, y) = (5 2)$
\end{multi}

%%%%%%%%%%%%%%%%%%%%%%%%%%%%%%%%%%%%%%%%
\begin{multi}[]%
    % - Identificador
    {Soluciones-09}
    % - Enunciado
    Resuelve el siguiente sistema de ecuaciones:
    \[
    \begin{cases}
    -x + 2y = 3 \\
    -2x - y = -4
    \end{cases}
    \]
    \item $(x, y) = (2, 3)$
    \item* $(x, y) = (1, 2)$
    \item $(x, y) = (2, 2)$
\end{multi}

%%%%%%%%%%%%%%%%%%%%%%%%%%%%%%%%%%%%%%%%
\begin{multi}[]%
    % - Identificador
    {Soluciones-10}
    % - Enunciado
    Resuelve el siguiente sistema de ecuaciones:
    \[
    \begin{cases}
    -2x + y = 3 \\
    x - y = -4
    \end{cases}
    \]
    \item $(x, y) = (3, 3)$
    \item* $(x, y) = (1, 5)$
    \item $(x, y) = (4, 2)$
\end{multi}

%%%%%%%%%%%%%%%%%%%%%%%%%%%%%%%%%%%%%%%%
\begin{multi}[]%
    % - Identificador
    {Soluciones-11}
    % - Enunciado
    Resuelve el siguiente sistema de ecuaciones:
    \[
    \begin{cases}
    -2x + 5y = 4 \\
    -x + y = -1
    \end{cases}
    \]
    \item $(x, y) = (3, 3)$
    \item* $(x, y) = (3, 2)$
    \item $(x, y) = (2, 3)$
\end{multi}

%%%%%%%%%%%%%%%%%%%%%%%%%%%%%%%%%%%%%%%%
\begin{multi}[]%
    % - Identificador
    {Soluciones-12}
    % - Enunciado
    Resuelve el siguiente sistema de ecuaciones:
    \[
    \begin{cases}
    -x + 5y = 2 \\
    2x + 5y = 11
    \end{cases}
    \]
    \item $(x, y) = (4, 3)$
    \item* $(x, y) = (3, 1)$
    \item $(x, y) = (2, 3)$
\end{multi}

%%%%%%%%%%%%%%%%%%%%%%%%%%%%%%%%%%%%%%%%
\begin{multi}[]%
    % - Identificador
    {Soluciones-13}
    % - Enunciado
    Resuelve el siguiente sistema de ecuaciones:
    \[
    \begin{cases}
    x + 5y = 7 \\
    x + 2y = 4
    \end{cases}
    \]
    \item $(x, y) = (4, 3)$
    \item* $(x, y) = (2, 1)$
    \item $(x, y) = (2, 4)$
\end{multi}

%%%%%%%%%%%%%%%%%%%%%%%%%%%%%%%%%%%%%%%%
\begin{multi}[]%
    % - Identificador
    {Soluciones-14}
    % - Enunciado
    Resuelve el siguiente sistema de ecuaciones:
    \[
    \begin{cases}
    -2x + 5y = 16 \\
    -x + 4y = 14
    \end{cases}
    \]
    \item $(x, y) = (4, 3)$
    \item* $(x, y) = (2, 4)$
    \item $(x, y) = (2, 1)$
\end{multi}

\end{quiz}



%%%%%%%%%%%%%%%%%%%%%%%%%%%%%%%%%%%%%%%%
\begin{quiz}{Conjunto solución}
%%%%%%%%%%%%%%%%%%%%%%%%%%%%%%%%%%%%%%%%
    
%%%%%%%%%%%%%%%%%%%%%%%%%%%%%%%%%%%%%%%%
\begin{multi}[]%
    % - Identificador
    {ConjuntoSol-01}
    % - Enunciado
    Al resolver un sistema de ecuaciones, se obtiene el siguiente sistema equivalente:
    \[
        \begin{cases}
            x + 3y = 5 \\
            0 = 0
        \end{cases}
    \]
    ¿Cuál es el conjunto solución del sistema de ecuaciones lineales?
    \item* $\{(5-3t, t) : t\in \mathbb{R}\}$
    \item $\{(5+3t, t) : t\in \mathbb{R}\}$
    \item $\{(t, 3t) : t\in \mathbb{R}\}$
    \item $\{(5, t) : t\in \mathbb{R}\}$
\end{multi}

%%%%%%%%%%%%%%%%%%%%%%%%%%%%%%%%%%%%%%%%
\begin{multi}[]%
    % - Identificador
    {ConjuntoSol-02}
    % - Enunciado
    Al resolver un sistema de ecuaciones, se obtiene el siguiente sistema equivalente:
    \[
        \begin{cases}
            x - 3y = 5 \\
            0 = 0
        \end{cases}
    \]
    ¿Cuál es el conjunto solución del sistema de ecuaciones lineales?
    \item $\{(5-3t, t) : t\in \mathbb{R}\}$
    \item* $\{(5+3t, t) : t\in \mathbb{R}\}$
    \item $\{(t, -3t) : t\in \mathbb{R}\}$
    \item $\{(5, t) : t\in \mathbb{R}\}$
\end{multi}

%%%%%%%%%%%%%%%%%%%%%%%%%%%%%%%%%%%%%%%%
\begin{multi}[]%
    % - Identificador
    {ConjuntoSol-03}
    % - Enunciado
    Al resolver un sistema de ecuaciones, se obtiene el siguiente sistema equivalente:
    \[
        \begin{cases}
            2x + 4y = 8 \\
            0 = 0
        \end{cases}
    \]
    ¿Cuál es el conjunto solución del sistema de ecuaciones lineales?
    \item $\{(4+2t, t) : t\in \mathbb{R}\}$
    \item* $\{(4-2t, t) : t\in \mathbb{R}\}$
    \item $\{(2t, 4t) : t\in \mathbb{R}\}$
    \item $\{(8, t) : t\in \mathbb{R}\}$
\end{multi}

%%%%%%%%%%%%%%%%%%%%%%%%%%%%%%%%%%%%%%%%
\begin{multi}[]%
    % - Identificador
    {ConjuntoSol-04}
    % - Enunciado
    Al resolver un sistema de ecuaciones, se obtiene el siguiente sistema equivalente:
    \[
        \begin{cases}
            2x - 4y = 8 \\
            0 = 0
        \end{cases}
    \]
    ¿Cuál es el conjunto solución del sistema de ecuaciones lineales?
    \item* $\{(4+2t, t) : t\in \mathbb{R}\}$
    \item $\{(4-2t, t) : t\in \mathbb{R}\}$
    \item $\{(2t, -4t) : t\in \mathbb{R}\}$
    \item $\{(8, t) : t\in \mathbb{R}\}$
\end{multi}

%%%%%%%%%%%%%%%%%%%%%%%%%%%%%%%%%%%%%%%%
\begin{multi}[]%
    % - Identificador
    {ConjuntoSol-05}
    % - Enunciado
    Al resolver un sistema de ecuaciones, se obtiene el siguiente sistema equivalente:
    \[
        \begin{cases}
            2x + 3y = 7 \\
            0 = 0
        \end{cases}
    \]
    ¿Cuál es el conjunto solución del sistema de ecuaciones lineales?
    \item* $\{\left(\frac{7}{2}-\frac{3}{2}t, t\right) : t\in \mathbb{R}\}$
    \item $\{\left(\frac{7}{2}+\frac{3}{2}t, t\right) : t\in \mathbb{R}\}$
    \item $\{\left(2t,3t\right) : t\in \mathbb{R}\}$
    \item $\{\left(3t, 2t\right) : t\in \mathbb{R}\}$
\end{multi}

%%%%%%%%%%%%%%%%%%%%%%%%%%%%%%%%%%%%%%%%
\begin{multi}[]%
    % - Identificador
    {ConjuntoSol-06}
    % - Enunciado
    Al resolver un sistema de ecuaciones, se obtiene el siguiente sistema equivalente:
    \[
        \begin{cases}
            -x - 5y = -9 \\
            0 = 0
        \end{cases}
    \]
    ¿Cuál es el conjunto solución del sistema de ecuaciones lineales?
    \item* $\{(5t-9, t) : t\in \mathbb{R}\}$
    \item $\{(5t+9, t) : t\in \mathbb{R}\}$
    \item $\{(t,5t) : t\in \mathbb{R}\}$
    \item $\{(5t,t) : t\in \mathbb{R}\}$
\end{multi}

%%%%%%%%%%%%%%%%%%%%%%%%%%%%%%%%%%%%%%%%
\begin{multi}[]%
    % - Identificador
    {ConjuntoSol-07}
    % - Enunciado
    Al resolver un sistema de ecuaciones, se obtiene el siguiente sistema equivalente:
    \[
        \begin{cases}
            3x - 2y = 12 \\
            0 = 0
        \end{cases}
    \]
    ¿Cuál es el conjunto solución del sistema de ecuaciones lineales?
    \item* $\{(4+\frac{3}{2}t, t) : t\in \mathbb{R}\}$
    \item $\{(4-\frac{3}{2}t, t) : t\in \mathbb{R}\}$
    \item $\{(2t,4t) : t\in \mathbb{R}\}$
    \item $\{(4t,2t) : t\in \mathbb{R}\}$
\end{multi}

%%%%%%%%%%%%%%%%%%%%%%%%%%%%%%%%%%%%%%%%
\begin{multi}[]%
    % - Identificador
    {ConjuntoSol-08}
    % - Enunciado
    Al resolver un sistema de ecuaciones, se obtiene el siguiente sistema equivalente:
    \[
        \begin{cases}
            2x +4y = 6 \\
            0 = 0
        \end{cases}
    \]
    ¿Cuál es el conjunto solución del sistema de ecuaciones lineales?
    \item* $\{(3-2t, t) : t\in \mathbb{R}\}$
    \item $\{(3+2t, t) : t\in \mathbb{R}\}$
    \item $\{(2t,4t) : t\in \mathbb{R}\}$
    \item $\{(4t,2t) : t\in \mathbb{R}\}$
\end{multi}

%%%%%%%%%%%%%%%%%%%%%%%%%%%%%%%%%%%%%%%%
\begin{multi}[]%
    % - Identificador
    {ConjuntoSol-09}
    % - Enunciado
    Al resolver un sistema de ecuaciones, se obtiene el siguiente sistema equivalente:
    \[
        \begin{cases}
            2x - 4y = 6 \\
            0 = 0
        \end{cases}
    \]
    ¿Cuál es el conjunto solución del sistema de ecuaciones lineales?
    \item* $\{(3+2t, t) : t\in \mathbb{R}\}$
    \item $\{(3-2t, t) : t\in \mathbb{R}\}$
    \item $\{(2t,4t) : t\in \mathbb{R}\}$
    \item $\{(4,2t) : t\in \mathbb{R}\}$
\end{multi}

%%%%%%%%%%%%%%%%%%%%%%%%%%%%%%%%%%%%%%%%
\begin{multi}[]%
    % - Identificador
    {ConjuntoSol-10}
    % - Enunciado
    Al resolver un sistema de ecuaciones, se obtiene el siguiente sistema equivalente:
    \[
        \begin{cases}
            6x + 5y = 2 \\
            0 = 0
        \end{cases}
    \]
    ¿Cuál es el conjunto solución del sistema de ecuaciones lineales?
    \item* $\{(\frac{1}{3}-\frac{5}{6}t, t) : t\in \mathbb{R}\}$
    \item $\{(\frac{1}{3}+\frac{5}{6}t, t) : t\in \mathbb{R}\}$
    \item $\{(6t,5t) : t\in \mathbb{R}\}$
    \item $\{(5t,6t) : t\in \mathbb{R}\}$
\end{multi}

%%%%%%%%%%%%%%%%%%%%%%%%%%%%%%%%%%%%%%%%
\begin{multi}[]%
    % - Identificador
    {ConjuntoSol-11}
    % - Enunciado
    Al resolver un sistema de ecuaciones, se obtiene el siguiente sistema equivalente:
    \[
        \begin{cases}
            6x - 5y = 2 \\
            0 = 0
        \end{cases}
    \]
    ¿Cuál es el conjunto solución del sistema de ecuaciones lineales?
    \item* $\{(\frac{1}{3}+\frac{5}{6}t, t) : t\in \mathbb{R}\}$
    \item $\{(\frac{1}{3}-\frac{5}{6}t, t) : t\in \mathbb{R}\}$
    \item $\{(6t,5t) : t\in \mathbb{R}\}$
    \item $\{(5t,6t) : t\in \mathbb{R}\}$
\end{multi}

%%%%%%%%%%%%%%%%%%%%%%%%%%%%%%%%%%%%%%%%
\begin{multi}[]%
    % - Identificador
    {ConjuntoSol-12}
    % - Enunciado
    Al resolver un sistema de ecuaciones, se obtiene el siguiente sistema equivalente:
    \[
        \begin{cases}
            3x + 2y = 2 \\
            0 = 0
        \end{cases}
    \]
    ¿Cuál es el conjunto solución del sistema de ecuaciones lineales?
    \item* $\{(\frac{2}{3}-\frac{2}{3}t, t) : t\in \mathbb{R}\}$
    \item $\{(\frac{2}{3}+\frac{2}{3}t, t) : t\in \mathbb{R}\}$
    \item $\{(3t,2t) : t\in \mathbb{R}\}$
    \item $\{(2t,3t) : t\in \mathbb{R}\}$
\end{multi}

%%%%%%%%%%%%%%%%%%%%%%%%%%%%%%%%%%%%%%%%
\begin{multi}[]%
    % - Identificador
    {ConjuntoSol-13}
    % - Enunciado
    Al resolver un sistema de ecuaciones, se obtiene el siguiente sistema equivalente:
    \[
        \begin{cases}
            3x - 2y = 2 \\
            0 = 0
        \end{cases}
    \]
    ¿Cuál es el conjunto solución del sistema de ecuaciones lineales?
    \item* $\{(\frac{2}{3}+\frac{2}{3}t, t) : t\in \mathbb{R}\}$
    \item $\{(\frac{2}{3}-\frac{2}{3}t, t) : t\in \mathbb{R}\}$
    \item $\{(3t,2t) : t\in \mathbb{R}\}$
    \item $\{(2t,3t) : t\in \mathbb{R}\}$
\end{multi}

%%%%%%%%%%%%%%%%%%%%%%%%%%%%%%%%%%%%%%%%
\begin{multi}[]%
    % - Identificador
    {ConjuntoSol-14}
    % - Enunciado
    Al resolver un sistema de ecuaciones, se obtiene el siguiente sistema equivalente:
    \[
        \begin{cases}
            3x - 6y = 2 \\
            0 = 0
        \end{cases}
    \]
    ¿Cuál es el conjunto solución del sistema de ecuaciones lineales?
    \item* $\{(\frac{2}{3}+2t, t) : t\in \mathbb{R}\}$
    \item $\{(\frac{2}{3}-2t, t) : t\in \mathbb{R}\}$
    \item $\{(3t,2t) : t\in \mathbb{R}\}$
    \item $\{(2t,3t) : t\in \mathbb{R}\}$
\end{multi}
\end{quiz}

%%%%%%%%%%%%%%%%%%%%%%%%%%%%%%%%%%%%%%%%
\begin{quiz}{Rouché–Frobenius 01}
%%%%%%%%%%%%%%%%%%%%%%%%%%%%%%%%%%%%%%%%

%%%%%%%%%%%%%%%%%%%%%%%%%%%%%%%%%%%%%%%%
\begin{multi}[]%
    % - Identificador
    {Rouché–Frobenius-01-01}
    % - Enunciado
    Dado un sistema lineal de ecuaciones de 3 incógnitas, si el rango de la matriz de coeficientes es 3 y el rango de la matriz ampliada es 3, ¿cuál de las siguientes afirmaciones es verdadera?
    \item* El sistema tiene solución única.
    \item El sistema tiene infinitas soluciones.
    \item El sistema no tiene solución.
\end{multi}

%%%%%%%%%%%%%%%%%%%%%%%%%%%%%%%%%%%%%%%%
\begin{multi}[]%
    % - Identificador
    {Rouché–Frobenius-01-02}
    % - Enunciado
    Dado un sistema lineal de ecuaciones de 3 incógnitas, si el rango de la matriz de coeficientes es 3 y el rango de la matriz ampliada es 4, ¿cuál de las siguientes afirmaciones es verdadera?
    \item El sistema tiene solución única.
    \item El sistema tiene infinitas soluciones.
    \item* El sistema no tiene solución.
\end{multi}

%%%%%%%%%%%%%%%%%%%%%%%%%%%%%%%%%%%%%%%%
\begin{multi}[]%
    % - Identificador
    {Rouché–Frobenius-01-03}
    % - Enunciado
    Dado un sistema lineal de ecuaciones de 4 incógnitas, si el rango de la matriz de coeficientes es 3 y el rango de la matriz ampliada es 3, ¿cuál de las siguientes afirmaciones es verdadera?
    \item El sistema tiene solución única.
    \item* El sistema tiene infinitas soluciones.
    \item El sistema no tiene solución.
\end{multi}

%%%%%%%%%%%%%%%%%%%%%%%%%%%%%%%%%%%%%%%%
\begin{multi}[]%
    % - Identificador
    {Sistemas-Lineales-01-04}
    % - Enunciado
    Para un sistema lineal de ecuaciones con 3 incógnitas, si el rango de la matriz de coeficientes es 2 y el rango de la matriz ampliada es 2, ¿cuál de las siguientes afirmaciones es verdadera?
    \item El sistema tiene solución única.
    \item* El sistema tiene infinitas soluciones.
    \item El sistema no tiene solución.
\end{multi}
    
%%%%%%%%%%%%%%%%%%%%%%%%%%%%%%%%%%%%%%%%
\begin{multi}[]%
    % - Identificador
    {Sistemas-Lineales-01-04}
    % - Enunciado
    Dado un sistema lineal de ecuaciones con 2 incógnitas, si el rango de la matriz de coeficientes es 2 y el rango de la matriz ampliada es 1, ¿cuál de las siguientes afirmaciones es correcta?
    \item El sistema tiene solución única.
    \item El sistema tiene infinitas soluciones.
    \item* El sistema no tiene solución.
\end{multi}
    
%%%%%%%%%%%%%%%%%%%%%%%%%%%%%%%%%%%%%%%%
\begin{multi}[]%
    % - Identificador
    {Sistemas-Lineales-01-05}
    % - Enunciado
    Para un sistema lineal de ecuaciones con 3 incógnitas, si el rango de la matriz de coeficientes es 3 y el rango de la matriz ampliada es 2, ¿cuál de las siguientes afirmaciones es verdadera?
    \item* El sistema no tiene solución.
    \item El sistema tiene solución única.
    \item El sistema tiene infinitas soluciones.
\end{multi}
    
%%%%%%%%%%%%%%%%%%%%%%%%%%%%%%%%%%%%%%%%
    \begin{multi}[]%
    % - Identificador
    {Sistemas-Lineales-01-06}
    % - Enunciado
    Para un sistema lineal de ecuaciones con 3 incógnitas, si el rango de la matriz de coeficientes es 3 y el rango de la matriz ampliada es 3, ¿qué se puede afirmar sobre el sistema?
    \item* El sistema tiene solución única.
    \item El sistema tiene infinitas soluciones.
    \item El sistema no tiene solución.
    \end{multi}
    
%%%%%%%%%%%%%%%%%%%%%%%%%%%%%%%%%%%%%%%%
\begin{multi}[]%
    % - Identificador
    {Sistemas-Lineales-01-07}
    % - Enunciado
    Para un sistema lineal de ecuaciones con 3 incógnitas, si el rango de la matriz de coeficientes es 1 y el rango de la matriz ampliada es 1, ¿cuál de las siguientes afirmaciones es verdadera?
    \item El sistema tiene solución única.
    \item* El sistema tiene infinitas soluciones.
    \item El sistema no tiene solución.
\end{multi}

%%%%%%%%%%%%%%%%%%%%%%%%%%%%%%%%%%%%%%%%
\begin{multi}[]%
    % - Identificador
    {Sistemas-Lineales-01-08}
    % - Enunciado
    Para un sistema lineal de ecuaciones con 3 incógnitas, si el rango de la matriz de coeficientes es 1 y el rango de la matriz ampliada es 2, ¿cuál de las siguientes afirmaciones es verdadera?
    \item El sistema tiene solución única.
    \item El sistema tiene infinitas soluciones.
    \item* El sistema no tiene solución.
\end{multi}

%%%%%%%%%%%%%%%%%%%%%%%%%%%%%%%%%%%%%%%%
\begin{multi}[]%
    % - Identificador
    {Sistemas-Lineales-01-09}
    % - Enunciado
    Para un sistema lineal de ecuaciones con 5 incógnitas, si el rango de la matriz de coeficientes es 4 y el rango de la matriz ampliada es 4, ¿cuál de las siguientes afirmaciones es verdadera?
    \item El sistema tiene solución única.
    \item* El sistema tiene infinitas soluciones.
    \item El sistema no tiene solución.
\end{multi}

%%%%%%%%%%%%%%%%%%%%%%%%%%%%%%%%%%%%%%%%
\begin{multi}[]%
    % - Identificador
    {Sistemas-Lineales-01-10}
    % - Enunciado
    Para un sistema lineal de ecuaciones con 5 incógnitas, si el rango de la matriz de coeficientes es 4 y el rango de la matriz ampliada es 3, ¿cuál de las siguientes afirmaciones es verdadera?
    \item El sistema tiene solución única.
    \item El sistema tiene infinitas soluciones.
    \item* El sistema no tiene solución.
\end{multi}

%%%%%%%%%%%%%%%%%%%%%%%%%%%%%%%%%%%%%%%%
\begin{multi}[]%
    % - Identificador
    {Sistemas-Lineales-01-11}
    % - Enunciado
    Para un sistema lineal de ecuaciones con 6 incógnitas, si el rango de la matriz de coeficientes es 6 y el rango de la matriz ampliada es 6, ¿cuál de las siguientes afirmaciones es verdadera?
    \item* El sistema tiene solución única.
    \item El sistema tiene infinitas soluciones.
    \item El sistema no tiene solución.
\end{multi}

%%%%%%%%%%%%%%%%%%%%%%%%%%%%%%%%%%%%%%%%
\begin{multi}[]%
    % - Identificador
    {Sistemas-Lineales-01-12}
    % - Enunciado
    Para un sistema lineal de ecuaciones con 4 incógnitas, si el rango de la matriz de coeficientes es 2 y el rango de la matriz ampliada es 2, ¿cuál de las siguientes afirmaciones es verdadera?
    \item El sistema tiene solución única.
    \item* El sistema tiene infinitas soluciones.
    \item El sistema no tiene solución.
\end{multi}

%%%%%%%%%%%%%%%%%%%%%%%%%%%%%%%%%%%%%%%%
\begin{multi}[]%
    % - Identificador
    {Sistemas-Lineales-01-13}
    % - Enunciado
    Para un sistema lineal de ecuaciones con 7 incógnitas, si el rango de la matriz de coeficientes es 7 y el rango de la matriz ampliada es 7, ¿cuál de las siguientes afirmaciones es verdadera?
    \item* El sistema tiene solución única.
    \item El sistema tiene infinitas soluciones.
    \item El sistema no tiene solución.
\end{multi}

%%%%%%%%%%%%%%%%%%%%%%%%%%%%%%%%%%%%%%%%
\begin{multi}[]%
    % - Identificador
    {Sistemas-Lineales-01-14}
    % - Enunciado
    Para un sistema lineal de ecuaciones con 4 incógnitas, si el rango de la matriz de coeficientes es 5 y el rango de la matriz ampliada es 5, ¿cuál de las siguientes afirmaciones es verdadera?
    \item* El sistema tiene solución única.
    \item El sistema tiene infinitas soluciones.
    \item El sistema no tiene solución.
\end{multi}


\end{quiz}

%%%%%%%%%%%%%%%%%%%%%%%%%%%%%%%%%%%%%%%%
\begin{quiz}{Rouché–Frobenius 02}
%%%%%%%%%%%%%%%%%%%%%%%%%%%%%%%%%%%%%%%%

%%%%%%%%%%%%%%%%%%%%%%%%%%%%%%%%%%%%%%%%
\begin{multi}[]%
    % - Identificador
    {Rouché–Frobenius-02-01}
    % - Enunciado
    Dado un sistema lineal de ecuaciones de 3 incógnitas, si su matriz ampliada es equivalente por filas a 
    \[
    \begin{pmatrix}
    1 & 0 & 0 & | & 3 \\
    0 & 1 & 0 & | & 2 \\
    0 & 0 & 1 & | & 1
    \end{pmatrix}
    \]
    entonces, se tiene que:
    \item* El sistema tiene solución única.
    \item El sistema tiene infinitas soluciones.
    \item El sistema no tiene solución.
    \item No se puede determinar la naturaleza del sistema.
\end{multi}

%%%%%%%%%%%%%%%%%%%%%%%%%%%%%%%%%%%%%%%%
\begin{multi}[]%
    % - Identificador
    {Rouché–Frobenius-02-02}
    % - Enunciado
    Dado un sistema lineal de ecuaciones de 3 incógnitas, si su matriz ampliada es equivalente por filas a 
    \[
    \begin{pmatrix}
    1 & 0 & 0 & | & 3 \\
    0 & 1 & 0 & | & 2 \\
    0 & 0 & 0 & | & 1
    \end{pmatrix}
    \]
    entonces, se tiene que:
    \item El sistema tiene solución única.
    \item El sistema tiene infinitas soluciones.
    \item* El sistema no tiene solución.
    \item No se puede determinar la naturaleza del sistema.
\end{multi}

%%%%%%%%%%%%%%%%%%%%%%%%%%%%%%%%%%%%%%%%
\begin{multi}[]%
    % - Identificador
    {Rouché–Frobenius-02-03}
    % - Enunciado
    Dado un sistema lineal de ecuaciones de 3 incógnitas, si su matriz ampliada es equivalente por filas a 
    \[
    \begin{pmatrix}
    1 & 0 & 0 & | & 3 \\
    0 & 1 & 0 & | & 2 \\
    0 & 0 & 0 & | & 0
    \end{pmatrix}
    \]
    entonces, se tiene que:
    \item El sistema tiene solución única.
    \item* El sistema tiene infinitas soluciones.
    \item El sistema no tiene solución.
    \item No se puede determinar la naturaleza del sistema.
\end{multi}

%%%%%%%%%%%%%%%%%%%%%%%%%%%%%%%%%%%%%%%%
\begin{multi}[]%
    % - Identificador
    {Rouché–Frobenius-02-04}
    % - Enunciado
    Dado un sistema lineal de ecuaciones de 3 incógnitas, si su matriz ampliada es equivalente por filas a 
    \[
    \begin{pmatrix}
    1 & 0 & 0 & | & 3 \\
    0 & 0 & 0 & | & -2 \\
    0 & 0 & 0 & | & 0
    \end{pmatrix}
    \]
    entonces, se tiene que:
    \item El sistema tiene solución única.
    \item El sistema tiene infinitas soluciones.
    \item* El sistema no tiene solución.
    \item No se puede determinar la naturaleza del sistema.
\end{multi}

%%%%%%%%%%%%%%%%%%%%%%%%%%%%%%%%%%%%%%%%
\begin{multi}[]%
    % - Identificador
    {Rouché–Frobenius-02-05}
    % - Enunciado
    Dado un sistema lineal de ecuaciones de 3 incógnitas, si su matriz ampliada es equivalente por filas a 
    \[
    \begin{pmatrix}
    1 & 0 & 0 & | & 3 \\
    0 & 1 & 0 & | & 2 \\
    0 & 0 & 0 & | & 1
    \end{pmatrix}
    \]
    entonces, se tiene que:
    \item El sistema tiene solución única.
    \item El sistema tiene infinitas soluciones.
    \item* El sistema no tiene solución.
    \item No se puede determinar la naturaleza del sistema.
\end{multi}

%%%%%%%%%%%%%%%%%%%%%%%%%%%%%%%%%%%%%%%%
\begin{multi}[]%
    % - Identificador
    {Rouché–Frobenius-02-06}
    % - Enunciado
    Dado un sistema lineal de ecuaciones de 3 incógnitas, si su matriz ampliada es equivalente por filas a 
    \[
    \begin{pmatrix}
    1 & 0 & 0 & | & 3 \\
    0 & 1 & 0 & | & 2 \\
    0 & 0 & 0 & | & 0
    \end{pmatrix}
    \]
    entonces, se tiene que:
    \item El sistema tiene solución única.
    \item* El sistema tiene infinitas soluciones.
    \item El sistema no tiene solución.
    \item No se puede determinar la naturaleza del sistema.
\end{multi}

%%%%%%%%%%%%%%%%%%%%%%%%%%%%%%%%%%%%%%%%
\begin{multi}[]%
    % - Identificador
    {Rouché–Frobenius-02-07}
    % - Enunciado
    Dado un sistema lineal de ecuaciones de 3 incógnitas, si su matriz ampliada es equivalente por filas a 
    \[
    \begin{pmatrix}
    1 & 0 & 0 & | & 3 \\
    0 & 0 & 0 & | & 2 \\
    0 & 0 & 0 & | & 0
    \end{pmatrix}
    \]
    entonces, se tiene que:
    \item El sistema tiene solución única.
    \item El sistema tiene infinitas soluciones.
    \item* El sistema no tiene solución.
    \item No se puede determinar la naturaleza del sistema.
\end{multi}

%%%%%%%%%%%%%%%%%%%%%%%%%%%%%%%%%%%%%%%%
\begin{multi}[]%
    % - Identificador
    {Rouché–Frobenius-02-08}
    % - Enunciado
    Dado un sistema lineal de ecuaciones de 3 incógnitas, si su matriz ampliada es equivalente por filas a 
    \[
    \begin{pmatrix}
    1 & 2 & 2 & | & 3 \\
    0 & 1 & 2 & | & 2 \\
    0 & 0 & 1 & | & 1 \\
    0 & 0 & 0 & | & 0
    \end{pmatrix}
    \]
    entonces, se tiene que:
    \item* El sistema tiene solución única.
    \item El sistema tiene infinitas soluciones.
    \item El sistema no tiene solución.
    \item No se puede determinar la naturaleza del sistema.
\end{multi}

%%%%%%%%%%%%%%%%%%%%%%%%%%%%%%%%%%%%%%%%
\begin{multi}[]%
    % - Identificador
    {Rouché–Frobenius-02-09}
    % - Enunciado
    Dado un sistema lineal de ecuaciones de 4 incógnitas, si su matriz ampliada es equivalente por filas a 
    \[
    \begin{pmatrix}
    1 & 2 & 2 & 0 & | & 3 \\
    0 & 1 & 2 & 0 & | & 2 \\
    0 & 0 & 1 & 0 & | & 1 \\
    0 & 0 & 0 & 0 & | & 0
    \end{pmatrix}
    \]
    entonces, se tiene que:
    \item El sistema tiene solución única.
    \item* El sistema tiene infinitas soluciones.
    \item El sistema no tiene solución.
    \item No se puede determinar la naturaleza del sistema.
\end{multi}

%%%%%%%%%%%%%%%%%%%%%%%%%%%%%%%%%%%%%%%%
\begin{multi}[]%
    % - Identificador
    {Rouché–Frobenius-02-10}
    % - Enunciado
    Dado un sistema lineal de ecuaciones de 4 incógnitas, si su matriz ampliada es equivalente por filas a
    \[
    \begin{pmatrix}
    1 & 2 & 2 & 0 & | & 3 \\
    0 & 1 & 2 & 0 & | & 2 \\
    0 & 0 & 1 & 0 & | & 1 \\
    0 & 0 & 0 & 1 & | & 0
    \end{pmatrix}
    \]
    entonces, se tiene que:
    \item* El sistema tiene solución única.
    \item El sistema tiene infinitas soluciones.
    \item El sistema no tiene solución.
    \item No se puede determinar la naturaleza del sistema.
\end{multi}


%%%%%%%%%%%%%%%%%%%%%%%%%%%%%%%%%%%%%%%%
\begin{multi}[]%
    % - Identificador
    {Rouché–Frobenius-02-10}
    % - Enunciado
    Dado un sistema lineal de ecuaciones de 4 incógnitas, si su matriz ampliada es equivalente por filas a
    \[
    \begin{pmatrix}
    1 & 2 & 2 & 0 & | & 3 \\
    0 & 1 & 2 & 0 & | & 2 \\
    0 & 0 & 0 & 0 & | & 1 \\
    0 & 0 & 0 & 0 & | & 0
    \end{pmatrix}
    \]
    entonces, se tiene que:
    \item El sistema tiene solución única.
    \item El sistema tiene infinitas soluciones.
    \item* El sistema no tiene solución.
    \item No se puede determinar la naturaleza del sistema.
\end{multi}


%%%%%%%%%%%%%%%%%%%%%%%%%%%%%%%%%%%%%%%%
\begin{multi}[]%
    % - Identificador
    {Rouché–Frobenius-02-11}
    % - Enunciado
    Dado un sistema lineal de ecuaciones de 4 incógnitas, si su matriz ampliada es equivalente por filas a
    \[
    \begin{pmatrix}
    1 & 0 & 2 & 0 & | & 3 \\
    0 & 0 & 4 & 0 & | & 2 \\
    0 & 0 & 0 & 0 & | & 4 \\
    0 & 0 & 0 & 0 & | & 0
    \end{pmatrix}
    \]
    entonces, se tiene que:
    \item El sistema tiene solución única.
    \item El sistema tiene infinitas soluciones.
    \item* El sistema no tiene solución.
    \item No se puede determinar la naturaleza del sistema.
\end{multi}


%%%%%%%%%%%%%%%%%%%%%%%%%%%%%%%%%%%%%%%%
\begin{multi}[]%
    % - Identificador
    {Rouché–Frobenius-02-12}
    % - Enunciado
    Dado un sistema lineal de ecuaciones de 3 incógnitas, si su matriz ampliada es equivalente por filas a 
    \[
    \begin{pmatrix}
    1 & 2 & 2 & | & 3 \\
    0 & 1 & 2 & | & 2 \\
    0 & 0 & 0 & | & 2 \\
    0 & 0 & 0 & | & 0
    \end{pmatrix}
    \]
    entonces, se tiene que:
    \item El sistema tiene solución única.
    \item El sistema tiene infinitas soluciones.
    \item* El sistema no tiene solución.
    \item No se puede determinar la naturaleza del sistema.
\end{multi}


%%%%%%%%%%%%%%%%%%%%%%%%%%%%%%%%%%%%%%%%
\begin{multi}[]%
    % - Identificador
    {Rouché–Frobenius-02-13}
    % - Enunciado
    Dado un sistema lineal de ecuaciones de 3 incógnitas, si su matriz ampliada es equivalente por filas a 
    \[
    \begin{pmatrix}
    1 & 2 & 0 & | & 3 \\
    0 & 1 & 0 & | & 2 \\
    0 & 0 & 0 & | & 0 \\
    0 & 0 & 0 & | & 0
    \end{pmatrix}
    \]
    entonces, se tiene que:
    \item El sistema tiene solución única.
    \item* El sistema tiene infinitas soluciones.
    \item El sistema no tiene solución.
    \item No se puede determinar la naturaleza del sistema.
\end{multi}

%%%%%%%%%%%%%%%%%%%%%%%%%%%%%%%%%%%%%%%%
\begin{multi}[]%
    % - Identificador
    {Rouché–Frobenius-02-14}
    % - Enunciado
    Dado un sistema lineal de ecuaciones de 3 incógnitas, si su matriz ampliada es equivalente por filas a 
    \[
    \begin{pmatrix}
    1 & 2 & 0 & | & 3 \\
    0 & 1 & 0 & | & 2 \\
    0 & 0 & 1 & | & 1 \\
    0 & 0 & 0 & | & 0
    \end{pmatrix}
    \]
    entonces, se tiene que:
    \item* El sistema tiene solución única.
    \item El sistema tiene infinitas soluciones.
    \item El sistema no tiene solución.
    \item No se puede determinar la naturaleza del sistema.
\end{multi}

\end{quiz}

%%%%%%%%%%%%%%%%%%%%%%%%%%%%%%%%%%%%%%%%
\begin{quiz}{Sistema paramétrico}
%%%%%%%%%%%%%%%%%%%%%%%%%%%%%%%%%%%%%%%%

%%%%%%%%%%%%%%%%%%%%%%%%%%%%%%%%%%%%%%%%
\begin{numerical}[tolerance=0.01]%
    % - Identificador
    {SisParametrico-01}
    % - Enunciado
    Dado un sistema lineal de ecuaciones, si su matriz ampliada es equivalente por filas a 
    \[
    \begin{pmatrix}
    1 & 0 & 0 & | & 3 \\
    0 & 1 & 0 & | & 2 \\
    0 & 0 & \beta & | & 1
    \end{pmatrix}
    \]
    entonces, ¿cuál debe ser el valor de $\beta$ para que el sistema no tenga solución?
    \item 0
\end{numerical}

%%%%%%%%%%%%%%%%%%%%%%%%%%%%%%%%%%%%%%%%
\begin{numerical}[tolerance=0.01]%
    % - Identificador
    {SisParametrico-02}
    % - Enunciado
    Dado un sistema lineal de ecuaciones, si su matriz ampliada es equivalente por filas a 
    \[
    \begin{pmatrix}
    1 & 0 & 0 & | & 3 \\
    0 & 1 & 0 & | & 2 \\
    0 & 0 & 4-2\beta & | & 1
    \end{pmatrix}
    \]
    entonces, ¿cuál debe ser el valor de $\beta$ para que el sistema no tenga solución?
    \item 2
\end{numerical}

%%%%%%%%%%%%%%%%%%%%%%%%%%%%%%%%%%%%%%%%
\begin{numerical}[tolerance=0.01]%
    % - Identificador
    {SisParametrico-03}
    % - Enunciado
    Dado un sistema lineal de ecuaciones, si su matriz ampliada es equivalente por filas a 
    \[
    \begin{pmatrix}
    1 & 0 & 0 & | & 4 \\
    0 & 1 & 0 & | & -1 \\
    0 & 0 & \alpha + 2 & | & 3
    \end{pmatrix}
    \]
    entonces, ¿cuál debe ser el valor de $\alpha$ para que el sistema no tenga solución?
    \item -2
\end{numerical}

%%%%%%%%%%%%%%%%%%%%%%%%%%%%%%%%%%%%%%%%
\begin{numerical}[tolerance=0.01]%
    % - Identificador
    {SisParametrico-04}
    % - Enunciado
    Dado un sistema lineal de ecuaciones, si su matriz ampliada es equivalente por filas a 
    \[
    \begin{pmatrix}
    1 & 0 & 0 & | & -2 \\
    0 & 1 & 0 & | & 5 \\
    0 & 0 & 0 & | & 2\beta+5
    \end{pmatrix}
    \]
    entonces, ¿cuál debe ser el valor de $\beta$ para que el sistema tenga solución?
    \item -2.5
\end{numerical}

\begin{numerical}[tolerance=0.01]%
    % - Identificador
    {SisParametrico-05}
    % - Enunciado
    Dado un sistema lineal de ecuaciones, si su matriz ampliada es equivalente por filas a 
    \[
    \begin{pmatrix}
    1 & 4 & 0 & | & 3 \\
    0 & 2 & 1 & | & 2 \\
    0 & 0 & 2\beta-3 & | & 0
    \end{pmatrix}
    \]
    entonces, ¿cuál debe ser el valor de $\beta$ para que el sistema no tenga solución?
    \item 1.5
\end{numerical}

\begin{numerical}[tolerance=0.01]%
    % - Identificador
    {SisParametrico-06}
    % - Enunciado
    Dado un sistema lineal de ecuaciones, si su matriz ampliada es equivalente por filas a 
    \[
    \begin{pmatrix}
    1 & 3 & 0 & | & 1 \\
    0 & 4 & 1 & | & -2 \\
    0 & 0 & 2-4\beta & | & 4
    \end{pmatrix}
    \]
    entonces, ¿cuál debe ser el valor de $\beta$ para que el sistema no tenga solución?
    \item 0.5
\end{numerical}

\begin{numerical}[tolerance=0.01]%
    % - Identificador
    {SisParametrico-07}
    % - Enunciado
    Dado un sistema lineal de ecuaciones, si su matriz ampliada es equivalente por filas a 
    \[
    \begin{pmatrix}
    1 & 0 & 4 & | & 3 \\
    0 & 1 & 2 & | & 5 \\
    0 & 0 & 0 & | & 5\alpha-1
    \end{pmatrix}
    \]
    entonces, ¿cuál debe ser el valor de $\alpha$ para que el sistema tenga solución?
    \item 0.2
\end{numerical}

\begin{numerical}[tolerance=0.01]%
    % - Identificador
    {SisParametrico-08}
    % - Enunciado
    Dado un sistema lineal de ecuaciones, si su matriz ampliada es equivalente por filas a 
    \[
    \begin{pmatrix}
    1 & 5 & 0 & | & 2 \\
    0 & 5 & 1 & | & 3 \\
    0 & 0 & 0 & | & 8+4\alpha
    \end{pmatrix}
    \]
    entonces, ¿cuál debe ser el valor de $\alpha$ para que el sistema tenga solución?
    \item -2
\end{numerical}

\begin{numerical}[tolerance=0.01]%
    % - Identificador
    {SisParametrico-09}
    % - Enunciado
    Dado un sistema lineal de ecuaciones, si su matriz ampliada es equivalente por filas a 
    \[
    \begin{pmatrix}
    1 & 0 & 5 & | & 4 \\
    0 & 1 & 2 & | & -3 \\
    0 & 0 & 6-2\alpha & | & 0
    \end{pmatrix}
    \]
    entonces, ¿cuál debe ser el valor de $\alpha$ para que el sistema no tenga solución?
    \item 3
\end{numerical}

\begin{numerical}[tolerance=0.01]%
    % - Identificador
    {SisParametrico-10}
    % - Enunciado
    Dado un sistema lineal de ecuaciones, si su matriz ampliada es equivalente por filas a 
    \[
    \begin{pmatrix}
    1 & 0 & 0 & | & 3 \\
    0 & 7 & 1 & | & 2 \\
    0 & 0 & 0 & | & 6+2\alpha
    \end{pmatrix}
    \]
    entonces, ¿cuál debe ser el valor de $\alpha$ para que el sistema tenga solución?
    \item -3
\end{numerical}

\begin{numerical}[tolerance=0.01]%
    % - Identificador
    {SisParametrico-11}
    % - Enunciado
    Dado un sistema lineal de ecuaciones, si su matriz ampliada es equivalente por filas a 
    \[
    \begin{pmatrix}
    1 & 5 & 0 & | & 2 \\
    0 & 5 & 1 & | & 3 \\
    0 & 0 & 0 & | & 7+21\alpha
    \end{pmatrix}
    \]
    entonces, ¿cuál debe ser el valor de $\alpha$ para que el sistema tenga solución?
    \item -0.33
\end{numerical}

\begin{numerical}[tolerance=0.01]%
    % - Identificador
    {SisParametrico-12}
    % - Enunciado
    Dado un sistema lineal de ecuaciones, si su matriz ampliada es equivalente por filas a 
    \[
    \begin{pmatrix}
    1 & 5 & 0 & | & 2 \\
    0 & 5 & 1 & | & 3 \\
    0 & 0 & 14 & | & 7+21\alpha
    \end{pmatrix}
    \]
    entonces, ¿cuál debe ser el valor de $\alpha$ para que el sistema tenga solución?
    \item 0.33
\end{numerical}

\begin{numerical}[tolerance=0.01]%
    % - Identificador
    {SisParametrico-13}
    % - Enunciado
    Dado un sistema lineal de ecuaciones, si su matriz ampliada es equivalente por filas a 
    \[
    \begin{pmatrix}
    1 & 5 & 0 & | & 2 \\
    0 & 5 & 1 & | & 3 \\
    0 & 0 & 2 & | & 3+8\alpha
    \end{pmatrix}
    \]
    entonces, ¿cuál debe ser el valor de $\alpha$ para que el sistema tenga solución?
    \item -0.125
\end{numerical}

\begin{numerical}[tolerance=0.01]%
    % - Identificador
    {SisParametrico-14}
    % - Enunciado
    Dado un sistema lineal de ecuaciones, si su matriz ampliada es equivalente por filas a 
    \[ 
    \begin{pmatrix}
    1 & 5 & 0 & | & 2 \\
    0 & 5 & 1 & | & 3 \\
    0 & 0 & 4 & | & 3+4\alpha
    \end{pmatrix}
    \]
    entonces, ¿cuál debe ser el valor de $\alpha$ para que el sistema tenga solución?
    \item 0.25
\end{numerical}
\end{quiz}

\end{document}