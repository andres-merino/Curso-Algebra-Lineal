\documentclass[a4,11pt]{aleph-notas}
% Actualizado en febrero de 2024
% Funciona con TeXLive 2022
% Para obtener solo el pdf, compilar con pdfLaTeX. 
%  latexmk -pdf 01\ Cuestionarios/01\ Matrices.tex -output-directory="01 Cuestionarios"
% Para obtener el xml compilar con XeLaTeX.
% latexmk -xelatex 01\ Cuestionarios/01\ Matrices.tex -output-directory="01 Cuestionarios"

% -- Paquetes adicionales
\usepackage{aleph-moodle}
\moodleregisternewcommands
% Todos los comandos nuevos deben ir luego del comando anterior
\usepackage{aleph-comandos}
\usepackage{listings}


% -- Datos 
\institucion{Escuela de Ciencias Físicas y Matemática}
\carrera{Ciencia de datos / Bioingeniería}
\asignatura{Álgebra Lineal}
\tema{Cuestionario: Valores propios de matrices de $3\times 3$}
\autor{Andrés Merino}
\fecha{Semestre 2024-1}

\logouno[0.14\textwidth]{Logos/logoPUCE_04_ac}
\definecolor{colortext}{HTML}{0030A1}
\definecolor{colordef}{HTML}{0030A1}
\fuente{montserrat}

% -- Otros comandos
\newcommand{\Bin}{\text{Bin}}
\decimalpoint

\begin{document}

\encabezado

\vspace*{-8mm}
%%%%%%%%%%%%%%%%%%%%%%%%%%%%%%%%%%%%%%%%
\section{Indicaciones}
%%%%%%%%%%%%%%%%%%%%%%%%%%%%%%%%%%%%%%%%

Se plantean preguntas para calcular ciertas probabilidades de una variable aleatoria binomial.


%%%%%%%%%%%%%%%%%%%%%%%%%%%%%%%%%%%%%%%%
%%%%%%%%%%%%%%%%%%%%%%%%%%%%%%%%%%%%%%%%
%% No editar nada de aquí en adelante
%%%%%%%%%%%%%%%%%%%%%%%%%%%%%%%%%%%%%%%%
%%%%%%%%%%%%%%%%%%%%%%%%%%%%%%%%%%%%%%%%

Se utilizó la siguiente pregunta base:
\begin{lstlisting}[breaklines]
\begin{numerical}[tolerance=0.01]%
    % - Indentificador
    {Valores propios 3 por 3 - [[id]]}
    % - Enunciado
    Determine los valores propios de la matriz
    \[
    A = \begin{pmatrix}
    [[a1]] & [[a2]] & [[a5]] \\
    [[a3]] & [[a4]] & [[a6]] \\
    [[a7]] & [[a8]] & [[a9]]
    \end{pmatrix}.
    \]
    Escriba en forma decimal, con 2 decimales, el valor propio más grande. En caso de ser un número complejo, tome en cuenta solo la parte real.
    \item[] [[N(re(max(list(Matrix([ [a1,a2,a5],[a3,a4,a6],[a7,a8,a9] ]).eigenvals().keys()), key=lambda x: re(x))),4)]]
\end{numerical}

\end{lstlisting}
\noindent
Con los siguientes parámetros:
\begin{itemize}
	\item $a1 \in \{-2, -1, 0, 1, 2\}$
	\item $a2 \in \{-2, -1, 1, 2\}$
	\item $a3 \in \{-2, -1, 1, 2\}$
	\item $a4 \in \{-2, -1, 0, 1, 2\}$

\end{itemize}
En total, se plantean 25 preguntas.


%%%%%%%%%%%%%%%%%%%%%%%%%%%%%%%%%%%%%%%%
\section{Banco de preguntas}
%%%%%%%%%%%%%%%%%%%%%%%%%%%%%%%%%%%%%%%%

%%%%%%%%%%%%%%%%%%%%%%%%%%%%%%%%%%%%%%%%
\begin{quiz}{Valores propios 3 por 3}
%%%%%%%%%%%%%%%%%%%%%%%%%%%%%%%%%%%%%%%%

%%%%%%%%%%%%%%%%%%%%%%%%%%%%%%%%%%%%%%%%
%%%%%%%%%%%%%%%%%%%%%%%%%%%%%%%%%%%%%%%%
%%%%%%%%%%%%%%%%%%%%%%%%%%%%%%%%%%%%%%%%
\begin{numerical}[tolerance=0.01]%
    % - Indentificador
    {Valores propios 3 por 3 - 1}
    % - Enunciado
    Determine los valores propios de la matriz
    \[
    A = \begin{pmatrix}
    2 & 2 & 0 \\
    -2 & 1 & -1 \\
    0 & 1 & 1
    \end{pmatrix}.
    \]
    Escriba en forma decimal, con 2 decimales, el valor propio más grande. En caso de ser un número complejo, tome en cuenta solo la parte real.
    \item[] 1.397
\end{numerical}

%%%%%%%%%%%%%%%%%%%%%%%%%%%%%%%%%%%%%%%%
\begin{numerical}[tolerance=0.01]%
    % - Indentificador
    {Valores propios 3 por 3 - 2}
    % - Enunciado
    Determine los valores propios de la matriz
    \[
    A = \begin{pmatrix}
    -2 & -1 & 2 \\
    -2 & -1 & -1 \\
    1 & 0 & -2
    \end{pmatrix}.
    \]
    Escriba en forma decimal, con 2 decimales, el valor propio más grande. En caso de ser un número complejo, tome en cuenta solo la parte real.
    \item[] 0.4605
\end{numerical}

%%%%%%%%%%%%%%%%%%%%%%%%%%%%%%%%%%%%%%%%
\begin{numerical}[tolerance=0.01]%
    % - Indentificador
    {Valores propios 3 por 3 - 3}
    % - Enunciado
    Determine los valores propios de la matriz
    \[
    A = \begin{pmatrix}
    1 & 1 & 2 \\
    -1 & 0 & -2 \\
    0 & 2 & -2
    \end{pmatrix}.
    \]
    Escriba en forma decimal, con 2 decimales, el valor propio más grande. En caso de ser un número complejo, tome en cuenta solo la parte real.
    \item[] -0.1424
\end{numerical}

%%%%%%%%%%%%%%%%%%%%%%%%%%%%%%%%%%%%%%%%
\begin{numerical}[tolerance=0.01]%
    % - Indentificador
    {Valores propios 3 por 3 - 4}
    % - Enunciado
    Determine los valores propios de la matriz
    \[
    A = \begin{pmatrix}
    2 & -2 & 2 \\
    -2 & 1 & -1 \\
    -2 & -1 & -2
    \end{pmatrix}.
    \]
    Escriba en forma decimal, con 2 decimales, el valor propio más grande. En caso de ser un número complejo, tome en cuenta solo la parte real.
    \item[] 3.172
\end{numerical}

%%%%%%%%%%%%%%%%%%%%%%%%%%%%%%%%%%%%%%%%
\begin{numerical}[tolerance=0.01]%
    % - Indentificador
    {Valores propios 3 por 3 - 5}
    % - Enunciado
    Determine los valores propios de la matriz
    \[
    A = \begin{pmatrix}
    2 & 2 & 2 \\
    -1 & 0 & 1 \\
    1 & 2 & -2
    \end{pmatrix}.
    \]
    Escriba en forma decimal, con 2 decimales, el valor propio más grande. En caso de ser un número complejo, tome en cuenta solo la parte real.
    \item[] 1.523
\end{numerical}

%%%%%%%%%%%%%%%%%%%%%%%%%%%%%%%%%%%%%%%%
\begin{numerical}[tolerance=0.01]%
    % - Indentificador
    {Valores propios 3 por 3 - 6}
    % - Enunciado
    Determine los valores propios de la matriz
    \[
    A = \begin{pmatrix}
    -2 & -2 & 2 \\
    -1 & 2 & 1 \\
    2 & -1 & -2
    \end{pmatrix}.
    \]
    Escriba en forma decimal, con 2 decimales, el valor propio más grande. En caso de ser un número complejo, tome en cuenta solo la parte real.
    \item[] 2.162
\end{numerical}

%%%%%%%%%%%%%%%%%%%%%%%%%%%%%%%%%%%%%%%%
\begin{numerical}[tolerance=0.01]%
    % - Indentificador
    {Valores propios 3 por 3 - 7}
    % - Enunciado
    Determine los valores propios de la matriz
    \[
    A = \begin{pmatrix}
    -1 & 1 & -1 \\
    2 & 1 & 0 \\
    1 & 1 & 2
    \end{pmatrix}.
    \]
    Escriba en forma decimal, con 2 decimales, el valor propio más grande. En caso de ser un número complejo, tome en cuenta solo la parte real.
    \item[] 1.839
\end{numerical}

%%%%%%%%%%%%%%%%%%%%%%%%%%%%%%%%%%%%%%%%
\begin{numerical}[tolerance=0.01]%
    % - Indentificador
    {Valores propios 3 por 3 - 8}
    % - Enunciado
    Determine los valores propios de la matriz
    \[
    A = \begin{pmatrix}
    1 & 2 & 0 \\
    1 & -2 & -2 \\
    2 & -2 & -1
    \end{pmatrix}.
    \]
    Escriba en forma decimal, con 2 decimales, el valor propio más grande. En caso de ser un número complejo, tome en cuenta solo la parte real.
    \item[] 1.075
\end{numerical}

%%%%%%%%%%%%%%%%%%%%%%%%%%%%%%%%%%%%%%%%
\begin{numerical}[tolerance=0.01]%
    % - Indentificador
    {Valores propios 3 por 3 - 9}
    % - Enunciado
    Determine los valores propios de la matriz
    \[
    A = \begin{pmatrix}
    2 & -2 & 1 \\
    -1 & 2 & 0 \\
    0 & 2 & 1
    \end{pmatrix}.
    \]
    Escriba en forma decimal, con 2 decimales, el valor propio más grande. En caso de ser un número complejo, tome en cuenta solo la parte real.
    \item[] 3.000
\end{numerical}

%%%%%%%%%%%%%%%%%%%%%%%%%%%%%%%%%%%%%%%%
\begin{numerical}[tolerance=0.01]%
    % - Indentificador
    {Valores propios 3 por 3 - 10}
    % - Enunciado
    Determine los valores propios de la matriz
    \[
    A = \begin{pmatrix}
    0 & -1 & -1 \\
    -1 & -1 & -1 \\
    -2 & -2 & 1
    \end{pmatrix}.
    \]
    Escriba en forma decimal, con 2 decimales, el valor propio más grande. En caso de ser un número complejo, tome en cuenta solo la parte real.
    \item[] 2.145
\end{numerical}

%%%%%%%%%%%%%%%%%%%%%%%%%%%%%%%%%%%%%%%%
\begin{numerical}[tolerance=0.01]%
    % - Indentificador
    {Valores propios 3 por 3 - 11}
    % - Enunciado
    Determine los valores propios de la matriz
    \[
    A = \begin{pmatrix}
    -1 & -1 & 0 \\
    -1 & 0 & 1 \\
    -1 & -2 & -2
    \end{pmatrix}.
    \]
    Escriba en forma decimal, con 2 decimales, el valor propio más grande. En caso de ser un número complejo, tome en cuenta solo la parte real.
    \item[] 0.2599
\end{numerical}

%%%%%%%%%%%%%%%%%%%%%%%%%%%%%%%%%%%%%%%%
\begin{numerical}[tolerance=0.01]%
    % - Indentificador
    {Valores propios 3 por 3 - 12}
    % - Enunciado
    Determine los valores propios de la matriz
    \[
    A = \begin{pmatrix}
    -2 & -1 & -2 \\
    -2 & 1 & 0 \\
    -2 & 2 & 2
    \end{pmatrix}.
    \]
    Escriba en forma decimal, con 2 decimales, el valor propio más grande. En caso de ser un número complejo, tome en cuenta solo la parte real.
    \item[] 3.519
\end{numerical}

%%%%%%%%%%%%%%%%%%%%%%%%%%%%%%%%%%%%%%%%
\begin{numerical}[tolerance=0.01]%
    % - Indentificador
    {Valores propios 3 por 3 - 13}
    % - Enunciado
    Determine los valores propios de la matriz
    \[
    A = \begin{pmatrix}
    2 & -1 & 0 \\
    -1 & 2 & -2 \\
    -1 & 2 & 0
    \end{pmatrix}.
    \]
    Escriba en forma decimal, con 2 decimales, el valor propio más grande. En caso de ser un número complejo, tome en cuenta solo la parte real.
    \item[] 2.000
\end{numerical}

%%%%%%%%%%%%%%%%%%%%%%%%%%%%%%%%%%%%%%%%
\begin{numerical}[tolerance=0.01]%
    % - Indentificador
    {Valores propios 3 por 3 - 14}
    % - Enunciado
    Determine los valores propios de la matriz
    \[
    A = \begin{pmatrix}
    2 & -2 & 2 \\
    -1 & 1 & 0 \\
    -1 & -1 & 1
    \end{pmatrix}.
    \]
    Escriba en forma decimal, con 2 decimales, el valor propio más grande. En caso de ser un número complejo, tome en cuenta solo la parte real.
    \item[] 2.696
\end{numerical}

%%%%%%%%%%%%%%%%%%%%%%%%%%%%%%%%%%%%%%%%
\begin{numerical}[tolerance=0.01]%
    % - Indentificador
    {Valores propios 3 por 3 - 15}
    % - Enunciado
    Determine los valores propios de la matriz
    \[
    A = \begin{pmatrix}
    0 & 1 & 2 \\
    2 & 2 & 2 \\
    -2 & 0 & 2
    \end{pmatrix}.
    \]
    Escriba en forma decimal, con 2 decimales, el valor propio más grande. En caso de ser un número complejo, tome en cuenta solo la parte real.
    \item[] 2.000
\end{numerical}

%%%%%%%%%%%%%%%%%%%%%%%%%%%%%%%%%%%%%%%%
\begin{numerical}[tolerance=0.01]%
    % - Indentificador
    {Valores propios 3 por 3 - 16}
    % - Enunciado
    Determine los valores propios de la matriz
    \[
    A = \begin{pmatrix}
    -2 & 1 & -2 \\
    1 & -1 & -1 \\
    0 & 2 & -2
    \end{pmatrix}.
    \]
    Escriba en forma decimal, con 2 decimales, el valor propio más grande. En caso de ser un número complejo, tome en cuenta solo la parte real.
    \item[] -0.9245
\end{numerical}

%%%%%%%%%%%%%%%%%%%%%%%%%%%%%%%%%%%%%%%%
\begin{numerical}[tolerance=0.01]%
    % - Indentificador
    {Valores propios 3 por 3 - 17}
    % - Enunciado
    Determine los valores propios de la matriz
    \[
    A = \begin{pmatrix}
    0 & -2 & 1 \\
    -1 & 0 & 2 \\
    -1 & -1 & -1
    \end{pmatrix}.
    \]
    Escriba en forma decimal, con 2 decimales, el valor propio más grande. En caso de ser un número complejo, tome en cuenta solo la parte real.
    \item[] 1.488
\end{numerical}

%%%%%%%%%%%%%%%%%%%%%%%%%%%%%%%%%%%%%%%%
\begin{numerical}[tolerance=0.01]%
    % - Indentificador
    {Valores propios 3 por 3 - 18}
    % - Enunciado
    Determine los valores propios de la matriz
    \[
    A = \begin{pmatrix}
    1 & -2 & -1 \\
    -2 & 0 & 2 \\
    -1 & -2 & 2
    \end{pmatrix}.
    \]
    Escriba en forma decimal, con 2 decimales, el valor propio más grande. En caso de ser un número complejo, tome en cuenta solo la parte real.
    \item[] 1.947
\end{numerical}

%%%%%%%%%%%%%%%%%%%%%%%%%%%%%%%%%%%%%%%%
\begin{numerical}[tolerance=0.01]%
    % - Indentificador
    {Valores propios 3 por 3 - 19}
    % - Enunciado
    Determine los valores propios de la matriz
    \[
    A = \begin{pmatrix}
    -2 & -1 & 2 \\
    -1 & 2 & 0 \\
    0 & 1 & -1
    \end{pmatrix}.
    \]
    Escriba en forma decimal, con 2 decimales, el valor propio más grande. En caso de ser un número complejo, tome en cuenta solo la parte real.
    \item[] 2.086
\end{numerical}

%%%%%%%%%%%%%%%%%%%%%%%%%%%%%%%%%%%%%%%%
\begin{numerical}[tolerance=0.01]%
    % - Indentificador
    {Valores propios 3 por 3 - 20}
    % - Enunciado
    Determine los valores propios de la matriz
    \[
    A = \begin{pmatrix}
    1 & 1 & -2 \\
    -2 & 0 & 1 \\
    -2 & -1 & 2
    \end{pmatrix}.
    \]
    Escriba en forma decimal, con 2 decimales, el valor propio más grande. En caso de ser un número complejo, tome en cuenta solo la parte real.
    \item[] 2.414
\end{numerical}

%%%%%%%%%%%%%%%%%%%%%%%%%%%%%%%%%%%%%%%%
\begin{numerical}[tolerance=0.01]%
    % - Indentificador
    {Valores propios 3 por 3 - 21}
    % - Enunciado
    Determine los valores propios de la matriz
    \[
    A = \begin{pmatrix}
    -1 & -2 & -2 \\
    1 & 1 & -1 \\
    2 & -2 & 0
    \end{pmatrix}.
    \]
    Escriba en forma decimal, con 2 decimales, el valor propio más grande. En caso de ser un número complejo, tome en cuenta solo la parte real.
    \item[] 2.000
\end{numerical}

%%%%%%%%%%%%%%%%%%%%%%%%%%%%%%%%%%%%%%%%
\begin{numerical}[tolerance=0.01]%
    % - Indentificador
    {Valores propios 3 por 3 - 22}
    % - Enunciado
    Determine los valores propios de la matriz
    \[
    A = \begin{pmatrix}
    0 & 2 & -1 \\
    2 & 1 & 0 \\
    -1 & 0 & -2
    \end{pmatrix}.
    \]
    Escriba en forma decimal, con 2 decimales, el valor propio más grande. En caso de ser un número complejo, tome en cuenta solo la parte real.
    \item[] 2.646
\end{numerical}

%%%%%%%%%%%%%%%%%%%%%%%%%%%%%%%%%%%%%%%%
\begin{numerical}[tolerance=0.01]%
    % - Indentificador
    {Valores propios 3 por 3 - 23}
    % - Enunciado
    Determine los valores propios de la matriz
    \[
    A = \begin{pmatrix}
    1 & -2 & -2 \\
    2 & -1 & 0 \\
    -1 & 0 & -2
    \end{pmatrix}.
    \]
    Escriba en forma decimal, con 2 decimales, el valor propio más grande. En caso de ser un número complejo, tome en cuenta solo la parte real.
    \item[] 0.1573
\end{numerical}

%%%%%%%%%%%%%%%%%%%%%%%%%%%%%%%%%%%%%%%%
\begin{numerical}[tolerance=0.01]%
    % - Indentificador
    {Valores propios 3 por 3 - 24}
    % - Enunciado
    Determine los valores propios de la matriz
    \[
    A = \begin{pmatrix}
    1 & 1 & 1 \\
    -2 & 0 & 0 \\
    1 & -2 & 2
    \end{pmatrix}.
    \]
    Escriba en forma decimal, con 2 decimales, el valor propio más grande. En caso de ser un número complejo, tome en cuenta solo la parte real.
    \item[] 2.913
\end{numerical}

%%%%%%%%%%%%%%%%%%%%%%%%%%%%%%%%%%%%%%%%
\begin{numerical}[tolerance=0.01]%
    % - Indentificador
    {Valores propios 3 por 3 - 25}
    % - Enunciado
    Determine los valores propios de la matriz
    \[
    A = \begin{pmatrix}
    0 & -1 & -1 \\
    1 & -2 & 1 \\
    -1 & 1 & -2
    \end{pmatrix}.
    \]
    Escriba en forma decimal, con 2 decimales, el valor propio más grande. En caso de ser un número complejo, tome en cuenta solo la parte real.
    \item[] 0
\end{numerical}



\end{quiz}




\end{document}