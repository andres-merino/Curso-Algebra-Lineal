\documentclass[a4,11pt]{aleph-notas}
% Actualizado en febrero de 2024
% Funciona con TeXLive 2022
% Para obtener solo el pdf, compilar con pdfLaTeX. 
%  latexmk -pdf 01\ Cuestionarios/01\ Matrices.tex -output-directory="01 Cuestionarios"
% Para obtener el xml compilar con XeLaTeX.
% latexmk -xelatex 01\ Cuestionarios/01\ Matrices.tex -output-directory="01 Cuestionarios"

% -- Paquetes adicionales
\usepackage{aleph-moodle}
\moodleregisternewcommands
% Todos los comandos nuevos deben ir luego del comando anterior
\usepackage{aleph-comandos}
\usepackage{listings}


% -- Datos 
\institucion{Escuela de Ciencias Físicas y Matemática}
\carrera{Ciencia de datos / Bioingeniería}
\asignatura{Álgebra Lineal}
\tema{Cuestionario: Valores propios de matrices de $2\times 2$}
\autor{Andrés Merino}
\fecha{Semestre 2024-1}

\logouno[0.14\textwidth]{Logos/logoPUCE_04_ac}
\definecolor{colortext}{HTML}{0030A1}
\definecolor{colordef}{HTML}{0030A1}
\fuente{montserrat}

% -- Otros comandos
\newcommand{\Bin}{\text{Bin}}
\decimalpoint

\begin{document}

\encabezado

\vspace*{-8mm}
%%%%%%%%%%%%%%%%%%%%%%%%%%%%%%%%%%%%%%%%
\section{Indicaciones}
%%%%%%%%%%%%%%%%%%%%%%%%%%%%%%%%%%%%%%%%

Se plantean preguntas para calcular ciertas probabilidades de una variable aleatoria binomial.


%%%%%%%%%%%%%%%%%%%%%%%%%%%%%%%%%%%%%%%%
%%%%%%%%%%%%%%%%%%%%%%%%%%%%%%%%%%%%%%%%
%% No editar nada de aquí en adelante
%%%%%%%%%%%%%%%%%%%%%%%%%%%%%%%%%%%%%%%%
%%%%%%%%%%%%%%%%%%%%%%%%%%%%%%%%%%%%%%%%

Se utilizó la siguiente pregunta base:
\begin{lstlisting}[breaklines]

\begin{numerical}[tolerance=0.01]%
    % - Indentificador
    {Valores propios 2 por 2 - [[id]]}
    % - Enunciado
    Determine los valores propios de la matriz
    \[
    A = \begin{pmatrix}
    [[a1]] & [[a2]] \\
    [[a3]] & [[a4]]
    \end{pmatrix}.
    \]
    Escriba en forma decimal, con 2 decimales, el valor propio mas grande. En caso de ser un numero complejo, tome en cuenta solo la parte real.
    \item[] [[N(re(max(list(Matrix([ [a1,a2],[a3,a4] ]).eigenvals().keys()), key=lambda x: re(x))),4)]]
\end{numerical}

\end{lstlisting}
\noindent
Con los siguientes parámetros:
\begin{itemize}
	\item $a1 \in \{-2, -1, 0, 1, 2\}$
	\item $a2 \in \{-2, -1, 1, 2\}$
	\item $a3 \in \{-2, -1, 1, 2\}$
	\item $a4 \in \{-2, -1, 0, 1, 2\}$

\end{itemize}
En total, se plantean 25 preguntas.


%%%%%%%%%%%%%%%%%%%%%%%%%%%%%%%%%%%%%%%%
\section{Banco de preguntas}
%%%%%%%%%%%%%%%%%%%%%%%%%%%%%%%%%%%%%%%%

%%%%%%%%%%%%%%%%%%%%%%%%%%%%%%%%%%%%%%%%
\begin{quiz}{Valores propios 2 por 2}
%%%%%%%%%%%%%%%%%%%%%%%%%%%%%%%%%%%%%%%%

%%%%%%%%%%%%%%%%%%%%%%%%%%%%%%%%%%%%%%%%
%%%%%%%%%%%%%%%%%%%%%%%%%%%%%%%%%%%%%%%%
\begin{numerical}[tolerance=0.01]%
    % - Indentificador
    {Valores propios 2 por 2 - 1}
    % - Enunciado
    Determine los valores propios de la matriz
    \[
    A = \begin{pmatrix}
    1 & 1 \\
    1 & 0
    \end{pmatrix}.
    \]
    Escriba en forma decimal, con 2 decimales, el valor propio más grande. En caso de ser un número complejo, tome en cuenta solo la parte real.
    \item[] 1.618
\end{numerical}

%%%%%%%%%%%%%%%%%%%%%%%%%%%%%%%%%%%%%%%%
\begin{numerical}[tolerance=0.01]%
    % - Indentificador
    {Valores propios 2 por 2 - 2}
    % - Enunciado
    Determine los valores propios de la matriz
    \[
    A = \begin{pmatrix}
    -2 & -2 \\
    2 & -1
    \end{pmatrix}.
    \]
    Escriba en forma decimal, con 2 decimales, el valor propio más grande. En caso de ser un número complejo, tome en cuenta solo la parte real.
    \item[] -1.500
\end{numerical}

%%%%%%%%%%%%%%%%%%%%%%%%%%%%%%%%%%%%%%%%
\begin{numerical}[tolerance=0.01]%
    % - Indentificador
    {Valores propios 2 por 2 - 3}
    % - Enunciado
    Determine los valores propios de la matriz
    \[
    A = \begin{pmatrix}
    0 & 1 \\
    2 & 2
    \end{pmatrix}.
    \]
    Escriba en forma decimal, con 2 decimales, el valor propio más grande. En caso de ser un número complejo, tome en cuenta solo la parte real.
    \item[] 2.732
\end{numerical}

%%%%%%%%%%%%%%%%%%%%%%%%%%%%%%%%%%%%%%%%
\begin{numerical}[tolerance=0.01]%
    % - Indentificador
    {Valores propios 2 por 2 - 4}
    % - Enunciado
    Determine los valores propios de la matriz
    \[
    A = \begin{pmatrix}
    1 & -2 \\
    -1 & 0
    \end{pmatrix}.
    \]
    Escriba en forma decimal, con 2 decimales, el valor propio más grande. En caso de ser un número complejo, tome en cuenta solo la parte real.
    \item[] 2.000
\end{numerical}

%%%%%%%%%%%%%%%%%%%%%%%%%%%%%%%%%%%%%%%%
\begin{numerical}[tolerance=0.01]%
    % - Indentificador
    {Valores propios 2 por 2 - 5}
    % - Enunciado
    Determine los valores propios de la matriz
    \[
    A = \begin{pmatrix}
    1 & 1 \\
    2 & -2
    \end{pmatrix}.
    \]
    Escriba en forma decimal, con 2 decimales, el valor propio más grande. En caso de ser un número complejo, tome en cuenta solo la parte real.
    \item[] 1.562
\end{numerical}

%%%%%%%%%%%%%%%%%%%%%%%%%%%%%%%%%%%%%%%%
\begin{numerical}[tolerance=0.01]%
    % - Indentificador
    {Valores propios 2 por 2 - 6}
    % - Enunciado
    Determine los valores propios de la matriz
    \[
    A = \begin{pmatrix}
    -2 & -2 \\
    -1 & 0
    \end{pmatrix}.
    \]
    Escriba en forma decimal, con 2 decimales, el valor propio más grande. En caso de ser un número complejo, tome en cuenta solo la parte real.
    \item[] 0.7320
\end{numerical}

%%%%%%%%%%%%%%%%%%%%%%%%%%%%%%%%%%%%%%%%
\begin{numerical}[tolerance=0.01]%
    % - Indentificador
    {Valores propios 2 por 2 - 7}
    % - Enunciado
    Determine los valores propios de la matriz
    \[
    A = \begin{pmatrix}
    -1 & -1 \\
    -1 & -2
    \end{pmatrix}.
    \]
    Escriba en forma decimal, con 2 decimales, el valor propio más grande. En caso de ser un número complejo, tome en cuenta solo la parte real.
    \item[] -0.3820
\end{numerical}

%%%%%%%%%%%%%%%%%%%%%%%%%%%%%%%%%%%%%%%%
\begin{numerical}[tolerance=0.01]%
    % - Indentificador
    {Valores propios 2 por 2 - 8}
    % - Enunciado
    Determine los valores propios de la matriz
    \[
    A = \begin{pmatrix}
    0 & 2 \\
    2 & -1
    \end{pmatrix}.
    \]
    Escriba en forma decimal, con 2 decimales, el valor propio más grande. En caso de ser un número complejo, tome en cuenta solo la parte real.
    \item[] 1.562
\end{numerical}

%%%%%%%%%%%%%%%%%%%%%%%%%%%%%%%%%%%%%%%%
\begin{numerical}[tolerance=0.01]%
    % - Indentificador
    {Valores propios 2 por 2 - 9}
    % - Enunciado
    Determine los valores propios de la matriz
    \[
    A = \begin{pmatrix}
    1 & -2 \\
    1 & -1
    \end{pmatrix}.
    \]
    Escriba en forma decimal, con 2 decimales, el valor propio más grande. En caso de ser un número complejo, tome en cuenta solo la parte real.
    \item[] 0
\end{numerical}

%%%%%%%%%%%%%%%%%%%%%%%%%%%%%%%%%%%%%%%%
\begin{numerical}[tolerance=0.01]%
    % - Indentificador
    {Valores propios 2 por 2 - 10}
    % - Enunciado
    Determine los valores propios de la matriz
    \[
    A = \begin{pmatrix}
    -1 & 2 \\
    -2 & 0
    \end{pmatrix}.
    \]
    Escriba en forma decimal, con 2 decimales, el valor propio más grande. En caso de ser un número complejo, tome en cuenta solo la parte real.
    \item[] -0.5000
\end{numerical}

%%%%%%%%%%%%%%%%%%%%%%%%%%%%%%%%%%%%%%%%
\begin{numerical}[tolerance=0.01]%
    % - Indentificador
    {Valores propios 2 por 2 - 11}
    % - Enunciado
    Determine los valores propios de la matriz
    \[
    A = \begin{pmatrix}
    2 & -2 \\
    1 & 2
    \end{pmatrix}.
    \]
    Escriba en forma decimal, con 2 decimales, el valor propio más grande. En caso de ser un número complejo, tome en cuenta solo la parte real.
    \item[] 2.000
\end{numerical}

%%%%%%%%%%%%%%%%%%%%%%%%%%%%%%%%%%%%%%%%
\begin{numerical}[tolerance=0.01]%
    % - Indentificador
    {Valores propios 2 por 2 - 12}
    % - Enunciado
    Determine los valores propios de la matriz
    \[
    A = \begin{pmatrix}
    -1 & -2 \\
    -2 & 0
    \end{pmatrix}.
    \]
    Escriba en forma decimal, con 2 decimales, el valor propio más grande. En caso de ser un número complejo, tome en cuenta solo la parte real.
    \item[] 1.562
\end{numerical}

%%%%%%%%%%%%%%%%%%%%%%%%%%%%%%%%%%%%%%%%
\begin{numerical}[tolerance=0.01]%
    % - Indentificador
    {Valores propios 2 por 2 - 13}
    % - Enunciado
    Determine los valores propios de la matriz
    \[
    A = \begin{pmatrix}
    -2 & -2 \\
    2 & 0
    \end{pmatrix}.
    \]
    Escriba en forma decimal, con 2 decimales, el valor propio más grande. En caso de ser un número complejo, tome en cuenta solo la parte real.
    \item[] -1.000
\end{numerical}

%%%%%%%%%%%%%%%%%%%%%%%%%%%%%%%%%%%%%%%%
\begin{numerical}[tolerance=0.01]%
    % - Indentificador
    {Valores propios 2 por 2 - 14}
    % - Enunciado
    Determine los valores propios de la matriz
    \[
    A = \begin{pmatrix}
    1 & -1 \\
    -1 & -1
    \end{pmatrix}.
    \]
    Escriba en forma decimal, con 2 decimales, el valor propio más grande. En caso de ser un número complejo, tome en cuenta solo la parte real.
    \item[] 1.414
\end{numerical}

%%%%%%%%%%%%%%%%%%%%%%%%%%%%%%%%%%%%%%%%
\begin{numerical}[tolerance=0.01]%
    % - Indentificador
    {Valores propios 2 por 2 - 15}
    % - Enunciado
    Determine los valores propios de la matriz
    \[
    A = \begin{pmatrix}
    1 & -2 \\
    1 & -2
    \end{pmatrix}.
    \]
    Escriba en forma decimal, con 2 decimales, el valor propio más grande. En caso de ser un número complejo, tome en cuenta solo la parte real.
    \item[] 0
\end{numerical}

%%%%%%%%%%%%%%%%%%%%%%%%%%%%%%%%%%%%%%%%
\begin{numerical}[tolerance=0.01]%
    % - Indentificador
    {Valores propios 2 por 2 - 16}
    % - Enunciado
    Determine los valores propios de la matriz
    \[
    A = \begin{pmatrix}
    0 & -2 \\
    -1 & 0
    \end{pmatrix}.
    \]
    Escriba en forma decimal, con 2 decimales, el valor propio más grande. En caso de ser un número complejo, tome en cuenta solo la parte real.
    \item[] 1.414
\end{numerical}

%%%%%%%%%%%%%%%%%%%%%%%%%%%%%%%%%%%%%%%%
\begin{numerical}[tolerance=0.01]%
    % - Indentificador
    {Valores propios 2 por 2 - 17}
    % - Enunciado
    Determine los valores propios de la matriz
    \[
    A = \begin{pmatrix}
    -2 & -1 \\
    2 & 1
    \end{pmatrix}.
    \]
    Escriba en forma decimal, con 2 decimales, el valor propio más grande. En caso de ser un número complejo, tome en cuenta solo la parte real.
    \item[] 0
\end{numerical}

%%%%%%%%%%%%%%%%%%%%%%%%%%%%%%%%%%%%%%%%
\begin{numerical}[tolerance=0.01]%
    % - Indentificador
    {Valores propios 2 por 2 - 18}
    % - Enunciado
    Determine los valores propios de la matriz
    \[
    A = \begin{pmatrix}
    -1 & 1 \\
    -1 & 0
    \end{pmatrix}.
    \]
    Escriba en forma decimal, con 2 decimales, el valor propio más grande. En caso de ser un número complejo, tome en cuenta solo la parte real.
    \item[] -0.5000
\end{numerical}

%%%%%%%%%%%%%%%%%%%%%%%%%%%%%%%%%%%%%%%%
\begin{numerical}[tolerance=0.01]%
    % - Indentificador
    {Valores propios 2 por 2 - 19}
    % - Enunciado
    Determine los valores propios de la matriz
    \[
    A = \begin{pmatrix}
    2 & 2 \\
    -2 & -1
    \end{pmatrix}.
    \]
    Escriba en forma decimal, con 2 decimales, el valor propio más grande. En caso de ser un número complejo, tome en cuenta solo la parte real.
    \item[] 0.5000
\end{numerical}

%%%%%%%%%%%%%%%%%%%%%%%%%%%%%%%%%%%%%%%%
\begin{numerical}[tolerance=0.01]%
    % - Indentificador
    {Valores propios 2 por 2 - 20}
    % - Enunciado
    Determine los valores propios de la matriz
    \[
    A = \begin{pmatrix}
    0 & -1 \\
    -2 & 2
    \end{pmatrix}.
    \]
    Escriba en forma decimal, con 2 decimales, el valor propio más grande. En caso de ser un número complejo, tome en cuenta solo la parte real.
    \item[] 2.732
\end{numerical}

%%%%%%%%%%%%%%%%%%%%%%%%%%%%%%%%%%%%%%%%
\begin{numerical}[tolerance=0.01]%
    % - Indentificador
    {Valores propios 2 por 2 - 21}
    % - Enunciado
    Determine los valores propios de la matriz
    \[
    A = \begin{pmatrix}
    -2 & -1 \\
    -2 & 0
    \end{pmatrix}.
    \]
    Escriba en forma decimal, con 2 decimales, el valor propio más grande. En caso de ser un número complejo, tome en cuenta solo la parte real.
    \item[] 0.7320
\end{numerical}

%%%%%%%%%%%%%%%%%%%%%%%%%%%%%%%%%%%%%%%%
\begin{numerical}[tolerance=0.01]%
    % - Indentificador
    {Valores propios 2 por 2 - 22}
    % - Enunciado
    Determine los valores propios de la matriz
    \[
    A = \begin{pmatrix}
    0 & 1 \\
    2 & -2
    \end{pmatrix}.
    \]
    Escriba en forma decimal, con 2 decimales, el valor propio más grande. En caso de ser un número complejo, tome en cuenta solo la parte real.
    \item[] 0.7320
\end{numerical}

%%%%%%%%%%%%%%%%%%%%%%%%%%%%%%%%%%%%%%%%
\begin{numerical}[tolerance=0.01]%
    % - Indentificador
    {Valores propios 2 por 2 - 23}
    % - Enunciado
    Determine los valores propios de la matriz
    \[
    A = \begin{pmatrix}
    -2 & 2 \\
    1 & -1
    \end{pmatrix}.
    \]
    Escriba en forma decimal, con 2 decimales, el valor propio más grande. En caso de ser un número complejo, tome en cuenta solo la parte real.
    \item[] 0
\end{numerical}

%%%%%%%%%%%%%%%%%%%%%%%%%%%%%%%%%%%%%%%%
\begin{numerical}[tolerance=0.01]%
    % - Indentificador
    {Valores propios 2 por 2 - 24}
    % - Enunciado
    Determine los valores propios de la matriz
    \[
    A = \begin{pmatrix}
    1 & 2 \\
    -1 & 1
    \end{pmatrix}.
    \]
    Escriba en forma decimal, con 2 decimales, el valor propio más grande. En caso de ser un número complejo, tome en cuenta solo la parte real.
    \item[] 1.000
\end{numerical}

%%%%%%%%%%%%%%%%%%%%%%%%%%%%%%%%%%%%%%%%
\begin{numerical}[tolerance=0.01]%
    % - Indentificador
    {Valores propios 2 por 2 - 25}
    % - Enunciado
    Determine los valores propios de la matriz
    \[
    A = \begin{pmatrix}
    0 & -1 \\
    -2 & -1
    \end{pmatrix}.
    \]
    Escriba en forma decimal, con 2 decimales, el valor propio más grande. En caso de ser un número complejo, tome en cuenta solo la parte real.
    \item[] 1.000
\end{numerical}




\end{quiz}




\end{document}