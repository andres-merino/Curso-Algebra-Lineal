\documentclass[a4,11pt]{aleph-notas}

% -- Paquetes adicionales
\usepackage{enumitem}
\usepackage{aleph-comandos}

% -- Datos  
\institucion{Escuela de Ciencias Físicas y Matemática}
\carrera{Ciencia de datos}
\asignatura{Álgebra lineal}
\tema{Resumen}
\autor[A. Merino]{Andrés Merino}
\fecha{Semestre 2024-1}

\logouno[0.14\textwidth]{Logos/logoPUCE_04_ac}
\definecolor{colortext}{HTML}{0030A1}
\definecolor{colordef}{HTML}{0030A1}
\fuente{montserrat}


% -- Comandos adicionales
\setlist[enumerate]{label=\roman*.}
% % Caga del paquete
\usepackage{listings}

% Definición de color y ambiente
\definecolor{colcod}{RGB}{174,218,255}
\newtcolorbox{pycodigo}
    {icono=\faKeyboardO,color=colcod,
     postit,top=-2mm,bottom=-2mm}

% Estilo para python
\lstloadlanguages{Python}
% Configuración de estilo
\lstset{
    language=Python,
    basicstyle=\scriptsize\ttfamily,
    numberstyle=\color{gray},
    stringstyle=\color[HTML]{933797},
    commentstyle=\color[HTML]{228B22}\ttfamily,
    emph={[2]from,import,pass,return}, emphstyle={[2]\color[HTML]{DD52F0}},
    emph={[3]range}, emphstyle={[3]\color[HTML]{D17032}},
    emph={[4]for,in,def}, emphstyle={[4]\color{blue}},
    showstringspaces=false,
    breaklines=true,
    prebreak=\mbox{{\color{gray}\tiny$\searrow$}},
    numbers=left,
    xleftmargin=25pt,
    inputencoding=utf8,
    extendedchars=true,
    literate=   {á}{{\'a}}1
                {é}{{\'e}}1 
                {í}{{\'i}}1 
                {ó}{{\'o}}1 
                {ú}{{\'u}}1 
                {ñ}{{\~n}}1,
}

\begin{document}

\encabezado

\vspace*{-10mm}
\tableofcontents

%%%%%%%%%%%%%%%%%%%%%%%%%%%%%%%%%%%%%%
\section{Matrices}
%%%%%%%%%%%%%%%%%%%%%%%%%%%%%%%%%%%%%%

En este curso, tomaremos:
\begin{itemize}
    \item
        $n,m\in \N$ con $n>0$ y $m>0$;
    \item 
        $I = \{ 1, 2, \ldots, m\}$ y $J =  \{1, 2 , \ldots, n \}$; y
    \item 
        $\K$ un cuerpo o campo (es decir, cualquier conjunto que cumpla con los axiomas de cuerpo), puede ser $\Q$, $\R$ o $\C$.
\end{itemize}

\begin{defi}[Matriz]
    Una matriz sobre un cuerpo $\mathbb{K}$ es una función 
    \[
        \funcion{A}{I\times J}{\mathbb{K}}
        {(i,j)}{A(i,j)=a_{ij},}
    \]
    la cual se representa por
    \[
        A = 
        \begin{pmatrix}
        a_{11} & a_{12} & \cdots & a_{1n} \\ 
        a_{21} & a_{22} & \cdots & a_{2n} \\ 
        \vdots & \vdots & \ddots & \vdots\\ 
        a_{m1} & a_{m2} & \cdots & a_{mn}
        \end{pmatrix}.
    \]
    Con esto, se dice que $A$ es de orden $m \times n$ y que tiene $m$ filas y $n$ columnas. Al conjunto de todas las matrices de orden $m \times n$ sobre el campo $\mathbb{K}$ se denota por $\Mat[\K]{m}{n}$.
\end{defi}

% Una manera para trabajar con matrices en Python, es utilizando la librería \texttt{sympy}, con la siguiente línea de comandos:

% \begin{pycodigo}
% \begin{lstlisting}[language=Python]
% from sympy import *
% \end{lstlisting}
% \end{pycodigo}

% Con esto, podemos almacenar y visualizar una matriz de la siguiente manera:

% \begin{pycodigo}
% \begin{lstlisting}[language=Python]
% A = Matrix([[1,2,3], [4,5,6],[-1,0,2]])
% A
% \end{lstlisting}
% \end{pycodigo}


\begin{advertencia}
    Dada una matriz $A$, esta se la representa también por $A=(a_{ij})$.
\end{advertencia}

\begin{advertencia}
    Otras notaciones usuales para el conjunto de la matrices, que se puede encontrar en la literatura, son $\M_{m\times n}$, $M_{m\times n}$, $\M_{mn}$ o $M_{mn}$.
\end{advertencia}

\begin{defi}[Filas y columnas]
    Sean $A\in\Mat[\K]{m}{n}$, $i\in I$ y $j\in J$. A la matriz
    \[
        \begin{pmatrix}
        a_{1j} \\ 
        a_{2j} \\ 
        \vdots \\ 
        a_{mj} 
        \end{pmatrix}
    \]
    se la llama la $j$-ésima columna de $A$ y a la matriz
    \[
        \begin{pmatrix}
        a_{i1}& a_{i2}&\cdots& a_{in}
        \end{pmatrix}
    \]
    se la llama la $i$-ésima fila de $A$.
\end{defi}

\begin{prop}[Igualdad de matrices]
    Sean $A \in \Mat[\K]{m}{n}$ y $B \in \Mat[\K]{p}{q}$. Se dice que las $A$ y $B$ son iguales, y se representa por $A=B$, si:
    \begin{itemize}
    \item 
        $m=p$ y $n=q$; y
    \item 
        $a_{ij}=b_{ij}$ para todo $i\in I$ y $j \in J$.
    \end{itemize}
\end{prop}

\begin{defi}[Vectores]
    El conjunto $\K^n$ es:
    \[
        \K^n=\underbrace{\K\times\K\times\cdots\times\K}_{n\text{ veces}},
    \]
    es decir,
    \[
        \K^n=\{(a_1,a_2,\ldots,a_n):a_i\in\K\text{ para todo }i \in J\}.
    \]
\end{defi}

\begin{advertencia}
    Por notación, si $a\in\K^n$, se asumirá $a=(a_1,a_2,\ldots,a_n)$.
\end{advertencia}

\begin{advertencia}
    Podemos identificar cada elemento de $\K^n$ con una matriz de $\Mat[\K]{n}{1}$ de la siguiente manera: si
    \[
        a = (a_1,a_2,\ldots,a_n) \in \K^n,
    \]
    entonces, visto como matriz es
    \[
        a = 
        \begin{pmatrix}
            a_1\\a_2\\
            \vdots\\a_n
        \end{pmatrix}
        \in \Mat[\K]{n}{1}.
    \]
\end{advertencia}

%%%%%%%%%%%%%%%%%%%%%%%%%%%%%%%%%%%%%%%%
\subsection{Operaciones de matrices}
%%%%%%%%%%%%%%%%%%%%%%%%%%%%%%%%%%%%%%

\begin{defi}[Suma de matrices]
    Sean $A, B \in \Mat[\K]{m}{n}$. La suma de las matrices $A$ y $B$ es la matriz $C\in \Mat[\K]{m}{n}$ tal que:
    \[ 
        c_{ij} = a_{ij} + b_{ij}
    \]
    para todo $i \in I$ y $j \in J$. A esta matriz se la denota por $A+B$.
\end{defi}

\begin{advertencia}
    Con esto, tenemos que $(a_{ij})+(b_{ij})=(a_{ij} + b_{ij})$.
\end{advertencia}

\begin{defi}[Multiplicación por un escalar]
    Sea $A\in \Mat[\K]{m}{n}$ y $\alpha \in \mathbb{K}$. El producto del escalar $\alpha$ por la matriz $A$ es la matriz $B\in \Mat[\K]{m}{n}$ tal que:
    \[ 
        b_{ij} = \alpha a_{ij}
    \]
    para todo $i \in I$ y $j \in J$. A esta matriz se la denota por $\alpha A$.
\end{defi}

\begin{advertencia}
    Con esto, tenemos que $\alpha (a_{ij}) =(\alpha a_{ij})$.
\end{advertencia}

\begin{defi}[Transpuesta de una matriz]
    Sea $A \in \Mat[\K]{m}{n}$. La matriz transpuesta de $A$ es la matriz $B\in \Mat[\K]{n}{m}$ tal que: 
    \[
        (b_{ij}) = (a_{ji})
    \]
    para todo $i\in I$ y $j \in J$. A esta matriz se la denota por $A^\intercal$.
\end{defi}

\begin{advertencia}
    Con esto, tenemos que si $A\in \Mat[\K]{m}{n}$, entonces $A^\intercal\in \Mat[\K]{n}{m}$ y $(a_{ij})^\intercal =(a_{ji})$.
\end{advertencia}

\begin{teo}
    Sean $A,B,C\in\Mat[\K]{m}{n}$. Se tiene que
    \begin{itemize}
    \item 
        $A + B = B + A$.
    \item 
        $A + (B+C) = (A + B)+C$.
    \item 
        Existe una única matriz $0$ de orden $m \times n$ tal que 
        \[ 
            A + 0 =  A 
        \]
        para cualquier matriz $A$ de orden $m \times n$. La matriz $0$ se denomina neutro aditivo de orden $m \times n$, también llamada la matriz nula.
    \item 
        Para cada matriz $A$ de orden $m \times n$, existe una única matriz, denotada por $-A$, de orden $m \times n$ tal que:
        \[
            A + (-A) = 0.
        \]
        A la matriz $-A$ se la denomina el inverso aditivo $A$.
    \end{itemize}
\end{teo}

\begin{teo}
    Sean $\alpha,\beta \in \K$ y $A,B\in\Mat[\K]{m}{n}$. Se tiene que
    \begin{itemize}
    \item 
        $\alpha(\beta A) = (\alpha\beta)A$;
    \item 
        $(\alpha+\beta)A = \alpha A + \beta A$; y
    \item 
        $\alpha(A+B) = \alpha A + \alpha B$.
    \end{itemize}
\end{teo}


%%%%%%%%%%%%%%%%%%%%%%%%%%%%%%%%%%%%%%
\subsubsection{Multiplicación de matrices}
%%%%%%%%%%%%%%%%%%%%%%%%%%%%%%%%%%%%%%

En esta sección consideramos  $p,q\in \N$ con $p>0$ y $q>0$.


% \begin{defi}[Producto punto]
%     Sean $a, b \in \K^n$ donde, se define el producto punto o el producto interno de $a$ y $b$ por
%     \[ 
%         a \cdot b = \sum_{j=1}^n a_j b_j
%         = a_1 b_1 + a_2 b_2 + \cdots + a_n b_n.
%     \]
% \end{defi}

\begin{defi}[Multiplicación de matrices]
    Sean $A \in \Mat[\K]{m}{n}$ y $B \in \Mat[\K]{n}{p}$, el producto de las matrices $A$ y $B$ es la matriz $C \in \Mat[\K]{m}{p}$ tal que
    \begin{align*}
        c_{ij} & = a_{i1}b_{1j} + a_{i2} b_{2j} + \cdots + a_{1p}b_{pj} \\
               & = \sum_{k=1}^p a_{ik} b_{kj}
    \end{align*}
    para todo $i \in I$ y $j \in J$. A esta matriz se la denota por $AB$.
\end{defi}



\begin{teo}
    Sean $A, B\in \mathbb{K}^{m \times n}$, $C,D\in \mathbb{K}^{n \times p}$, $E\in \mathbb{K}^{p \times q}$ y $\alpha \in\K$. Se tiene que
    \begin{itemize}
    \item 
        $A(DE) = (AD)E$;
    \item 
        $A(C+D) = AC + AD$; 
    \item 
        $(A+B)C = AC + BC$; 
    
    \item 
        $A(\alpha C) = \alpha (AC) = (\alpha A) C$.
    \end{itemize}
\end{teo}

\begin{advertencia}
    En general, $AB\neq BA$.
\end{advertencia}

\begin{defi}[Matriz cuadrada]
    Si $A$ es un elemento de $\K^{n \times n}$, se dice que $A$ es una matriz cuadrada de orden $n$. 
\end{defi}

\begin{defi}[Matriz identidad de orden $n$]
    Se define la matriz identidad de orden $n$ al elemento $A$ de $\K^{n \times n}$ tal que
    \[
        a_{ij} =
            \begin{cases}
            1 & \text{si } i=j, \\
            0 & \text{si } i \neq j,
            \end{cases}
    \]
    para todo $i,j\in I$. A esta matriz se la denota por $I_n$.
\end{defi} 

\begin{prop}
    Sea $A\in\Mat[\K]{m}{n}$ se tiene que
    \[ 
        I_m A = A I_n = A.
    \]
\end{prop}

\begin{defi}
    Sean $A \in \mathbb{K}^{n \times n}$ y $r\in\N$. Se define $A$ a la potencia $r$ por
    \begin{itemize}
    \item 
        $A^0 = I_n$; y
    \item 
        $A^{r+1} = A^rA$.
    \end{itemize}
\end{defi}

\begin{teo}
    Sean $A \in \mathbb{K}^{n \times n}$ y $r,s\in\N$. Se tiene que
    \begin{itemize}
    \item 
        $A^r A^s = A^{r+s}$; y 
    \item 
        $(A^r)^s = A^{rs}$.
    \end{itemize}
\end{teo}

\begin{advertencia}
    Note que en general $(AB)^r \neq A^r B^r$.
\end{advertencia}


\begin{teo}
    Sean $\alpha \in \mathbb{K}$, $A,B\in\mathbb{K}^{m \times n}$ y  $C\in\mathbb{K}^{n \times p}$. Se tiene que
    \begin{itemize}
    \item 
        $\left(A^\intercal\right)^\intercal = A$;
    \item 
        $(A+B)^\intercal = A^\intercal + B^\intercal$;
    \item 
        $(AC)^\intercal = C^\intercal A^\intercal$;
    \item 
        $(\alpha A)^\intercal = \alpha A^\intercal$.
    \end{itemize}
\end{teo}

%%%%%%%%%%%%%%%%%%%%%%%%%%%%%%%%%%%%%%
\subsection{Tipos de matrices}
%%%%%%%%%%%%%%%%%%%%%%%%%%%%%%%%%%%%%%

\begin{defi}[Matriz diagonal]
    Sea $A\in \mathbb{K}^{n \times n}$. Se dice que $A$ es una matriz diagonal si verifica que:
    \[ 
        a_{ij} = 0
    \]
    para todo $i\in I$ y $j\in J$ con $i\neq j$.
\end{defi}

\begin{defi}[Matriz escalar]
    Sea $A\in \mathbb{K}^{n \times n}$. Se dice que $A$ es una matriz escalar si es una matriz diagonal que verifica:
    \[ 
        a_{ii} = \alpha
    \]
    para todo $i\in I$, con $\alpha\in\K$.
\end{defi}

\begin{defi}[Matriz simétrica]
    Sea $A\in \mathbb{K}^{n \times n}$. Se dice que $A$ es simétrica si $A^\intercal = A$.
\end{defi}

\begin{defi}[Matriz antisimétrica]
    Sea $A\in \mathbb{K}^{n \times n}$. Se dice que $A$ es simétrica si $A^\intercal = -A$.
\end{defi}

\begin{defi}[Matriz triangular]
    Sea $A\in \mathbb{K}^{n \times n}$. Se dice que $A$ es
    \begin{itemize}
    \item 
        una matriz triangular superior si $a_{ij} = 0 $ para todo $i>j$ con $i,j \in J$; y
    \item 
        una matriz triangular inferior si $a_{ij} = 0 $ para todo $i<j$ con $i,j \in J$.
    \end{itemize}
\end{defi}

\begin{defi}[Matriz nilpotente]
    Sea $A\in \mathbb{K}^{n \times n}$ y $r \in \N$. Se dice que $A$ es nilpotente de orden $r$ si $r$ es el menor entero positivo tal que
    \[ 
        A^r = 0.
    \]
\end{defi}

\begin{defi}[Matriz idempotente]
    Sea $A\in \mathbb{K}^{n \times n}$. Se dice que $A$ es una matriz idempotente si $A^2 = A$.
\end{defi}

%%%%%%%%%%%%%%%%%%%%%%%%%%%%%%%%%%%%%%
\section{Soluciones de sistemas de ecuaciones lineales}
%%%%%%%%%%%%%%%%%%%%%%%%%%%%%%%%%%%%%%
\begin{defi}[Matriz aumentada]
    Dado un sistema de ecuaciones lineales de $m$ ecuaciones lineales en las cuales figuran $n$ incógnitas:
    \begin{align*}
        a_{11}x_1 + a_{12}x_2 + \cdots + a_{1n}x_n &= b_1,\\
        a_{21}x_1 + a_{22}x_2 + \cdots + a_{2n}x_n &= b_2,\\
            & \hspace{2mm} \vdots\\
        a_{m1}x_1 + a_{m2}x_2 + \cdots + a_{mn}x_n &= b_m.
    \end{align*}
    donde $a_{ij}\in\R$ y $b_i\in \R$ con $i\in I$ y $j\in J$. A la matriz
    \[
        A=\begin{pmatrix}
        a_{11} & a_{12} & \cdots & a_{1n}\\
        a_{21} & a_{22} & \cdots & a_{2n}\\
        \vdots & \vdots & \ddots & \vdots\\
        a_{m1} & a_{m2} & \cdots & a_{mn}
        \end{pmatrix}
    \]
    se la llama matriz de coeficientes del sistema y a 
    \[
        b=\begin{pmatrix}
        b_1 \\ b_2 \\ \vdots \\ b_m
        \end{pmatrix}
        \texty
        x=\begin{pmatrix}
        x_1 \\ x_2 \\ \vdots \\ x_n
        \end{pmatrix}
    \]
    se las llama columnas de constantes y de incógnitas, respectivamente.
\end{defi}

\begin{advertencia}
    Bajo estas definiciones, dado un sistema de ecuaciones lineales, se dice que
    \[
        Ax=b
    \]
    es la representación matricial del sistema de ecuaciones.
\end{advertencia}

\begin{defi}[Matriz ampliada]
    Sean $A\in\Mat[\K]{m}{n}$ y $B\in\Mat[\K]{m}{p}$, se define la matriz ampliada de $A$ y $B$ al elemento de $\Mat[\K]{m}{(m+p)}$ dado por:
    \[
        \begin{pmatrix}
        a_{11}&a_{12}&\cdots&a_{1n}&|&b_{11}&b_{12}&\cdots&b_{1p}\\
        a_{21}&a_{22}&\cdots&a_{2n}&|&b_{21}&b_{22}&\cdots&b_{2p}\\
        \vdots&\vdots&\ddots&\vdots&|&\vdots&\vdots&\ddots&\vdots\\
        a_{m1}&a_{m2}&\cdots&a_{mn}&|&b_{m1}&b_{m2}&\cdots&b_{mp}
        \end{pmatrix}
    \]
    y se la denota por $(A|B)$.
\end{defi}

\begin{defi}
    Dado el sistema de ecuaciones lineales en forma matricial
    \[
        Ax=b
    \]
    con $A\in \Mat[\K]{m}{n}$ y $b\in\K^m$, se dice que 
    \[
        (A|b)
    \]
    es la matriz ampliada asociada al sistema.
\end{defi}

\begin{defi}[Matriz escalonada reducida por filas]
    Se dice que una matriz está en forma escalonada reducida por filas cuando satisface las siguientes propiedades:
    \begin{enumerate}
    \item 
        Todas las filas que constan de ceros, si las hay, están en la parte inferior de la matriz. 
    \item 
        La primera entrada distinta de cero de la fila, al leer de izquierda a derecha, es un $1$. Esta entrada se denomina entrada principal o uno principal de su fila. 
    \item 
        Para cada fila que no consta sólo de ceros, el uno principal aparece a la derecha y abajo de cualquier uno principal en las filas que le preceden. 
    \item 
        Si una columna contiene un uno principal, el resto de las entradas de dicha columna son iguales a cero. 
    \end{enumerate}
    Se dice que una matriz está en forma escalonada por filas si satisface las primeras tres propiedades . 
\end{defi}

\begin{defi}[Operaciones elementales de fila]
    Dada una matriz $A\in\K^{m \times n}$, una operación elemental por filas sobre $A$ es una de las siguientes:
    \begin{itemize}
    \item   
        \textbf{Intercambio de filas:} dados $i\in I$ y $j\in J$, intercambiar la fila $i$ por la fila $j$, denotado por
        \[
            F_i \leftrightarrow F_j,
        \]
        es reemplazar la fila
        \[
            \begin{pmatrix}a_{i1}&a_{i2}&\ldots& a_{in}\end{pmatrix}
        \]
        por la fila
        \[
            \begin{pmatrix}a_{j1}&a_{j2}&\ldots& a_{jn}\end{pmatrix}
        \]
        y viceversa.
    \item 
        \textbf{Multiplicar una fila por un escalar:} dados $i\in I$ y $\alpha\in \K$, con $\alpha\neq 0$, multiplicar la fila $i$ por $\alpha$, denotado por
        \[
            \alpha F_i \rightarrow F_i,
        \]
        es reemplazar la fila
        \[
            \begin{pmatrix}a_{i1}&a_{i2}&\ldots& a_{in}\end{pmatrix}
        \]
        por 
        \[
            \begin{pmatrix}\alpha a_{i1}&\alpha a_{i2}&\ldots&\alpha a_{in}\end{pmatrix}.
        \]
    \item 
        \textbf{Sumar un múltiplo de una fila con otra:} dados $i,j\in I$ y $\alpha\in \K$, multiplicar la fila $i$ por $\alpha$ y sumarlo a la fila $j$, denotado por
        \[
            \alpha F_i + F_j\rightarrow F_j,
        \]
        es reemplazar la fila
        \[
            \begin{pmatrix}a_{j1}&a_{j2}&\ldots& a_{jn}\end{pmatrix}
        \]
        por 
        \[
            \begin{pmatrix}\alpha a_{i1} + a_{j1}&\alpha a_{i2} + a_{j2}&\ldots&\alpha a_{in} + a_{jn}\end{pmatrix}.
        \]
\end{itemize}
\end{defi}

\begin{defi}[Matriz elemental]
    Dada una operación elemental por fila, se llama matriz elemental correspondiente a la operación al resultado de aplicar dicha operación a la matriz identidad.
\end{defi}

\begin{teo}
    Sean $E\in \Mat[\K]{n}{n}$ una matriz elemental y $A\in\Mat[\K]{n}{m}$, el resultado aplicar la operación por filas correspondiente a $E$ a la matriz $A$ es $EA$.
\end{teo}

\begin{defi}[Equivalente por filas]
    Sean $A,B\in \Mat[\K]{m}{n}$, se dice que la matriz $A$ es equivalente por filas a una matriz $B$, denotado por $A\sim B$, si $B$ se puede obtener al aplicar a la matriz $A$ una sucesión de operaciones elementales por fila.
\end{defi}

\begin{teo}
    Toda matriz $A\in\Mat[\K]{m}{n}$ es equivalente por filas a una única matriz en forma escalonada reducida por filas. 
\end{teo}

\begin{advertencia}
    El proceso para obtener una matriz escalonada reducida por filas a partir de una matriz cualquiera se conoce como eliminación de Gauss-Jordan.
\end{advertencia}

\begin{defi}
    Sean $A\in\Mat[\K]{m}{n}$ y $B\in\Mat[\K]{m}{n}$ la única matriz escalonada reducida por filas equivalente a $A$. El rango de $A$, denotado por $\rang(A)$, es el número de filas no nulas que tiene la matriz $B$.  
\end{defi}

\begin{prop}
    Sean $A, B\in\Mat[\K]{m}{n}$. Se tiene que si $A\sim B$, entonces 
    \[
        \rang(A) = \rang(B).
    \]
\end{prop}

\begin{prop}
    Sean $A\in\Mat[\K]{m}{n}$. Se tiene que si $A$ es una matriz escalonada, entonces $\rang(A)$ es el número de filas no nulas que tiene $A$.
\end{prop}

\subsection{Resolución de sistemas lineales}

\begin{teo}
    Sean $A,C\in \Mat[\K]{m}{n}$ y $b,d\in \K^m$, se tiene que los sistemas de ecuaciones lineales
    \[
        Ax=b
        \texty
        Cx=d
    \]
    tienen las mismas soluciones si y solo si
    \[
        (A|b)\sim (C|d),
    \]
    es decir, si las matrices aumentadas de los sistemas son equivalentes por filas. 
\end{teo}


\begin{defi}
    Sean $A\in \Mat[\K]{m}{n}$ y $b\in \K^m$, dado el sistema de ecuaciones lineales
    \[
        Ax=b,
    \]
    se dice que
    \begin{itemize}
    \item
        el sistema es \textbf{inconsistente} si no tiene solución;
    \item
        el sistema es \textbf{consistente} si tiene solución.
    \end{itemize}
\end{defi}

\begin{teo}
    Sean $A\in \Mat[\K]{m}{n}$ y $b\in \K^m$, dado el sistema de ecuaciones lineales
    \[
        Ax=b,
    \]
    se tiene una y solo una de las siguientes
    \begin{itemize}
    \item
        el sistema es inconsistente;
    \item
        el sistema es consistente y tiene una única solución; o
    \item
        el sistema es consistente y tiene infinitas soluciones.
    \end{itemize}
\end{teo}


\begin{teo}[Teorema de Rouché–Frobenius]
    Sean $A\in \Mat[\K]{m}{n}$ y $b\in \K^m$, dado el sistema de ecuaciones lineales
    \[
        Ax=b,
    \]
    se tiene que
    \begin{itemize}
    \item
        el sistema es consistente si y solo si $\rang(A)=\rang(A|b)$;
    \item
        en caso de que el sistema sea consistente, la solución es única si y solo si $\rang(A)=n$.
    \end{itemize}
\end{teo}

%%%%%%%%%%%%%%%%%%%%%%%%%%%%%%%%%%%%%%
\subsection{Sistemas homogéneos}

\begin{defi}[Sistema homogéneo]
    Sea $A\in \Mat[\K]{m}{n}$ al sistema
    \[ 
        Ax = 0
    \]
    se lo denomina sistema de ecuaciones lineales homogéneo.
\end{defi}

\begin{defi}
    Sea $A\in \Mat[\K]{m}{n}$, dado el sistema homogéneo
    \[ 
        Ax = 0,
    \]
    entonces
    \begin{itemize}
    \item 
        a $x = 0$ se la llama la solución trivial del sistema;
    \item 
        a $x \neq 0$ tal que $Ax=0$ se la llama una solución no trivial. 
    \end{itemize}
\end{defi}

\begin{teo}
    Un sistema homogéneo de $m$ ecuaciones con $n$ incógnitas siempre tiene una solución no trivial si $m<n$, es decir, si el número de incógnitas es mayor que el número de ecuaciones. 
\end{teo}

%%%%%%%%%%%%%%%%%%%%%%%%%%%%%%%%%%%%%%
\section{Inversa de una matriz}
%%%%%%%%%%%%%%%%%%%%%%%%%%%%%%%%%%%%%%

\begin{defi}
    Sea $A\in \Mat[\K]{n}{n}$ es no singular o invertible si existe una matriz $B\in \Mat[\K]{n}{n}$ tal que 
    \[
        AB = BA = I_n.
    \]
    A la matriz $B$ se la denomina inversa de $A$ y se la denota por $A^{-1}$. Si no existe tal matriz, entonces se dice que $A$ es singular o no invertible. 
\end{defi}

\begin{teo}
    Si una matriz tiene inversa, la inversa es única. 
\end{teo}

\begin{teo}
    Sea $A\in \Mat[\K]{n}{n}$. 
    \begin{itemize}
    \item 
        Si $A$ es una matriz no singular, entonces $A^{-1}$ es no singular y 
        \[ 
            (A^{-1})^{-1} = A.
        \]
    \item 
        Si $A$ y $B$ son matrices no singulares, entonces $AB$ es no singular y 
        \[ 
            (AB)^{-1} = B^{-1}A^{-1}.
        \]
    \item 
        Si $A$ es una matriz no singular, entonces
        \[
            {(A^\intercal)}^{-1} = {(A^{-1})}^\intercal.
        \]
    \end{itemize}
\end{teo}

\begin{teo}
    Sean $p\in\N^*$ y $A_1, A_2, \ldots, A_p \in \Mat[\K]{n}{n}$ matrices no singulares. Se tiene que $A_1 A_2 \cdots A_p$ es no singular y 
    \[
        (A_1 A_2 \cdots A_p)^{-1} = A_p^{-1} A_{p-1}^{-1} \cdots A_1^{-1}.
    \]
\end{teo}

\begin{teo}
    Sean $A, B \in \Mat[\K]{n}{n}$. Se tiene que si $AB=I_n$, entonces $BA=I_n$.
\end{teo}

\begin{teo}
    Sea $A \in \Mat[\K]{n}{n}$, se tiene que $A$ es no singular si y solo si es equivalente por filas a $I_n$. Es más
    \[
        (A|I_n) \sim (I_n|A^{-1}).
    \]
\end{teo}

\begin{teo}
    Sea $A \in \Mat[\K]{n}{n}$, el sistema homogéneo 
    \[
        Ax = 0
    \]
    tiene una solución no trivial si y solo si $A$ es singular. 
\end{teo}

\begin{teo}
    Sea $A \in \Mat[\K]{n}{n}$. Se tiene que $A$ es no singular si y solo si el sistema lineal $Ax=b$ tiene una solución única para cada vector $b \in \K^{n}$.
\end{teo}

\begin{teo}
    Sea $A \in \Mat[\K]{n}{n}$, se tienen que las siguientes son equivalentes:
    \begin{enumerate}
    \item 
        $A$ es no singular;
    \item 
        el sistema $Ax=0$ solamente tiene la solución trivial;
    \item 
        $A$ es equivalente por filas a $I_n$;
    \item 
        $\rang(A)=n$; y 
    \item 
        el sistema lineal $Ax=b$ tiene una solución única para cada vector $b \in \K^{n}$. 
    \end{enumerate}
\end{teo}

\section{Aplicaciones de sistemas de ecuaciones}

%%%%%%%%%%%%%%%%%%%%%%%%%%%%%%%%%%%%%%
\subsection{Factorización LU}
%%%%%%%%%%%%%%%%%%%%%%%%%%%%%%%%%%%%%%

Sea matriz $U\in\Mat[\K]{n}{n}$ una matriz triangular superior y cuyas entradas en la diagonal sean distintas de $0$, dado $b\in\K^n$, el sistema 
\[
    Ux = b,
\]
cuya matriz ampliada es
\[
    \begin{pmatrix}
        u_{11} & u_{12} & u_{13} & \cdots & u_{1n} & | & b_1 \\
        0      & u_{22} & u_{23} & \cdots & u_{2n} & | & b_2 \\
        0      & 0      & u_{33} & \cdots & u_{3n} & | & b_3 \\
        \vdots & \vdots & \vdots & \ddots & \vdots & | & \vdots \\
        0      & 0      & 0      & \cdots & u_{nn} & | & b_n 
    \end{pmatrix},
\]
se puede resolver por \emph{sustitución regresiva}, obteniendo que la solución del sistema es
\[
    x_n = \frac{b_n}{u_{nn}}
    \texty
    x_k = \dfrac{b_k - \displaystyle\sum_{j=k+1}^n u_{kj}x_j}{u_{kk}},
\]
para $k=n-1,\ldots,1$.

De manera análoga, Sea matriz $L\in\Mat[\K]{n}{n}$ una matriz triangular inferior y cuyas entradas en la diagonal sean distintas de $0$, dado $b\in\K^n$, el sistema 
\[
    Lx = b,
\]
cuya matriz ampliada es
\[
    \begin{pmatrix}
        l_{11} & 0      & 0      & \cdots & 0      & | & b_1 \\
        l_{21} & l_{22} & 0      & \cdots & 0      & | & b_2 \\
        l_{31} & l_{32} & l_{33} & \cdots & 0      & | & b_3 \\
        \vdots & \vdots & \vdots & \ddots & \vdots & | & \vdots \\
        l_{n1} & l_{n2} & l_{n3} & \cdots & l_{nn} & | & b_n 
    \end{pmatrix},
\]
se puede resolver por \emph{sustitución progresiva}, obteniendo que la solución del sistema es
\[
    x_1 = \frac{b_1}{l_{nn}}
    \texty
    x_k = \dfrac{b_k - \displaystyle\sum_{j=1}^{k-1} u_{kj}x_j}{u_{kk}},
\]
para $k=2,\ldots,n$.

Finalmente, si $A \in \Mat[\K]{n}{n}$ se puede escribir como el producto de una matriz triangular inferior $L$, un una matriz triangular superior $U$, es decir, 
\[
    A = LU,
\]
entonces se dice que $A$ tiene una factorización $LU$, y dado $b\in\K^n$, se tiene que el sistema
\[
    Ax=b
\]
equivale al sistema
\[
    L(Ux) = b,
\]
con esto, podríamos resolver el sistema inicial, resolviendo los sistemas
\[
    Ly = b
    \texty
    Ux = y.
\]


\begin{teo}[Factorización LU]
    Sean $A\in \Mat[\K]{n}{n}$ una matriz tal que sea equivalente por filas a una matriz $U\in \Mat[\K]{n}{n}$ triangular superior, entonces se tiene que
    \[
        U = E_{m}\cdots E_2 E_1 A,
    \]
    donde $E_k$, para $k=1,\ldots,m$, son las matrices elementales correspondientes a las operaciones por filas. Si no se ha utilizado intercambio de filas y únicamente se han eliminado los elementos bajo la diagonal, entonces
    \[
        L = E_{1}^{-1} E_2^{-1} \cdots E_m^{-1}
    \]
    es una matriz triangular inferior y se tiene que $A=LU$.
\end{teo}


%%%%%%%%%%%%%%%%%%%%%%%%%%%%%%%%%%%%%%
\subsection{Mínimos cuadrados}
%%%%%%%%%%%%%%%%%%%%%%%%%%%%%%%%%%%%%%

\begin{defi}[Problema de la recta de mínimos cuadrados]
    En $\R^2$, dados $n$ puntos $(x_1,y_1)$, $(x_2,y_2)$, \ldots, $(x_n,y_n)$, se plantea encontrar la ecuación de la recta que minimice la suma de los cuadrados de las distancias verticales de los puntos a la recta, es decir, se busca la pendiente $m$ y el intercepto $b$ de una recta que minimice la función definida por
    \[
        L(b,m)=\sum_{k=1}^n(y_k-mx_k-b)^2.
    \]
    A la recta de pendiente $m$ e intercepto $b$ que minimizan la función anterior se la llama recta de mínimos cuadrados.
\end{defi}

\begin{teo}
    Dados los puntos $(x_1,y_1)$, $(x_2,y_2)$, \ldots, $(x_n,y_n)$, definimos
    \[
        y = \begin{pmatrix}
            y_1 \\ y_2 \\ \vdots \\ y_n
        \end{pmatrix},
        \qquad
        A = \begin{pmatrix}
            1 & x_1 \\ 1 & x_2 \\ \vdots & \vdots \\ 1 & x_n
        \end{pmatrix}
        \texty
        u = \begin{pmatrix}
            b \\ m
        \end{pmatrix}.
    \]
    Se tiene que el problema de la recta de mínimos cuadrados es equivalente a resolver el sistema
    \[
        A^\intercal A u = A^\intercal y.
    \]
\end{teo}

\begin{teo}
    Sea $A\in\Mat{m}{n}$ tal que $\rang(A)=n$, entonces, $A^\intercal A$ es no singular.
\end{teo}

%%%%%%%%%%%%%%%%%%%%%%%%%%%%%%%%%%%%%%
\section{Determinantes}
%%%%%%%%%%%%%%%%%%%%%%%%%%%%%%%%%%%%%%

En esta sección tomaremos $n\in \N$, con $n>0$, e $I = \{ 1, 2, \ldots, n\}$.

\begin{defi}[Menor]
    Sean $A\in \Mat[\K]{n}{n}$ e $i,j\in I$. A la matriz de $\Mat[\K]{(n-1)}{(n-1)}$ que se obtiene eliminar la fila $i$ y la columna $j$ de $A$ se la llama el menor $ij$ de $A$, denotado por $A_{ij}$.
\end{defi}

\begin{advertencia}
    En la literatura se puede encontrar la notación de $M_{ij}$ para el menor de $ij$ de $A$.
\end{advertencia}

\begin{defi}[Menor principal]
    Sean $A\in \Mat[\K]{n}{n}$ y $k\in I$. A la matriz de $\Mat[\K]{k}{k}$ que se obtiene eliminar las $n-k$ últimas filas y columnas de $A$, se la llama el menor principal $k$ de $A$, denotado por $M_{k}$.
\end{defi}

\begin{advertencia}
    En la literatura se puede encontrar la notación de $A_{k}$ para el menor principal $k$ de $A$.
\end{advertencia}


\begin{defi}[Determinantes] 
    Sea $A\in \Mat[\K]{n}{n}$ se define el determinante de $A$, denotado por $\det(A)$ (o por $|A|$), de manera inductiva, como sigue:
    \begin{itemize}
    \item 
        Si $n=1$ y $A=(a_{11})$, entonces $\det(A) = a_{11}$.
    \item
        Si $n>1$, entonces
        \begin{align*}
            \det(A) 
            & = \sum_{k=1}^n a_{1k}(-1)^{1+k}\det(A_{1k})\\
            & = a_{11}\det(A_{11}) - a_{12}\det(A_{12}) 
            + \ldots +  (-1)^{1+n} a_{1n}\det(A_{1n}).
        \end{align*}
    \end{itemize}
\end{defi}

Ejemplos:
\begin{itemize}
\item 
    Sea $A$ una matriz de orden $2 \times 2$ de la forma
    \[ 
        A=
        \begin{pmatrix}
         a_{11} & a_{12}\\
         a_{21} & a_{22}
        \end{pmatrix},
    \]
    se tiene que
    \[
        A_{11} = (a_{22})
        \texty
        A_{12} = (a_{21}),
    \]
    por lo tanto
    \[
        \det(A_{11}) = a_{11}
        \texty
        \det(A_{12}) = a_{12},
    \]
    de esta forma,
    \[
        \det(A) = a_{11}\det(A_{11}) - a_{12}\det(A_{12})
            = a_{11}a_{22} - a_{12}a_{21},
    \]
    es decir,
    \[
        \det(A) = \begin{vmatrix}
        a_{22} & a_{23} \\
        a_{32} & a_{33}
        \end{vmatrix}
        = a_{11}a_{22} - a_{12}a_{21}.
    \]
\item Sea $A$ una matriz de orden $3 \times 3$ de la forma
         \[ A=
        \begin{pmatrix}
         a_{11} & a_{12} & a_{13}\\
         a_{21} & a_{22} & a_{23}\\
         a_{31} & a_{32} & a_{33}\\
        \end{pmatrix}
        \]
        \noindent el determinante de la matriz $A$ está dado por:
        \[
        \det(A) = a_{11}
        \begin{vmatrix}
        a_{22} & a_{23} \\
        a_{32} & a_{33}
        \end{vmatrix} - a_{12}
        \begin{vmatrix}
        a_{21} & a_{23} \\
        a_{31} & a_{33} \\
        \end{vmatrix}+ a_{13}
        \begin{vmatrix}
        a_{21} & a_{22} \\
        a_{31} & a_{32}
        \end{vmatrix}.
        \]
\end{itemize}

%%%%%%%%%%%%%%%%%%%%%%%%%%%%%%%%%%%%%%
\subsection{Propiedades de los determinantes}
%%%%%%%%%%%%%%%%%%%%%%%%%%%%%%%%%%%%%%

\begin{teo}
    Sea $A \in \Mat[\K]{n}{n}$. Si una fila o columna de $A$ contiene solo ceros, entonces $\det (A) =0$.
\end{teo}


\begin{teo}
    Sea $A \in \Mat[\K]{n}{n}$ una matriz triangular superior o triangular inferior, entonces
    \[
        \det(A) = a_{11}a_{22} \cdots a_{nn},
    \]
    es decir, el determinante de una matriz triangular es el producto de los elementos de su diagonal principal.
\end{teo}


\begin{teo}
    Sea $E\in\Mat[\K]{n}{n}$ una matriz elemental, $i,j\in I$ y $\alpha\in\K$, con $\alpha\neq 0$. Se tiene que
    \begin{itemize}
    \item   
        si la operación es $F_i \leftrightarrow F_j$, entonces $\det(E) = - 1$;
    \item 
        si la operación es $\alpha F_i \rightarrow F_i$, entonces $\det(E) = \alpha$; y
    \item 
        si la operación es $\alpha F_i + F_j\rightarrow F_j$, entonces $\det(E) = 1$.
    \end{itemize}
\end{teo}

\begin{teo}
    Sea $A \in \Mat[\K]{n}{n}$. El determinante de una matriz $A$ y de su transpuesta son iguales, es decir, 
    \[
        \det(A^\intercal) = \det(A).
    \]
\end{teo}

\begin{teo}
    Sean $A, B\in \Mat[\K]{n}{n}$, se tiene que:
    \[
        \det(AB) = \det(A)\det(B).
    \]
\end{teo}


\begin{teo}
    Sean $A, B \in \Mat[\K]{n}{n}$. Si la matriz $B$ se obtiene intercambiando dos filas o columnas de $A$ entonces
    \[  
        \det(B) = - \det(A).
    \]
\end{teo}

\begin{teo}
    Sea $A \in \Mat[\K]{n}{n}$. Si dos filas o columnas de $A$ son iguales, entonces 
    \[
        \det(A) = 0
    \]
\end{teo}

\begin{teo}
    Sean $A, B \in \Mat[\K]{n}{n}$. Si $B$ se obtiene al multiplicar una fila o columna de $A$ por un escalar $\alpha \in \K$, entonces
    \[
        \det(B) = \alpha \det(A).
    \]
\end{teo}


\begin{teo}
    Sean $A\in \Mat[\K]{n}{n}$ y $\alpha \in\K$. Se tiene que
    \[
        \det(\alpha A) = \alpha^n \det(A).
    \]
\end{teo}

\begin{teo}
    Sean $A, B \in \Mat[\K]{n}{n}$, $\alpha\in\K$ e $i,j\in I$, con $i\neq j$. Si $B$ se obtiene al aplicar una operación de fila $\alpha F_i + F_j \to F_j$, entonces 
    \[
        \det(B) = \det(A).
    \]
\end{teo}
 

\begin{teo}
    Sea $A \in \Mat[\K]{n}{n}$. Si $A$ es no singular, entonces $\det(A) \neq 0$ y 
    \[
        \det(A^{-1}) = \dfrac{1}{\det(A)}.
    \]
\end{teo}

\begin{teo}
    Sean $A, B, C \in \Mat[\K]{n}{n}$ y $j\in I$ tales que $A$, $B$ y $C$ difieran únicamente en la columna $j$. Si la columna $j$ de $C$ es el resultado de sumar las columnas $j$ de $A$ y $B$, entonces 
    \[
        \det(C) = \det(A) + \det(B).
    \]
\end{teo}

%%%%%%%%%%%%%%%%%%%%%%%%%%%%%%%%%%%%%%
\subsection{Cofactores}
%%%%%%%%%%%%%%%%%%%%%%%%%%%%%%%%%%%%%%

\begin{defi}[Cofactores]
    Sean $A\in \Mat[\K]{n}{n}$ e $i,j\in I$. El cofactor $ij$ de $A$, denotado $C_{ij}$, está dado por
    \[
        C_{ij} = (-1)^{i+j} \det (A_{ij})
    \]
     donde $A_{ij}$ es el menor $ij$ de $A$.
\end{defi}

\begin{advertencia}
    En la literatura, también se suele llamar menor al determinante de $A_{ij}$ en lugar de a la matriz, como lo haremos en este texto. Además, al cofactor, se lo suele denotar por $A_{ij}$.
\end{advertencia}

\begin{teo}
    Sea $A \in \Mat[\K]{n}{n}$. Se tiene que para todo $i\in I$,
    \[
        \det(A) 
        = \sum_{k=1}^n a_{ik}C_{ik} 
        = a_{i1}C_{i1} + a_{i2}C_{i2} + \ldots + a_{in}C_{in}
    \]
    y
    \[
        \det(A) 
        = \sum_{k=1}^n a_{ki}C_{ki} 
        = a_{1i}C_{1i} + a_{2i}C_{2i} + \ldots + a_{ni}C_{ni}.
    \]
    El lado derecho de las igualdades toma el nombre de expansión por cofactores del determinante de $A$.
\end{teo}

%%%%%%%%%%%%%%%%%%%%%%%%%%%%%%%%%%%%%%
\subsection{Más sobre la inversa de una matriz}
%%%%%%%%%%%%%%%%%%%%%%%%%%%%%%%%%%%%%%

\begin{defi}[Matriz de cofactores]
    Sea $A \in \Mat[\K]{n}{n}$. La matriz de cofactores de $A$, que se denota por $\cof(A)$, es la matriz de $\Mat[\K]{n}{n}$ que está formada por los cofactores de $A$, es decir, 
    \[
        \cof(A) = (C_{ij}) = 
        \begin{pmatrix}
          C_{11} & C_{12} & \cdots & C_{1n} \\
          C_{21} & C_{22} & \cdots & C_{2n} \\
          \vdots & \vdots & \ddots &\vdots  \\
          C_{n1} & C_{n2} & \cdots & C_{nn}
        \end{pmatrix}.
    \]
\end{defi}


\begin{defi}
    Sea $A \in \Mat[\K]{n}{n}$. La matriz adjunta de $A$, que se denota por $\adj(A)$, es la matriz de $\Mat[\K]{n}{n}$ que está formada por la transpuesta de la matriz de los cofactores de $A$, es decir, 
    \[
        \adj(A) = \cof(A)^\intercal.
    \]
\end{defi}

\begin{teo}
    Sea $A \in \Mat[\K]{n}{n}$, entonces
    \[
        A (\adj(A)) = (\adj(A))A = \det(A)I_n.
    \]
\end{teo}

\begin{cor}
    Sea $A \in \Mat[\K]{n}{n}$. Si $\det(A) \neq 0$, entonces $A$ es invertible y
    \[
        A^{-1} = \dfrac{1}{\det(A)} \adj(A).
    \]
\end{cor}

\begin{teo}
    Sea $A \in \Mat[\K]{n}{n}$. Una matriz $A$ es no singular si y sólo si $\det(A) \neq 0$.
\end{teo}

\begin{teo}
    Sea $A \in \Mat[\K]{n}{n}$. El sistema homogéneo $Ax = 0$ tiene una solución no trivial si y sólo si $\det(A) = 0$.
\end{teo}

\begin{teo}
    Sean $A \in \Mat[\K]{n}{n}$ y $b\in\K^n$. El sistema $Ax = b$ tiene una solución única si y sólo si $\det(A) \neq 0$ y su solución es
    \[
        A^{-1}b.
    \]
\end{teo}

\begin{teo}[Equivalencias no singulares]
    
    Sea $A \in \Mat[\K]{n}{n}$, se tienen que las siguientes son equivalentes:
    \begin{enumerate}
    \item 
        $A$ es no singular;
    \item 
        el sistema $Ax=0$ tiene solamente la solución trivial;
    \item 
        $A$ es equivalente por filas a $I_n$;
    \item 
        $\rang(A)=n$; 
    \item 
        el sistema lineal $Ax=b$ tiene una solución única para cada vector $b \in \K^{n}$; y
    \item 
        $\det(A) \neq 0$.
    \end{enumerate}
\end{teo}

%%%%%%%%%%%%%%%%%%%%%%%%%%%%%%%%%%%%%%
\section{El espacio $\R^n$}
%%%%%%%%%%%%%%%%%%%%%%%%%%%%%%%%%%%%%%

En esta sección, consideramos $n\in\N$ con $n\geq 1$.

\begin{defi}[El conjunto $\R^n$]
    El conjunto $\R^n$ es
    \[
        \R^n=\underbrace{\R\times\R\times\cdots\times\R}_{n\text{ veces}},
    \]
    es decir
    \[
        \R^n=\{(x_1,x_2,\ldots,x_n):x_i\in\R\text{ para todo }i=1,2,\ldots,n\}.
    \]
\end{defi}

\begin{advertencia}
    Recordemos que, por notación, si $y\in\R^n$, se asumirá $y=(y_1,y_2,\ldots,y_n)$.
\end{advertencia}

\begin{advertencia}
    Recordemos que podemos identificar cada elemento de $\R^n$ con una matriz de $\Mat{n}{1}$ de la siguiente manera: si
    \[
        a = (a_1,a_2,\ldots,a_n) \in \R^n,
    \]
    entonces, visto como matriz es
    \[
        a = 
        \begin{pmatrix}
            a_1\\a_2\\
            \vdots\\a_n
        \end{pmatrix}
        \in \Mat{n}{1}.
    \]
\end{advertencia}

\begin{teo}
    En $\R^n$ se cumplen las siguientes propiedades:
    \begin{enumerate}
    \item \textbf{asociativa de la suma:}
        para todo $x,y,z\in \R^n$ se tiene que
        \[
            (x + y) + z = x + (y + z);
        \]
    \item \textbf{conmutativa de la suma:}
        para todo $x,y\in \R^n$ se tiene que
        \[
            x + y = y + x;
        \]
    \item \textbf{elemento neutro de la suma:}
        existe un elemento de $\R^n$, denotado por $0$, tal que para todo $x\in \R^n$ se tiene que 
        \[
            x + 0 = 0 + x = x;
        \]
    \item \textbf{inverso de la suma:}
        para todo $x\in\R^n$, existe un elemento de $\R^n$, denotado por $-x$, tal que
        \[
            x + (-x)= 0;
        \]
    \item \textbf{distributiva del producto I:}
        para todo $x,y\in\R^n$ y todo $\alpha\in \R$ se tiene que
        \[
            \alpha (x + y)=\alpha x + \alpha y
        \]
    \item \textbf{distributiva del producto II:}
        para todo $x\in\R^n$ y todo $\alpha,\beta\in \R$ se tiene que
        \[
            (\alpha+\beta) x=\alpha x + \beta x;
        \]
    \item \textbf{asociativa del producto:}
        para todo $x\in\R^n$ y todo $\alpha,\beta\in \R$ se tiene que
        \[
            (\alpha\beta) x=\alpha(\beta x);
        \]
    \item \textbf{elemento neutro del producto:}
        para todo $x\in\R^n$ se tiene que
        \[
            1 x=x.
        \]
    \end{enumerate}
\end{teo}

\begin{defi}[Base canónica]
    En $\R^n$, el conjunto $\{e^1,e^2,\ldots,e^n\}\subset\R^n$, definidos por 
    \[
        e^i_j=
        \begin{cases}
            0& \text{si }i\neq j,\\
            1& \text{si }i= j,
        \end{cases}
    \]
    para todo $i,j\in\{1,2,\ldots,n\}$, se lo denomina \emph{base canónica} de $\R^n$.
\end{defi}

\begin{teo}
    Sea $x\in\R^n$, se tiene que existen únicos
    \[
        \alpha_1,\alpha_2,\ldots,\alpha_n\in\R
    \]
    tales que
    \[
        x = \alpha_1e^1 + \alpha_2 e^2 + \cdots + \alpha_n e^n.
    \]
\end{teo}

\begin{defi}[Producto punto]
    La función 
    \[
        \funcion{\cdot}{\R^n\times \R^n}{\R}
        {(x,y)}{\sum_{i=1}^n x_iy_i}
    \]
    se denomina producto punto de $\R^n$ o producto interno de $\R^n$.
\end{defi}

\begin{teo}[Propiedades del producto punto]
    Sean $x,y,z\in \R^n$ y $\alpha\in\R$, se tiene que
    \begin{itemize}
    \item
        $x\cdot x \geq 0$;
    \item
        $x\cdot x = 0$ si y solo si $x=0$;
    \item
        $x \cdot y = y \cdot x$;
    \item 
        $(x + y)\cdot z = (x\cdot z) + (y\cdot z)$; y
    \item
        $(\alpha x)\cdot y = x\cdot(\alpha y) = \alpha(x\cdot y)$.
    \end{itemize}
\end{teo}

\begin{defi}[Norma]
    La función
    \[
        \funcion{\|\cdot\|}{\R^n}{\R}
        {x}{\sqrt{x\cdot x}}
    \]
   se la llama la norma de $\R^n$. Para $x\in\R^n$ a $\|x\|$ se le llama norma, módulo o longitud de $x$.
\end{defi}





\begin{teo}[Desigualdad de Cauchy-Schwarz]
    Para todo $x,y\in\R^n$,
    \[
        |x\cdot y|\leq \|x\|\,\|y\|.
    \]
\end{teo}

\begin{teo}[Propiedades de la norma]
    Sean $x,y\in \R^n$ y $\alpha\in\R$, se tiene que
    \begin{itemize}
    \item
        $\|x\| \geq 0$;
    \item
        $\|x\| = 0$ si y solo si $x=0$;
    \item 
        $\|\alpha x\| = |\alpha|\|x\|$; y
    \item
        $\|x + y\| \leq \|x\|+\|y\|$
    \end{itemize}
\end{teo}

\begin{advertencia}
    Dados $x,y\in\R^n$, se define la distancia entre $x$ y $y$ por $\|x-y\|$.
\end{advertencia}

\begin{defi}[Vector unitario]
    Dado $x\in\R^n$, se dice que $x$ es un vector unitario si $\|x\|=1$.
\end{defi}

    

\begin{defi}[Ángulo entre vectores]
    Sean $x,y\in\R^n$, ambos diferentes de 0, se define el ángulo entre estos vectores por
    \[
       \theta =  \arccos\left( \frac{x\cdot y}{\|x\|\,\|y\|} \right).
    \]
\end{defi}

\begin{defi}
    Sean $x,y\in\R^n$, se dice que 
    \begin{itemize}
    \item 
        son ortogonales si $x\cdot y =0$;
    \item
        son paralelos si $|x\cdot y| = \|x\|\,\|y\|$; y
    \item
        tienen la misma dirección si $x\cdot y = \|x\|\,\|y\|$.
    \end{itemize}
\end{defi}

\begin{teo}
    Sean $x,y\in\R^n$ tal que $y\neq 0$. Si $x$ y $y$ son paralelos, entonces existe $\alpha\in\R$ tal que $x=\alpha y$.
\end{teo}

\begin{defi}[Proyección ortogonal]
    Sean $x,y\in\R^n$, con $y\neq 0$, la proyección ortogonal de $x$ sobre $y$ es
    \[
        \proy_y(x) = \frac{x\cdot y}{\|y\|^2} y
    \]
    y la componente normal de $x$ respecto a $y$ es
    \[
        \norm_y(x) = x - \proy_y(x).
    \]
\end{defi}


%%%%%%%%%%%%%%%%%%%%%%%%%%%%%%%%%%%%%%
\subsection{Geometría de $\R^n$}
%%%%%%%%%%%%%%%%%%%%%%%%%%%%%%%%%%%%%%

En esta sección, a menos que se indique lo contrario, asumiremos $n\in\N$ con $n\geq 2$.
%%%%%%%%%%%%%%%%%%%%%%%%%%%%%%%%%%%%%%

\begin{defi}[Puntos de $\R^n$]
    Se llama punto de $\R^n$ a cualquier elemento de $\R^n$. 
\end{defi}


\begin{defi}[Recta de $\R^n$]
    Dados $a,b\in\R^n$, con $b\neq0$, la recta que pasa por $a$ con dirección $b$ es el conjunto
    \[
        L(a;b)=\{a+t b:t\in\R\}.
    \]
\end{defi}

\begin{proof}[Ejemplo]
    En $\R^2$, tenemos la recta que pasa por $(1,-2)$ con dirección $(3,2)$ es
    \begin{align*}
        L((1,-2);(3,2)) 
            & = \{(1,-2)+t(3,2):t \in\R\} \\
            & = \{(1+3t ,-2+2t ): t \in\R\} \\
            & = \{(x,y)\in\R^2: x = 1+3t ,\ y=-2+2t ,\ t \in\R\} \\
            & = \{(x,y)\in\R^2: 2x-3y=4\}
    \end{align*}
    Con esto, tenemos que la recta está definida por
    \begin{align*}
        x& = 1+3t ,\\
        y& = -2+2t ,
    \end{align*}
    con $t \in \R$; a esta se la llama la ecuación paramétrica de la recta. Por otro lado, también tenemos que la recta está definida por 
    \[
        2x-3y=4
    \]
    con $(x,y)\in\R^2$; a esta se la llama la ecuación cartesiana de la recta.
\end{proof}

\begin{defi}[Plano de $\R^n$]
    Dados $a,b,c\in\R^n$, con $b$ y $c$ no paralelos y diferentes de $0$, el plano que pasa por $a$ con dirección $b$ y $c$ es el conjunto
    \[
        P(a;b,c)=\{a+s  b+t c:s ,t \in\R\}.
    \]
\end{defi}

\begin{proof}[Ejemplo]
    En $\R^3$, tenemos la recta que pasa por $(1,-2,-1)$ con direcciones $(3,2,1)$ y $(1,0,2)$ es
    \begin{align*}
        L&((1,-2,-1);(3,2,1),(1,0,2)) \\
            & = \{(1,-2,-1)+s (3,2,1) + t (1,0,2) :s ,t \in\R\} \\
            & = \{(1+3s +t ,-2+2s ,-1+s +2t ): s ,t \in\R\} \\
            & = \{(x,y,z)\in\R^3: x = 1+3s +t ,\ y=-2+2s ,\ z=-1+s +2t ,\ s ,t \in\R\} \\
            & = \{(x,y,z)\in\R^3: 4x-5y-2z=8\}
    \end{align*}
    Con esto, tenemos que la recta está definida por
    \begin{align*}
        x& = 1+3s +t ,\\
        y& = -2+2s ,\\
        z& = -1+s +2t 
    \end{align*}
    con $s ,t \in \R$; a esta se la llama la ecuación paramétrica del plano. Por otro lado, también tenemos que la recta está definida por 
    \[
        4x-5y-2z=8
    \]
    con $(x,y,z)\in\R^3$; a esta se la llama la ecuación cartesiana de la recta.
\end{proof}

\begin{advertencia}
    Podemos pasar de la ecuación paramétrica a la ecuación cartesiana si, al verlas como un sistema de ecuaciones en sus parámetros, analizamos cuándo el sistema es consistente.
    
    Además, podemos pasar de la ecuación cartesiana a la ecuación paramétrica si, al verla como un sistema de ecuaciones, resolvemos el sistema.
\end{advertencia}

\begin{advertencia}
    En $\R^2$, dados dos puntos $(x_1,y_1),(x_2,y_2)\in\R^2$, la ecuación cartesiana de la recta que pasa por estos dos puntos es
    \[
        \begin{vmatrix}
            x & y & 1\\
            x_1 & y_1 & 1\\
            x_2 & y_2 & 1
        \end{vmatrix}
        = 0,
    \]
    para $(x,y)\in\R^2$.
\end{advertencia}

\begin{advertencia}
    En $\R^3$, dados tres puntos $(x_1,y_1,z_1),(x_2,y_2,z_2),(x_3,y_3,z_3)\in\R^3$, la ecuación cartesiana del plano que pasa por estos tres puntos es
    \[
        \begin{vmatrix}
            x & y & z & 1\\
            x_1 & y_1 & z_1 & 1\\
            x_2 & y_2 & z_2 & 1\\
            x_3 & y_3 & z_3 & 1
        \end{vmatrix}
        = 0,
    \]
    para $(x,y,z)\in\R^3$.
\end{advertencia}

%%%%%%%%%%%%%%%%%%%%%%%%%%%%%%%%%%%%%%
\section{Espacios vectoriales}
%%%%%%%%%%%%%%%%%%%%%%%%%%%%%%%%%%%%%%

\begin{defi}[Espacio Vectorial]
    Dados un campo $\K$, un conjunto no vacío $E$ y dos operaciones
    \[
        \funcion{\oplus}{E\times E}{E}{(x,y)}{x\oplus y,}
        \texty
        \funcion{\odot}{\K\times E}{E}{(\alpha,x)}{\alpha\odot x}
    \]
    llamadas suma y producto, respectivamente; se dice que $(E,\oplus,\odot,\K)$ es un espacio vectorial si cumplen las siguientes propiedades
    \begin{enumerate}
    \item \textbf{asociativa de la suma:}
        para todo $x,y,z\in E$ se tiene que
        \[
            (x \oplus y) \oplus z = x \oplus (y \oplus z);
        \]
    \item \textbf{conmutativa de la suma:}
        para todo $x,y\in E$ se tiene que
        \[
            x \oplus y = y \oplus x;
        \]
    \item \textbf{elemento neutro de la suma:}
        existe un elemento de $E$, denotado por $0_E$ o simplemente $0$, tal que para todo $x\in E$ se tiene que 
        \[
            x \oplus 0 = 0 \oplus x = x;
        \]
    \item \textbf{inverso de la suma:}
        para todo $x\in E$, existe un elemento de $E$, denotado por $-x$, tal que
        \[
            x \oplus (-x)= 0;
        \]
    \item \textbf{distributiva del producto I:}
        para todo $x,y\in E$ y todo $\alpha\in \K$ se tiene que
        \[
            \alpha\odot (x \oplus y)=\alpha\odot x + \alpha\odot y
        \]
    \item \textbf{distributiva del producto II:}
        para todo $x\in E$ y todo $\alpha,\beta\in \K$ se tiene que
        \[
            (\alpha+\beta)\odot x=\alpha\odot x \oplus \beta\odot x;
        \]
    \item \textbf{asociativa del producto:}
        para todo $x\in E$ y todo $\alpha,\beta\in \K$ se tiene que
        \[
            (\alpha\beta)\odot x=\alpha\odot(\beta\odot x);
        \]
    \item \textbf{elemento neutro del producto:}
        para todo $x\in E$ se tiene que
        \[
            1\odot x=x,
        \]
        donde $1\in\K$ es el elemento neutro multiplicativo de $\K$
    \end{enumerate}
\end{defi}

\begin{advertencia}
    Utilizamos los símbolos $\oplus$ y $\odot$ para enfatizar el hecho de que, en general, las operaciones definidas no son la suma y el producto estándar que utilizamos. Si no existe riesgo de confusión, utilizaremos la notación
    \[
        x\oplus y = x+y
        \texty
        \alpha \odot x = \alpha x
    \]
    y diremos que el espacio vectorial es $(E,+,\cdot,\K)$, es más, en caso de que no exista ambigüedad en las operaciones utilizadas se dirá simplemente que $E$ es un espacio vectorial.
\end{advertencia}

\begin{teo}
    Los siguientes conjuntos son espacios vectoriales en el campo $\R$:
    \begin{itemize}
    \item
        $\R^n$, con $n\in \N^*$;
    \item
        $\Mat[\R]{m}{n}$, con $m,n\in \N^*$;
    \item
        $\mathcal{F}(I)=\{\func{f}{I}{\R}: f\text{ es una función}\}$, con $I\subseteq \R$.
    % \item
    %     $\mathcal{C}(I)=\{\func{f}{I}{\R}: f\text{ es continua en }I\}$, con $I\subseteq \R$.
    % \item
    %     $\mathcal{C}^k(I)=\{\func{f}{I}{\R}: f\text{ es $k$ veces derivable en }I\text{ y }f^{k}\in\mathcal{C}^k(I)\}$, con $I\subseteq \R$.
    \item
        $\R_n[x]$ el conjunto de todos los polinomio de grado menor igual que $n$ en la variable $x$, con $n\in \N$;
    \end{itemize}
\end{teo}

\begin{teo}
    Sea $(E,+,\cdot,\K)$ un espacio vectorial, se tiene que para todo $u \in E$ y todo $\alpha \in \K$, se tiene que
    \begin{enumerate}
        \item $0u = 0$;
        \item $\alpha 0 = 0$;
        \item si $\alpha u=0$ entonces $\alpha  = 0$ o $u = 0$;
        \item $(-1) u = -u$.
    \end{enumerate}
\end{teo}

%%%%%%%%%%%%%%%%%%%%%%%%%%%%%%%%%%%%%%
\subsection{Subespacios}
%%%%%%%%%%%%%%%%%%%%%%%%%%%%%%%%%%%%%%

\begin{defi}[Subespacio Vectorial]
    Sea $(E,+,\cdot,\K)$ un espacio vectorial y $W \subset E$ un conjunto no vacío de $E$. Si $W$ es un espacio vectorial con respecto a las operaciones de $E$, entonces se dice que $W$ es un subespacio de $E$.
\end{defi}

\begin{teo}
    Sean $(E,+,\cdot,\K)$ un espacio vectorial y $W$ un subconjunto no vacío de $E$. Entonces $W$ es un subespacio de $E$ si y sólo si se cumplen las siguientes condiciones:
    \begin{itemize}
    \item 
        si $u,v \in W$, entonces $u + v \in W$; u
    \item 
        si $\alpha \in \K$ y $u \in W$, entonces $\alpha u \in W$.
    \end{itemize}
\end{teo}

\begin{defi}[Combinación lineal]
    Sean $(E,+,\cdot,\K)$ un espacio vectorial, $k\in\N^*$ y $v_1, v_2, \ldots, v_k \in E$. Se dice que un vector $v \in E$ es una combinación lineal de 
    $v_1, v_2, \ldots, v_k$ si 
    \[
        v = \alpha_1 v_1 + \alpha_2 v_2 + \cdots + \alpha_k v_k
    \]
    para algunos $\alpha_1, \alpha_2, \ldots, \alpha_k \in \K$.
\end{defi}

\begin{defi}
    Sean $(E,+,\cdot,\K)$ un espacio vectorial y $S = \{v_1, v_2, \ldots, v_k\} \subset E$. Al conjunto conjunto de todos los vectores en $E$ que son combinaciones lineales de los vectores de $S$ se lo llama cápsula de $S$ y se denota por $\spn(S)$, es decir
    \[
        \spn(S) = \{\alpha_1 v_1 + \alpha_2 v_2 + \cdots + \alpha_k v_k:\alpha_1, \alpha_2, \ldots, \alpha_k \in \K\}.
    \]
\end{defi}

\begin{advertencia}
    A este conjunto también se lo conoce como \emph{clausura lineal} y su notación viene de su nombre en inglés, \emph{linear span}. También se utiliza la notación $\langle S\rangle$.
\end{advertencia}

\begin{teo}
   Sean $(E,+,\cdot,\K)$ un espacio vectorial y $S\subset E$. Se tiene que $\spn(S)$ es un subespacio vectorial de $E$.
\end{teo}

%%%%%%%%%%%%%%%%%%%%%%%%%%%%%%%%%%%%%%
\subsection{Conjunto generador}
%%%%%%%%%%%%%%%%%%%%%%%%%%%%%%%%%%%%%%

\begin{defi}
    Sean $(E,+,\cdot,\K)$ un espacio vectorial y $S\subseteq E$. Al subespacio vectorial más pequeño que contiene a $S$ (es decir, la intersección de todos los subespacios que contienen a $S$) se lo denomina espacio generado por $S$ y se denota por $\gen(S)$.
\end{defi}

\begin{teo}
    Sean $(E,+,\cdot,\K)$ un espacio vectorial y $S\subseteq E$. Se tiene que
    \[
        \spn(S)=\gen(S).
    \]
\end{teo}


\begin{defi}
    Sean $(E,+,\cdot,\K)$ un espacio vectorial y $S=\{v_1,v_2,\ldots,v_k\}\subseteq E$. Se dice que $S$ genera el espacio vectorial $E$ si cada vector en $E$ es una combinación lineal de los elementos de $S$, es decir, si para todo $v\in E$, existen $\alpha_1, \alpha_2, \ldots, \alpha_k\in K$ tales que
    \[
        v = \alpha_1v_1+ \alpha_2v_2+ \cdots+ \alpha_k v_k.
    \]
\end{defi}


\begin{teo}
    Sean $(E,+,\cdot,\K)$ un espacio vectorial y $S=\{v_1,v_2,\ldots,v_k\}\subseteq E$. Se tiene que $S$ genera el espacio vectorial $E$ si y solo si
    \[
        E = \gen(S) = \spn(S).
    \]
\end{teo}

%%%%%%%%%%%%%%%%%%%%%%%%%%%%%%%%%%%%%%
\subsection{Independencia lineal}
%%%%%%%%%%%%%%%%%%%%%%%%%%%%%%%%%%%%%%

\begin{defi}
    Sean $(E,+,\cdot,\K)$ un espacio vectorial y $S=\{v_1,v_2,\ldots,v_k\}\subseteq E$. Se dice que $S$ es un conjunto linealmente dependiente si existen $\alpha_1, \alpha_2, \ldots, \alpha_k \in \K$, no todos iguales a cero, tales que:
    \[
        \alpha_1 v_1 + \alpha_2 v_2 + \cdots + \alpha_k v_k = 0
    \]
    en caso contrario, se dice que $S$ es un conjunto linealmente independiente. 
\end{defi}

\begin{advertencia}
    De esta definición, se tiene que $\{v_1,v_2,\ldots,v_k\}$ es linealmente independiente si y solo si
    \[
        \alpha_1 v_1 + \alpha_2 v_2 + \cdots + \alpha_k v_k = 0
    \]
    implica que
    \[
        \alpha_1 = \alpha_2 = \cdots = \alpha_k = 0.
    \]
\end{advertencia}

\begin{advertencia}
    Se puede extender esta definición para conjuntos infinitos diciendo que $S$ es linealmente independiente si todo subconjunto finito de $S$ es linealmente independiente.
\end{advertencia}

\begin{teo}
    Sean $(E,+,\cdot,\K)$ un espacio vectorial y $S\subseteq E$. Si $0\in S$, entonces $S$ es linealmente dependiente.
\end{teo}

\begin{teo}
    Sean $(E,+,\cdot,\K)$ un espacio vectorial y $S=\{v_1,v_2,\ldots,v_k\}\subseteq E$. Se tiene que $S$ es un conjunto linealmente dependiente si y sólo si alguno de los vectores $v_j\in S$ es una combinación lineal de otros elementos de $S$.
\end{teo}

%%%%%%%%%%%%%%%%%%%%%%%%%%%%%%%%%%%%%%
\subsection{Bases}
%%%%%%%%%%%%%%%%%%%%%%%%%%%%%%%%%%%%%%

\begin{defi}
    Sean $(E,+,\cdot,\K)$ un espacio vectorial y $B\subseteq E$. Se dice que $B$ es una base de $E$ si 
    \begin{itemize}
        \item $B$ genera a $E$ y 
        \item $B$ es linealmente independiente. 
    \end{itemize}
\end{defi}


\begin{teo}[Base canónica de $\R^n$]
    En $\R^n$, el conjunto $\{e^1,e^2,\ldots,e^n\}\subset\R^n$, definidos por 
    \[
        e^i_j=
        \begin{cases}
            0& \text{si }i\neq j,\\
            1& \text{si }i= j,
        \end{cases}
    \]
    para todo $i,j\in\{1,2,\ldots,n\}$, es una base de $\R^n$.
\end{teo}

\begin{teo}[Base canónica de {$\R_n[x]$}]
    En $\R_n[x]$, el conjunto $\{1,x,\ldots,x^{n-1},x^n\}$ es una base de $\R_n[x]$. A esta base se la denomina la base canónica de $\R_n[x]$.
\end{teo}

\begin{teo}
    Sean $(E,+,\cdot,\K)$ un espacio vectorial y $B\subseteq E$ una base de $E$. Se tiene que todo elemento de $E$ se puede escribir, de manera única, como combinación lineal de elementos de $B$.
\end{teo}

\begin{teo}
    Sean $(E,+,\cdot,\K)$ un espacio vectorial y $B\subseteq E$. Se tiene que si todo elemento de $E$ se puede escribir, de manera única, como combinación lineal de elementos de $B$, entonces $B$ es una base de $E$.
\end{teo}


\begin{teo}
    Sean $(E,+,\cdot,\K)$ un espacio vectorial y $S\subseteq E$ un conjunto que genera a $E$. Se tiene que algún subconjunto de $S$ es base de $E$.
\end{teo}


\begin{teo}
    Todo espacio vectorial tiene una base.
\end{teo}

%%%%%%%%%%%%%%%%%%%%%%%%%%%%%%%%%%%%%%
\subsection{Dimensión}
%%%%%%%%%%%%%%%%%%%%%%%%%%%%%%%%%%%%%%


\begin{teo}
    Sean $(E,+,\cdot,\K)$ un espacio vectorial y $B\subseteq E$ una base de $E$. Si $S\subseteq E$ es un conjunto linealmente independiente, entonces 
    \[
        |S|\leq |B|.
    \]
\end{teo}

\begin{teo}
    Sean $(E,+,\cdot,\K)$ un espacio vectorial y $B\subseteq E$ una base de $E$. Si $S\subseteq E$ es un conjunto que genera a $E$, entonces 
    \[
        |B|\leq |S|.
    \]
\end{teo}


\begin{teo}
    Sean $(E,+,\cdot,\K)$ un espacio vectorial y $B,T\subseteq E$ bases de $E$. Se tiene que
    \[
        |B|=|T|.
    \]
\end{teo}

\begin{defi}[Dimensión]
    Sean $(E,+,\cdot,\K)$ un espacio vectorial y $B\subseteq E$ una base de $E$. Se define la dimensión de $E$, denotado por $\dim(E)$, por la cantidad de elementos de $B$.
\end{defi}

\begin{teo}
    Se tiene que
    \begin{itemize}
    \item
        $\dim(\R^n)=n$, con $n\in \N^*$;
    \item
        $\dim(\Mat[\R]{m}{n})=mn$, con $m,n\in \N^*$;
    \item
        $\dim(\R_n[x])=n+1$, con $n\in \N$;
    \item
        $\dim(\{0\}) = 0$.
    \end{itemize}
\end{teo}


\begin{teo}
    Sean $(E,+,\cdot,\K)$ un espacio vectorial y $S\subseteq E$ un conjunto linealmente independiente. Se tiene que existe una base $B$ de $E$ que contiene a $S$.
\end{teo}


\begin{teo}
    Sean $(E,+,\cdot,\K)$ un espacio vectorial, $B\subseteq E$ y $n\in\N^*$ tal que $\dim(E)=n$ y $|B|=n$. Se tiene que
    \begin{itemize}
    \item
        si $B$ es linealmente independiente, entonces $B$ es una base de $E$.
    \item
        si $B$ genera a $E$, entonces $B$ es una base de $E$.
    \end{itemize}
\end{teo}


%%%%%%%%%%%%%%%%%%%%%%%%%%%%%%%%%%%%%%
\subsection{Componentes y Cambio de base}
%%%%%%%%%%%%%%%%%%%%%%%%%%%%%%%%%%%%%%

En esta sección, asumiremos $n\in\N^*$.

\begin{defi}[Base ordenada]
    Sean $(E,+,\cdot,\K)$ un espacio vectorial y $B\subseteq E$ una base de $E$. Si numeramos los vectores de $B$, obteniendo
    \[
        B = \{v_1,v_2,\ldots,v_n\},
    \]
    se la llama una base ordenada.
\end{defi}

\begin{defi}[Vector de componentes]
    Sean $(E,+,\cdot,\K)$ un espacio vectorial y 
    \[
        B = \{v_1,v_2,\ldots,v_n\}\subseteq E
    \]
    una base ordenada. Para $u\in E$, se define el vector de componentes de $u$ por
    \[
        [u]_B = \begin{pmatrix}
            \alpha_1\\\alpha_2\\\vdots\\\alpha_n
        \end{pmatrix}
    \]
    donde $\alpha_1,\alpha_2,\ldots,\alpha_n\in\K$ son tales que
    \[
        u = \alpha_1 v_1+\alpha_2 v_2+\cdots+\alpha_n v_n.
    \]
\end{defi}

\begin{advertencia}
    En la literatura, también se puede encontrar la definición anterior bajo el nombre de vector de coordenadas.
\end{advertencia}

\begin{defi}[Matriz de cambio de base]
    Sean $(E,+,\cdot,\K)$ un espacio vectorial tal que $\dim(E)=n$ y $S,T\subseteq E$ bases ordenadas de $E$. Se llama matriz de cambio de base $T$ a $S$ a la única matriz de $\Mat[\K]{n}{n}$, denotada por $P_{S\gets T}$, que cumple que
    \[
        [v]_S = P_{S\gets T} [v]_T
    \]
    para todo $v\in E$.
\end{defi}

\begin{advertencia}
    En la literatura, también se puede encontrar la definición anterior bajo el nombre de matriz de transición.
\end{advertencia}


\begin{teo}
    Sean $(E,+,\cdot,\K)$ un espacio vectorial tal que $\dim(E)=n$, $S,T\subseteq E$ bases ordenadas de $E$ y $P_{S\gets T}$ la matriz de cambio de base $T$ a $S$. Se tiene que $P_{S\gets T}$ es no singular y
    \[
        (P_{S\gets T})^{-1} = P_{T\gets S}.
    \]
\end{teo}


%%%%%%%%%%%%%%%%%%%%%%%%%%%%%%%%%%%%%%
%% Espacios con producto interno
%%%%%%%%%%%%%%%%%%%%%%%%%%%%%%%%%%%%%%
\section{Espacios con producto interno}

\begin{defi}[Producto interno]
    Sea $(E,+,\cdot,\K)$ un espacio vectorial. Un producto interno sobre $E$ es una función
    \[
        \funcion{\langle \cdot, \cdot \rangle}{E\times E}{\K}{(u,v)}{\langle u, v \rangle}
    \]
    tales que cumple:
    \begin{enumerate}
        \item 
            $\langle v, v \rangle \geq 0$ para todo $ v \in E$;
        \item 
            $\langle v,v \rangle = 0$ si y solo si $ v = 0$;
        \item 
            $\langle u + v, w \rangle = \langle u,w \rangle + \langle v,w\rangle $ para todo $u,v,w \in E$;
        \item
            $\langle \alpha v,w \rangle = \alpha \langle v,w \rangle$ para todo $v,w \in E$ y $\alpha\in\K$.
        \item
            $\langle v,w \rangle = \overline{\langle w, v \rangle}$ para todo $v,w \in E$.
    \end{enumerate}
\end{defi}


\begin{advertencia}
    Otra notación para el producto interno es
    \[
        \langle u , v \rangle
        = (u|v).
    \]
\end{advertencia}

\begin{advertencia}
    En caso de tratarse de un espacio vectorial sobre el campo de los números reales, la propiedad 5 se transforma en
    \begin{enumerate}[start=5]
        \item
            $\langle v,w \rangle = \langle w, v \rangle$ para todo $v,w \in E$.
    \end{enumerate}
\end{advertencia}


\begin{advertencia}
    Se puede demostrar utilizando la propiedad 4 que
    \begin{itemize}
        \item
            si $v = 0$, entonces $\langle v,v \rangle = 0$;
    \end{itemize}
    por lo cual, podríamos cambiar la propiedad 2 por
    \begin{enumerate}[start=2]
        \item
            si $\langle v,v \rangle = 0$, entonces $ v = 0$
    \end{enumerate}
\end{advertencia}


Si se define un producto interno sobre un espacio vectorial $E$, a este se lo denomina espacio con producto interno o pre-Hilbertiano. 

\begin{teo}
    Sea $(E,+,\cdot,\K)$ un espacio vectorial provisto de producto interno $\langle \cdot, \cdot \rangle$, entonces:
    \begin{enumerate}
        \item 
            Para todo $u,v,w \in E$ 
            \[\langle u, v+w\rangle = \langle u,v\rangle + \langle u, w \rangle.\]
        \item 
            Para todo $u,v \in E$ y $\alpha \in \K$
            \[\langle u, \alpha v\rangle =  \overline{\alpha}\langle u,v\rangle.\]
        \item
            Para todo $u\in E$ 
            \[\langle u, 0 \rangle =  \langle 0, u \rangle  = 0.\]
    \end{enumerate}
\end{teo}

\begin{advertencia}
    En caso de tratarse de un espacio vectorial sobre el campo de los números reales, la propiedad 2 se transforma en
    \begin{enumerate}[start=2]
        \item
            Para todo $u,v \in E$ y $\alpha \in \K$
            \[\langle u, \alpha v\rangle =  \alpha\langle u,v\rangle\]
    \end{enumerate}
\end{advertencia}

\subsection{Productos internos usuales}

\begin{enumerate}
    \item En $(\R^n,+, \cdot, \R)$, para $x,y\in \R^n$:
    \[
        \langle x, y \rangle = \sum_{k=1}^n x_k y_k.
    \]
    
    \item En $(\C^n,+, \cdot, \C)$, para $x,y\in \C^n$:
    \[
        \langle x, y \rangle = \sum_{k=1}^n x_k \overline{y_k}.
    \]
    
    \item En $(\R_n[x],+, \cdot, \R)$, para $p(x),q(x)\in \R_n[x]$, si 
    \[  
        p(x) = a_0 + a_1x + \cdots + a_nx^n
        \texty
        q(x) = b_0 + b_1x + \cdots + b_nx^n
    \]
    entonces
    \[
        \langle p(x), q(x) \rangle = \sum_{k=0}^n a_k b_k.
    \]
    
    \item En $(\Mat[\K]{m}{n},+, \cdot, \R)$, para $A,B\in \Mat[\K]{m}{n}$:
    \[
        \langle A, B \rangle = \tr(AB^\intercal).
    \]
    
    \item En $(\mathcal{C}([a,b]),+, \cdot, \R)$, para $f,g\in \mathcal{C}([a,b])$:
    \[
        \langle f, g \rangle = \int_a^b f(x) g(x) dx.
    \]
    \end{enumerate}

\begin{defi}
Sea $(E,+,\cdot,\K)$ un espacio vectorial provisto de producto interno y suponga que $u, v \in E$.
Entonces:
\begin{enumerate}
    \item $u$ y $v$ son ortogonales si $\langle u, v \rangle = 0$.
    \item La norma de $u$, denotada por $\|u \|$, está dada por
    \[
        \| u\| = \sqrt{\langle u, u \rangle}
    \]
\end{enumerate}
\end{defi}

\begin{defi}
    Sea $(E,+,\cdot,\K)$ un espacio vectorial provisto de producto interno. La distancia en el espacio se define por
    \[
        \funcion{d}{E\times E}{\R}{(u,v)}{\| u - v\|.}
    \]
\end{defi}


\begin{teo}[Vectores ortogonales]
    Sea $(E,+,\cdot,\K)$ un espacio vectorial provisto de producto interno. Para  $u,v\in E$, se dice que son ortogonales si
    \[
        \langle u , v \rangle = 0.
    \]
\end{teo}


\begin{teo}[Teorema de Pitágoras]
    Sea $(E,+,\cdot,\K)$ un espacio vectorial provisto de producto interno.
    Si $u,v$ son vectores ortogonales de $E$, entonces
    \[
        \| u + v \|^2 = \|u\|^2 + \|v\|^2.
    \]
\end{teo}

\begin{teo}[Desigualdad de Cauchy-Schwartz]
    Sea $(E,+,\cdot,\K)$ un espacio vectorial provisto de producto interno.
    Para todo $u,v\in E$ se cumple que
    \[
        | \langle u , v \rangle | \leq \|u\|  \|v\|.
    \]
\end{teo}

\begin{teo}[Desigualdad triangular]
    Sea $(E,+,\cdot,\K)$ un espacio vectorial provisto de producto interno.
    Para todo $u,v\in E$ se cumple que
    \[
        \| u + v \| \leq \|u\| + \|v\|.
    \]
\end{teo}


\begin{defi}[Conjunto ortogonal]
    Sea $(E,+,\cdot,\K)$ un espacio vectorial provisto de producto interno y
    \[C=\{v_1, v_2, \ldots, v_n\}\subseteq E.\] Se dice que $C$ es un conjunto ortogonal en $E$ si
    \[
        \langle v_i, v_j \rangle = 0
    \]
    para todo $i \neq j$.
\end{defi}

\begin{defi}[Conjunto ortonormal]
    Sea $(E,+,\cdot,\K)$ un espacio vectorial provisto de producto interno y
    \[C = \{v_1, v_2, \ldots, v_n\} \subseteq E.\] Se dice que $C$ es un conjunto ortonormal en $E$ si es ortogonal y 
    \[
        \| v_k\| =  1
    \]
    para todo $k \in \{1, \ldots, n\}$.
\end{defi}


\begin{teo}
    Sea $(E,+,\cdot,\K)$ un espacio vectorial provisto de producto interno. Si $C \subseteq E$ es un conjunto ortogonal de vectores no nulos, entonces es linealmente independiente.
\end{teo}

%%%%%%%%%%%%%%%%%%%%%%%%%%%%%%%%%%%%%%
%% Bases ortogonales
%%%%%%%%%%%%%%%%%%%%%%%%%%%%%%%%%%%%%%

A partir de aquí, siempre consideraremos espacios vectoriales provistos con un producto interno.


\subsection{Bases ortogonales}

\begin{defi}[Base ortogonal (ortonormal)]
    En un espacio vectorial, una base ortogonal (ortonormal) es una base cuyos vectores forman un conjunto ortogonal (ortonormal). 
\end{defi}

\begin{teo}
    Sean $(E,+,\cdot,\K)$ un espacio vectorial y $\{u_1, u_2, \ldots, u_n\}$ una base ortogonal para $E$. Se tiene que, para $v \in E$,
    \[
        v = \alpha_1 u_1 + \alpha_2 u_2 +\cdots + \alpha_n u_n
    \]
    donde
    \[
        c_k = \dfrac{\langle v, u_k \rangle}{\langle u_k, u_k \rangle}
    \]
    para todo $k \in \{1, 2, \ldots, n\}$.
\end{teo}


\begin{teo}
    Sean $(E,+,\cdot,\K)$ un espacio vectorial y $\{u_1, u_2, \ldots, u_n\}$ una base ortonormal para $E$. Se tiene que, para $v \in E$,
    \[
        v = \alpha_1 u_1 + \alpha_2 u_2 +\cdots + \alpha_n u_n
    \]
    donde
    \[
        c_k = \langle v, u_k \rangle 
    \]
    para todo $k \in \{1, 2, \ldots, n\}$.
\end{teo}


\begin{teo}[Proceso de Gram-Schmidt]
    Sean $(E,+,\cdot,\K)$ un espacio vectorial y $\{u_1, u_2, \ldots, u_n\}$ un conjunto linealmente independiente de $E$. Definamos
    \begin{enumerate}
    \item 
        $v_1 = u_1$ y
    \item
        $\displaystyle v_k = 
            u_k 
            - \sum_{i=1}^{k-1} \dfrac{\langle u_k, v_i\rangle}{\langle v_i,v_i\rangle} v_i$, para $k=2,\ldots,n$.
    \end{enumerate}
    Se tiene que el conjunto
    \[
        \{v_1, v_2, \ldots, v_n\}
    \]
    es un conjunto ortogonal. Además, si definimos
    \[
        w_k = \dfrac{v_k}{\|v_k\|}
    \]
    para $k=1,\ldots,n$, se tiene que el conjunto
    \[
        \{w_1, w_2, \ldots, w_n\}
    \]
    es un conjunto ortonormal.
\end{teo}


\begin{teo}
    Todo espacio vectorial de dimensión finita tiene una base ortogonal y una base ortonormal.
\end{teo}


\begin{ejem}
    Considere la base $S = \{u_1, u_2, u_3\}$ de $\R^3$, donde 
    \[
        u_1 = (1,1,0), \qquad u_2 = (0, 1,1), \qquad u_3 = (1,0,1)
    \]
    mediante el proceso de Gram-Schmidt, se obtiene la base ortonormal $T=\{w_1, w_2, w_2\}$ para $\R^3$ donde
    \[
        w_1 = \left(\dfrac{1}{\sqrt{2}},\dfrac{1}{\sqrt{2}},0\right), \qquad
        w_2 = \left(-\dfrac{1}{\sqrt{6}},\dfrac{1}{\sqrt{6}},
        \dfrac{2}{\sqrt{6}}\right), \qquad
        w_3 = \left(\dfrac{1}{\sqrt{3}},-\dfrac{1}{\sqrt{3}},
        \dfrac{1}{\sqrt{3}}\right)
    \]
\end{ejem}

%%%%%%%%%%%%%%%%%%%%%%%%%%%%%%%%%%%%%% 13
%% Complemento ortogonal
%%%%%%%%%%%%%%%%%%%%%%%%%%%%%%%%%%%%%%
\subsection{Complemento ortogonal}

A partir de aquí, siempre consideraremos espacios vectoriales provistos con un producto interno.

\begin{defi}[Complemeto ortogonal]
    Sea $(E,+,\cdot,\K)$ un espacio vectorial y
    $H \subset E$. El complemento ortogonal de $H$, denotado por $H^\perp$, se define por:
    \[
        H^\perp = \{ x \in E: \langle x, h \rangle = 0, \text{ para todo } h \in H\}.
    \]
\end{defi}

\begin{teo}
    Sea $(E,+,\cdot,\K)$ un espacio vectorial y
    $H$ subespacio vectorial de $H$, entonces:
    \begin{enumerate}
        \item $H^\perp$ es un subespacio vectorial de $E$.
        \item $H \cap H^\perp = \{0\}$.
        \item Si $\dim(E) = n$, entonces $\dim(H^\perp) = n - \dim(H)$
    \end{enumerate}
\end{teo}


\begin{teo}
    Sea $(E,+,\cdot,\K)$ un espacio vectorial de dimensión finita, $W$ un subespacio vectorial de $E$, entonces 
    \[
    E = W \oplus W^\perp.
    \]
\end{teo}

\begin{teo}
    Sea $(E,+,\cdot,\K)$ un espacio vectorial de dimensión finita, $W$ un subespacio vectorial de $E$, entonces 
    \[
    \left(W^\perp\right)^\perp =  W.
    \]
\end{teo}

\begin{defi}[Proyección ortogonal]
    Sea $(E,+,\cdot,\K)$ un espacio vectorial y
    $H$ un subespacio vectorial de $E$, con base ortogonal $\{u_1, u_2, \ldots, u_n \}$. Para $v \in E$, la proyección ortogonal de $v$ sobre $H$, denotado por $\proy_H v$, se define por:
    \[
       \proy_H( v) = 
       \dfrac{\langle v,u_1 \rangle}{\langle u_1, u_1 \rangle}u_1 +
       \dfrac{\langle v,u_2 \rangle}{\langle u_2, u_2 \rangle}u_2 +
       \cdots +
       \dfrac{\langle v,u_n \rangle}{\langle u_n, u_n \rangle}u_n,
    \]
    donde $\proy_H (v) \in H$.
\end{defi}


\begin{teo}[Teorema de la proyección]
    Sea $(E,+,\cdot,\K)$ un espacio vectorial de dimensión finita, $H$ un subespacio vectorial de $E$ y $v \in E$. Se tiene que
    \[
        v = \proy_H( v) + \proy_{H^\perp}( v).
    \]
\end{teo}

\begin{teo}
    Sea $(E,+,\cdot,\K)$ un espacio vectorial de dimensión finita, $H$ un subespacio vectorial de $E$ y $v \in E$. Se tiene que el vector en $H$ más cercano a $v$ es $\proy_H(v)$, es decir,
    \[
    \|v-w\| \quad \text{ es mínima cuando } \quad w = \proy_W(v).
    \]
\end{teo}


%%%%%%%%%%%%%%%%%%%%%%%%%%%%%%%%%%%%%% 14
%% Aplicaciones lineales
%%%%%%%%%%%%%%%%%%%%%%%%%%%%%%%%%%%%%%
\section{Aplicaciones lineales}

\begin{defi}[Aplicación lineal]
    Sean $(E,+_1,\cdot_1,\K)$ y $(F,+_2,\cdot_2,\K)$ espacios vectoriales. A una función $\func{T}{E}{F}$ se la llama una aplicación lineal (transformación lineal) si satisface que para todo $\alpha \in \K$, y todo $u,v \in E$ se cumple
    \begin{enumerate}
    \item 
        $T(u +_1 v) = T(u) +_2 T(v)\quad$ y
    \item 
        $T(\alpha \cdot_1 v ) = \alpha \cdot_2 T(v)$.
    \end{enumerate}
\end{defi}

En adelante, consideraremos $(E,+_1,\cdot_1,\K)$ y $(F,+_2,\cdot_2,\K)$ espacios vectoriales. Notaremos por $\mathcal{L}(E,F)$ el espacio de las aplicaciones lineales de $E$ en $F$.

\begin{advertencia}
    En caso de que no exista riesgo de ambigüedad, dado $T \in \mathcal{L}(E,F)$, se denotará
    \begin{enumerate}
    \item 
        $T(u + v) = T(u) + T(v)\quad$ y
    \item 
        $T(\alpha v ) = \alpha T(v)$,
    \end{enumerate}
    para $\alpha \in \K$ y $u,v \in E$.
\end{advertencia}
    
\begin{teo}
     Sea $T \in \mathcal{L}(E,F)$. Para todo $u,v, v_1, v_2, \ldots, v_n \in E$ y $\alpha_1, \alpha_2, \ldots, \alpha_n \in \K$
     \begin{enumerate}
         \item $T(0_E) = 0_F$;
         \item $T(u-v) = T(u) - T(v)$: y
         \item $T\left(\displaystyle \sum_{k=1}^n \alpha_k v_k \right)= 
         \displaystyle \sum_{k=1}^n \alpha_k T\left(v_k \right)$.
     \end{enumerate}
\end{teo}

\subsection{Ejemplos de transformaciones lineales}
\begin{enumerate}
\item 
    Transformación nula, \[\funcion{T}{F}{F}{v}{0.}\]
\item 
    Transformación identidad, \[\funcion{T}{F}{F}{v}{v.}\]
\item 
    \[
        \funcion{T}{\R^3}{\R^2}
            {(x,y,z)}{(x, y).}
    \]
\item 
    \[
        \funcion{T}{\R^2}{\R^3}
            {(x,y)}{(x, y,0).}
    \]
\item 
    \[
        \funcion{T}{\R^3}{\R^3}
            {(x,y,z)}{(x+y, y + z,x).}
    \]
\item 
    \[
        \funcion{T}{\R^2}{\Mat{2}{2}}
            {(x,y)}{\begin{pmatrix}x& 0\\ 0 &y\end{pmatrix}.}
    \]
\item 
    \[
        \funcion{T}{\R^3}{\R_2[t]}
            {(x,y,z)}{x+(x-y)t + zt^2.}
    \]
\end{enumerate}



\begin{teo}
    Sea $T \in \mathcal{L}(E,F)$. Si $\{u_1,u_2,\ldots,u_n\}$ es una base de $E$, entonces $T$ está completamente determinada por
    \[
        \{T(u_1),T(u_2),\ldots,T(u_n)\}.
    \]
    Es decir, si se conoce el valor de $\{T(u_1),T(u_2),\ldots,T(u_n)\}$, entonces se conoce $T(u)$ para todo $u\in E$.
\end{teo}


%%%%%%%%%%%%%%%%%%%%%%%%%%%%%%%%%%%%%% 14
%% Núcleo e imagen
%%%%%%%%%%%%%%%%%%%%%%%%%%%%%%%%%%%%%%
\subsection{Núcleo e imagen}

\begin{defi}[Núcleo e imagen de una aplicación lineal]
    Sea $T \in \mathcal{L}(E,F)$.
    \begin{enumerate}
    \item 
        El núcleo de $T$, denotado por $\ker(T)$, está definida por:
        \[
            \ker(T) = \{v \in E: T(v) = 0\}.
        \]
    \item 
        La imagen de $T$, denotada por $\img(T)$, está definida por:
        \[
            \img(T) = \{ w \in F: w = T(v) \text{ para algún } v \in E\}.
        \]
    \end{enumerate}
\end{defi}

\begin{teo}
    Sea $T \in \mathcal{L}(E,F)$, entonces
    \begin{enumerate}
    \item 
        $\ker(T)$ es un subespacio vectorial de $E$.
    \item 
        $\img(T)$ es un subespacio vectorial de $F$.
    \end{enumerate}
\end{teo}
    
\begin{teo}
    Sea $T \in \mathcal{L}(E,F)$. Se tiene que $T$ es inyectiva si y solo si $\ker(T)=\{0\}$.
\end{teo}

\begin{defi}[Nulidad y rango de una aplicación lineal]
    Sea $T \in \mathcal{L}(E,F)$.
    \begin{enumerate}
    \item 
        Se llama nulidad de $T$ a $\dim(\ker(T))$.
    \item 
        Se llama rango de $T$ a $\dim(\img(T))$.
    \end{enumerate}
\end{defi}

\begin{teo}
    Sea $T \in \mathcal{L}(E,F)$ con $E$ un espacio de dimensión finita. Se tiene que
    \[
        \dim(E) = \dim(\ker(T)) +  \dim(\img(T)).
    \]
\end{teo}   

\begin{teo}
    Sea $T \in \mathcal{L}(E,F)$ con $E$ un espacio de dimensión finita. Si $\dim(E)=\dim(F)$, entonces se tiene que la siguientes son equivalentes
    \begin{itemize}
        \item $T$ es inyectiva,
        \item $T$ es sobreyectiva.
    \end{itemize}
\end{teo}   

%%%%%%%%%%%%%%%%%%%%%%%%%%%%%%%%%%%%%% 14
%% Aplicaciones lineales
%%%%%%%%%%%%%%%%%%%%%%%%%%%%%%%%%%%%%%
\subsection{Propiedades de aplicaciones lineales}

\begin{teo}
    Se tiene que $\mathcal{L}(E,F)$ es un espacio vectorial.
\end{teo}


\begin{teo}
    Sean $T_1,T_2 \in \mathcal{L}(E,F)$ y $B= \{v_1, v_2, \ldots, v_n\}$ una base para $E$. Si
    \[
        T_1(v_i) = T_2(v_i)
    \]
    para todo $i \in \{1, 2, \ldots, n\}$, entonces se tiene que $T_1(v)=T_2(v)$, para todo $v\in E$, es decir, 
    \[
        T_1 = T_2.
    \]
\end{teo}

\begin{teo}
    Sean $B= \{v_1, v_2, \ldots, v_n\}$ una base para $E$ y $w_1, w_2, \ldots, w_n\in F$. Se tiene que existe una única transformación lineal $T \in \mathcal{L}(E,F)$ tal que
    \[
        T(v_i) = w_i
    \]
    para todo $i\in \{1, 2, \ldots, n\}$.
\end{teo}


\begin{teo}
    Sea $T \in \mathcal{L}(E,F)$. Supongamos que $\dim(E) = n$ y $\dim(F) = m$, se tiene que:
    \begin{enumerate}
        \item si $n > m$, entonces $T$ no es inyectiva; y
        \item si $m > n$, entonces $T$ no es sobreyectiva.
    \end{enumerate}
\end{teo} 

%%%%%%%%%%%%%%%%%%%%%%%%%%%%%%%%%%%%%%
\subsection{Isomorfismos}
%%%%%%%%%%%%%%%%%%%%%%%%%%%%%%%%%%%%%%


\begin{defi}[Isomorfismo]
    Sea $T \in \mathcal{L}(E,F)$. Se dice que $T$ es un isomorfismo de $E$ en $F$ si $T$ es biyectiva.
\end{defi}

\begin{defi}[Espacios isomorfos]
    Sean $(E,+,\cdot,\K)$ y $(F,+,\cdot,\K)$. Se dice que $E$ y $F$ son isomorfos si existe un isomorfismo $T$ de $E$ en $F$, se lo denota por $E \cong F$.
\end{defi}

\begin{teo}
    Sea $T \in \mathcal{L}(E,F)$ un isomorfismo, se tiene que
    \begin{enumerate}
    \item 
        si $\{v_1, v_2, \ldots, v_n\}$ genera a $E$, entonces 
        \[
            \{T(v_1), T(v_2), \ldots, T(v_n)\}
        \]
        genera a $F$;
    \item 
        si $\{v_1, v_2, \ldots, v_n\}$ son linealmente independientes en $E$, entonces 
        \[
            \{T(v_1), T(v_2), \ldots, T(v_n)\}
        \] 
        es linealmente independientes en $F$;
    \item 
        si $\{v_1, v_2, \ldots, v_n\}$ es base de $E$, entonces
        \[
            \{T(v_1), T(v_2), \ldots, T(v_n)\}
        \]
        es base de $F$; 
    \item 
        si $E$ es de dimensión finita, entonces $F$ es de dimensión finita y 
        \[
            \dim(E) = \dim(F).
        \]
    \end{enumerate}
\end{teo} 

\begin{teo}
    Sean $(E,+,\cdot,\R)$ y $(F,+,\cdot,\R)$ espacios de dimensión finita tales que $\dim(E) = \dim(F)$, entonces $E \cong F$.
\end{teo}

\begin{teo}
    Se tiene que
    \begin{enumerate}
        \item $\Mat{m}{n} \cong \R^{mn}$, con $m,n\in\N^*$;
        \item $P_n[x] \cong \R^{n+1}$, con $n\in\N^*$.
    \end{enumerate}
\end{teo}


%%%%%%%%%%%%%%%%%%%%%%%%%%%%%%%%%%%%%% 14
%% Matriz asociada
%%%%%%%%%%%%%%%%%%%%%%%%%%%%%%%%%%%%%%
\subsection{Matriz asociada}

En adelante, consideraremos $E$ y $F$ espacios vectoriales de dimensión finita tales que $\dim(E)=n$ y $\dim(F)=m$.

\begin{defi}
    Sean $T \in \mathcal{L}(E,F)$ y $B$, $D$ bases para $E$ y $F$, respectivamente. Se tiene que existe una única matriz de $\Mat[\K]{m}{n}$, denotada por $[T]_{D,B}$, tal que
    \[
        [T(v)]_{D} = [T]_{D,B} [v]_{B}
    \]
    para todo $v\in E$. A esta matriz se la llama matriz asociada a la aplicación lineal $T$.
\end{defi}


\begin{teo}
    Sean $T \in \mathcal{L}(E,F)$ y 
    \[
        B = \{v_1,v_2,\ldots,v_n\}
        \texty
        D = \{u_1,u_2,\ldots,u_m\}
    \]
    bases para $E$ y $F$, respectivamente. Se tiene que las columnas de la matriz $[T]$ son los vectores de coordenadas de $T(v_j)$ en la base $D$, para $j\in\{1,2,\ldots,n\}$, es decir
    \[
        [T]_{D,B} = \begin{pmatrix}
            [T(v_1)]_D & [T(v_2)]_D & \cdots & [T(v_n)]_D
        \end{pmatrix}.
    \]
\end{teo}

\begin{advertencia}
    En caso de no existir riesgo de confusión en las bases que se utiliza, se nota simplemente $[T]$ a la matriz asociada a la aplicación lineal $T$. 
\end{advertencia}


Se pueden ver ejemplos del procedimiento para encontrar una matriz asociada a una aplicación lineal entre las páginas 522 y 527 del libro de Kolman.


\begin{teo}
    Sean $I \in \mathcal{L}(E,E)$ la aplicación identidad y $B,D$ dos bases para $E$, considerando $B$ para el conjunto de salida y $D$ para el conjunto de llegada. Se tiene que
    \[
        [I]_{D,B} = P_{D\leftarrow B}.
    \]
\end{teo}



\begin{teo}
    Sea $T \in \mathcal{L}(E,F)$. Se tiene que $T$ es biyectiva si y solo si $[T]$ es invertible.
\end{teo}



\begin{teo}
    Sean $T_1 \in \mathcal{L}(E,F)$ y $T_2\in\mathcal{L}(F,G)$, con $G$ un espacio vectorial de dimensión finita. Se tiene que
    \[
        [T_2\circ T_1] = [T_2][T_1].
    \]
\end{teo}


\begin{teo}
    Sea $T\in \mathcal{L}(E,F)$ una aplicación lineal invertible. Se tiene que
    \[
        [T^{-1}] = [T]^{-1}.
    \]
\end{teo}


\begin{teo}
    Se tiene que $\mathcal{L}(E,F)$ es isomorfo a $\Mat[\K]{m}{n}$, es decir
    \[
        \mathcal{L}(E,F)
        \cong
        \Mat[\K]{m}{n}.
    \]
\end{teo}

%%%%%%%%%%%%%%%%%%%%%%%%%%%%%%%%%%%%%% 14
%% Valores y vectores propios
%%%%%%%%%%%%%%%%%%%%%%%%%%%%%%%%%%%%%%
\section{Valores y vectores propios}

\begin{defi}
    Sea $A\in \Mat[\K]{n}{n}$, se tiene que $\lambda\in \K$ es un valor propio de $A$ si existe $v\in \R^n$, con $v\neq 0$, tal que
    \[
        Av = \lambda v,
    \]
    además, $v$ es un vector propio de $A$ asociado al valor propio $\lambda$.
\end{defi}


\begin{teo}
   Sean $A\in \Mat[\K]{n}{n}$ y $\lambda \in \K$. Se tiene que $\lambda$ es un valor propio de $A$ si y solo si
   \[
        \det(A - \lambda I) = 0,
   \]
   donde $I$ es la matriz identidad.
\end{teo}

\begin{teo}
    Los valores propios de una matriz triangular son los elementos de la diagonal de la matriz. 
\end{teo}


\begin{defi}
    Sea $A\in \Mat[\K]{n}{n}$. Al polinomio
    \[
        p_A(\lambda) = \det(A - \lambda I)
    \]
    se lo denomina polinomio característico de $A$.
\end{defi}


\begin{teo}[Teorema de Cayley–Hamilton]
    Sea $A\in \Mat[\K]{n}{n}$ y $p_A(\lambda)$ su polinomio característico. Se tiene que
    \[
        p_A(A) = 0.
    \]
\end{teo}

%%%%%%%%%%%%%%%%%%%%%%%%%%%%%%%%%%%%%% 14
%% Aplicaciones lineales
%%%%%%%%%%%%%%%%%%%%%%%%%%%%%%%%%%%%%%
\subsection{Valores y vectores propios de aplicaciones lineales}

\begin{defi}
     Sean $T \in \mathcal{L}(E,E)$ y $\lambda \in\K$. Se dice que $\lambda$ es un valor propio de $T$ si existe $v\in E$, con $v\neq 0$, tal que
     \[
        T(v) = \lambda v.
     \]
     A este vector $v$ se lo llama vector propio de $T$ asociado al valor propio $\lambda$.
\end{defi}

\begin{advertencia}
    Note que, por definición, el vector $0$ no puede ser un vector propio asociado a una aplicación lineal.
\end{advertencia}

\begin{teo}
    Sean $T \in \mathcal{L}(E,E)$ y $\lambda \in\K$ un valor propio de $T$. Se tiene que el conjunto
    \[
        E_\lambda = \{ v\in E : T(v) = \lambda v\} 
    \]
    es un subespacio vectorial de $E$. Se lo denomina el espacio propio de $T$ asociado al valor propio $\lambda$.
\end{teo}


\begin{teo}
    Sean $T \in \mathcal{L}(E,E)$ y $\lambda_1, \lambda_2, \ldots, \lambda_m$ valores propios de $T$ distintos entre sí. Si tomamos $v_1, v_2, \ldots, v_m$ vectores propios asociados a los valores propios $\lambda_1, \lambda_2, \ldots, \lambda_m$, respectivamente, se tiene que $\{v_1, v_2, \ldots, v_n\}$ es un conjunto linealmente independiente. 
\end{teo}


En adelante, consideraremos a $E$ como un espacio vectorial de dimensión finita.

\begin{defi}
     Sea $T \in \mathcal{L}(E,E)$. Se denomina espectro de $T$ al conjunto de todos los valores propios de $T$.
\end{defi}


\begin{teo}
    Sea $T \in \mathcal{L}(E,E)$. Se tiene 
    \begin{enumerate}
    \item
        $\lambda\in\K$ es un valor propio de $T$ si y solo si es un valor propio de $[T]$; y
    \item
        $v\in E$ es un vector propio de $T$ si y solo si $[v]$ es un vector propio de $[T]$.
    \end{enumerate}
\end{teo}


\begin{advertencia}
    De este resultado, se sigue que, para determinar los valores, vectores y espacios propios de un operador lineal, basta determinar los valores, vectores y espacios propios de una matriz.
\end{advertencia}

%%%%%%%%%%%%%%%%%%%%%%%%%%%%%%%%%%%%%% 14
%% Diagonalización de matrices
%%%%%%%%%%%%%%%%%%%%%%%%%%%%%%%%%%%%%%
\subsection{Diagonalización de Matrices}

\begin{defi}
    Sean $A,B\in \Mat[\K]{n}{n}$. Se dice que $A$ es semejante a $B$ si existe una matriz invertible $P\in \Mat[\K]{n}{n}$ tal que
    \[
        A = P^{-1} B P.
    \]
\end{defi}


\begin{teo}
    Sean $A,B,C\in \Mat[\K]{n}{n}$. Se tiene que
    \begin{enumerate}
    \item 
        $A$ es semejante a $A$.
    \item
        Si $A$ es semejante a $B$, entonces $B$ es semejante a $A$.
    \item
        Si $A$ es semejante a $B$ y $B$ es semejante a $C$, entonces $A$ es semejante a $C$.
    \end{enumerate}
\end{teo}

\begin{defi}
    Sea $A\in \Mat[\K]{n}{n}$, se dice que $A$ es diagonalizable si $A$ es semejante a una matriz diagonal. Es decir, si existe una matriz invertible $P\in \Mat[\K]{n}{n}$ y una matriz diagonal $D\in \Mat[\K]{n}{n}$ tal que
    \[
        A = P^{-1} D P.
    \]
\end{defi}

\begin{teo}
    Sean $A,B\in \Mat[\K]{n}{n}$. Si $A$ y $B$ son semejantes, entonces $A$ y $B$ tienen los mismos valores propios, es decir, tienen el mismo espectro.
\end{teo}

\begin{teo}
    Sea $A\in \Mat[\K]{n}{n}$. Se tiene que $A$ es diagonalizable si y solo si tiene $n$ vectores propios linealmente independientes.
\end{teo}

\begin{teo}
    Sea $A\in \Mat[\K]{n}{n}$. Si $A$ tiene $n$ valores propios distintos entre sí, entonces $A$ es diagonalizable.
\end{teo}

\begin{teo}
    Sea $A\in \Mat[\K]{n}{n}$. Si $A$ es diagonalizable, entonces tomando $D$ la matriz diagonal formada por los valores propios de $A$, y $P$ la matriz cuyas columnas son los vectores propios de $A$, se tiene que
    \[
        A = P^{-1} D P.
    \]
\end{teo}

\begin{teo}
    Sea $A\in \Mat[\K]{n}{n}$. Si $A$ es simétrica, entonces $A$ todos sus valores propios son reales; además, $A$ es diagonalizable.
\end{teo}

\begin{defi}
    Sea $A\in \Mat[\K]{n}{n}$. Se dice que $A$ es ortogonal si $A^\intercal = A^{-1}$.
\end{defi}


\begin{teo}
    Sea $A\in \Mat[\K]{n}{n}$. Si $A$ es simétrica, y $D,P\in \Mat[\K]{n}{n}$ son tales que
    \[
        A = P^{-1} D P,
    \]
    entonces $P$ es ortogonal.
\end{teo}


\end{document}