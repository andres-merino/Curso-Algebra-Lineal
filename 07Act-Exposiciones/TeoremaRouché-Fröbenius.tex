\documentclass[a4,11pt]{aleph-notas}

% -- Paquetes adicionales
\usepackage{aleph-comandos}


% -- Datos 
\institucion{Escuela de Ciencias Físicas y Matemática}
\carrera{Ciencia de datos}
\asignatura{Álgebra lineal}
\tema{Video exposición no. 1: El Teorema de Rouché-Fröbenius}
\autor{Andrés Merino}
\fecha{Semestre 2024-1}

\logouno[0.14\textwidth]{Logos/logoPUCE_04_ac}
\definecolor{colortext}{HTML}{0030A1}
\definecolor{colordef}{HTML}{0030A1}
\fuente{montserrat}


% -- Otros comandos




\begin{document}

\encabezado

\vspace*{-10mm}
%%%%%%%%%%%%%%%%%%%%%%%%%%%%%%%%%%%%%%%%
\section{Resultado de aprendizaje}
%%%%%%%%%%%%%%%%%%%%%%%%%%%%%%%%%%%%%%%%

\begin{itemize}
\item 
    Identifica los elementos de los sistemas de ecuaciones lineales y la clasificación de estos por su tipo de soluciones.
\end{itemize}


%%%%%%%%%%%%%%%%%%%%%%%%%%%%%%%%%%%%%%%%
\section{Instrucciones}
%%%%%%%%%%%%%%%%%%%%%%%%%%%%%%%%%%%%%%%%

\begin{enumerate}
    \item Preparación del Material: Se deberá preparar una presentación en formato de video donde expliquen de manera clara y concisa los conceptos relacionados con los tipos de soluciones de sistemas de ecuaciones y la aplicación del Teorema de Rouché-Frobenius para determinar dichos tipos.
    \item Contenido del Video:
    \begin{itemize}
        \item Introducción: Se deberá introducir el tema explicando qué son los sistemas de ecuaciones lineales y por qué es importante estudiar sus soluciones.
        \item Tipos de Soluciones: Se deberá explicar detalladamente los tres posibles tipos de soluciones que puede tener un sistema de ecuaciones: única, infinitas y ninguna solución. Cada tipo debe ser ilustrado con ejemplos concretos acompañado de una representación gráfica en tres dimensiones.
        \item Teorema de Rouché-Frobenius: Se deberá presentar el Teorema de Rouché-Frobenius, explicando sus condiciones de aplicabilidad y cómo se utiliza para determinar los tipos de soluciones de un sistema de ecuaciones.
        \item Ejemplos Prácticos: Se deberá incluir ejemplos numéricos o casos específicos donde se aplique el Teorema de Rouché-Frobenius para identificar los tipos de soluciones de sistemas de ecuaciones.
    \end{itemize}
    \item Claridad y Profundidad: Se valorará la claridad en la exposición, así como la profundidad en la explicación de los conceptos. Los estudiantes deben asegurarse de que su presentación sea comprensible para un público que tenga conocimientos básicos de álgebra lineal.
    \item Duración del Video: La duración del video no deberá exceder los 7 minutos.
\end{enumerate}

%%%%%%%%%%%%%%%%%%%%%%%%%%%%%%%%%%%%%%%%
\section{Entrega de la actividad}
%%%%%%%%%%%%%%%%%%%%%%%%%%%%%%%%%%%%%%%%

Se deberá entregar el enlace al video de la exposición a través del aula virtual antes de la fecha límite establecida.


%%%%%%%%%%%%%%%%%%%%%%%%%%%%%%%%%%%%%%%%
\section{Guía de calificación}
%%%%%%%%%%%%%%%%%%%%%%%%%%%%%%%%%%%%%%%%

\begin{itemize}
\item
    Explicación de los Tipos de Soluciones (11 puntos):
    \begin{itemize}
        \item 11 puntos: La exposición comprende una explicación clara y completa de los tres tipos de soluciones de sistemas de ecuaciones lineales: única, infinitas y ninguna solución. Se proporcionan ejemplos concretos y se ilustran gráficamente.
        \item 6 puntos: La exposición presenta una explicación adecuada de los tipos de soluciones, pero puede haber algunas deficiencias en la claridad o completitud de la explicación o en la presentación de ejemplos.
        \item 0 puntos: La explicación de los tipos de soluciones es insuficiente o incorrecta en la mayoría de los casos.
    \end{itemize}

\item
    Ejemplo gráfico (11 puntos):
    \begin{itemize}
        \item 11 puntos: Se incluye un ejemplo gráfico, claro y preciso, que ilustra uno o más tipos de soluciones de sistemas de ecuaciones lineales. El ejemplo gráfico complementa la explicación verbal.
        \item 6 puntos: Se presenta un ejemplo gráfico, pero puede haber algunas deficiencias en la claridad o relevancia del ejemplo en relación con los conceptos explicados.
        \item 0 puntos: No se incluye un ejemplo gráfico o el ejemplo presentado no es relevante para los conceptos explicados.
    \end{itemize}

\item
    Explicación del Teorema de Rouché-Frobenius (11 puntos):
    \begin{itemize}
        \item 11 puntos: Se proporciona una explicación clara y precisa del Teorema de Rouché-Frobenius, incluyendo sus condiciones de aplicabilidad y cómo se utiliza para determinar los tipos de soluciones de sistemas de ecuaciones lineales.
        \item 6 puntos: La explicación del teorema es adecuada, pero puede haber algunas deficiencias en la claridad o completitud de la explicación.
        \item 0 puntos: La explicación del teorema es insuficiente o incorrecta en la mayoría de los casos.
    \end{itemize}

\item
    Ejemplos prácticos (11 puntos):
    \begin{itemize}
        \item 11 puntos: Se presentan ejemplos prácticos que muestran la aplicación del Teorema de Rouché-Frobenius para determinar los tipos de soluciones de sistemas de ecuaciones lineales. Los ejemplos son claros, relevantes y están bien explicados.
        \item 6 puntos: Se incluyen ejemplos prácticos, pero puede haber algunas deficiencias en la claridad o relevancia de los ejemplos o en la explicación asociada.
        \item 0 puntos: No se presentan ejemplos prácticos o los ejemplos presentados no son relevantes o están mal explicados.
    \end{itemize}


\item
    Excelencia (6 puntos), este puntaje es dado únicamente si se tiene el puntaje completo en los otros puntos. Se premiará la creatividad en la presentación del contenido, la originalidad en la elección de ejemplos o la inclusión de elementos visuales innovadores que enriquezcan la experiencia del espectador. Se evaluará la claridad en la comunicación de los conceptos, la fluidez en la presentación y la capacidad para mantener el interés del espectador a lo largo del video.
    \begin{itemize}
        \item 6 puntos: El video es excelente.
        \item 0 puntos: El video no es excelente.
    \end{itemize}

\end{itemize}



\end{document}