\documentclass[a4,11pt]{aleph-notas}

% -- Paquetes adicionales
\usepackage{aleph-comandos}


% -- Datos  
\institucion{Escuela de Ciencias Físicas y Matemática}
\carrera{Ciencia de datos}
\asignatura{Álgebra lineal}
\tema{Taller no. 1: ¿Cómo sumar dos matrices?}
\autor{Andrés Merino - Mario Cueva}
\fecha{Semestre 2023-1}

\logouno[0.14\textwidth]{Logos/logoPUCE_04_ac}
\definecolor{colortext}{HTML}{0030A1}
\definecolor{colordef}{HTML}{0030A1}
\fuente{montserrat}


% -- Otros comandos




\begin{document}

\encabezado

%%%%%%%%%%%%%%%%%%%%%%%%%%%%%%%%%%%%%%%%
\section{Objetivos}
%%%%%%%%%%%%%%%%%%%%%%%%%%%%%%%%%%%%%%%%

\begin{itemize}
\item 
    Aplicar la definición de suma de matrices.
\item
    Conocer los comandos básicos de la programación en Python.
\end{itemize}


%%%%%%%%%%%%%%%%%%%%%%%%%%%%%%%%%%%%%%%%
\section{Descripción}
%%%%%%%%%%%%%%%%%%%%%%%%%%%%%%%%%%%%%%%%

Por alguna razón, la librería \texttt{sympy} presenta un problema al momento de sumar dos matrices. Dado que esta función nos es de gran utilidad y la necesitaremos para rendir el primer examen de la asignatura de Álgebra Lineal de la carrera de Ciencia de Datos, cae en nuestras manos el reto de generar el código necesario para sumar dos matrices de manera automática en Python.

Para guiarnos, consideremos las siguientes preguntas:
\begin{itemize}
\item 
    ¿Existen condiciones que se deban cumplir para sumar dos matrices?
\item
    Al momento de sumar dos matrices, ¿existe algún proceso que se repita?
\item
    ¿Cómo puedo hacer que Python determine si una condición se cumple o no?
\item
    ¿Cómo puedo hacer que Python repita un proceso de manera automática?
\end{itemize}

Cabe recalcar que no se puede utilizar otra librería y que únicamente se deberá usar el operador $+$ para sumar dos números.


%%%%%%%%%%%%%%%%%%%%%%%%%%%%%%%%%%%%%%%%
\section{Producto}
%%%%%%%%%%%%%%%%%%%%%%%%%%%%%%%%%%%%%%%%

Se deberá producir una Jupyter Notebook, en el cual, en una celda, se ingresen dos matrices y en otra celda se genere el resultado de sumar estas matrices. Además, se deberá generar un video explicando la utilización del producto, con la presentación de ejemplos.

Se debe entregar el archivo .ipynb mediante el aula virtual, dentro del archivo, debe constar un enlace al video de presentación.

%%%%%%%%%%%%%%%%%%%%%%%%%%%%%%%%%%%%%%%%
\section{Rúbrica de evaluación}
%%%%%%%%%%%%%%%%%%%%%%%%%%%%%%%%%%%%%%%%

\begin{itemize}
\item
    Jupyter Notebook: calificado sobre 6 puntos bajo los siguientes parámetros:
    \begin{itemize}
        \item El producto da la respuesta correcta, en todos los casos, 6 puntos.
        \item El producto da la respuesta correcta, en la mayoría de casos, pero no en todos, 3 punto.
        \item El producto no da la respuesta correcta, en la mayoría de casos, 0 puntos.
    \end{itemize}

\item
    Documentación del Jupyter Notebook: calificado sobre 3 puntos bajo los siguientes parámetros:
    \begin{itemize}
        \item Existen comentarios que explican detalladamente lo que hace el producto, 3 puntos.
        \item No se comenta a detalle la totalidad de lo que hace el producto, 1 punto.
        \item No existen comentarios, 0 puntos.
    \end{itemize}

\item
    Presentación: calificado sobre 3 puntos bajo los siguientes parámetros:
    \begin{itemize}
        \item Se expone de manera comprensible lo que hace el producto y se presenta ejemplos, 3 puntos.
        \item La presentación del producto no está completa, 1 punto.
        \item No se existe presentación, 0 puntos.
    \end{itemize}
    
\end{itemize}

%%%%%%%%%%%%%%%%%%%%%%%%%%%%%%%%%%%%%%%%
\section{Grupos}
%%%%%%%%%%%%%%%%%%%%%%%%%%%%%%%%%%%%%%%%

\begin{itemize}
\item 
    Grupo A:\\
    	JONATHAN DAVID CARRILLO TORRES,\\ ANDRéS NICOLáS NOLIVOS GIRALDO,\\ ARELLYS ANAHí SORIA QUINGA 
\item 
    Grupo B:\\
    	MARTIN EDUARDO MOLINA POLO,\\ ARLETH DANIELA RODRíGUEZ
\item 
    Grupo C:\\
    	ADRIAN NICOLAS RUBIO TERAN,\\ ETHAN ALEJANDRO TAIPE BONILLA 
\end{itemize}

\end{document}