\documentclass[a4,11pt]{aleph-notas}

% -- Paquetes adicionales
\usepackage{aleph-comandos}
\hypersetup{
    urlcolor=colordef!70,
}

% -- Datos  
\institucion{Escuela de Ciencias Físicas y Matemática}
\carrera{Ciencia de datos}
\asignatura{Álgebra lineal}
\tema{Reto no. 2: Análisis de Componentes Principales}
\autor{Andrés Merino}
\fecha{Semestre 2024-1}

\logouno[0.14\textwidth]{Logos/logoPUCE_04_ac}
\definecolor{colortext}{HTML}{0030A1}
\definecolor{colordef}{HTML}{0030A1}
\fuente{montserrat}


% -- Otros comandos




\begin{document}

\encabezado

%%%%%%%%%%%%%%%%%%%%%%%%%%%%%%%%%%%%%%%%
\section{Resultado de aprendizaje}
%%%%%%%%%%%%%%%%%%%%%%%%%%%%%%%%%%%%%%%%

\begin{itemize}
\item 
    Aplica métodos de Álgebra Lineal para resolver sistemas de ecuaciones en contextos variados, demostrando habilidad para seleccionar y emplear el método más adecuado.
\end{itemize}


%%%%%%%%%%%%%%%%%%%%%%%%%%%%%%%%%%%%%%%%
\section{Descripción}
%%%%%%%%%%%%%%%%%%%%%%%%%%%%%%%%%%%%%%%%

El análisis de componentes principales (ACP) es una técnica estadística poderosa utilizada para reducir la dimensionalidad de grandes conjuntos de datos, preservando al mismo tiempo la mayor cantidad posible de la variabilidad de los datos. Esta técnica es fundamental en el campo del álgebra lineal y se basa en conceptos de espacios vectoriales y transformaciones lineales. Al transformar las variables originales en nuevas variables ortogonales llamadas componentes principales, ACP facilita la visualización y la interpretación de datos complejos y multidimensionales. Esta metodología no solo es relevante para el análisis exploratorio de datos, sino que también juega un papel crucial en la mejora del rendimiento de modelos predictivos al eliminar redundancias y reducir el ruido. Su aplicación se extiende a una variedad de campos como la genómica, la economía, la psicología y más, donde la interpretación y reducción eficiente de datos es esencial.

El reto propuesto culminará en la creación de tres productos: un documento detallado, un video explicativo y un notebook de Jupyter. El documento proporcionará una explicación clara y concisa sobre qué es el Análisis de Componentes Principales (ACP) y cómo se lleva a cabo, abarcando tanto la teoría subyacente como su aplicación práctica. Este será complementado por un video, que visualizará los conceptos y procedimientos descritos en el documento, facilitando así una comprensión más intuitiva y dinámica del tema. Por último, el notebook de Jupyter permitirá la experimentación directa con el ACP, incluyendo códigos ejecutables para análisis de datos reales. Juntos, estos materiales no solo serán de utilidad para los estudiantes y profesionales que buscan profundizar en técnicas avanzadas de reducción de datos, sino que también servirán como referencia educativa accesible y práctica para cualquier persona interesada en aplicaciones estadísticas y de ciencia de datos.


\subsection{Parte 1: Investigación}

Vas a explorar qué es el Análisis de Componentes Principales (ACP), sus fundamentos matemáticos y el procedimiento para ejecutarlo, desarrollando un minitutorial (video) que sirva como guía de aprendizaje. En esta parte, debes analizar la relación (donde interviene) cada uno de estos temas en el ACP:

\begin{itemize}
\item
    Bases y dimensión.
\item
    Proyección ortogonal.
\item
    Transformación lineal.
\item 
    Valores y vectores propios.
\end{itemize}

Para este punto, puedes usar estas referencias, pero es necesario también incluir referencias a libros o artículos: 
\begin{itemize}
    \item \href{https://youtu.be/FgakZw6K1QQ?si=a9cxs4ePio-H6MXm}{StatQuest: Análisis de componentes principales (PCA), paso a paso}.
    \item \href{https://youtu.be/7My_PBhxeP4?si=zQf5Hr18NOb6cerX}{Análisis de componentes principales (PCA)}.
\end{itemize}


\subsection{Parte 2: Aplicación Práctica}

el Análisis de Componentes Principales (ACP) a un conjunto de datos de emisiones de vehículos en condiciones reales, demostrando comprensión y habilidad en el uso de este método.

Para esto, descarga el conjunto de datos desde \href{https://www.kaggle.com/datasets/konradb/real-world-vehicle-emissions}{Kaggle}, específicamente el archivo llamado «2021 Cars Raw.csv». Luego, selecciona las columnas cuantitativas del dataset y realiza un ACP para dos y tres componentes utilizando el paquete de Python, Scikit-learn. Es fundamental que presentes gráficos de los resultados, mostrando la distribución de los puntos en el espacio de los componentes principales.

Además, es importante que describas las bases vectoriales para cada uno de los casos de componentes principales y verifiques manualmente la proyección de al menos uno de los datos sobre estas bases.

%%%%%%%%%%%%%%%%%%%%%%%%%%%%%%%%%%%%%%%%
\section{Producto}
%%%%%%%%%%%%%%%%%%%%%%%%%%%%%%%%%%%%%%%%

El producto final de este reto será un documento, redactado en \LaTeX, que constará de tres secciones principales. A continuación, se detallan las instrucciones específicas para cada sección del documento.


\subsection{Parte 1: Investigación y Minitutorial}

\begin{enumerate}
\item
    Título de la Sección: Comienza esta sección con un título claro y directo, por ejemplo, «Introducción al Análisis de Componentes Principales».
\item
    Definición y explicación: Inicia con una introducción breve a qué es el Análisis de Componentes Principales (ACP) y su relevancia en el contexto de la reducción de dimensionalidad y el análisis de datos multivariados. Explica los principios matemáticos del ACP y su relación con el mismo, incluyendo bases y dimensión, proyección ortogonal, transformación lineal, valores y vectores propios.
\item
    Minitutorial (video): Bajo un nuevo subtítulo «Minitutorial sobre el Análisis de Componentes Principales», incluye un enlace a un video tutorial que hayas creado para complementar el contenido escrito. Utiliza \verb+\href{url}{texto enlace}+ para añadir el enlace. Asegúrate de que el enlace sea accesible para tus revisores y que el video sea claro y educativo.
\end{enumerate}

\subsection{Parte 2: Aplicación Práctica}

\begin{enumerate}
\item
    Título de la Sección: Emplea el título «Aplicación Práctica del ACP» para señalar el comienzo de esta sección.
\item
    Describe el conjunto de datos que utilizas, e indica los principales resultados obtenidos al hacer en ACP.
\item 
    Incluye el enlace a un Jupyter Notebook de Google Colab donde se realicen todo lo solicitado en este reto.
\end{enumerate}

\subsection{Parte de Conclusión}

    Concluye con una sección breve que resuma lo aprendido a través del reto, la utilidad del ACP en la exploración de grandes datasets y su impacto en la toma de decisiones basada en datos. Subraya cómo este método se integra en el ámbito más amplio del análisis de datos. No olvides incluir una bibliografía en formato APA para citar todas las referencias utilizadas.

%%%%%%%%%%%%%%%%%%%%%%%%%%%%%%%%%%%%%%%%
\section{Rúbrica de evaluación}
%%%%%%%%%%%%%%%%%%%%%%%%%%%%%%%%%%%%%%%%

\begin{itemize}
\item
    Definición y explicación (15 puntos): Se evaluará la claridad y precisión en la definición del ACP y la explicación de sus componentes.
\item 
    Minitutorial (15 puntos): Se valorará la calidad del tutorial para explicar el proceso del ACP.
\item 
    Aplicación práctica (15 puntos): Se evaluará la claridad y explicación del código del Jupyter Notebook.

\item
    Excelencia (5 puntos), este puntaje es dado únicamente si se tiene el puntaje completo en los otros puntos. Se premiará la creatividad en la presentación del contenido.
\end{itemize}

\end{document}